
%!TEX root = Animismus_in_Anime.tex
Die Verbindung von Animismus und Anime ist nicht nur dem Wort nach gegeben. Sie darf jedoch auch nicht einfach impliziert werden. Ob ein Zusammenhang besteht und in welcher Weise dieser gegeben ist, soll im Nachfolgenden untersucht werden. Dazu werden Filme von Regisseur und Produzent Hayao Miyazaki auf animistische Elemente untersucht und dabei festgehalten werden, was sie als solche ausmacht.

Angestossen wurde das Interesse vor allem daran, dass ein Film wie \textsc{Chihiros Reise ins Zauberland}\footnote{\textsc{Sen to Chihiro no kamikakushi / Chihiros Reise ins Zauberland}. (2001). Japan: Ghibli Studio. Drehbuch, Storyboard und Regie: Hayao Miyazaki. Produzent: Toshio Suzuki.} international einen sehr grossen Anklang gefunden hat, obwohl sehr viele darin vorkommende Elemente spezifisch für die japanische Kultur sind. Es stellt sich die Frage wie es kommt, dass ein solcher Film dennoch von Personen ausserhalb dieses Kulturkreises verstanden werden kann. Wenn man sich etwas mit Miyazakis Filmen beschäftigt so kann man verschiedene immer wieder auftauchende Motive erkennen. Zu den bedeutensten gehört die Faszination des Fliegens und der Kampf zwischen der Natur und den Menschen. In dieser Arbeit spielt vorallem letzteres eine zentrale Rolle. Es soll im Zusammenhang von Kultur und Religion, noch etwas genauer gesagt im Zusammenhang mit Animismus untersucht werden. Der Animismus stellt gerade heute einen schwer greifbaren Begriff dar. Wie viele Themen und Begriffe, welche der Kolonialzeit entsprungen sind, ist eine saubere Definition oder Abgrenzung schwierig. Dennoch ist es naheliegend von Animismus zu sprechen, wenn Wolf- und Keilergötter ihren Wald schützen oder ein von einem Feuergeist animiertes Schloss durch die Welt zieht. Wir finden hier also Dinge, aber insbesondere die Natur selbst als beseelt. 

Die in der Wissenschaft gebräuchlichen Beispiele von Geistern, Hexen und anderen übernatürlichen Phänomenen, sind für die Ethnien, aus denen diese Beispiele stammen, real. Im Unterschied dazu nimmt der westliche Betrachter eine vom Glauben unabhängige, distanziert-analytische Position ein. Die Betrachter von Filmen hingegen bilden eine dritte Herangehensweise, weil sie sich mit den Charakteren identizifieren und die fiktive Umgebungswelt leben können. Filme schaffen somit einen Rahmen, um zumindest temporär in ein System von übernatürlichen und phantastischen Phänomenen abzutauchen, ohne seine rationale und faktische Weltsicht aufgeben zu müssen. Diese Kombination aus Distanz und Nähe gibt uns die Möglichkeit uns selbst zu studieren, obwohl sich die Vorstellung von Animismus auf die Untersuchung von Völker und Ethnien konzentriert, die sich dem westlichen Kolonialismus zum Trotz ihre Kultur und Religion erhalten konnten. Das Studium des Animismus ist somit fast zwangsläufig eines, das die Studierenden aus ihrer unmittelbaren Umwelt weg führt. 

Es mag nach dieser Argumentation etwas widersprüchlich scheinen, dass ausgerechnet japanische Animationsfilme Gegenstand der Untersuchung sind. Es gibt im Bereich der Animationsfilme genügend Alternativen\footnote{Um ein Beispiel zu nennen: \textsc{Song of the Sea} (2014) und \textsc{The Book of Kells} (2009) produziert von Cartoon Saloon, einem irischen Animations Studio.}, welche kulturell näher stehen. Doch am Ende ist die kulturelle Prägung des Rezeptionisten für diese Untersuchung entscheidend, und nicht die des Filmes. 

In den Filmen \textsc{Prinziessin Mononoke}\footnote{\textsc{Mononokehime / Prinzessin Mononoke}. (1997). Japan: Ghibli Studio. Drehbuch, Storyboard und Regie: Hayao Miyazaki. Produzent: Toshio Suzuki.} und \textsc{Das wandelnde Schloss}\footnote{\textsc{Hauro no Ugoku Shiro / Das wandelnde Schloss}. (2004). Japan: Ghibli Studio. Drehbuch, Storyboard und Regie: Hayao Miyazaki. Produzent: Toshio Suzuki.} haben wir einerseits ist das alte Japan, reich an Mythen und Legenden. Hier wandeln Götter und Geister unter den Sterblichen. Wenn auch die Welt angereichert ist mit phantastischen Wesen, so erkennen wir darin doch unsere eigene Vergangenheit in dieser Welt. Andrerseits haben wir eine Märchenwelt, ein gutes Jahrhundert jüngeres Europa, wo Magie und Zauberei zum Alltag gehören. Diese Welt ist uns zeitlich zwar näher, doch gerade der breite Gebrauch von Magie in einer Zeit, an die wir uns noch zu erinnern glauben, entfremdet sie für den Betrachter. In beide Welten ist das Übernatürliche Grund zur Furcht oder Faszination, jedoch ist niemand über das Vorhandensein dessen überrascht.

Um die Frage zu beantworten, warum diese Filme, welche für ein japanisches Publikum gemacht sind auch bei uns Anklang finden, werde ich mit Pascal Boyers Ansatz der kognitiven Religionswissenschaft arbeiten. Eine ganz andere Sicht bietet hingegen Harvey. Seine Interpretation von Animismus basiert auf einem gegenseitigen respektvollen Umgang, nicht nur zwischen Menschen sondern zwischen allem was belebt ist.

\medskip

Im ersten Kapitel soll dargelegt werden wie Animismus für die vorliegende Arbeit zu verstehen sei. Dazu wird zuerst eine kurzere Übersicht zum historischen Begriff des Animismus und seine Verwendung gegeben. Es folgt ein kurzer Abschnitt über den japanischen Animismus, den Shintoismus. Danach werden wir Graham Harves Werk \emph{Animism. Respecting the Living World} betrachten um eine moderne Interpretation des Animismus kennen zu lernen.\footnote{Harvey, Graham. (2006). \emph{Respecting the Living World.} New York: Columbia University Press.} Mit Pascal Boyers \emph{Und Mensch schuf Gott} als Vertreter der kognitiven Religionswissenschaft bekommen wir weitere Kriterien zur Untersuchung animistischer Elemente.\footnote{Boyer, Pascal. (2004).\emph{Und Mensch schuf Gott.} Stuttgart: Klett-Cotta.} Dem Teil über Animismus folgt im zweiten Kapitel eine Vertiefung über japanische Animationsfilme und ein biografischer Abriss über Hayao Miyazaki, sowie die Analyse der Filme \textsc{Prinzessin Mononoke} und \textsc{Das wandelnde Schloss}. In Kapitel drei folgt die Anwendung der Methoden nach Boyer und Harvey. Schlussendlich steht die Frage ob es diese animistischen Elemente sind, welche den Filmen von Miyazaki helfen, auch bei einem nicht japanischen Publikum erfolgreich zu sein.
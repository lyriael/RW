%!TEX root = Animismus_in_Anime.tex
\section{Einleitung}
\subsection{Fragestellung}
In dieser Arbeit werden Filme von Regisseur und Produzent Miyazaki Hayao auf animistische Elemente untersucht und dabei festgehalten werden was sie als solche ausmacht.
Angestossen wurde das Interesse vor allem daran, dass ein Film wie Chihiros Reise ins Zauberland (orig. Titel) international einen sehr grossen Anklang gefunden hat, obwohl sehr viele darin vorkommende Elemente spezifisch für die japanische Kultur sind. Es stellt sich die Frage wie es kommt, dass ein solcher Film dennoch von Personen ausserhalb dieses Kulturkreises verstanden werden kann. Wenn man sich etwas mit Miyazakis Filmen beschäftigt so kann man verschiedene immer wieder auftauchende Motive erkennen. Zu den stärksten gehört die Kampf der Natur gegen die Menschen und die Faszination des Fliegens. Auf das zweitere soll hier nicht eingegangen werden, doch das erste soll in dieser Arbeit im Zusammenhang von Kultur und Religion, noch etwas genauer gesagt im Zusammenhang mit Animismus untersucht werden. Der Animismus stellt gerade heute einen schwer greifbaren Begriff dar. Wie viele Themen und Begriffe welche der Kolonialzeit entsprungen sind, ist eine saubere Definition und Abgrenzung schwierig. Dennoch ist es naheliegend von Animismus zu sprechen, wenn Wolf- und Keilergötter ihren Wald schützen (Mononoke), ein von einem Feuergeist animiertes Schloss durch die Welt zieht (das wandelnde Schloss), ein Fluss eine Badeanstalt besucht um seine Verschmutzung loszuwerden (Chihiros Reise ins Zauberland) oder ein Buss zwei Mädchen in Form einer grossen Katze transportiert (Mein Nachbar Totoro). Wir finden Dinge, aber insbesondere die Natur als beseelt. Um die Frage zu beantworten, warum diese Filme, welche für ein japanisches Publikum gemacht sind auch bei uns Anklang finden, werde ich mit Boyers Ansatz der \emph{Cognitive Religion} arbeiten. Eine ganz andere Sicht bietet hingegen Harvey.

Im ersten Teil soll ein kurzer Geschichtlicher Abriss über den Animismus gegeben werden. Danach folgt eine kurze Einführung über Harvey und seine moderne Interpretation von Animismus, gefolgt von Boyers Auffassung von Religion ausgelegt auf den Animismus. Im zweiten Teil folgt eine Analyse ausgewählter Elemente aus Miyazakis Filmen. Es wurden die vier Animes Mononoke, Totoro, Chihiro und Howl ausgewählt. Schlussendlich soll Boyer und Harvey auf die Filme angewendet werden.
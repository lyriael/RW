%!TEX root = Animismus_in_Anime.tex
In dieser Arbeit werden Filme von Regisseur und Produzent Miyazaki Hayao auf animistische Elemente untersucht und dabei festgehalten werden was sie als solche ausmacht.
Angestossen wurde das Interesse vor allem daran, dass ein Film wie \textsc{Chihiros Reise ins Wunderland} international einen sehr grossen Anklang gefunden hat, obwohl sehr viele darin vorkommende Elemente spezifisch für die japanische Kultur sind. Es stellt sich die Frage wie es kommt, dass ein solcher Film dennoch von Personen ausserhalb dieses Kulturkreises verstanden werden kann. Wenn man sich etwas mit Miyazakis Filmen beschäftigt so kann man verschiedene immer wieder auftauchende Motive erkennen. Zu den bedeutensten gehört die Faszination des Fliegens und der Kampf zwischen der Natur und den Menschen. In dieser Arbeit spielt vorallem letzteres eine zentrale Rolle. Es soll im Zusammenhang von Kultur und Religion, noch etwas genauer gesagt im Zusammenhang mit Animismus untersucht werden. Der Animismus stellt gerade heute einen schwer greifbaren Begriff dar. Wie viele Themen und Begriffe, welche der Kolonialzeit entsprungen sind, ist eine saubere Definition oder Abgrenzung schwierig. Dennoch ist es naheliegend von Animismus zu sprechen, wenn Wolf- und Keilergötter ihren Wald schützen (Mononoke), ein von einem Feuergeist animiertes Schloss durch die Welt zieht (das wandelnde Schloss), ein Fluss eine Badeanstalt besucht um seine Verschmutzung loszuwerden (Chihiros Reise ins Zauberland)\todo{fehlt} oder ein Buss zwei Mädchen in Form einer grossen Katze transportiert (Mein Nachbar Totoro)\todo{fehlt}. Wir finden hier also Dinge, aber insbesondere die Natur selbst als beseelt. 

Um die Frage zu beantworten, warum diese Filme, welche für ein japanisches Publikum gemacht sind auch bei uns Anklang finden, werde ich mit Pascal Boyers Ansatz der kognitiven Religionswissenschaft arbeiten. Eine ganz andere Sicht bietet hingegen Harvey.

Im zweiten Kapitel geht es darum eine genauere Vorstellung von der Bedeutung von Animismus zu erlangen. Es wird zuerst eine kurzere Übersicht zum historischen Begriff des Animismus und seine Verwendung gegeben, gefolgt von einem kurzen Abschnitt über den japanischen Animismus: den Shintoismus. Danach werden wir Graham Harves\footnote{Kommt hier irgendwas?} Werk \emph{Animism. Respecting the Living World} betrachten um eine moderne Interpretation des Animismus kennen zu lernen. Mit Pascal Boyers \emph{Und Mensch schuf Gott} als Vertreter der kognitiven Religionswissenschaft bekommen wir letztlich noch einmal eine ganz andere Sicht präsentiert.

Dem Teil über Animismus folgt im dritten Kapitel eine Vertiefung über japanische Animationsfilme und dem Starproduzent Hayao Miyazaki.

Da es im Rahmen dieser Arbeit unmöglich ist, sich mit Miyazakis Gesamtwerk zu beschäftigen, behandeln Kapitel vier und fünf mit den beiden ausgewählten Anime \textsc{Prinzessin Mononoke} und \textsc{Das wandelnde Schloss}. Der erste Film spielt in einem alten Japan, wo Wälder noch Geister und Götter beherbergten. Der zweite Film, welcher auf dem Buch einer Britin basiert spielt hingegen in einer Märchenwelt.

Im Kapitel 6 folgt dann die Analyse. Hier soll untersucht werden, wo Elemente von Harveys Animismus zu finden sind und ob sich Boyers Beschreibung vom Üernatürlichen anwenden lässt.

Schlussendlich steht die Frage ob es diese animistischen Elemente sind, welche den Filmen von Miyazaki helfen, auch bei einem nicht japanischen Publikum erfolgreich zu sein.
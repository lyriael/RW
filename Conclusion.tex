%!TEX root = Animismus_in_Anime.tex
\section*{Schlusswort}
Die Kriterien von Graham Harvey und Pascal Boyer haben es möglich gemacht animistische Elemente in den Filmen zu erkennen. Ein anderer Ansatz hätte auch sein können, den \emph{japanischen} Animismus zu untersuchen anhand der Filme zu untersuchen und zu schauen, wie genau er verwendet wird und was für Unterschiede zwischen dem was man im Film und dem historischen Shintoismus findet. Doch Ziel dieser Arbeit war, die Filme, als auch den Animismus interkulturell zu betrachten. Aus diesem Grund ist weder auf den Shintoismus noch auf Elemente der japanischen Kultur in der Analyse eingegangen geworden. Die dahinterliegende Absicht dieser Arbeit ist, eine mögliche Begründung zu finden, weshalb die Filme Miyazakis auch ausserhalb ihres Kulturkreises, wo viele Hinweise und Anspielungen unverstanden bleiben, trotzdem auf so grossen Anklang stossen. Gibt es eine Zusammenhang zwischen den animistischen Elementen und dem Erfolg dieser Geschichten?

Diese Frage lässt sich aus den Untersuchungen nicht beantworten. Wir sehen aber, dass der Animismus eine wesentliche Rolle in den Filmen spielt. Einerseits lässt sich das aus Harveys Sicht so begründen, dass wir in einer Welt, wo wir die Konsequenzen unseres gierigen und respektlosen Handelns gegenüber unserer Umwelt langsam zu spüren bekommen, die Vorstellung einer Utopie, wo in Harmonie mit der Umwelt mit all seinen Bewohnern gelebt wird, erstreben. Anderseits bietet uns Boyer eine Erklärung, warum wir Geschichte welche von Übernatürlichem Handeln (das schliesst Animismus mit ein), so faszinierend finden. Es könnte hierbei hilfreich sein, weitere Filme Miyazakis zu untersuchen. Zum Beispiel sein Erstlingswerk \textsc{Nausicaä aus dem Tal der Winde}, das in einer Welt spielt, in der Menschen die Erde nahezu zerstört haben und es um die Wiederversöhnung mit der Natur geht. Oder \textsc{Chihiros Reise ins Zauberland}, welches wie der Titel schon sagt, eine Fülle an übernatürlichen Wesen und Dinge bietet. Zudem müsste für eine weiterführende Interpredation auch japanische Ausdrücke und Namen vermehrt berücksichtigt werden.
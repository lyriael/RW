%!TEX root = Animismus_in_Anime.tex
\section*{Schlusswort}

Mag Anime und Miyazaki -> vielleicht doch besser ins Vorwort

Die Seherin vollführt ein Ritual um sich und ihr Dorf vor dem Zorn des Gefallen Gottes zu schützen. Diese beiden helfen aber nur ein religiöses Phänomen zu identifizieren, jedoch nicht umbedingt zu erklären, warum das auch für uns westler so interessant ist. Oder ist es gerade, dass das Ritual anders ist, aber die Notwendigkeit (Schablone davon) in uns allen vorhanden und wir es deswegen verstehen? Letztlich noch kurz die falschen Wildschweine: Unheimlich auch für den Betrachter. Umgekehrt von den Tiergöttern: PERSON die aussieht wie TIER.

Wir wissen, dass Calcifier das Schloss belebt - animiert. Doch wie wird dieser Eindruck, dass eine Konstruktion belebt ist an den Rezeptionist vermittelt?
Obwohl Calcifer das Schloss gebaut hat, steuert und auch seine Emotionen durch das Schloss geäussert werden, so es für den Rezipienten doch ein eigenes Wesen.

wir wollen alle heile-welt-animismus

boyers methoden bringen auch die rädchen in unseren hirn zum laufen, unabhängig von kultur

Ansatz hätte auch sein können: den \emph{japanischen} Animismus zu untersuchen. Schauen wie genau er verwendet wird und ob es änderungen gibt.

- H2 siehe da, wir können was machen wenn wir die sprache verstehen (!= Japanisch isch)

Musik wurde nicht berücksichtig

Filmanalyse nicht so erfolgreich. Hilft höchstens beim Detail aufmerksam machen.

Beantworte: Um die Frage zu beantworten, warum diese Filme, welche für ein japanisches Publikum gemacht sind auch bei uns Anklang finden, werde ich mit Pascal Boyers Ansatz der kognitiven Religionswissenschaft arbeiten. Eine ganz andere Sicht bietet hingegen Harvey. Seine Interpretation von Animismus basiert auf einem gegenseitigen respektvollen Umgang, nicht nur zwischen Menschen sondern zwischen allem was belebt ist.
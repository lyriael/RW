\documentclass[a4paper]{article}

%\usepackage[english]{babel}
\usepackage[utf8]{inputenc}
\usepackage{amsmath}
\usepackage{graphicx}
\usepackage[colorinlistoftodos]{todonotes}
\usepackage{enumitem, color, amssymb} % needed for roman numbering in enumerate enviroment
\usepackage[hyperfootnotes=false]{hyperref}
% deutsch
\usepackage[ngerman]{babel}
%\usepackage[ansinew]{inputenc}
\usepackage{graphicx}
\usepackage{caption}
\usepackage{subcaption}

\begin{document}
\begin{titlepage}
% Generiere Titel
	\begin{center}
	{\scshape\LARGE Institut für Religionswissenschaft Universität Bern \par}
	\vspace{1cm}
	{\huge\bfseries Animismus und Anime\par}
	\vspace{.5cm}
	{\large\bfseries Untersuchung zweier japanischer Animationsfilme von Hayao Miyazaki auf animistische Elemente.\par}
	\vspace{2cm}
	{\Large\itshape Judith Fuog\par}
	\vfill
	\end{center}

% Titelseiten Text
\begin{minipage}{\textwidth}
	\begin{large}
	Art der Arbeit: Grosser religionswissenschaftlicher Essay\par
	Studiengang laut RSP: BA Min. (30 ECT) Science of Religion\par
	Fachsemester: 4.\par
	\hfill

	eingereicht bei: Prof. Dr. Jens Schlieter\par
	Abgabedatum: \today\par
	\hfill

	Mat.-Nr.: 09-926-809\par
	Anschrift: Sängergasse 25, CH-4054 Basel\par
	E-mail: judith.fuog@students.unibe.ch\par
	Tel.: 076 572 14 19\par
	\end{large}
\end{minipage}
\hfill
\end{titlepage}
\newpage
\tableofcontents
\newpage

%%%%%%%%%%%%%%%%%%%%%%%%%%%%%%%%%%%%%%%%%%%%%%%%%%%%%%%%%
% Text
\addcontentsline{toc}{section}{Einleitung}
\section*{Einleitung}

%!TEX root = Animismus_in_Anime.tex
Die Verbindung von Animismus und Anime ist nicht nur dem Wort nach gegeben. Sie darf jedoch auch nicht einfach impliziert werden. Ob ein Zusammenhang besteht und in welcher Weise dieser gegeben ist, soll im Nachfolgenden untersucht werden. Dazu werden Filme von Regisseur und Produzent Hayao Miyazaki auf animistische Elemente untersucht. In der Folge soll festgehalten werden, was diese als solche ausmacht.

Angestossen wurde das Interesse vor allem daran, dass ein Film wie \textsc{\mbox{Chihiros} Reise ins Zauberland}\footnote{\textsc{Sen to Chihiro no kamikakushi / Chihiros Reise ins Zauberland}. (2001). Japan: Ghibli Studio. Drehbuch, Storyboard und Regie: Hayao Miyazaki. Produzent: Toshio Suzuki.} international einen sehr grossen Anklang gefunden hat, obwohl sehr viele darin vorkommende Elemente spezifisch für die japanische Kultur sind. Es stellt sich die Frage, wie es kommt, dass ein solcher Film dennoch von Personen ausserhalb dieses Kulturkreises verstanden werden kann. Wenn man sich etwas mit Miyazakis Filmen beschäftigt, so kann man verschiedene immer wieder auftauchende Motive erkennen. Zu den bedeutendsten gehört die Faszination des Fliegens und der Kampf zwischen der Natur und den Menschen. In dieser Arbeit spielt Letzteres eine zentrale Rolle. Es soll im Zusammenhang von Kultur und Religion, genauer im Zusammenhang mit Animismus untersucht werden. Der Animismus stellt gerade heute einen schwer greifbaren Begriff dar. Bei wievielen Themen und Begriffen, welche der Kolonialzeit entsprungen sind, ist eine saubere Definition und Abgrenzung schwierig. Dennoch ist es naheliegend von Animismus zu sprechen, wenn Wolf- und Keilergötter ihren Wald schützen oder ein von einem Feuergeist animiertes Schloss durch die Welt zieht. Wir finden hier also Dinge, aber insbesondere die Natur selbst, als beseelt vor. 

Die in der Wissenschaft gebräuchlichen Beispiele von Geistern, Hexen und anderen übernatürlichen Phänomenen, sind für die Ethnien, aus denen diese Beispiele stammen, real. Im Unterschied dazu nimmt der westliche Betrachter eine vom Glauben unabhängige, distanziert-analytische Position ein. Die Be\-trachter von Filmen hingegen nehmen eine dritte Position ein, weil sie sich mit den Charakteren identizifieren und die fiktive Umgebungswelt leben können. Filme schaffen somit einen Rahmen, um zumindest temporär in ein System von übernatürlichen und phantastischen Phänomenen abzutauchen, ohne eine rationale und faktische Weltsicht aufgeben zu müssen. Diese Kombination aus Distanz und Nähe bietet die Möglichkeit uns selbst zu studieren, obwohl sich die Vorstellung von Animismus auf die Untersuchung von Ethnien konzentriert, die sich dem westlichen Kolonialismus zum Trotz ihre Kultur und Religion erhalten konnten. Das Studium des Animismus ist somit fast zwangsläufig eines, das die Studierenden aus ihrer unmittelbaren Umwelt weg führt. 

Es mag nach dieser Argumentation etwas widersprüchlich scheinen, dass ausgerechnet japanische Animationsfilme Gegenstand der Untersuchung sind. Es gibt im Bereich der Animationsfilme genügend Alternativen\footnote{Um ein Beispiel zu nennen: \textsc{Song of the Sea} (2014) und \textsc{The Book of Kells} (2009) produziert von Cartoon Saloon, einem irischen Animationsstudio.}, welche kulturell näher stehen. Doch am Ende ist die kulturelle Prägung des Rezeptionisten für diese Untersuchung entscheidend, und nicht die des Filmes. 

In den Filmen \textsc{Prinzessin Mononoke}\footnote{\textsc{Mononokehime / Prinzessin Mononoke}. (1997). Japan: Ghibli Studio. Drehbuch, Story\-board und Regie: Hayao Miyazaki. Produzent: Toshio Suzuki.} und \textsc{Das wandelnde Schloss}\footnote{\textsc{Hauro no Ugoku Shiro / Das wandelnde Schloss}. (2004). Japan: Ghibli Studio. Drehbuch, Storyboard und Regie: Hayao Miyazaki. Produzent: Toshio Suzuki.} haben wir einerseits das alte Japan, reich an Mythen und Legenden. Hier wandeln Götter und Geister unter den Sterblichen. Wenn auch diese Welt ange\-reichert ist mit phantastischen Wesen, so erkennen wir darin doch unsere eigene Vergangenheit. Andrerseits haben wir eine Märchenwelt, ein gutes Jahrhundert jüngeres Europa, wo Magie und Zauberei zum Alltag gehören. Diese Welt ist uns zeitlich zwar näher, doch gerade der breite Gebrauch von Magie in einer Zeit, an die wir uns noch zu erinnern glauben, entfremdet sie für den Betrachter. In beiden Welten ist das Übernatürliche Grund zur Furcht oder Faszination, jedoch ist niemand über das Vorhandensein dessen überrascht.

Um die Frage zu beantworten, warum diese Filme, die ja für ein japanisches Publikum gemacht sind, auch bei uns Anklang finden, werde ich mit Pascal Boyers Ansatz der kognitiven Religionswissenschaft arbeiten. Eine ganz andere Sicht bietet hingegen Harvey. Seine Interpretation von Animismus basiert auf einem gegenseitigen respektvollen Umgang, nicht nur zwischen Menschen, sondern zwischen allem was belebt ist.

\subsection*{Vorgehensweise}
Im ersten Kapitel soll dargelegt werden, wie Animismus für die vorliegende Arbeit zu verstehen sei. Dazu wird zuerst eine Übersicht zum historischen Begriff des Animismus und seiner Verwendung gegeben. Es folgt ein kurzer Abschnitt über den japanischen Animismus, den Shintoismus. Danach werden wir Graham Harveys Werk \emph{Animism. Respecting the Living World} betrachten, um eine mo\-derne Interpretation des Animismus kennen zu lernen.\footnote{Harvey, Graham. (2006). \emph{Respecting the Living World.} New York: Columbia University Press.} Mit Pascal Boyers \emph{Und Mensch schuf Gott} als Vertreter der kognitiven Religionswissenschaft gewinnen wir weitere Kriterien zur Untersuchung animistischer Elemente.\footnote{Boyer, Pascal. (2004). \emph{Und Mensch schuf Gott.} Stuttgart: Klett-Cotta.} Dem Teil über Animismus folgt im zweiten Kapitel eine Vertiefung über japanische Animationsfilme und ein biografischer Abriss über Hayao Miyazaki, sowie die Analyse der Filme \textsc{Prinzessin Mononoke} und \textsc{Das wandelnde Schloss}. In Kapitel drei folgt die Anwendung der Methoden nach Boyer und Harvey. Zum Schluss stellt sich die Frage, ob es diese animistischen Elemente sind, welche den Filmen von Miyazaki helfen, auch bei einem nicht japanischen Publikum erfolgreich zu sein.
\newpage

\section{Animismus}
%!TEX root = Animismus_in_Anime.tex
\section{Animismus in der Moderne}
Das lateinische Wort \emph{anima}\footnote{Das kann man auch anders übersetzten!! wiki sagt: Wind, Hauch} für Seele lässt den Animismus wurde von Stahl erstmals eingeführt und durch Edward Tylor in \emph{Primitive Culture} in 1871 eingeführt. Wie bei vielen Begriffen in der Religionswissenschaft, trägt der Begriff mehrere Bedeutung und kann verschieden verwendet werden. In der Regel wird bei einer animistischen Religion von einer schriftlosen Religion aus. Früher wurden die gerne als Natur-, achaische oder primitive Religionen bezeichnet. In diesem Zusammenhang, aber nicht deckend, versteht man unter Animismus auch den Glauben an eine beseelte Umwelt. Somit ist der Mensch nicht das einzige beseelte Wesen, sonder auch Tiere und Naturobjekte können beseelt sein. Letztlich kann mit Animismus auch einfach der Glaube an Geister und Seelen verstanden werden.~\footnote{RGG 1: 504} 
Religionssoziologie\todo{sollte vielleicht erwähnt werden?}

Es soll an dieser Stelle zunächst ein kurzer Historischer Abriss des Animismus gegeben werden. Danach wird Graham Harveys Animismus (\emph{Respecting the Living World}) als Vertreter eines modernen Animismus vorgestellt. Schliesslich verlassen wir das ausdrückliche Gebiet des Animismus um eine ganz andere Perspektive auf Religion zu haben und wenden uns noch der kognitiven Religionswissenschaft zu. 

\subsection{Der alte Animismus}

Der erste, welcher den Begriff Animismus verwendet hat war Georg Ernst Stahl gewesen. Er stellte die Theorie auf, dass es ein physikalisches Element gäbe, welches belebt. Eine lebendige Person hat demensprechend viel anima, während eine tote Person, oder ein Stein kein anima (mehr) hat. Dabei gibt es eine Abstufung, so dass Tiere und Pflanzen auch anima besitzen, jedoch weniger als der Mensch.~\footnote{\textsc{Harvey 06: 3-4}}

James Frazer (1854-1914) stellt die Theorie auf, dass die Wilden (savage) Pflanzen und Tiere genau so beseelt glaubten wie die Menschen. Der Animismus werde dann zum Polytheismus, wenn dann die Wilden beginnen zu glauben, dass Pflanzen und Tiere nur temporär durch eine andere Wesenheit beseelt seien.~\footnote{\textsc{Harvey 06: 3-4}} 

Edward Tylor (1932-1917) beschreibt in seinem Werk \glqq Primitive Culture\grqq (Die Anfänge der Cultur) den Animismus als der Ursprung der Religion. Der Animismus würde dann, im Laufe der Weiterentwicklung und Zivilisierung einer Kultur durch verschiedene andere Formen der Religion abgelöst werden. Doch auch in einer hochentwickelten und komplexen Religion würden sich noch Überreste der alten Religion in Form von Aberglaube finden.

Robert R. Marett (1866-1943) kritisierte Tylors Religionstheorie welche den Animismus als Ursprungsreligion setzten weil diese Phänomene wie Ehrfurcht vor Tieren, Blut oder Naturgewalten nicht berücksichtigten.Marett führt die Dichotomie vom Alltäglichen und vom Ausseraltäglichen ein, wobei letzteres durch Religion erklärt und verarbeitet würde. Das Ausseralltägliche teilt er weiter in die Begriffe Mana und Tabu. Mana beschreibe die Begegnung mit einer übermenschlichen Macht, während Tabu für Furcht und Kontaktvermeidung aufgrund von Gefahr steht. Er ordnet dann Religion dem Mana an, während er das was mit Tabu verbunden wird als Magie bezeichnet.\footnote{\texttt{https://de.wikipedia.org/wiki/Robert\_Ranulph\_Marett} 25.08.15} 

Emil Durkheim (1858-1917) setzt Anstelle des Animismus den Totmismus als Ursprungsreligion. Seiner Meinung nach ständen soziale Aspekte über den Erfahrungen eines Individuums.

Zusammenfassend kann man sagen, dass der \glqq alte\grqq Animismus als Vorstufe für eine bessere Religion gesehen wurde. In der Geburtsstunde der Religionssoziologie und der Anthropologie war der Westen überzeugt, dass es eine lineare Entwicklung gibt, wobei der Westen ganz oben an dieser Skala steht. Diese Ansicht gilt heute als veraltet und somit haben auch frühere Werke über Animismus ihre Bedeutung in der heutigen Religionswissenschaft eingebüsst. Es gibt aber auch Versuche den Animismus neu zu definieren, - ihm eine Bedeutung in der Moderne zu geben. 

\subsection{Ansätze für einen Modernen Animismus}
\subsubsection*{Moderner Animismus}
Der Animismus steckt heute in so fern in einer Krise, da auf den \glqq alten\grqq Animismus nicht weiter aufgebaut werden kann. Anderseits sind die Phänomene des Animismus weiterhin interkulturell präsent. Die Phänomene sind weiter hin von Ethnologen und Psychologen ernst genommen worden. Als Konsequenz werden sie als kognitiver Fehler\footnote{In diesem Zusammenhang siehe nächstes Kapitel.}, als Projektion, als Produktion einer überproduktiven Phantasie oder einer mangelnden Trennung von subjektiver und objektiver Welt eingeschätzt. 

Auf der anderen Seite wirkt die Moderne zu Gunsten des Animismus. Früher wurden Kulturen belächelt und man bezeichnete sie als primitiv, wenn sie an Naturgeister glauben und diese anbeteten. Hundert Jahre später sehen wir unsere Existenzgrundlage bedroht, weil wir unseren Umgebung rücksichtslos ausgenommen haben. Es ist daher verständlich, dass Haltungen welche die Natur in ein Gegenüber stellt, mit dem man (respektvoll) interagieren kann eine gewisse Sympathie erfährt.

\subsubsection*{Graham Harvey}
\begin{quote}
	Animists are people who recognise that the world is full of persons, only some of whom are human, and that life is always lived in relationship with others. Animism is lived out in various ways that are all about learning to act respectfully towards and among other persons.\cite{harvey06}
\end{quote}

Mit dieser Aussage beginnt Graham Harvey sein Buch \emph{Animism. Respecting the Living World.}\footnote{\textsc{harvey06}}. Es ist hier bereits erkennbar, dass es Harvey in erster Linie darum geht eine Lebenshaltung zu postulieren. Er erklärt anhand vieler Beispiel welche er im Laufe seiner Forschung bei den *** gemacht wie dieser Animismus zu verstehen ist. Aber wichtiger scheint es zu sein, dass sein Anliegen, eine Aufforderung zu mehr Respekt auf dem Gegenüber was wir nicht (er)kennen zu zollen, beim Leser ankommt. So redet er auch weniger vom Animismus, als von Animisten.

Harvey macht ebenfalls eine Unterscheidung vom \glqq alten\grqq und \glqq neuen\grqq Animismus. In der alten Vorstellung ging man davon aus, dass Animisten Menschen sind, welche nicht zwischen Objekten und Subjektion unterscheiden konnten oder wollten. Neue Animisten hingegen suchen Wege und Ansichten wie sie mit andern Leuten richtig und respektvoll interagieren können. Zentral in Harvey Buch ist das Zusammenfassen von Menschen (humans) und Andere-als-Menschen zu einer Übergruppe von Leuten (people). Es gibt also Leute, welche nicht Menschen sind. Es handelt sich aber dennoch um Leute mit denen man interagieren kann (oder muss). Unter den Leuten gibt es hinterlistige und verschlagene Personen (Menschen und Andere-als-Menschen) und es ist wichtig allfällige Masken, Täuschungen und falsche Aussagen zu durchschauen zu können. Im Wissen, dass es Leute gibt, welche uns gerne essen möchten, ist es weise sowohl vorsichtig als auch konstruktiv im respektvollen Umgang mit andern zu sein.

Ich werde hier auf zwei seiner Beispiele eingehen. Das erste handelnd von den Ojibwe, einem Nordamerikanischen Indianerstamm. Seine Überlegungen stützen sich hauptsächlich auf Irving Hallowells Beobachtungen und Untersuchungen wobei die Sprache im Zentrum steht.~\footnote{Zitiere Hallowells Buch} Als zweites soll Harveys ÜBerlegungen zu Maori Kunst dargestellt werden. Nebst seinen persönlichen Erfahrungen\todo{Stimmt das?} benutzt er XX\todo{wer bitte?} als primäre Quelle.

\subsubsection*{Die Sprache der Ojibwa}
Die Ojibwa geben uns ein Beispiel dafür, dass sich Animismus in der Grammatik der Sprache zeigen kann. So wie es im Deutschen (und ähnlichen Sprachen) eine Untescheidung zwischen männlich und weiblich gibt, unterscheidet die ojibwe Grammatik zwischen belebt (animated) und leblos (inanimated). Die Unterscheidung ist aber nicht selbstredend. So wie wir bei uns \glqq die Tasse\grqq oder \glqq der Hund\grqq sagen, ist das nicht immer eine eindeutige Aussage über das Geschlecht des Beschriebenen. In der ojibwe Sprache geschreibt die Grammatik die Steine als animiert. Doch als Antwort auf die Frage ob denn alle Steine leben würde, antwortete ein alter Ojibwe mit: \glqq Nein. Aber ein paar schon.\grqq\footnote{\textsc{Harvey 06: 33}}. Aus diesem Beispiel geht hervor, dass der Animismus hier kein dogmatisches Glaubenssystem ist. Es ist möglich, dass ein Stein animiert ist, jedoch lässt sich diese Aussage nicht auf alle Steine übertragen. Für diese Animisten ist also nicht grundsätzlich alles belebt.

Eine weitere Anekdote erzählt von einem Stein, der durch einem weissen Händler ausgegraben wurden. Der Händler dachte dass er zu einer zeremoniellen Pavillon gehöre, also suchte er einen Ojibwa namens John auf. John beuge sich zum Stein und frage den Stein leise, ob er zu diesem Pavillon gehöre. Laut John antwortete der Stein, dass dem nicht so sei. Das wichtige was wir hier sehen ist, dass mit dem Stein wie mit einer Person umgegangen wurde. John sprach nicht \emph{zu} sondern \emph{mit} dem Stein.\footnote{\textsc{Harvey 06: 37}} 

Es gibt auch Erzählungen bei den Ojibwa von Steinen welche belebt sind und anthropomorphe Merkmale besitzen. Zum Beispiele Steine die so geformt sind, dass es aussieht als ob sie einen Mund, oder Augen haben. Solche Merkmale werden aber nicht zwingend als Hinweis zur Beseeltheit des Steines gelesen. Das Aussehen kann trügen. Ein Stein gilt als animiert, wenn mit ihm gesprochen werden kann. Wenn man mit ihm wie mit andern Personen interagieren kann.

Ein anderes und weitaus abstrakteres Beispiel findet sich bei den Saisongeschichten (Seasonal Stories). Der Umgang mit diesen Geschichten entspricht dem respektvollen Umgang mit einer Person. Tatsächlich werden diese Geschichten Grossvater genannt und sind entsprechend auch ehrwürdig.\footnote{\textsc{Harvey 06: 42}} Man beschäftigt sich nicht leichtfertig mit diesen Geschichten, und auch wenn sie auch lustig sein können, so nimmt man sie doch ernst. Sie vermitteln Dinge grosser Wichtigkeit, wenn man sich ihnen respektvoll annähert.

\subsubsection*{Die Kunst der Maori}
Maori sind für ihre kunstvollen Schnitzereien von Pounamu Steinen\footnote{Sammelnbezeichnung der Maori für Nephrit-Jade und Bowenit. Im Englischen werden diese Steine schlicht \emph{greenstone} genannt.}, Knochen und Holz berühmt. Harvey möchte zeigen, dass diese Kunstwerke selbst (durch den Macher) beseelt sind. 

Die Maori fühlen eine tiefe Verwandtschaft mit dem Ort an dem sie leben. Ein junger Mensch entwickelt sich in Abhängigkeit seiner Familie und seines Clans, aber auch die Natur gehört zu seinen Vorvätern. Das Land wird als Quelle der Identität betrachtet. Es gehört und wird kontinuierlich geteilt von den Toten, den Lebenden und den Ungeborenen.\footnote{\texttt{http://www.justice.govt.nz/publications/publications-archived/2001/he-hinatore-ki-te-ao-maori-a-glimpse-into-the-maori-world/part-1-traditional-maori-concepts/whenua}} Es ist zum Beispiel Brauch, dass bei der Geburt eines Kindes die Plazenta vergraben wird. Somit ist das Neugeborene mit dem Ort verbunden.

Die Maori sehen in der Süsskartoffel nahe Verwandte, ohne deren Hilfe den Maori eine wichtige Nahrungsgrundlage fehlen würde. Ohne die Hilfe der MAori würde die Pflanze jedoch gar nicht erst wachsen und gedeien können. Die Kartoffeln aus zugraben und zu essen grenzt daher an Kannibalismus.\footnote{Kannibalismus ist unter Maori durchaus üblich. Dabei geht es in keiner Weise darum sich vom Menschenfleisch zu ernähren. Die Einverleibung fand von Freunden und Feinden statt.}

Das Schnitzen von Knochen, welche in jedem Menschen vorhanden sind, stellt keinen grösseren Eingriff dar, als das Fällen und Schnitzen von Bäumen und das Schnitzen von Holz. Eine Schnitzerei steht somit immer im Zusammenhang dem Nehmen von Leben. Die Überreste einer Schnitzerei werden jeweils zurückgegeben. Die kunstvolle Schnitzerei ist nicht dafür da, um davon abzulenken. Durch das Schnitzen findet eine Transformation statt, in der der Künstler das Potential das im Holz, Stein oder Knochen schlummert hervor bringt. Ein Pounamu Anhänger ist belebt und nicht einfach nur Schmuck oder Identität für den Träger. Er hat ein Geschlecht, einen Namen und verdient Respekt. 

Ein Wharenui, das Gemeinschaftshaus einer Maori Gruppe, gilt als belebt. Durch die kunstvollen Schnitzereien wird der Vorfahre enthüllt und präsentiert. Das Haus \emph{isst} jene, welche eintreten und transformiert das \emph{tapu}-Neuheit in eine \emph{noa}-Normalität.\footnote{Tapu und Noa?} Der Gast ist somit nicht zum Einheimischen geworden, aber statt der Befindlichkeit findet man eine Normalität, selbst wenn diese nicht alltäglich ist. 

\smallskip
Die beiden aufgezeigten Kulturen in welchen Harvey von Animismus redet zeigen, dass es sehr grosse Unterschiede darin gibt, wie Animisten mit der Welt um sie herum agieren. Dabei decken diese beiden Beispiele nur einen sehr kleinen Teil der Aspekte ab, welche Harvey dem Begriff Animismus zusammenfasst. Weitere Beispiele sollen bei der Filmanalyse an passender Stelle gezeigt werden.

\subsection{Die Sicht der kognitiven Religionswissenschaft}
In dieser Underdisziplin der Religionswissenschaft wird Religion oder religiöse Phänomene aus der Perspektive der Kognitions- und Evolutionswissenschaft betrachtet. Es wird versucht zu erklären, weshalb religiöse Praktiken und und Denkweisen universell verbreitet sind.~\footnote{wikipedia: \texttt{https://en.wikipedia.org/wiki/Cognitive\_science\_of\_religion} und \texttt{https://de.wikipedia.org/wiki/Kognitive\_Religionswissenschaft} 24.08.15}

Die kognitive Religionswissenschaft ist eine eher junge Disziplin, welche sich erst Ende des zwanzigsten Jahrhunderts etabliert hat. Zu ihren Begründern gehören unter anderen E. Thomas Lawson und Robert McCauley (\emph{Rethinking Religion: Connecting Cognition and Culture and Bringing Ritual to Mind: Psychological Foundations of Cultural Forms}), Pascal Boyer (\emph{Naturalness of Religous Ideas}) und Guthrie (\emph{Faces in the cloud}).

An dieser Stelle soll Pascal Boyers \emph{Und Mensch schuf Gott}~\footnote{\cite{boyer04}} als Stellvertreter für andere kognitive Ansätze dienen um die Animationsfilme auf Animismus zu untersuchen. 

Im Geiste des Menschen ist zu suchen. Jeder menschlicher Geist hat das Zeug religiös zu sein. Boyers Theorien stützen sich auf Funktionieren des Geistes im Allgemeinen, unabhänig von Kultur. Das scheint zu nächst im Widerspruch, da es ja gerade kulturell grosse Unterschiede bezüglich religiöser Praktiken und Vorstellungen gibt. Das geniale hier sei, dass sich etwas so \emph{vielschichtiges} wie Religion durch etwas erklären lässt, was überall gleich ist (das Gehirn). Es ist jedoch notwendig zunächst mehr darüber zu wissen, wie das Gehirn Informationen aufnimmt und verarbeitet.[S.11]

Die Arbeit die ein Gehirn leistet ist lange unterschätzt worden. Einerseits muss man von der verbreiteten Annahme wegkommen, dass es sich beim Geist um ein leeres Gefäss handelt welches beliebig mit Informationen (Erziehung, Bildung und persönliche Erlebnissen) gefüllt werden kann. Anderseits auf von der Idee, dass der Geist mit wahllosen Informationen abgefüllt werden kann. Wir können uns bei weitem nicht alles merken und das ist auch gut so. Es braucht also etwas im Hirn, das relevante Informationen aus der Umwelt identifiziert und auf eine spezifische Weise zu verarbeiten mag. [S. 12]

Die relevanten Informationen werden nicht mit den Genen weiter gegeben. Aber das System, welches die Arbeit welche dahinter liegt verrichtet schon. Denn wenn man ein normale menschliches Gehirn besitzt kann daraus noch nicht geschlossen werden, dass dieser Mensch auch Religion hat. Es bedeutet lediglich, dass sich dieser Mensch Religion zu eigen machen kann. Daher ist die wichtige Frage: Wie muss der Nährboden für Informationen aussehen, damit sie als relevant gelten und erfolgreich verarbeitet werden. 

Bei der Frage nach dem Ursprung der Religion tauchen immer wieder ähnliche intuitive Begründungen auf: \emph{Die Religion bietet Erklärungen, Die Religion spendet Trost, Die Religion sichert die gesellschaftliche Ordnung, Die Religion ist eine kognitive Täuschung}~\footnote{\textsc{Boyer 04: 14-15}}. Laut Boyer sind diese intuitiven Gewissheiten in ihrer Existenz zwar berechtigt, jedoch nicht dienlich dabei den Ursprung zu finden. Einen Ursprung im Sinne eines historischen Ereignis ist eine Wunschvorstellung, welche aus dem Wunsch entspringt eine Ursache zu haben, aus der sich alle weiteren Phänomene ableiten lassen würden. Ausführlich zeigt Boyer, wie man in jeder dieser Vorstellung Widersprüche findet oder sie schlicht nicht befriedigend sind, welche sie als Ursprung ungeeignet machen. Doch für jedes Gebiet, das er abarbeitet fügt er am Ende \glqq[einen] andere[n] Blickwinkel\grqq  hinzu, in dem er aus kognitiver Sicht den Wert dieser intuitiven Annahme beschreibt:

\begin{itemize}
	\item Das Erkenntnissystem des Gehirns produziert Erklärungen, oft ohne dass wir uns dessen gewahr sind. - Religion als Erklärung.
	\item Emotionale Programme sind für uns lebenswichtig~\footnote{Hier das Beispiel der Angst vor einem Raubtier. [S.34]} und somit ein Aspekt unseres entwicklungsgeschichtlichen Erbes. - Emotionen in der Religion.
	\item Die Untersuchung des sozialen Bewusstseins (soziale Intelligenz) kann Antworten auf die Frage nach den Erwartungen an das gesellschaftliche Leben und Moral geben. - Religion, Moral und Gesellschaft.
	\item Bei all den Übernatürlichen Informationen welcher der Geist bekommt, werden nur manche als plausibel erachtet und so angeeignet. - Religion und Denken.
\end{itemize}

Boyer ist sehr grosszügig mit einleuchtenden Beispielen, jedoch ist nicht immer ganz klar, wie viel von dem was er sagt wissenschaftlich erwiesen ist, und wie viel davon Spekulation ist. 

Jedenfalls führt er nun weitere Konzepte ein, welche für seinen Ansatz unausschliessbar sind. Er zählt diese als den Inhalt eines Werkzeugkasten auf. Für die weitere Arbeit werden diese Werkzeuge eine zentrale Rolle spielen, da diese die Methode zur Analyse liefern aus Sicht der Kognitiven Religionswissenschaft.

\subsubsection*{Meme}
Die Bezeichnung Meme als Kulturelemente, also Vorstellungen, Werte, Geschichten und dergleiche, welche Menschen in ihrem Handeln beeinflusst und weitergegeben werden,~\footnote{\textsc{Boyer} 04: 50} wurde vom Evolutionsbiologen Richard Dawkins erstmals vorgestellt. Ein Meme bezeichnet demnach einen Bewusstseinsinhalt\footnote{wiki: \texttt{https://de.wikipedia.org/wiki/Mem} 24.08.15}, welcher durch Kommunikation in der Gesellschaft weitergegeben und somit vervielfältigt werden kann. Es ist das soziokulturelle Pendant zu den biologischen Genen in der Evolutions. So dann lassen sich die Meme auch ähnlich wie die Gene beschreiben: Information wird durch Kommunikation weiter gegeben (replizieren). Dadurch werden die Inhalte aber nicht einfach verbreitet, sonder auch leicht (oder schwerwiegend) abgeändert (mutieren).~\footnote{Die Information ändert sich nicht erst durch deren Weitergabe. Etwas das wir erfahren wird bei verschiedenen Menschen bereits anders verarbeitet. Somit können zwei Personen genau das gleiche hören und eine andere Version des Inhalts in ihrem Bewusst sein haben.} Schliesslich werden nur jene Memes tatsächlich weiter gegeben oder überhaupt erst erinnert, welche einprägsam sind (selektieren). Die Frage wäre nun, wo oder was entscheidet welche Memes weitergegeben werden und welche durch das Raster fallen, weil sie nicht relevant sind? 

Boyer beschreibt die Meme als wunderbare Ausgangslage, will ihnen aber nicht mehr als genau das zugestehen. Seine Kritik liegt darauf, dass es keine Replikation der eigentlichen Informationen gibt. Inhalte werden nicht faktisch übergeben, sondern jeweils neu konstruiert. Zwei Menschen können zwei faktisch identische Aussagen machen, aber jeder hat die Information, welche er wiedergibt auf seine eigene Art rekonstruiert. Entsprechend stellt Boyer als nächstes seine Theorie zum einfangen von Vorstellungen durch Schablonen vor. 

\subsubsection*{Schablonen und Vorstellungen}
Ein grosser Teil dessen, was wir wissen musste uns niemand faktisch erzählen. Eine erstaunliche Eigenschaft unserer unserer geistigen Fähigkeit ist es durch die Kombination bereits existierendem Wissen und der Hinzugabe einer neuen Information mehr Wissen zu generieren. Boyer zeigt dies anhand eines Beispiels mit einem Kind auf, mit dem Kind als eine Person, dessen Wissen erweitert wird. Zeigen wir einen Kind zum ersten Mal ein Walross, so hat es keine weiteren Informationen darüber als den Namen und seine äussere Erscheinung. Dennoch wird das Kin erwartet, dass das Walross isst, schläft und dass es Kinder haben kann. Diese Information über das Walross hat das Kind geschlussfogert, indem es eine Annahme gemacht hat: Das Walross ist ein TIER. Tiere essen, schlafen und bekommen Kinder. Boyer bezeichnet TIER als eine \emph{Schablone}. Mit dieser TIER-\emph{Schablone} hat das Kind eine Walross-Vorstellung gebildet.\footnote{\textsc{Boyer 04: 59}}   

Boyer stellt es zur Grundannahme, dass es deutlich weniger Schablonen\footnote{Beispiele die er nennt und auch öfters gebraucht: TIER, WERKZEUG, UNREINE SUBSTANZ, NATUROBJEKT, PERSON, PFLANZE. Bei diesen Schablonen handelt es sich vorwiegend um konkrete Dinge. Boyer erwähnt aber auch GESICHT als eine Schablone im abstrakten Sinne (\emph{das Gesicht verlieren}). \textsc{Boyer 04: 61}} als Vorstellungen gibt und dass diese Schablonen universell sind, im Gegensatz zu den Vorstellungen. 

\subsubsection*{epidemiologisches Modell}
Als weiteres Element in seinem Werkzeugkasten stellt Boyer die Kulturepidemie~\footnote{\textsc{Boyer 04: 62ff}} vor. Religiöse Vorstellungen und Phänomene betreffen in der Regel eine beliebig grosse Gruppe von Menschen.
\todo{Haaaa?}
Man kann also Religion als eine besondere Form der mentalen Epidemie erklären. Durch die Ausbreitung der Epidemie formen Menschen (auf Basis von unterschiedlichen Informationen) ähnlich strukturierte Formen religiöser Vorstellungen und Normen.\footnote{\textsc{Boyer 04: 64}} Hier setzt nun das Konzept der Schablonen und Vorstellungen an. Die Vorstellungen welche von Angehörigen einer gleichen Gruppe anhand einer Schablone hergestellt werden sind sich in der Regel ähnlich. Die Vorstellungen einer anderen Gruppe kann stark davon abweichen, trotzdem die gleiche Schablone benutzt wurde. So lasse sich das Beispiel mit den Tier Vorstellungen auch auf religiöse Vorstellungen übertragen. Demnach gibt es eine Schablone für religiöse Vorstellungen. Wie bei der Tierschablone können religiöse Vorstellungen übereinstimmen (einigermassen ähnlich strukturiert sein), obwohl die Information auf deren sie aufbaut von Mensch zu Mensch verschieden ist. Letztlich muss berücksichtig werden, dass die kulturelle Varianz in der Regel geringer ist, als man annimmt. Beim Übermitteln findet durch die Schablonen ein Filtern der gegebenen Informationen statt, so dass daraus voraussagbare Strukturen gebaut werden.[65]

\subsubsection*{Beschaffenheit des übernatürlichen}
Es folgt also die Suche nach dem mentalen Rezept für religiöse Vorstellungen. Mit einem Versuch verschiedener mehr oder weniger potenten religiösen Aussagen versucht Boyer dem Leser zu zeigen, dass man der Intuition folgend gewisse Aussagen direkt ausschliessen kann, während andere absolut denkbar wären als religiöse Vorstellung. 

Folgende Anleitung kristallisiert sich mit der Zeit heraus: In einer Aussage wird ein Vertreter einer ontologischen Kategorie\footnote{Whats that precious?} gewählt und mit einem Merkmal/Bemerkung behaftet, welche kontraintuitive bezüglich der ontologischen Kategorie ist. Ein Beispiel von Kontraintuitivität ist zum Beispiel die Raupe, welche nach der Metamorphose zu einem Schmetterling wird. Die Erwartung für ein TIER ist, dass es sich im Laufe des Wachstums nur durch grösse und Masse verändert, jedoch nicht, dass es zu einem anderen TIER wird. Als fiktives Beispiel einer religiösen Aussage nennt Boyer hier. 
\begin{quote}Manche Ebenholzbäume behalten Gespräche in Erinnerung, die Menschen in ihrem Schatten geführt haben.\end{quote} 

Hier ist der Ebenholzbaum ein Vertreter der onologischen Kategorie PFLANZE und hat das kontraintuitiv Merkmal eine geistige Präsenz zu haben. Dabei ist es wichtig, dass tatsächlich gegen die ontologische Kategorie verstossen wird und nicht nur einfach eine Merkwürdigkeit. Eine PERSON welche ihre Hautfarbe ändert, ist dem zu Folge weniger erfolgreich, als eine PERSON, welche durch Wände gehen kann. 

Diese Verstösse bilden den Kern der religiösen Aussage. Es ist aber auch üblich, den ontologischen Kategorienverstoss mit weiteren Verstössen auszuschmücken, welche aber nicht mehr kontraintuitiv bezüglich der Ontologischen Kategorie sind.

\begin{itemize}
	\item Nur eine Stufe der Ontologie brechen. (PERSON zu TIER, NATUR OBJEKT zu TIER)
	\item Wichtigkeit der Information. INformation die von der Gruppe von Menschen verwaltet wird. Wichtigkeit der Kommunikation. Wichtigkeit andere Menschen zu verstehen.
\end{itemize}
\subsubsection*{komplexität des Hirns: was damit gemacht wird}

TODO\todo{Fertig machen -> Google Drive: Boyer Methodik}

\subsection{Shinto: Animismus in Japan?}
TODO
Traditioneller Glaube von Japan ist ein Geister, Vorfahre und Naturglaube bla.

\subsection{Pascal Boyer: Methoden der kognitiven Religionswissenschaft}
%!TEX root = Animismus_in_Anime.tex
In dieser Underdisziplin der Religionswissenschaft wird Religion oder religiöse Phänomene aus der Perspektive der Kognitions- und Evolutionswissenschaft betrachtet. Es wird versucht zu erklären, weshalb religiöse Praktiken und und Denkweisen universell verbreitet sind.~\footnote{wikipedia: \texttt{https://en.wikipedia.org/wiki/Cognitive\_science\_of\_religion} und \texttt{https://de.wikipedia.org/wiki/Kognitive\_Religionswissenschaft} 24.08.15}

Die kognitive Religionswissenschaft ist eine eher junge Disziplin, welche sich erst Ende des zwanzigsten Jahrhunderts etabliert hat. Zu ihren Begründern gehören unter anderen E. Thomas Lawson und Robert McCauley (\emph{Rethinking Religion: Connecting Cognition and Culture and Bringing Ritual to Mind: Psychological Foundations of Cultural Forms}), Pascal Boyer (\emph{Naturalness of Religous Ideas}) und Guthrie (\emph{Faces in the cloud}). An dieser Stelle soll Pascal Boyers \emph{Und Mensch schuf Gott}~\footnote{\cite{boyer04}} als Stellvertreter für andere kognitive Ansätze dienen um die Animationsfilme auf Animismus zu untersuchen. In den nächsten Abschnitten 

\subsection*{Boyer: Und Mensch schuf Gott}
Boyer erklärt in der Einleitung seines Buches, dass es im Geiste des Menschen sei zu suchen. Jeder menschlicher Geist habe das Zeug religiös zu sein. Es ist somit nicht sein Ziel zu beweisen, dass es Gott nicht gibt und nur ein Produkt unserer Fantasie ist, sonder er möchte erklären können, warum wir das glauben was wir glauben.

Bei der Frage nach dem Ursprung der Religion tauchen immer wieder ähnliche intuitive Begründungen auf: \emph{Die Religion bietet Erklärungen, Die Religion spendet Trost, Die Religion sichert die gesellschaftliche Ordnung, Die Religion ist eine kognitive Täuschung}~\footnote{\textsc{Boyer 04: 14-15}}. Laut Boyer sind diese intuitiven Gewissheiten in ihrer Existenz zwar berechtigt, jedoch nicht dienlich dabei den Ursprung zu finden. Einen Ursprung im Sinne eines historischen Ereignis ist eine Wunschvorstellung, welche aus dem Wunsch entspringt eine Ursache zu haben, aus der sich alle weiteren Phänomene ableiten lassen würden. Ausführlich zeigt Boyer, wie man in jeder dieser Vorstellung Widersprüche findet oder sie schlicht nicht befriedigend sind, welche sie als Ursprung ungeeignet machen. Doch für jedes Gebiet, das er abarbeitet fügt er am Ende \glqq[einen] andere[n] Blickwinkel\grqq  hinzu, in dem er aus kognitiver Sicht den Wert dieser intuitiven Annahme beschreibt:

\begin{itemize}
	\item Das Erkenntnissystem des Gehirns produziert Erklärungen, oft ohne dass wir uns dessen gewahr sind. - Religion als Erklärung.
	\item Emotionale Programme sind für uns lebenswichtig~\footnote{Hier das Beispiel der Angst vor einem Raubtier. [S.34]} und somit ein Aspekt unseres entwicklungsgeschichtlichen Erbes. - Emotionen in der Religion.
	\item Die Untersuchung des sozialen Bewusstseins (soziale Intelligenz) kann Antworten auf die Frage nach den Erwartungen an das gesellschaftliche Leben und Moral geben. - Religion, Moral und Gesellschaft.
	\item Bei all den Übernatürlichen Informationen welcher der Geist bekommt, werden nur manche als plausibel erachtet und so angeeignet. - Religion und Denken.
\end{itemize}

Für den ersten Punkt das Beispiel vom Donnergrollen, welches das Grollen der Geister/Götter/Ahnen über ein Fehlverhalten gibt für Boyer zu wenig her für einen zu grossen Aufwand. Um laute, grollende, dumpfe Geräusche bei Stürmen zu erklären, muss eine komplett imaginäre Welt mit übernatürlichen Mächten vorausgesetzt werden, welche selbst noch mehr Fragen auf werfen: Woher sind sie gekommen? Wo sind sie? Warum sieht man sie nicht? Haben sie einen riesigen Mund um diese Geräusche zu erzeugen? Ist ein solcher Glauben verbreitet, so finden sich zu diesen Fragen auch Antworten. Diese Antworten sind aber teilweise so weit her geholt, dass es die Ausgangslage, die Geräusche eines Gewitters erklären zu wollen unwahrscheinlich macht.

Boyer ist sehr grosszügig mit einleuchtenden Beispielen, jedoch ist nicht immer ganz klar, wie viel von dem was er sagt wissenschaftlich erwiesen ist, und wie viel davon Spekulation ist. 

Jedenfalls führt er nun weitere Konzepte ein, welche für seinen Ansatz unausschliessbar sind. Er zählt diese als den Inhalt eines Werkzeugkasten auf. Für die weitere Arbeit werden diese Werkzeuge eine zentrale Rolle spielen, da diese die Methode zur Analyse liefern aus Sicht der Kognitiven Religionswissenschaft.

In seiner Erklärung geht Boyer von der Funktionsweise des Geistes im Allgemeinen aus, welche unabhängig von der Kultur überall gleich ist. Das scheint zu nächst eine Sackgasse, da kulturell grosse Unterschiede bezüglich religiöser Praktiken und Vorstellungen zu finden sind. Das geniale hier sei, argumentiert er, dass sich etwas so \emph{vielschichtiges} wie Religion durch etwas erklären lässt, was überall gleich ist (das Gehirn). Es ist jedoch notwendig zunächst mehr darüber zu wissen, wie das Gehirn Informationen aufnimmt und verarbeitet.[S.11]

Die Arbeit die ein Gehirn leistet ist lange unterschätzt worden. Einerseits muss man von der verbreiteten Annahme wegkommen, dass es sich beim Geist um ein leeres Gefäss handelt welches beliebig mit Informationen (Erziehung, Bildung und persönliche Erlebnissen) gefüllt werden kann. Anderseits auf von der Idee, dass der Geist mit wahllosen Informationen abgefüllt werden kann. Wir können uns bei weitem nicht alles merken und das ist auch gut so. Es braucht also etwas im Hirn, das relevante Informationen aus der Umwelt identifiziert und auf eine spezifische Weise zu verarbeiten mag. [S. 12]

\subsubsection*{Meme}
Boyer geht nun der Frage nach, wie religiöse Konzepte überhaupt entstehen. Die Bezeichnung Meme als Kulturelemente, also Vorstellungen, Werte, Geschichten und dergleichen, welche Menschen in ihrem Handeln beeinflusst und weitergegeben werden,~\footnote{\textsc{Boyer} 04: 50} wurde vom Evolutionsbiologen Richard Dawkins erstmals vorgestellt. Ein Meme bezeichnet demnach einen Bewusstseinsinhalt\footnote{wiki: \texttt{https://de.wikipedia.org/wiki/Mem} 24.08.15}, welcher durch Kommunikation in der Gesellschaft weitergegeben und somit vervielfältigt werden kann. Es ist das soziokulturelle Pendant zu den biologischen Genen in der Evolution. So dann lassen sich die Meme auch ähnlich wie die Gene beschreiben: Information wird durch Kommunikation weiter gegeben (replizieren). Dadurch werden die Inhalte aber nicht einfach verbreitet, sonder auch leicht (oder schwerwiegend) abgeändert (mutieren).~\footnote{Die Information ändert sich nicht erst durch deren Weitergabe. Etwas das wir erfahren wird bei verschiedenen Menschen bereits anders verarbeitet. Somit können zwei Personen genau das gleiche hören und eine andere Version des Inhalts in ihrem Bewusst sein haben.} Schliesslich werden nur jene Memes tatsächlich weiter gegeben oder überhaupt erst erinnert, welche einprägsam sind (selektieren).

Boyer beschreibt die Meme als wunderbare Ausgangslage, will ihnen aber nicht mehr als genau das zugestehen. Seine Kritik liegt darauf, dass es keine Replikation der eigentlichen Informationen gibt. Inhalte werden nicht faktisch übergeben, sondern jeweils neu konstruiert. Zwei Menschen können zwei faktisch identische Aussagen machen, aber jeder hat die Information, welche er wiedergibt auf seine eigene Art rekonstruiert. Entsprechend stellt Boyer als nächstes seine Theorie zum Einfangen von Vorstellungen durch Schablonen vor. 

\subsubsection*{Schablonen, Vorstellungen und das Schlussfolgerungssystem}
Ein grosser Teil dessen, was wir wissen musste uns niemand faktisch erzählen. Eine erstaunliche Eigenschaft unserer unserer geistigen Fähigkeit ist es durch die Kombination bereits existierendem Wissen und der Hinzugabe einer neuen Information mehr Wissen zu generieren. Boyer zeigt dies anhand eines Beispiels mit einem Kind auf, mit dem Kind als eine Person, dessen Wissen erweitert wird. Zeigen wir einen Kind zum ersten Mal ein Seehund, so hat es keine weiteren Informationen darüber als den Namen und seine äussere Erscheinung. Dennoch wird das Kind erwartet, dass das Seehund isst, schläft und dass es Kinder haben kann. Diese Information über das Seehund hat das Kind geschlussfolgert, indem es eine Annahme gemacht hat: Das Seehund ist ein Säugetier. Diese essen, schlafen und bekommen Kinder. Boyer bezeichnet das Säugetier als eine \emph{Schablone}. Mit dieser Säugetier-\emph{Schablone} hat das Kind eine Seehund-Vorstellung gebildet.\footnote{\textsc{Boyer 04: 59}}   

Über diesen Schablonen, welche dazu da sind verschiedene Konzepte zusammen zufassen und dabei helfen aus einem Informationsstück mehr Information zu schaffen, stehen die ontologischen Kategorien. Boyer nennt fünf davon: \emph{Person, Tier, Pflanze, Naturobjekt, Werkzeug}. ie Schablonen, und im stärkerem Masse die ontologischen Kategorien seinen das, was über Kulturen hinweg universell sei. Erst bei den konkreten Konzepten ergäbe sich die Varianz. Die Vorstellungen welche von Angehörigen einer gleichen Gruppe anhand einer Schablone hergestellt werden sind sich in der Regel ähnlich. Die Vorstellungen einer anderen Gruppe kann stark davon abweichen, trotzdem die gleiche Schablone benutzt wurde.

Dieses System der Schablonen welche mit Schlussfolgerungen zu Vorstellungen beziehungsweise zu Konzepten führen überträgt Boyer nun auf die Religion. Es gibt demnach Schablonen für religiöse Vorstellungen. Die Schablonen von religiösen Vorstellungen teilen wir uns universell, doch die Konzepte variieren stark regional. Dass übernatürliche Kräfte unsichtbar sind/sein können ist man sich breitgänig einig. Wenn es aber darum geht, was und warum übernatürliche Kräfte etwas tun, so gehen die Vorstellungen weit auseinander.

Letztlich müsse berücksichtigt werden, dass die kulturelle Varianz in der Regel geringer ist, als man annimmt. Beim Übermitteln findet durch die Schablonen ein Filtern der gegebenen Informationen statt, so dass daraus voraussagbare Strukturen gebaut werden.[65]

\subsubsection*{Beschaffenheit des Übernatürlichen}
Es folgt also die Suche nach dem mentalen Rezept für religiöse Vorstellungen. Mit einem Versuch verschiedener mehr oder weniger potenten religiösen Aussagen versucht Boyer dem Leser zu zeigen, dass man der Intuition folgend gewisse Aussagen über übernatürliche Wesen direkt ausschliessen kann, während man bei andern sofort glauben würde, dass es sich um eine existierende religiöse  Vorstellung handelt. 

Folgende Kriterien für eine erfolgreiche (religiöse) Vorstellung kommen dabei heraus:

Erstens habe jedes übernatürliche Konzept die Tendenz, eine seiner ontologischen Annahmen zu verletzen. Ein Geist gehört zur ontologischen Kategorie PERSON, doch dass er keinen physikalischen Körper hat bricht diese Ontologie. Ein anderes Beispiel für diese Kontraintuitivität findet sich auch in der Natur, bei einer Raupe, welche nach der Metamorphose zu einem Schmetterling wird. Die Erwartung für ein TIER ist, dass es sich im Laufe des Wachstums nur durch Grösse und Masse verändert, jedoch nicht, dass es zu einem anderen TIER wird. Dabei ist es wichtig, dass tatsächlich gegen die ontologische Kategorie verstossen wird und es sich nicht nur um eine Merkwürdigkeit handelt. Eine PERSON welche ihre Hautfarbe ändert, ist dem zu Folge weniger erfolgreich, als eine PERSON, welche durch Wände gehen kann. 

Zweitens hat religiöses Denken die Tendenz auf Leute-ähnliche übernatürliche Wesen zu fokussieren, welche Zugang zu sozial-relevanten Informationen haben. Der Austausch von Informationen ist für den Menschen kritisch so Boyer. Wir sind darauf angewiesen, dass andere in der Gruppe Dinge wissen und uns diese übertragen können. Bei religiösen Vorstellungen wird in der Regel davon ausgegangen, dass dieses Wesen die moralische Haltung eines Individiums oder auch der ganzen Gruppe teilt. Dieses Wesen weiss Bescheid darüber, wenn schlechte Dinge geschehen. Die Person erwartet dann, dass dieses Wesen wertet und allenfalls böse wird und die Person für deren Verhalten bestraft. Eine solche Vorstellung macht diese Wesen zu wichtigen Subjekten für Gedanken und Diskussionen in einer Gruppe.

Im dritten Punkt werden religiöse Rituale mit aus den Reinigungsritualen hergeleitet. Unser mentales System behandelt Krankheit mit Abscheu zum eigenen Schutz vor einer unsichtbaren Gefahr. So sollen auch Rituale in religiösen Vorstellungen vor allem zum Schutz vor unsichtbaren Gefahren und Mächten sein.

Letztlich setzt Boyer noch eine Fokus auf die Leiche und wie Menschen damit umgehen. Im Prinzip sind Leichen eine Spezialfall des ersten Punktes, wo es um die Verletzung der ontologischen Kategorie geht. Der Mensch, den wir sehen oder sogar kannten wird in unserem System der ontologischen Kategorie PERSON zugeordnet. Gleichzeitig verletzt die Leiche dieses Menschens jegliche Kriterien der PERSON Kategorie und entspräche demnach der Kategorie NATUROBJEKT und sollte unsere Abscheu vor Krankheit wecken. Dieser Widerspruch macht die menschliche Leichte zum Prototyp für religiöse Objekte überall auf der Welt.

\smallskip
Zusammenfassendes bla?

Zusammenfassung: http://serendip.brynmawr.edu/exchange/node/1581
Zusammenfassung 2: http://mason.gmu.edu/~rhanson/religion.html
\newpage

\section{Animes}
%!TEX root = Animismus_in_Anime.tex
\newpage
\subsection{Hayao Miyazaki}
Hayao Miyazaki, einer der bekanntesten Animations Produzent Japans, wurde während dem 2. Weltkrieg im Januar 1941 unweit von Tokyo geboren. Sein Vater arbeitete für seinen Bruder in der Maschinenbau Firma Miyazaki Airplanes.~\footnote{Dieser Hintergrund wird gerne gebraucht um Miyazakis Faszination vom Fliegen zu begründen.} 

Miyazakis Interesse an der Animiation wurde sehr früh schon durch den Animationsfilm \textsc{Panda and the Magic Serpent} geweckt, welcher in seinen Jugendjahren veröffentlich wurde. Obwohl er zunächst Politik und Wirtschaft an einer renommierten Universität studierte, zog es ihn nach Abschluss seines Diplomes ins Animationsgeschäft. Als Hintergrundzeicher fand er bei der derzeit führendem Studio Toei-Animation eine Anstellung und machte sich schnell einen Namen. Nebenbei veröffentlichte er unter einem Pseudonym er eine eigene Manga-Serie\todo{welche?} und sammelte wo immer möglich Erfahrung in Storyentwicklung und Produktion. Toei-Animation schickte seine Animatoren gelegentlich auf Reisen um um Skizzenstudien der Landschaften oder Städte zu machen.~\footnote{So zum Beispiel reiste Miyazaki für die Produktion von \textsc{Alpine Girl Heidi} nach Europa.} 

Sein Freund Isao Takahata verfilmte 1972 Miyazakis erste Kurzgeschichte (\textsc{Adventures of Panda and Friends}). 1978 übernimmt Miyazaki dann erstmals die Inszinierung einer Anime-Serie (\textsc{Boy Conan}) und im darauf folgenden Jahr führte er zum ersten Mal Filmregie (\textsc{Schloss des Caliostro}). Von da an bekam Miyazaki immer öfters die Leitung für die Inszenierungen von Anime-Serien.

1982 begann Miyazaki mit dem Manga \textsc{Nausicaä aus dem Tal der Winde}. Die einzelnen Teile erschienen mit zahlreichen Unterbrechungen in einem monatlich erscheinendem Magazin. Erst 1994 fand der Manga einen Abschluss. Doch bereits 1983 begannen die Vorarbeiten für eine Verfilmung der Geschichte unter der Leitung von Miyazaki. Der Film erschien im derauf folgendem Jahr in den japanischen Kinos. 

Miyazaki machte sich im Jahr darauf mit ein paar Kollegen von Toei-Animation selbständig und gründete das Animations Studio Ghibli\footnote{Warmer Wüstenwind}. Bei der Position als Regisseur und als Produzent wechseln sich Miyazaki und Takahata ab. Das Studio war finanziell nicht abgesichert und riskierte zunächst mit jeder Produktion seinen Ruin. Etwas mehr Sicherheit gewann das Ghibli Studio nach der Veröffentlichung des Filmes \textsc{My Neighbor Totoro}, da insbesondere der Merchandising ein grosser Erfolg (auch heute noch) verbuchen kann. Mit den weiteren Produktionen stieg die Firma Ghibli zu den erfolgreichsten und bekanntesten in Japan auf.

Miyazaki wollte sich, mit der Fertigstellung von \textsc{Prinzessin Mononoke} in 1997 eigentlich in den Ruhestand setzen,\footnote{Zitat von Miyazaki: +/- zu anstrengend} begann aber dann mit den Vorbereitungen für \textsc{Chihiros Reise ins Zauberland} welcher 2001 in die japanischen Kinos kam. Während der Film nicht nur nationale sonder auch internationale Preise gewann\footnote{Darunter Oscar in der Kategorie Bester Animationsfilm}, arbeitete Miyazaki auch schon am nächsten Projekt. \textsc{Das wandelnde Schloss} welches 2004 in Japan veröffentlicht wurde, bescherte erneut einen Einnahmerekord dar in den japanischen Kinos, erlangte jedoch international nicht mehr ganz so viel Aufmerksamkeit wie die beiden Vorgänger Filme. 

Mit \textsc{Wie der Wind sich hebt} kündete Miyazaki erneut seinen Rücktritt an. An einer Abendkonferenz in September 2013 erklärte Miyazaki, dass er keine abendfüllenden Anime-Filme mehr machen werde.\footnote{\texttt{http://asienspiegel.ch/2014/08/grosse-ehre-fur-hayao-miyazaki/}} Die Zeit der traditionellen Animation, wo man noch mit Hand zeichne sei vorbei. Auch bei seinen späteren Filmen wurden Computergrafiken nur vereinzelnd eingesetzt. Überraschenderweise wagt der 74 jährige Japaner doch den Sprung ins neue Zeitalter der Animation. In seinem neusten Projekt arbeitet er an einem 10 minütigen Kurzanime, welcher auf der Kurzgeschichte \textsc{Boro, die Raupe}. Erstmal will er einen vollständig computer animierten Anime machen.\footnote{\texttt{http://asienspiegel.ch/2015/07/miyazaki-arbeitet-an-kurzanime/}}

Natur und so?\todo{wo und was soll noch dazu kommen?}

\subsection{Japanische Animationsfilme}
\subsubsection{Geschichte}
Die Bezeichnung \emph{Anime} für japanische Animationsfilme ist eine Fremdbezeichnung, welche sich erst nach dem Krieg durch die amerikanische Übernahme, gegen d\={o}ga, den japanischen Begriff für Animation, durchgesetzt hat. Die Symbiose zwischen Anime und Manga (Comic) ist spezielle in Japan. Es ist häufig so, dass ein erfolgreicher Manga verfilmt wird, oder aber dass zu einem Anime hinter ein Manga gezeichnet wird. Anders als im europäischen Raum zielen Manga und Anime auch auf ein viel breiteres Publikum ab. Zwar sind viele Geschichten für Kinder gedacht, aber auch Erwachsene werden als Zielpublikum ernst genommen. Daher finden wir in den japanischen Animationsfilmen in der Regel eine grössere Inhaltliche Palette als in den amerikanischen Produktionen. Durch das weite Spektrum des Zielpublikums finden sich auch Produktionen an allen Genres. Nebst den Geschichten welche typischerweise japanische Märchen, Mythen und Legenden oder Science Fiction thematisieren, finden sich auch Horror, Historie, Romanzen, Komödien und Erotik ihren Platz. Beim Inhalt zeigt sich, dass obwohl häufig Elemente aus der Japanischen Kultur eine zentrale Rolle spielen, nicht davor gescheut wird auch westliche Elemente zu integrieren. Dennoch sind japanische Animationsfilme üblicherweise für ein japanisches Publikum gedacht.

Gerade nach der Kriegszeit übten die amerikanische Filmindustrie einen grossen Einfluss auf. Einerseits wurde versucht dem erfolgreichen Beispiel zu folgen, anderseits bestand auch der Drang sich davon abzugrenzen und sich auf die eigene Kultur zu konzentrieren. Dem US-Beispiel folgend entstanden in den 1950er Jahre etliche Animiations Studios. Anders als die amerikanischen Studios wie Disney setzte die japanische Filmindustrie mehr auf Quantität als auf Qualität. Das führt dazu, dass typische Anime in der Regel einfacher, weniger hyperrealistisch gestaltet sind als die amerikanischen. Neben der Machart unterscheiden sich die japanischen Animationen von amerikanischen oder auch europäischen durch ihren kulturellen Hintergrund. 

Im Shinto finden wir einige der Erklärungen für Eigenart der japanischen Animations Filme. Der traditionelle Shinto ist eine unorganisierte, schriftlose Religion. Im Shinto wird das Universum zudem prinzipiell als mehrdeutig betrachtet. Das macht sie ideal dafür, verschiedene (auch widersprüchliche) Konzepte in sich aufzunehmen. So wurde zum Beispiel der Buddhismus in den Shinto integriert und auch Elemente des Christentums, welches später nach Japan gelangte fanden ihren Platz im religiösen Gesamtverständnis der Japaner. Des weiteren wird nicht scharf zwischen Götter und Dämonen, gut und böse getrennt. Nach dem Kodex der Samurai steht die Absicht auch über der Handlung. Als Folge all dessen fehlt in den japanischen Geschichten in der Regel auch die Unterscheidung von Gut und Böse. Vielmehr steht der Protagonist und der Antagonist sich in einem Interessenskonflikt gegenüber. Die Interessen können sich in ihrer Essenz widersprechen und dennoch glaubhaft sein. 

Im Zentrum stehen die Motive der Charakteren. Ein Bildwechsel wird daher nicht unbedingt benutzt um den zeitlichen Verlauf zu markieren, sondern um einen Perspektivenwechsel zu ermöglichen. Für westliche Zuschauer gibt das den Eindruck einer Verlangsamung der Handlung. Dieser Fokus, zusammen mit der Eigenschaft, dass auch japanische Serien in der Regel abgeschlossen sind, geben den Charaktern der Geschichte die Möglichkeit dramatische Veränderungen zu durch gehen. In amerikanischen Produktionen fallen die Charakteren vergleichsweise flach und statisch mit wenig Möglichkeit zur Entwicklung auf Grund der episodenhaften Art aus.

Das ästhetische Prinzip von Wabi und Sabi\footnote{Definition, oder zumindest Andeutung} ist ebenfalls in den japanischen Filmen zu beobachtet. Auslassung ist genau so Teil eines Kunstwerks wie seine andern Bestandteile. Damit begründet sich auch das hohe Mass an Abstraktion zum Beispiel beim Charakterdesign.  

Eine weitere wichtige Rolle spielen symbolische Darstellungen. Gerade bei einer Analyse und einer damit verbunden Interpretation ist es wichtig sich aber bewusst zu sein, dass die japanische Kultur Grundlage der Interpretation sein muss. Im Gegensatz zu den Walt Disney Märchen wo die Standard Prinzessin blondes Haar trägt, ist die typische Haarfarbe für einen guten Charakter in japanischen Geschichten dunkel.~\footnote{Im Anime \textsc{Das wandelnde Schloss} welches später genauer betrachtet werden soll, trägt der Protagonist Hauro zunächst Blonde Haare. Doch als er endlich zu sich selbst und seinen Mut findet trägt er dunkle Haare.} 
\subsection{Prinzessin Mononoke}
%!TEX root = Animismus_in_Anime.tex
\subsection{Filmhintergrund} 
Mit \textsc{Prinzessin Mononoke} gelingt Hayao Miyazaki der internationale Durchbruch. Der Film kommt am 17. Juli 1997 in die japanischen Kinos. Mit 18,65 Millarden Yen (das entspricht damals 242,45 Mio CHF) spielt er in Japan mehr ein als \emph{Titanic} von James Cameron. Damit ist Prinzessin Mononoke der bisher erfolgreichste Film in Japan. Nachdem er 1998 auf der 48. Berlinale das erste Mal in Deutschland vorgeführt wird und \todo{zahlreiche Preise} gewinnt, kommt er 1999 in den Vereinigten Staaten und Kanada und im Jahr darauf auch in Europa in die Kinos. Trotz der vielen Auszeichnungen, die der Film gewonnen hat, blieb der Film, sowie das Ghibli Studio und Hayao Miyazaki, vorwiegend nur unter Anime-Fankreisen bekannt. Dies hängt sicherlich damit zusammen, dass ausserhalb von Japan kaum Werbung gemacht wurde. Auf internationale Vermarktung wurde lange verzichtet, da Miyazaki und sein Team darüber entsetzt waren, wie starkt \textsc{Nausicaä aus dem Tal der Winde} im Ausland geschnitten wurde. 
Nach dem riesigen Erflog an den japanischen Kassen zeigte das US-Studios Disney Interesse und sicherte sich die Verhandlungsposition der japanischen Filmemacher. Im Vertrag über die internationale Vermarktung der Ghibli-Filme, welche bald darauf geschlossen wurde, konnte sich Miyazaki und sein Team das Recht sichern, über allfällige Schnittstellen selbst entscheiden zu dürfen. 

\subsubsection*{Geschichtlicher Hintergrund und Miyazaki} 
Erste Ideen für \textsc{Prinzessin Mononoke} hatte Miyazaki bereits 1970. Damals diente ihm als Plotvorlage das Märchen von der Schönen und dem Biest.\todo{http://home.comcast.net/~rocksunner/miya\_e.html} Wegen dem leichtsinnigen Versprechen ihres Vaters, muss die Tochter des Fürsten ein Waldmonster [mononoke]\todo{Geist/Monster/Gespenst -> http://nausicaa.net/miyazaki/mh/faq.html\#translation} heiraten. Weil Miyazaki mit der Geschichte nicht weiter kam, schob er ihre Bearbeitung hinaus. 

\begin{quote} \glqq Actually, in the beginning I wanted to do a fantasy rather than a period drama set in Japan. However, when I said "Now let's do it", I didn't have the heart for it.\grqq \todo{http://home.comcast.net/~rocksunner/miya\_e.html} 
\end{quote}

Er entscheidet sich schliesslich gegen die anfänglich geplante Fantasy-Umwelt und setzt die Geschichte im traditionellen feudalen Japan an. Doch in der Zeit, in der er sich mit den Hintergründen auseinander setzt, ändert er auch das Herzstück der Geschichte. In Miyazaki erwächst der Wunsch einen tiefgründigeren, authentischeren Film zu machen. Dadurch findet er wieder zum Thema zurück, das er schon mit \textsc{Nausicaä aus dem Tal der Winde} im Fokus hat: Das (konfliktreiche) zusammenleben von Mensch und Natur einerseits, und von Mensch und Mensch andrerseits. So bleibt am Ende von der ursprünglichen Geschichte nicht mehr übrig als der Name \emph{Mononoke}.\todo{Das Märchen von der Prinzessin und dem Biest hat Miyazaki später in Form eines Bilderbuches veröffentlicht.} 

Infos aus:\todo{https://de.wikipedia.org/wiki/Prinzessin\_Mononoke\#Hintergr.C3.BCnde http://nausicaa.net/miyazaki/mh/filminfo.html http://home.comcast.net/~rocksunner/mono\_e.html}

\subsection{Zusammenfassung}
Die Geschichte spielt im feudalen Japan zu einer Zeit, die der Muromachi-Ära (1392-1573) ähnelt. Historisch wichtige Figuren bleiben im Hintergrund. Ein König bzw. Shogun wird zwar erwähnt, jedoch bekommen die Zuschauer nur die Folgen der Interessenskonflikte der Mächtigen zu sehen. Nicht bei ihnen liegt der Fokus der Geschichte, sondern bei den einfachen Menschen und mehr noch bei den Aussenseitern der Gesellschaft. Sie leben in einer Zeit der kulturellen Blüte, sozialer Umwälzungen und politischer Unruhe. 

Ashitaka, der letzte Prinz eines in Harmonie mit der Natur lebenden Volkes namens Emishi\todo{Alte Bezeichnung für ein japanisches Urvolk. Nach der Heian-Zeit wurde das Volk Ezo genannt.}, muss seine Heimat verlassen, weil er einen Dämon\todo{Tatarigame = curse god (http://home.comcast.net/~rocksunner/mono\_e.html\#ashitaka)}, der das Dorf angegriffen hat, tötete und so dessen Fluch auf sich zog. Ashitaka macht sich auf, den Wald des Shishigamis\todo{Bedeutung} zu finden. Denn von dort kommt der Dämon, welcher einst ein Keilergott war und erst durch den Schmerz und den daraus erwachsenen Hass zu einem Dämon wurde. 

Bald findet sich Ashitaka zwischen verhärteten Fronten. Es herrscht Krieg zwischen den Waldgöttern und den Menschen, welche in einer Erzschmiede arbeiten. Ashitaka trifft auf San, die von den anderen Mensch \emph{Prinzessin Mononoke} genannt wird, weil sie von der Wolfgöttin Moro aufgezogen wurde und sich wie ein Tier verhält. Ashitaka versucht zwischen San, die für ihre Familie und ihren Lebensraum kämpft, und den Bewohnern der Schmiede, welche den Wald abholzen um Erz zu gewinnen, was ihre Lebensgrundlage ist, zu vermitteln. Erschwerend kommt hinzu, dass die Anführerin der Schmiedebewohner, Eboshi, systematisch versucht die Waldgötter zu vernichten, um ihre Untertanen vor ihnen zu schützen. Darüberhinaus hat sie dem Kaiser den Kopf des Shishigami versprochen. Sie erhofft sich so den Schutz für ihre Schmiede zu sichern, da die Gefahr besteht, dass sie vom selbigen Kaiser angegriffen werden könnte. 

Trotz Ashitakas Bemühungen kann dem bewaffneten Konflikt nicht ausgewichen werden. Eboshi schiesst dem Waldgott Shishigami den Kopf ab. Als Folge dessen droht der nun kopflos tätige Rumpf des Gottes alles zu zerstören. Durch das selbstlose Eingreiffen von San und Ashitaka kann das Verderbnis im letzten Moment gestoppt werden. Doch vieles der alten Welt ist danach für immer verloren.

Obwohl man durch ästhetische Gestaltung einfach erraten kann, bei wem und wo Miyazakis Sympathie liegt, so ist es doch erstaunlich, dass der Geschichte jegliches Schwarz-Weiss denken fehlt. Jeder, der kämpft, hat seine Gründe. Ein durchgehendes Motiv ist der destruktive Hass. Somit birgt das Film-Ende zwar Hoffnung in sich, die Problematik bleibt jedoch bestehen. 
Obwohl Ashitaka und San sich weiterhin sehen wollen, kann das Wolfmädchen den Mensch nicht vergeben und bleibt bei ihren Wolfbrüdern im Wald. Ashitaka, der zwar grossen Respekt vor der Natur und den Geistern zeigt, kehrt dennoch in die Schmiede zurück, um dort mit den Menschen zu leben.  

\subsection{Figuren Analyse}
In Japans Altertum angesiedelt passt sich die belebte Natur gut ins Bild ein. In jenen Städten und Dörfern, die Ashitaka auf seinem Weg zum Shishigami Wald besucht, haben die Leute so gut wie keinen Kontakt zu den Naturgeistern. Zu sehr sind sie mit den politischen Problemen beschäftigt. Als Ashitaka endlich im Reich des Shishigami ankommt, ist das Verhältnis zwischen Mensch und Natur, oder vielmehr zwischen Mensch und Waldgöttern ein anderes. Ein friedliches Nebeneinander scheint unmöglich. 

\subsubsection*{Die Tiergötter: Moro, Nago und Okkoto}
Moro ist eine Wolfsgöttin die zusammen mit ihren Söhnen und ihrer adoptierten Menschentochter San im Wald des Shishigami lebt. Mit ihrer Grösse überragt sie alle Menschen. Sie ist weiss und hat einen doppelten Schwanz. Ihre Stimme ist tief und klingt nicht weiblich. Moro hasst die Menschen; Eboshi am meisten von allen. Für ihr Ziel, Eboshi zu töten, riskiert Moro viel, nicht zu letzt ihr eigenes Leben. Selbst im Tod, als ihr Kopf vom Körper getrennt ist, schafft es Moro noch Eboshi den Arm ab zu beissen. Im Gegensatz zu anderen Akteuren in der Geschichte (insbesondere den Wildschweinen) jedoch macht der Hass sie nicht blind. Und so stellt sie sich auch gegen den aufgebrachten Keilergott Okkoto um San zu retten.

\begin{quote}
\glqq Humans who attacked the forest threw a baby to me in order to escape my fangs. That was San...! She can't be human, neither can she fully be a wolf. She's my poor, ugly, loveable daughter.\grqq \todo{http://home.comcast.net/~rocksunner/mono3e.html\#moro}
\end{quote}

Ein anderes Bild von den Tiergöttern bekommen wir durch die Keiler. Der Dämon, den Ashitaka am Anfang des Filmes bezwingt um sein Dorf zu schützen, war einst ein mächtiger Keilergott. Auch er stammt aus dem Shishigami-Wald. Als er zu den Emishi kommt, hat der Schmerz und der Hass ihn bereits so wütend gemacht, dass er zu einem Dämon [Tatarigame]\todo{Begriffserklärung} wurde. Ashitaka bittet den rasenden Gott Umkehr zu machen und sein Dorf zu verschonen. Erst als sich Ashitaka zwischen seinem Dorf und dem Dämon entscheiden muss, tötet er ihn. Die Dorfseherin verrichtet gleich darauf ein Versöhnungsritual, in dessen Verlauf sie dem Dämon verspricht einen Schrein zu errichten und indem sie ihn bittet er möge nicht länger hassen. Doch eine Stimme erklingt aus dem Toten Keiler und verflucht alle Menschen (\glqq  Loathsome humans! You will know my wrath well for causing me pain\dots \grqq \todo{http://home.comcast.net/~rocksunner/mono\_e.html\#ashitaka}) 

Im späteren Verlauf der Geschichte begegnet Ashitaka dem mächtigen Gott Okkoto, dem Herr aller Keiler. Okkoto ist weiss wie Moro und ihre Söhne. Graue Stellen in seinem borstigem Fell und seine trüben Augen weisen jedoch darauf hin, dass er sehr alt ist. In seiner Grösse überragt der Keilergott sogar Moro. So, wie sie zwei Schwänze hat, verfügt er über eine zweite Reihe von mächtigen Hauern.  

Von Ashitaka erfährt Okkoto von Nagos Schicksal. Er zeigt sich beschämt, dass einer aus seinem Klan ein so übles Schicksal wiederfahren ist. Er zeigt jedoch weder Mitleid noch Vergeben gegenüber Ashitaka und teilt ihm mit, dass er ihn bei der nächsten Begegnung umbringen werde. Obwohl er einsichtiger als die andern Keiler zu sein scheint, ist Okkoto so sturr, dass von seiner Weisheit am Ende nichts übrig bleibt. Von den Menschen durch Waldbrände und Lärm provoziert, rennen die Wildschweine unter Okkotos führung geradewegs in ein Massaker. Moro durchschaut die Absichten der Menschen und auch Okkoto vermutet was dahinter steckt, trotzdem setzt er alles auf eine letzte Schlacht.  

\begin{quote}
\glqq Moro, look at my clan! Little by little they are becoming smaller and more stupid. If this keeps up, humans will be able to hunt us down like common meat\dots\grqq \todo{http://home.comcast.net/~rocksunner/miya\_e.html} 
\end{quote}

Auch Moros Söhne sind nicht so prächtig und gross wie sie selbst. Zwar sind sie immer noch grösser als normale Wölfe, und weiss, es fehlen aber ein doppelter Schwanz und sonstige Merkmale, die sie von normalen Wölfen unterscheidet. Aus Moros Handeln und dem was sie sagt geht hervor, dass sie keine Hoffnung für die Zukunft des Waldes und der darin lebenden Götter sieht. 

Okkoto kehrt später schwer verletzt zurück. Menschen, versteckt unter den Fellen der gefallenen Keiler, umringen ihn und verletzten ihn weiter. Dem zuwider glaubt Okkoto im Wahn seine Armee sei von den Toten auferstanden. Obgleich er bereits dabei ist zu einem Dämon zu werden, führt er die Jäger noch zum Zentrum des Waldes, zum Teich, an welchem der Shishigami erscheint.  

\subsubsection*{Der Wald: Shishigami und die Kodama} 
Als Ashitaka das erste Mal den Wald des Shishigami betritt, begegnen ihm die Kodama, kleine geisterhafte Geschöpfe, denen ein kindliches Gemüt inne wohnt. Sie sind eigentlich ein Zeichen dafür, dass der Wald gesund ist. Sie sind sozusagen seine weissen Blutkörperchen. In diesem Sinne weisen sie ihm auch den weg zum Herz des Waldes. Die verletzten Schmiedebewohner, welche Ashitaka aus dem Fluss gezogen hat und die er zu ihrem Dorf bringen möchte, fürchten sich vor den kleinen Wesen. Als Ashitaka jedoch sieht, dass sein Reittier auch in der Anwesenheit der Kodama ruhig bleibt, sieht er in ihnen keine direkte Gefahr. Nichtsdestotrotz schliesst er die Möglichkeit nicht aus, dass sie ihn in die Irre führen könnten. 

In der Mitte des Waldes findet sich ein seichter See. Hier wohnt der Waldgott Shishigami. Er hat die Erscheinung eines mächtigen Hirsches. An der Stelle eines Tierkopfs hat er aber menschliches Gesicht. Wenn die Sonne sich senkt und es Nacht wird kommt der Shishigami zur Lichtung beim See und verwandelt sich in den Nachtwandler. In einer ansatzweise humanoiden Form wächst er in der Grösse über den Wald hinaus und schreitet substanzlos durch den Wald. Mit den ersten Sonnenstrahlen kehrt er zur Lichtung zurück und verwandelt sich wieder in seine Hirschform. 

Der Shishigami ist der Herr dieses Waldes, als solcher kann er Leben geben oder Leben nehmen. Besonders deutlich wird das bei einer Nahaufnahme, als er über die Lichtung schreitet. Mit jedem Auftreten wachsen Blumen und Pflanzen, es wuchert regelrecht. Doch in dem Moment, wo sich der Fuss wieder hebt verwelkt alles und zurück bleibt ein kleiner Fleck toter Erde. Nicht nur die Tiere und Götter des Waldes wissen von den seltsamen Fähigkeiten von Shishigami. Der Kaiser beauftragt Eboshi den Kopf des Shishigami zu bringen, da er glaubt, der Kopf könne jede Wunde heilen. In der Tat rettet Shishigami Ashitaka das Leben, in dem er die Schusswunde heilt, die Ashitaka abbekommen hat, als er versuchte San von den Schmiedebewohnern zu beschützen. Doch zu Ashitakas Leid erkennt er, dass der Shishigami den Fluch des wütenden Keilerdämons nicht entfernt hat, und dass ihm somit immer noch ein schmerzlicher Tod bevorsteht. Shishigami ist kein allgültiger und alles heilender Gott. Wer ihn aufsucht, kann auf Heilung hoffen, muss aber auch den Tod erwarten. Moro nennt den Tatarigame feige, dass er sich dem Shishigami nicht gestellt hat. Der Waldgott hätte ihn heilen können, und wenn nicht, dann hätte er ihn getötet und Nago hätte nicht zu dem werden müssen, was er am Ende war. Es scheint jedoch, dass Nagos Furcht vor dem Shishigami berechtig gewesen war. Okkoto stirbt unter Shishigamis Berührung. 

Wo die Tiergötter gegen die Menschen und ihre Zerstörung des Waldes kämpfen, scheint der Waldgott selbst geradezu gleichgültig. Die Wildschweine warten bis am Ende darauf, dass Shishigami ihnen hilft die Menschen zu verjagen. Doch der Shishigami tut nichts dergleichen. Selbst als ihn Eboshi anschiesst, passiert nicht mehr, als dass er einerseits für einen kurzen Moment im Wasser, über welches er sonst läuft, einsinkt, bevor er seinen Gang fortsetzt. Zweitens lässt er aus dem Gewehr, mit welchem Eboshi auf ihn zielt, Pflanzen wachsen. Einmal mehr gibt er ein Bild von Leben und Tod. Als Eboshi es endlich schafft dem Waldgott den Kopf vom Rumpf zu schiessen, quillt das Innere des Waldgottes aus ihm heraus und zerstört alles auf seinem Weg. Im gleichen Moment fallen auch die Kodama von den Bäumen: der Wald stirbt. Im kopflosen Zustand bringt der Shishigami nur Zerstörung und Tod hervor. Erst als San und Ashitaka es fertig bringen den Kopf zurück zugeben, findet die Vernichtung ein Ende und in einer mächtigen Erschütterung wird aus der Zerstörrung neues Leben geschaffen. 
\subsection{Das wandelnde Schloss}
%!TEX root = Animismus_in_Anime.tex
%%%%%%%%%% HOWLS MOVING CASTLE %%%%%%%%%%%%%%%
\subsubsection{Filmhintergrund} 
\textsc{Das wandelnde Schloss} ist Miyazakis Adaption des Buches \glqq Sophie im Schloss des Zauberers\grqq~ von Diana Wynne Jones. Im Buch spielt die Geschichte wenigstens zum Teil in Wales. Also hat sich auch Miyazaki für ein europäisches Setting entschieden. So sieht man des Öfteren scharfkantige mit Schnee be\-deckte Bergspitzen, raue, aber saftig grüne Alpenwiesen und wunderschöne Täler mit glasklaren Bächen und Seen. Als Vorlage dienten unter anderem die europäischen Städte: Cardiff, Colmar, Heidelberg und Paris.\footnote{\textsc{Nieder} 2006: 107.} Obwohl es sich bei dem Film um eine Adaption handelt, weicht Miyazaki so stark von der Vorlage ab, das manche Dinge nur mit Hilfe des Buches verständlich zu sein scheinen. 
Die grösste Veränderung betrifft den Charakter der Hexe aus dem Ödland. Im Buch trägt sie eindeutig die Rolle der bösen Antagonistin, während im Film die Bürde des Gegenspielers auf verschiedene Charakter verteilt wird. Dies ist typisch für Miyazaki. In seinen Filmen gibt es keine klare Linie zwischen Gut und Böse. 

\subsubsection{Zusammenfassung} 
Der Zauberer Hauro zieht in seinem wandelnden Schloss umher. Es wird ge\-munkelt, er fresse die Herzen hübscher Mädchen. Sophie, die sich im Schatten ihrer hübschen Schwester und Mutter sieht, ist unzufrieden mit sich und ihrem Leben als Hutmacherin in dem Laden, den sie von ihrem verstorbenen Vater übernommen hat.

Eines Tages eilt ihr ein fremder Schönling zu Hilfe, um zwei übergriffigen Männern zu entkommen. Sophie verliebt sich in den jungen Mann, von dem sie aber vermutet, dass es sich um den Zauberer Hauro handelt. Diese kurze Begegnung reicht bereits aus, um die Aufmerksamkeit der Hexe aus dem Ödland auf sich zu ziehen. Diese hat scheinbar noch eine offene Rechnung mit Hauro. In der Folge wird Sophie durch einen Fluch in eine 80-jährige Greisin verwandelt. 

Auf der Suche nach etwas, was ihren Fluch brechen kann, findet sich Sophie bald darauf im wandelnden Schloss des Zauberers Hauro wieder. Sophie heuert kurzerhand als Hausdame und Putzfrau im Schloss an. Sie schliesst einen Handel mit Calcifer, dem Feuerdämon, welcher das Schloss steuert und bewegt: Er verspricht ihren Fluch zu brechen, wenn sie ihn von dem Packt mit Hauro befreit. In der Zwischenzeit ist ein offener Krieg zwischen den Nachbarländern ausgebrochen. Hauro, in den beiden Ländern unter verschiedenen Namen bekannt, soll auf beiden Seiten mitkämpfen. Erst mit Sophies Hilfe findet Hauro den Mut sich nicht mehr zu verstecken; Er stellt sich seiner Verantwortung. 

Es müssen aber erst noch viele Abenteuer bestanden werden, bis die beiden erkennen, dass sie einander lieben. Erst dann kann Sophie dem Zauberer sein flammendes Herz wieder zurück in seine Brust drücken und zugleich den Packt mit Calcifer lösen und ihren eigenen Fluch brechen. 

\subsubsection{Figurenanalyse} 
Auffallend in diesem Film ist, dass praktisch alle Charakteren eine Metamorphose im Verlauf der Geschichte durchgehen. Aus dem eitlen blonden und gleich\-zeitig feigen Zauberer Hauro wird ein verantwortungsvoller liebender junger Mann. Die Hexe aus dem Ödland verwandelt sich zu einer tattrigen Frau. Eine Vogelscheuche mit einer Rübe als Kopf wird zum Prinzen. Und aus der alten Sophie wird wieder ein junges Mädchen.  

Obwohl viele der Charaktere interessante Untersuchungsgegenstände ergeben würden, beschränken wir uns hier auf das wandelnde Schloss und den Feuerdämon. Die Geschichte zwischen Sophie und dem Zauberer Hauro steht zwar inhaltlich im Zentrum des Filmes, für unser Interesse sind aber das Schloss und der Feuerdämon ergiebiger. Auf die lebendige Vogelscheuche wird nicht weiter eingegangen, da sie nur eine auslassbare Nebenhandlung darstellt. 

\subsubsection*{Calcifer und das wandelnde Schloss} 
Auch das titelgebende Schloss durchläuft mehrere Schritte der Verwandlung. So ist es am Anfang eine riesige Maschinerie mit zahllosen Türmen, Röhren, Kammern und Öffnungen, deren Sinn und Zweck man nicht erraten kann. Es scheppert und klappert, pfeifft und knarrt mit jedem Schritt. Das Schloss läuft auf vier Vogelbeinen, die aus Metall hergestellt sind. Rostrot bis -braun dominiert das Konstrukt. Den Rumpf kann man in zwei Hauptteile unterteilen: Oben finden sich Schornsteine, Hausteile, Masten und schwere Kuppeln mit Guckrohren. Der untere Teil sieht aus wie ein Fisch mit Rübennase auf vier stelzigen Vogelbeinen. Ein langer Schlitz, der in zwei Gucklöchern endet, erzeugt die Illusion von Augen und Mund. Auf der Hinterseite des unteren Teiles ist eine senkrecht stehende Schwanzflosse angebracht. Die Last der oberen Teile schwankt bei jedem Schritt. Es ist vielmehr ein wandelndes Ungetüm als ein wandelndes Schloss. 

In dem Moment, in dem Sophie Calcifer aus dem Schloss trägt, verliert das Gebäude seine Integrität und sackt in sich zusammen. Sophie kehrt wieder mit Calcifer zurück und bittet ihn das Schloss erneut mobil zu machen. Erst nachdem sie ihm ihren Zopf opfert, verfügt Calcifer wieder über genügend Energie, um wenigstens einen Teil des Schlosses zu beleben. Was dabei heraus kommt ist eine viel kleinere und agilere Version des vorigen Baus. Dazu wurde viel unnützer Ballast abgeworfen, doch mangelt es nun auch an Komfort und Sicherheit. 
Im weiteren Verlauf der Geschichte erlischt Calcifer nahezu. Mit dem Leben, welches dem Feuerdämon entweicht, zerfällt auch das wandelnde Schloss. Nach einer Nacht, in der Calcifer zu einer kleine blauen Flamme reduziert wird, ist alles, was vom Schloss noch übrig bleibt eine hölzerne Plattform, getragen von zwei Beinen. 

Das Wesen von Calcifer und dem wandelnden Schloss ist untrennbar miteinander verschmolzen. Calcifer gibt dem Schloss Leben, und das Schloss ist der Körper, in dem Calcifer wohnt. 
\newpage

\section{Analyse}
%!TEX root = Animismus_in_Anime.tex
\subsection{Animismus in Anime}
Alt oder neuer Animismus. Japanischer Shinto. Wo zu finden und was heisst das?

\subsubsection*{Mononoke}
Da historisch passt sehr gut in Shinto und somit in den Geschichtlichen Animismus (in wie fern den alten). Harveys Ansatz vom respektvollen Leben mit andern people ist zentrale Botschaft des Filmes. Boyer lässt sich zwar Anwendung, aber ohne überraschende Erkenntnis. Tiergötter sind TIERE mit PERSONEN-Eigenschaften. Moro: ontologischer Bruch: hat eine Menschentochter. Interessant hier Shishigami: Auch wenn die gründe nicht erklärt werden, so geht man davon aus, dass der Gott entscheidetet wen er leben lässt und wen er tötet. Er muss also eine Art von Information besitzen. Anderes Beispiel aus Boyer: Tote und das Ritual -> Nago. Das NATUR OBJEKT toter Nago bewegt sich nicht mehr, aber er spricht! Die Seherin vollführt ein Ritual um sich und ihr Dorf vor dem Zorn des Gefallen Gottes zu schützen. Diese beiden helfen aber nur ein religiöses Phänomen zu identifizieren, jedoch nicht umbedingt zu erklären, warum das auch für uns westler so interessant ist. Oder ist es gerade, dass das Ritual anders ist, aber die Notwendigkeit (Schablone davon) in uns allen vorhanden und wir es deswegen verstehen? Letztlich noch kurz die falschen Wildschweine: Unheimlicfh auch für den Betrachter. Umgekehrt von den Tiergöttern: PERSON die aussieht wie TIER.

\subsubsection*{Wandelndes Schloss}
Typisch Animistisch gibt es wenig hier anzusetzten. Aus einer Zeit (Europa Epoche) wo man den Animismus nicht mehr gefolgt ist. Animismus gehört nicht mehr zur Narration. Man kann höchstens Animistische Elemente identifizieren. 

Wir wissen, dass Calcifier das Schloss belebt - animiert. Doch wie wird dieser Eindruck, dass eine Konstruktion belebt ist an den Rezeptionist vermittelt?

Obwohl Calcifer das Schloss gebaut hat, steuert und auch seine Emotionen durch das Schloss geäussert werden, so es für den Rezipienten doch ein eigenes Wesen.


Calcifer redet und interagiert mit den anderen Bewohnern des Schlosses. Das macht ihn, wenn man Harveys Ansatz folgt eindeutig zu einer Person. Um es für den Betrachter einfacher zu machen, das Feuer als Person zu sehen bekommt Calcifer zwei grosse Glupschaugen und ein Mund von variabler Grösse, in das er sich gerne Holzstücke reinstopft. Die Augen sind in der Regel rund, weisse Kreise mit schwarzen Pupillen. Doch mit dem flackern seines ganzen Körpers verändert sich die Form der Augen zwischen durch unmerklich, so dass sie schlitzförmiger werden und einen gefährlichen (dämonischen) Eindruck machen. Seine Flammenform, welche keine feste Umrisslinie hat flackert beständig und von Zeit zu Zeit lösen sich kleine Flämmchen von ihm.

\subsection{Kognitiver Ansatz}
Welten unterscheidung: Fiktive Welt ist eventuell der Welt welcher viele Beispiele entnommen wurden näher. Unser rationales westliches Denken schliesst übernatürliches in der Regel aus. In ethnischen Gruppen kann Magie und übernatürliches zum Alltag gehören. Daher eignet sich Boyers Methode also trotzdem, oder erst recht?

\subsubsection*{Wandelnde Schloss}
Haus als zur ontologischen Kategorie Werkzeug gehörig. Damit erreichen wir den Bruch/Verletzung dadurch, dass das Haus eine Selbstbestimmung hat.

Schablone vom Gebäude, Fixe Umgebung welche sich dennoch bewegt. Noch deutlicher wird das beim Rübenkopf, welcher in keiner Weise die Lebendigkeit des Schlosses zeigt (Keine Mimik, nichts bewegliches) aber dennoch in die gleiche Schublage gehört, da er sich zumindest selbständig bewegen kann.
Obwohl ihm vorigen Kapitel mehrheitlich auf das Schloss und nicht auf Calcifer eingegangen wurde, trotzdem noch eine Bemerkung: In der Szene, wo man sieht, wie die Lichter vom Himmel fallen, von denen auch Calcifer eines ist, verpuffen die meisten. Die springen vielleicht noch ein paar wenige Male vom Boden ab, bevor sie endgültig verglühen. Diese Lichter alleine reichen nicht aus, um unsere Aufmerksamkeit auf sich zu ziehen (trotz unheimlich schöner Animation), da ihnen der Bruch ihrer Ontologie fehlt. \emph{Farbige Lichter fallen vom Himmel und springen in Gestalt von Menschen ein paar Mal auf dem Boden, bevor sie verglühen.} Anders verhält sich das mit Calcifer, welcher nachdem ihm von Hauro das Herz gegeben wurde und später durch Sophie ein Leben ohne Herz ermöglicht wurde.

Abgesehen von den äusseren Merkmale welche dem Schloss tierähnlicher machen kommen seine Bewegungen und seine Reaktionen. Durch das individuelle Bewegen und langsame Vergrössern und Verkleinern einzelner Teile gibt dem Schloss etwas organisches. 

Es stellt sich natürlich die Frage, inwiefern sich aus Calcifers Beseeltheit auch die des Schlosses schliessen lässt, da das Schloss von Calcifer gebaut und gesteuert wird, ist es im Prinzip Spiegel von Calcifer Zustand.

\subsubsection*{Mononoke} 
\newpage

\addcontentsline{toc}{section}{Schlusswort}
%!TEX root = Animismus_in_Anime.tex
\section*{Schlusswort}
Die Kriterien von Graham Harvey und Pascal Boyer haben es möglich gemacht animistische Elemente in den Filmen zu erkennen. Ein anderer Ansatz hätte auch sein können, den \emph{japanischen} Animismus zu untersuchen anhand der Filme zu untersuchen und zu schauen, wie genau er verwendet wird und was für Unterschiede zwischen dem was man im Film und dem historischen Shintoismus findet. Doch Ziel dieser Arbeit war, die Filme, als auch den Animismus interkulturell zu betrachten. Aus diesem Grund ist weder auf den Shintoismus noch auf Elemente der japanischen Kultur in der Analyse eingegangen geworden. Die dahinterliegende Absicht dieser Arbeit ist, eine mögliche Begründung zu finden, weshalb die Filme Miyazakis auch ausserhalb ihres Kulturkreises, wo viele Hinweise und Anspielungen unverstanden bleiben, trotzdem auf so grossen Anklang stossen. Gibt es eine Zusammenhang zwischen den animistischen Elementen und dem Erfolg dieser Geschichten?

Diese Frage lässt sich aus den Untersuchungen nicht beantworten. Wir sehen aber, dass der Animismus eine wesentliche Rolle in den Filmen spielt. Einerseits lässt sich das aus Harveys Sicht so begründen, dass wir in einer Welt, wo wir die Konsequenzen unseres gierigen und respektlosen Handelns gegenüber unserer Umwelt langsam zu spüren bekommen, die Vorstellung einer Utopie, wo in Harmonie mit der Umwelt mit all seinen Bewohnern gelebt wird, erstreben. Anderseits bietet uns Boyer eine Erklärung, warum wir Geschichte welche von Übernatürlichem Handeln (das schliesst Animismus mit ein), so faszinierend finden. Es könnte hierbei hilfreich sein, weitere Filme Miyazakis zu untersuchen. Zum Beispiel sein Erstlingswerk \textsc{Nausicaä aus dem Tal der Winde}, das in einer Welt spielt, in der Menschen die Erde nahezu zerstört haben und es um die Wiederversöhnung mit der Natur geht. Oder \textsc{Chihiros Reise ins Zauberland}, welches wie der Titel schon sagt, eine Fülle an übernatürlichen Wesen und Dinge bietet. Zudem müsste für eine weiterführende Interpredation auch japanische Ausdrücke und Namen vermehrt berücksichtigt werden.
\newpage

\addcontentsline{toc}{section}{Bibliographie}


%%%%%%%%%%%%%%%%%%%%%%%%%%%%%%%%%%%%%%%%%%%%%%%%%%%%%%%%%
% Bibliographie
\newpage
\renewcommand\refname{Bibliographie}
\begin{thebibliography}{9}

\subsection*{Literatur}

\bibitem{rgg4}
	Betz, Hans Dieter. (Hrsg.). (1998-2007).
	\emph{Religion und Geschichte in der Gegenwart 4. Handwörterbuch für Theologie und Religionswissenschaft.} 4., völlig neu bearb. Aufl. Tübingen: Mohr Siebeck.

\bibitem{boyer04}
	Boyer, Pascal.
	(2004).
	\emph{Und Mensch schuf Gott.}
	Stuttgart: Klett-Cotta.

\bibitem{faulstich13}
	Faulstich, Werner.
	(2013).
	\emph{Grundkurs Filmanalyse.} 
	3. Aufl. 
	Paderborn: Wilhelm Fink Verlag.

\bibitem{hallowell}
	Hallowell, A. Irving. (1960). \glqq Ojibwa Ontology, Behavior, and World View\grqq. \emph{Culture in History: Essays in Honor of Paul Radin}, hg. v. Stanley Diamond. Columbia University Press. 

\bibitem{harvey06}
	Harvey, Graham.
	(2006).
	\emph{Respecting the Living World.}
	New York: Columbia University Press.

\bibitem{nieder06}
	Nieder, Julia. 
	(2006). 
	\emph{Die Filme von Hayao Miyazaki.}
	Marburg: Schüren-Verlag.

\bibitem{thomas12}
	Thomas, Jolyon.
	(2012).
	\emph{Drawing on tradition. Manga, Anime, and Religion in Contemporary Japan.}
	Honolulu: University of Hawai'i Press.

\subsection*{Internetquellen}

\bibitem{hanson}
	Hanson, Robin. (2001). \emph{A Review of Religion Explained: The Evolutionary Origins of Religious Thought}. \\ \url{http://mason.gmu.edu/~rhanson/religion.html} 10. September 2015.

\bibitem{asianspiegel1}
	Knüsel, Jan. (2015). \emph{Asienspiegel. News aus Japan.}\\ \url{http://asienspiegel.ch/2015/07/miyazaki-arbeitet-an-kurzanime} 10. September 2015.\\ \url{http://asienspiegel.ch/2014/08/grosse-ehre-fur-hayao-miyazaki/} 10. September 2015.

\bibitem{philo}
	Mittelstrass, Jürgen. (Hrsg.). (2005-2014).
	\emph{Enzyklopädie Philosophie und Wissenschaftstheorie. Gesamtwerk in acht Bänden.} 2. Aufl., Bd. 5. Stuttgart: Metzler, als online-PDF:\\ \url{http://www.uni-konstanz.de/FuF/Philo/Philosophie/philosophie/files/mem.pdf} 10. September 2015.  

\bibitem{nzjustis}
	New Zealand Ministry of Justice.
	(ohne Jahr).
	\emph{Whenua}.\\
	\url{http://www.justice.govt.nz/publications/publications-archived/2001/he-hinatore-ki-te-ao-maori-a-glimpse-into-the-maori-world/part-1-traditional-maori-concepts/whenua}	07. September 2015.

\bibitem{encyclo}
	Ruel, M. J. 
	(2008). 
	\glqq Marett, Robert Ranulph\grqq. \emph{International Encyclopedia of the Social Sciences}, in: Cengage Learning. (2015).\\ \url{http://www.encyclopedia.com/doc/1G2-3045000765.html} 07. September 2015.

\bibitem{tallman}
	Tallman, Dave. (ohne Jahr). \emph{Dave Tallman's Home Page. Santa Fe, New Mexico, U.S.A.}\\ \url{http://home.comcast.net/~rocksunner/miya_e.html} 10. September 2015.\\ \url{http://home.comcast.net/~rocksunner/mono3e.html#moro} 10. September 2015.

\bibitem{miyazakiweb}
	Team Ghiblink. (ohne Jahr). \emph{The Hayao MIYZAKI Web}.\\  
	\url{http://nausicaa.net/miyazaki/} 17. August 2015.\\
	\url{http://nausicaa.net/miyazaki/mh/faq.html} 10. September 2015.

\subsection*{Filmografie}

\bibitem{wandelndeSchloss}
	\textsc{Hauro no Ugoku Shiro / Das wandelnde Schloss}.\\
	Japan 2004. \\
	Drehbuch, Storyboard und Regie: Hayao Miyazaki.\\
	Produzent: Toshio Suzuki. \\
	Musik: Joe Hisaishi. \\
	Laufzeit: 119 Minuten. \\

\bibitem{nausicaa}
	\textsc{Kaze no Tani no Nausicaä / Nausicaä aus dem Tal der Winde}.\\ Japan 1984. \\
	Drehbuch, Storyboard und Regie: Hayao Miyazaki.\\
	Produzent: Isao Takahata. \\
	Musik: Joe Hisaishi. \\
	Laufzeit: 116 Minuten. \\

\bibitem{mononoke}
	\textsc{Mononokehime / Prinzessin Mononoke}.\\
	Japan 1997. \\
	Drehbuch, Storyboard und Regie: Hayao Miyazaki.\\
	Produzent: Toshio Suzuki. \\
	Musik: Joe Hisaishi. \\
	Laufzeit: 133 Minuten. \\

\bibitem{chihiro}
	\textsc{Sen to Chihiro no kamikakushi / Chihiros Reise ins Zauberland}.\\ Japan 2001. \\
	Drehbuch, Storyboard und Regie: Hayao Miyazaki.\\
	Produzent: Toshio Suzuki. \\
	Musik: Joe Hisaishi. \\
	Laufzeit: 125 Minuten. \\

\end{thebibliography}

\end{document}

\documentclass[a4paper]{article}

\usepackage[english]{babel}
\usepackage[utf8]{inputenc}
\usepackage{amsmath}
\usepackage{graphicx}
\usepackage[colorinlistoftodos]{todonotes}
\usepackage{enumitem, color, amssymb} % needed for roman numbering in enumerate enviroment


% Titel
\title{Animismus in Anime}
% Autor
\author{Judith Fuog}
% Datum (automatisch)
\date{\today}

\begin{document}
% Generiere Titel
\maketitle
% Generiere Inhalt
\tableofcontents
\newpage
%%%%%%%%%%%%%%%%%%%%%%%%%%%%%%%%%%%%%%%%%%%%%%%%%%%%%%%%%
% Text


%!TEX root = Animismus_in_Anime.tex
Die Verbindung von Animismus und Anime ist nicht nur dem Wort nach gegeben. Sie darf jedoch auch nicht einfach impliziert werden. Ob ein Zusammenhang besteht und in welcher Weise dieser gegeben ist, soll im Nachfolgenden untersucht werden. Dazu werden Filme von Regisseur und Produzent Hayao Miyazaki auf animistische Elemente untersucht und dabei festgehalten werden, was sie als solche ausmacht.

Angestossen wurde das Interesse vor allem daran, dass ein Film wie \textsc{Chihiros Reise ins Zauberland}\footnote{\textsc{Sen to Chihiro no kamikakushi / Chihiros Reise ins Zauberland}. (2001). Japan: Ghibli Studio. Drehbuch, Storyboard und Regie: Hayao Miyazaki. Produzent: Toshio Suzuki.} international einen sehr grossen Anklang gefunden hat, obwohl sehr viele darin vorkommende Elemente spezifisch für die japanische Kultur sind. Es stellt sich die Frage wie es kommt, dass ein solcher Film dennoch von Personen ausserhalb dieses Kulturkreises verstanden werden kann. Wenn man sich etwas mit Miyazakis Filmen beschäftigt so kann man verschiedene immer wieder auftauchende Motive erkennen. Zu den bedeutensten gehört die Faszination des Fliegens und der Kampf zwischen der Natur und den Menschen. In dieser Arbeit spielt vorallem letzteres eine zentrale Rolle. Es soll im Zusammenhang von Kultur und Religion, noch etwas genauer gesagt im Zusammenhang mit Animismus untersucht werden. Der Animismus stellt gerade heute einen schwer greifbaren Begriff dar. Wie viele Themen und Begriffe, welche der Kolonialzeit entsprungen sind, ist eine saubere Definition oder Abgrenzung schwierig. Dennoch ist es naheliegend von Animismus zu sprechen, wenn Wolf- und Keilergötter ihren Wald schützen oder ein von einem Feuergeist animiertes Schloss durch die Welt zieht. Wir finden hier also Dinge, aber insbesondere die Natur selbst als beseelt. 

Die in der Wissenschaft gebräuchlichen Beispiele von Geistern, Hexen und anderen übernatürlichen Phänomenen, sind für die Ethnien, aus denen diese Beispiele stammen, real. Im Unterschied dazu nimmt der westliche Betrachter eine vom Glauben unabhängige, distanziert-analytische Position ein. Die Betrachter von Filmen hingegen bilden eine dritte Herangehensweise, weil sie sich mit den Charakteren identizifieren und die fiktive Umgebungswelt leben können. Filme schaffen somit einen Rahmen, um zumindest temporär in ein System von übernatürlichen und phantastischen Phänomenen abzutauchen, ohne seine rationale und faktische Weltsicht aufgeben zu müssen. Diese Kombination aus Distanz und Nähe gibt uns die Möglichkeit uns selbst zu studieren, obwohl sich die Vorstellung von Animismus auf die Untersuchung von Völker und Ethnien konzentriert, die sich dem westlichen Kolonialismus zum Trotz ihre Kultur und Religion erhalten konnten. Das Studium des Animismus ist somit fast zwangsläufig eines, das die Studierenden aus ihrer unmittelbaren Umwelt weg führt. 

Es mag nach dieser Argumentation etwas widersprüchlich scheinen, dass ausgerechnet japanische Animationsfilme Gegenstand der Untersuchung sind. Es gibt im Bereich der Animationsfilme genügend Alternativen\footnote{Um ein Beispiel zu nennen: \textsc{Song of the Sea} (2014) und \textsc{The Book of Kells} (2009) produziert von Cartoon Saloon, einem irischen Animations Studio.}, welche kulturell näher stehen. Doch am Ende ist die kulturelle Prägung des Rezeptionisten für diese Untersuchung entscheidend, und nicht die des Filmes. 

In den Filmen \textsc{Prinziessin Mononoke}\footnote{\textsc{Mononokehime / Prinzessin Mononoke}. (1997). Japan: Ghibli Studio. Drehbuch, Storyboard und Regie: Hayao Miyazaki. Produzent: Toshio Suzuki.} und \textsc{Das wandelnde Schloss}\footnote{\textsc{Hauro no Ugoku Shiro / Das wandelnde Schloss}. (2004). Japan: Ghibli Studio. Drehbuch, Storyboard und Regie: Hayao Miyazaki. Produzent: Toshio Suzuki.} haben wir einerseits ist das alte Japan, reich an Mythen und Legenden. Hier wandeln Götter und Geister unter den Sterblichen. Wenn auch die Welt angereichert ist mit phantastischen Wesen, so erkennen wir darin doch unsere eigene Vergangenheit in dieser Welt. Andrerseits haben wir eine Märchenwelt, ein gutes Jahrhundert jüngeres Europa, wo Magie und Zauberei zum Alltag gehören. Diese Welt ist uns zeitlich zwar näher, doch gerade der breite Gebrauch von Magie in einer Zeit, an die wir uns noch zu erinnern glauben, entfremdet sie für den Betrachter. In beide Welten ist das Übernatürliche Grund zur Furcht oder Faszination, jedoch ist niemand über das Vorhandensein dessen überrascht.

Um die Frage zu beantworten, warum diese Filme, welche für ein japanisches Publikum gemacht sind auch bei uns Anklang finden, werde ich mit Pascal Boyers Ansatz der kognitiven Religionswissenschaft arbeiten. Eine ganz andere Sicht bietet hingegen Harvey. Seine Interpretation von Animismus basiert auf einem gegenseitigen respektvollen Umgang, nicht nur zwischen Menschen sondern zwischen allem was belebt ist.

\medskip

Im ersten Kapitel soll dargelegt werden wie Animismus für die vorliegende Arbeit zu verstehen sei. Dazu wird zuerst eine kurzere Übersicht zum historischen Begriff des Animismus und seine Verwendung gegeben. Es folgt ein kurzer Abschnitt über den japanischen Animismus, den Shintoismus. Danach werden wir Graham Harves Werk \emph{Animism. Respecting the Living World} betrachten um eine moderne Interpretation des Animismus kennen zu lernen.\footnote{Harvey, Graham. (2006). \emph{Respecting the Living World.} New York: Columbia University Press.} Mit Pascal Boyers \emph{Und Mensch schuf Gott} als Vertreter der kognitiven Religionswissenschaft bekommen wir weitere Kriterien zur Untersuchung animistischer Elemente.\footnote{Boyer, Pascal. (2004).\emph{Und Mensch schuf Gott.} Stuttgart: Klett-Cotta.} Dem Teil über Animismus folgt im zweiten Kapitel eine Vertiefung über japanische Animationsfilme und ein biografischer Abriss über Hayao Miyazaki, sowie die Analyse der Filme \textsc{Prinzessin Mononoke} und \textsc{Das wandelnde Schloss}. In Kapitel drei folgt die Anwendung der Methoden nach Boyer und Harvey. Schlussendlich steht die Frage ob es diese animistischen Elemente sind, welche den Filmen von Miyazaki helfen, auch bei einem nicht japanischen Publikum erfolgreich zu sein.
%!TEX root = Animismus_in_Anime.tex
Der Ausdruck \emph{Animismus}, von lateinisch \emph{anima} für Seele\footnote{Auch: Atem, Leben.}, wurde von Stahl erstmals verwendet und durch im Jahr 1871 von Edward Tylor in Primitive Culture eingeführt. Wie bei vielen Begriffen in der Religionswissenschaft, trägt auch dieser mehrere Bedeutungen und kann verschieden verwendet werden. In der Regel wird unter einer animistischen Religion eine schriftlose Religion verstanden. Früher wurden diese gerne als Naturreligionen oder als achaische oder primitive Religionen bezeichnet. In diesem Zusammenhang, aber nicht deckend, versteht man unter Animismus auch den Glauben an eine beseelte Umwelt. Somit ist der Mensch nicht das einzige beseelte Wesen, sondern auch Tiere und Naturobjekte können beseelt sein. Letztlich kann mit Animismus auch einfach der Glaube an Geister und Seelen verstanden werden.\footnote{RGG, 4. Aufl., 1 (2008). Animismus: 504-506.}

Es soll an dieser Stelle zunächst ein kurzer historischer Abriss des Animismus gegeben werden. Danach wird Graham Harveys Animismus \emph{Respecting the Living World} als Vertreter eines modernen Animismus vorgestellt. Schliesslich verlassen wir das ausdrückliche Gebiet des Animismus um eine ganz andere Perspektive auf Religion einzunehmen. Dazu wenden wir uns der kognitiven Religionswissenschaft zu. 

\subsection{Der alte Animismus}
Die nachfolgende Zusammenfassung basiert primär auf jener von Graham Harvey in \emph{Animism. Respecting the Living World}. 

Wie schon festgehalten, wird der Begriff des Animismus erstmals von Georg Ernst Stahl (1659-1734) verwendet. Er stellt die Theorie auf, dass es ein physikalisches Element gäbe, welches belebt. Je mehr Anima vorhanden ist, desto belebter ist ein Objekt. Während eine tote Person oder ein Stein keine Anima (mehr) aufweist, besitzt eine lebendige Person viel Anima. Auch Tiere und Pflanzen besitzen dem nach Anima, jedoch weniger als der Mensch.

James Frazer (1854-1914) stellt später die Theorie auf, dass die Wilden (savage) glauben, dass Pflanzen und Tiere gleichermassen beseelt seien wie die Menschen. Wenn nun die Wilden zu glauben beginnen, dass Pflanzen und Tiere nur temporär und durch eine andere Wesenheitbeseelt seien, entwickelt sich der Animismus in einer Folgestufe zum Polytheismus.

Edward Tylor (1932-1917) beschreibt in seinem Werk \emph{Primitive Culture} (Die Anfänge der Cultur) den Animismus als der Ursprung der Religion. Der Animismus würde im Laufe der Weiterentwicklung und Zivilisierung einer Kultur durch verschiedene andere Formen der Religion abgelöst werden. Nichtsdestotrotz würden sich auch in einer hochentwickelten und komplexen Religion noch Überreste der alten Religion in Form von Aberglaube finden.

Diese Religionstheorie Tylors kritisiert Robert R. Marett (1866-1943), weil Phänomene wie Ehrfurcht vor den Tieren, vor Blut oder Naturgewalten nicht berücksichtigt würde. Stattdessen führt Marett eine Dichotomie vom Alltäglichen und vom Ausseralltäglichen ein, wobei Letzteres durch Religion erklärt und verarbeitet würde. Das Ausseralltägliche teilt er weiter in die Begriffe Mana und Tabu ein. Mana beschreibt dabei die Begegnung mit einer übermenschlichen Macht, während Tabu für Furcht und Kontaktvermeidung aufgrund von Gefahr steht. Folglich ordnet er Religion dem Mana zu, während er das, was mit Tabu verbunden wird, als Magie bezeichnet.\footnote{Ruel, M. J. (2008). \glqq Marett, Robert Ranulph\grqq. \emph{International Encyclopedia of the Social Sciences}, in: Cengage Learning. (2015). \url{http://www.encyclopedia.com/doc/1G2-3045000765.html} 07. September 2015.} 

Ebenso wie Tylor, geht Emil Durkheim (1858-1917) von einer Ursprungsreligion aus. Im Unterschied dazu setzt er jedoch den Totemismus an die Stelle des Animismus. Er tut dies, weil er die sozialen Aspekte über die Erfahrungen des Individuums stellt.

Zusammenfassend kann gesagt werden, dass der \emph{alte} Animismus als Vorstufe für eine höher entwickelte Religion gesehen wurde. In der Geburtsstunde der Religionssoziologie und der Anthropologie war der Westen überzeugt, dass es eine lineare Entwicklung gebe, wobei der Westen auf der höchsten Stufe stehe. Weile diese Ansicht heutzutage als veraltet gilt, haben viele der früheren Werke über den Animismus ihre Bedeutung für die heutige Religionswissenschaft eingebüsst. Es gibt jedoch Versuche den Animismus neu zu definieren und ihm auf diese Weise eine neue Bedeutung in der Moderne zu geben. 

\subsubsection{Animismus und Shintoismus}
Shinto, oder auch Shintoismus bezeichnet die native japanische Religion. Ähnlich wie auch das Wort Bud\={o} setzt sich das Wort aus zwei japanischen Schriftzeichen zusammen, wobei das Zweite jeweils mit Weg übersetzt werden kann. Das erste Zeichen steht für \emph{Götter} oder \emph{Geister}, japanisch Kami oder ehrfürchtig Kami-sama. Dabei handelt es sich um übernatürliche Wesen, die in japanischen Mythen und Legenden vorkommen. Sie können auch natürlichen Phänomenen innewohnen oder Schutzpatron für eine jeweilige Region sein. Wird Kami als Terminus Technicus übernommen, kann Shinto als der \emph{Weg der Kami} übersetzt werden.\footnote{RGG, 4. Aufl., 7 (2004). Sintoismus: 1283-1286.}

Es wäre sicherlich interessant einen Abgleich zwischen dem Shintoismus und den animistischen Elemente, welche man in den Filmen findet, zu machen. Man könnte vergleichen wie nahe sich der historische Shintoismus und die moderne japanische Vorstellung sind. Das liegt jedoch nicht im Sinne dieser Arbeit, da hier der Animismus losgelöst von seiner Kultur gesucht wird. Deswegen dir der Shinto hier nur zur Vollständigkeit erwähnt, spielt aber für die weitere Untersuchung eine Nebensächliche Rolle.

\subsection{Harvey Graham: Ansätze für einen Modernen Animismus}
Der Animismus steckt heute insofern in einer Krise, als dass auf den \emph{alten} Animismus nicht einfach aufgebaut werden kann. Andrerseits sind die Phänomene des Animismus weiterhin interkulturell präsent. Die Phänomene an sich werden weiterhin von Ethnologen und Psychologen ernst genommen. Als Konsequenz werden sie jedoch als kognitiver Fehler, als Projektion, als Produkt einer überproduktiven Phantasie oder einer mangelnden Trennung von subjektiver und objektiver Welt eingeschätzt. 

Auf der anderen Seite wirkt die Moderne zu Gunsten des Animismus. Früher wurden fremde Kulturen belächelt. Man bezeichnete die Völker als primitiv, die an Naturgeister glaubten und diese anbeteten. Hundert Jahre später sehen wir unsere Existenzgrundlage bedroht, weil wir unsere Umwelt rücksichtslos ausgebeutet haben. Es ist daher verständlich, dass Haltungen welche die Natur in ein Gegenüber stellen und so eine respektvolle Interaktion ermöglichen, eine gewisse Sympathie erfahren. 

\subsubsection*{Graham Harvey}
Ein Beispiel dafür ist in Graham Harvey zu finden. Er sagt: 

\begin{quote}
	\glqq Animists are people who recognise that the world is full of persons, only some of whom are human, and that life is always lived in relationship with others. Animism is lived out in various ways that are all about learning to act respectfully towards and among other persons.\grqq ~(Harvey 2006: xi)
\end{quote}

Mit dieser Aussage beginnt Graham Harvey sein Buch \emph{Animism. Respecting the Living World}. Es ist bereits an dieser Stelle erkennbar, dass es Harvey in erster Linie darum geht eine bestimmte Lebenshaltung zu postulieren. Anhand zahlreicher Beispiele, welche er im Laufe seiner Forschung in Neuseeland, Australien, Hawaii, Neufundland, Niger, und Amerkia gesammelt hat, erklärt er, wie der Animismus zu verstehen sei. 

Harvey nimmt eine Unterscheidung zwischen \emph{altem} und \emph{neuem} Animismus vor. In der alten Vorstellung wird davon ausgegangen, dass Animisten Menschen sind, die nicht zwischen Objekt und Subjekt unterscheiden, sei es, weil sie es nicht können, sei es, weil sie es nicht wollen. Im Unterschied dazu suchen Neue Animisten Wege und Ansichten, wie sie mit andern Leuten richtig und respektvoll interagieren können. Zentral in Harveys Buch ist das Zusammenfassen von Menschen (humans) und Anderen-als-Menschen zu einer Übergruppe von Leuten (people). Es gibt also Leute, welche nicht Menschen sind. Dennoch kann und muss mit ihnen interagiert werden. Allerdings gibt es unter den Leuten auch hinterlistige und verschlagene Personen (Menschen und Andere-als-Menschen). Deshalb ist es wichtig allfällige Masken, Täuschungen und falsche Aussagen durchschauen zu können. Im Wissen etwa, dass es Leute gibt, die uns fressen möchten, ist es weise sowohl vorsichtig als auch konstruktiv im respektvollen Umgang mit den Anderen zu sein. 

Ich werde hier auf zwei seiner Beispiele eingehen. Das erste handelt von den Ojibwa, einem nordamerikanischen Indianerstamm. Harveys Überlegungen stützen sich dabei hauptsächlich auf Irving Hallowells Beobachtungen und Untersuchungen, wobei die Sprache im Zentrum steht\todo{Zitiere Hallowells Buch}. Als zweites soll Harveys Überlegungen zur Maori-Kunst dargestellt werden.

\subsubsection*{Die Sprache der Ojibwa}
Die Ojibwa geben uns ein Beispiel dafür, dass sich Animismus in der Grammatik der Sprache zeigen kann. So wie es im Deutschen (und verwandten Sprachen) eine Untescheidung zwischen männlich und weiblich gibt, unterscheidet die Ojibwe-Grammatik zwischen belebt (animated) und leblos (inanimated). Diese Unterscheidung ist keineswegs selbstredend. So wie wir „die Tasse“ oder „der Hund“ sagen, entspricht das grammatische Geschlecht nicht immer mit dem Geschlecht des Beschriebenen überein. In der Ojibwe-Sprache beschreibt die Grammatik auch die Steine als animiert. Doch als Antwort auf die Frage ob denn alle Steine leben würde, antwortete ein alter Ojibwe mit: „Nein. Aber ein paar schon“ (Harvey 06: 33). Aus diesem Beispiel geht hervor, dass der Animismus hier kein dogmatisches Glaubenssystem darstellt. Es ist möglich, dass ein Stein animiert ist, und doch lässt sich diese Aussage nicht auf alle Steine übertragen. Für diese Animisten ist also nicht grundsätzlich alles belebt. 

Eine weitere Anekdote erzählt von einem Stein, der durch einen weissen Händler ausgegraben wurde. Der Händler dachte, er gehöre zu einem zeremoniellen Pavillon. Also suchte er einen Ojibwa namens John auf. John beuge sich zum Stein und frage den Stein leise, ob er zu diesem Pavillon gehöre. Laut John antwortete der Stein, dass dem nicht so sei. Dieses Beispiel zeigt, dass mit dem Stein wie mit einer Person umgegangen wird. John spricht nicht \emph{zu} sondern \emph{mit} dem Stein (Harvey 06: 37). 

Weiter gibt es bei den Ojibwa auch Erzählungen von Steinen, die belebt sind und darüberhinaus anthropomorphe Merkmale besitzen. Dies sind zum Beispiel Steine, die so geformt sind, dass es aussieht, als ob sie einen Mund oder Augen haben. Gleichzeitig sind solche Merkmale nicht zwingend als Hinweis zur Beseeltheit des jeweiligen Steines zu verstehen. Das Aussehen kann trügen. Ein Stein gilt als animiert, wenn mit ihm gesprochen werden kann. Wenn man mit ihm, wie mit anderen Personen, interagieren kann. 

Ein nächstes und weitaus abstrakteres Beispiel findet sich bei den Saison\-geschichten (Seasonal Stories). Der Umgang mit diesen Geschichten entspricht dem respektvollen Umgang mit einer Person. Tatsächlich werden diese Geschichten Grossvater genannt und gelten entsprechend auch als ehrwürdig (Harvey 06: 42). Mit diesen Geschichten beschäftigt man sich nicht leichtfertig. Und auch wenn sie mitunter lustig sein können, so nimmt man sie doch ernst. Sie vermitteln Dinge von grosser Wichtigkeit, wenn man sich ihnen respektvoll annähert. 

\subsubsection*{Die Kunst der Maori}
Maori sind für ihre kunstvollen Schnitzereien von Pounamu Steinen\footnote{Sammelnbezeichnung der Maori für Nephrit-Jade und Bowenit. Im Englischen werden diese Steine schlicht \emph{greenstone} genannt.}, Knochen und Holz berühmt. Harvey möchte zeigen, dass diese Kunstwerke selbst (durch den Macher) beseelt sind.

Die Maori fühlen eine tiefe Verwandtschaft mit dem Ort an dem sie leben. Ein junger Mensch entwickelt sich nicht nur in Abhängigkeit seiner Familie und seines Clans, aber auch die Natur gehört zu seinen Vorvätern. Das Land wird als Quelle der Identität betrachtet. Es gehört und wird kontinuierlich geteilt von den Toten, den Lebenden und den Ungeborenen.\footnote{New Zealand Ministry of Justice. (ohne Jahr). \emph{Whenua}. 07. September 2015.

	\url{http://www.justice.govt.nz/publications/publications-archived/2001/he-hinatore-ki-te-ao-maori-a-glimpse-into-the-maori-world/part-1-traditional-maori-concepts/whenua}}
Es ist zum Beispiel Brauch, dass bei der Geburt eines Kindes die Plazenta vergraben wird. Somit ist das Neugeborene mit dem Ort verbunden.

Die Maori sehen in der Süsskartoffel nahe Verwandte, ohne deren Hilfe den Maori eine wichtige Nahrungsgrundlage fehlen würde. Ohne die Hilfe der Maori würde die Pflanze jedoch gar nicht erst wachsen und gedeihen können. Die Kartoffeln auszugraben und zu essen grenzt daher an Kannibalismus.\footnote{Kannibalismus ist unter Maori durchaus üblich. Dabei geht es in keiner Weise darum sich vom Menschenfleisch zu ernähren. Die Einverleibung fand von Freunden und Feinden statt.}

Ebenso stellt das Schnitzen von Knochen, welche ja in jedem Menschen vorhanden sind, keinen grösseren Eingriff dar, als das Fällen und Schnitzen von Bäumen und das Schnitzen von Holz. Eine Schnitzerei steht somit immer im Zusammenhang mit dem Nehmen von Leben. Die Überreste einer Schnitzerei werden jeweils zurückgegeben. Die kunstvolle Schnitzerei ist nicht dafür da, um davon abzulenken. Durch das Schnitzen findet eine Transformation statt, in der der Künstler das Potential das im Holz, Stein oder Knochen schlummert hervor bringt. Ein Pounamu Anhänger ist belebt und nicht einfach nur Schmuck oder Identität für den Träger. Er hat ein Geschlecht, einen Namen und verdient Respekt. 

\medskip
Die beiden aufgezeigten Kulturen in welchen Harvey von Animismus redet, zeigen, dass es sehr grosse Unterschiede darin gibt, wie Animisten mit der Welt um sie herum agieren. Dabei decken diese beiden Beispiele nur einen sehr kleinen Teil der Aspekte ab, welche Harvey dem Begriff Animismus zusammenfasst. Weitere Beispiele sollen bei der Filmanalyse an passender Stelle gezeigt werden. 





%!TEX root = Animismus_in_Anime.tex
\newpage
\subsection{Hayao Miyazaki}
Hayao Miyazaki, einer der bekanntesten Animations Produzent Japans, wurde während dem 2. Weltkrieg im Januar 1941 unweit von Tokyo geboren. Sein Vater arbeitete für seinen Bruder in der Maschinenbau Firma Miyazaki Airplanes.~\footnote{Dieser Hintergrund wird gerne gebraucht um Miyazakis Faszination vom Fliegen zu begründen.} 

Miyazakis Interesse an der Animiation wurde sehr früh schon durch den Animationsfilm \textsc{Panda and the Magic Serpent} geweckt, welcher in seinen Jugendjahren veröffentlich wurde. Obwohl er zunächst Politik und Wirtschaft an einer renommierten Universität studierte, zog es ihn nach Abschluss seines Diplomes ins Animationsgeschäft. Als Hintergrundzeicher fand er bei der derzeit führendem Studio Toei-Animation eine Anstellung und machte sich schnell einen Namen. Nebenbei veröffentlichte er unter einem Pseudonym er eine eigene Manga-Serie\todo{welche?} und sammelte wo immer möglich Erfahrung in Storyentwicklung und Produktion. Toei-Animation schickte seine Animatoren gelegentlich auf Reisen um um Skizzenstudien der Landschaften oder Städte zu machen.~\footnote{So zum Beispiel reiste Miyazaki für die Produktion von \textsc{Alpine Girl Heidi} nach Europa.} 

Sein Freund Isao Takahata verfilmte 1972 Miyazakis erste Kurzgeschichte (\textsc{Adventures of Panda and Friends}). 1978 übernimmt Miyazaki dann erstmals die Inszinierung einer Anime-Serie (\textsc{Boy Conan}) und im darauf folgenden Jahr führte er zum ersten Mal Filmregie (\textsc{Schloss des Caliostro}). Von da an bekam Miyazaki immer öfters die Leitung für die Inszenierungen von Anime-Serien.

1982 begann Miyazaki mit dem Manga \textsc{Nausicaä aus dem Tal der Winde}. Die einzelnen Teile erschienen mit zahlreichen Unterbrechungen in einem monatlich erscheinendem Magazin. Erst 1994 fand der Manga einen Abschluss. Doch bereits 1983 begannen die Vorarbeiten für eine Verfilmung der Geschichte unter der Leitung von Miyazaki. Der Film erschien im derauf folgendem Jahr in den japanischen Kinos. 

Miyazaki machte sich im Jahr darauf mit ein paar Kollegen von Toei-Animation selbständig und gründete das Animations Studio Ghibli\footnote{Warmer Wüstenwind}. Bei der Position als Regisseur und als Produzent wechseln sich Miyazaki und Takahata ab. Das Studio war finanziell nicht abgesichert und riskierte zunächst mit jeder Produktion seinen Ruin. Etwas mehr Sicherheit gewann das Ghibli Studio nach der Veröffentlichung des Filmes \textsc{My Neighbor Totoro}, da insbesondere der Merchandising ein grosser Erfolg (auch heute noch) verbuchen kann. Mit den weiteren Produktionen stieg die Firma Ghibli zu den erfolgreichsten und bekanntesten in Japan auf.

Miyazaki wollte sich, mit der Fertigstellung von \textsc{Prinzessin Mononoke} in 1997 eigentlich in den Ruhestand setzen,\footnote{Zitat von Miyazaki: +/- zu anstrengend} begann aber dann mit den Vorbereitungen für \textsc{Chihiros Reise ins Zauberland} welcher 2001 in die japanischen Kinos kam. Während der Film nicht nur nationale sonder auch internationale Preise gewann\footnote{Darunter Oscar in der Kategorie Bester Animationsfilm}, arbeitete Miyazaki auch schon am nächsten Projekt. \textsc{Das wandelnde Schloss} welches 2004 in Japan veröffentlicht wurde, bescherte erneut einen Einnahmerekord dar in den japanischen Kinos, erlangte jedoch international nicht mehr ganz so viel Aufmerksamkeit wie die beiden Vorgänger Filme. 

Mit \textsc{Wie der Wind sich hebt} kündete Miyazaki erneut seinen Rücktritt an. An einer Abendkonferenz in September 2013 erklärte Miyazaki, dass er keine abendfüllenden Anime-Filme mehr machen werde.\footnote{\texttt{http://asienspiegel.ch/2014/08/grosse-ehre-fur-hayao-miyazaki/}} Die Zeit der traditionellen Animation, wo man noch mit Hand zeichne sei vorbei. Auch bei seinen späteren Filmen wurden Computergrafiken nur vereinzelnd eingesetzt. Überraschenderweise wagt der 74 jährige Japaner doch den Sprung ins neue Zeitalter der Animation. In seinem neusten Projekt arbeitet er an einem 10 minütigen Kurzanime, welcher auf der Kurzgeschichte \textsc{Boro, die Raupe}. Erstmal will er einen vollständig computer animierten Anime machen.\footnote{\texttt{http://asienspiegel.ch/2015/07/miyazaki-arbeitet-an-kurzanime/}}

Natur und so?\todo{wo und was soll noch dazu kommen?}

\subsection{Japanische Animationsfilme}
\subsubsection{Geschichte}
Die Bezeichnung \emph{Anime} für japanische Animationsfilme ist eine Fremdbezeichnung, welche sich erst nach dem Krieg durch die amerikanische Übernahme, gegen d\={o}ga, den japanischen Begriff für Animation, durchgesetzt hat. Die Symbiose zwischen Anime und Manga (Comic) ist spezielle in Japan. Es ist häufig so, dass ein erfolgreicher Manga verfilmt wird, oder aber dass zu einem Anime hinter ein Manga gezeichnet wird. Anders als im europäischen Raum zielen Manga und Anime auch auf ein viel breiteres Publikum ab. Zwar sind viele Geschichten für Kinder gedacht, aber auch Erwachsene werden als Zielpublikum ernst genommen. Daher finden wir in den japanischen Animationsfilmen in der Regel eine grössere Inhaltliche Palette als in den amerikanischen Produktionen. Durch das weite Spektrum des Zielpublikums finden sich auch Produktionen an allen Genres. Nebst den Geschichten welche typischerweise japanische Märchen, Mythen und Legenden oder Science Fiction thematisieren, finden sich auch Horror, Historie, Romanzen, Komödien und Erotik ihren Platz. Beim Inhalt zeigt sich, dass obwohl häufig Elemente aus der Japanischen Kultur eine zentrale Rolle spielen, nicht davor gescheut wird auch westliche Elemente zu integrieren. Dennoch sind japanische Animationsfilme üblicherweise für ein japanisches Publikum gedacht.

Gerade nach der Kriegszeit übten die amerikanische Filmindustrie einen grossen Einfluss auf. Einerseits wurde versucht dem erfolgreichen Beispiel zu folgen, anderseits bestand auch der Drang sich davon abzugrenzen und sich auf die eigene Kultur zu konzentrieren. Dem US-Beispiel folgend entstanden in den 1950er Jahre etliche Animiations Studios. Anders als die amerikanischen Studios wie Disney setzte die japanische Filmindustrie mehr auf Quantität als auf Qualität. Das führt dazu, dass typische Anime in der Regel einfacher, weniger hyperrealistisch gestaltet sind als die amerikanischen. Neben der Machart unterscheiden sich die japanischen Animationen von amerikanischen oder auch europäischen durch ihren kulturellen Hintergrund. 

Im Shinto finden wir einige der Erklärungen für Eigenart der japanischen Animations Filme. Der traditionelle Shinto ist eine unorganisierte, schriftlose Religion. Im Shinto wird das Universum zudem prinzipiell als mehrdeutig betrachtet. Das macht sie ideal dafür, verschiedene (auch widersprüchliche) Konzepte in sich aufzunehmen. So wurde zum Beispiel der Buddhismus in den Shinto integriert und auch Elemente des Christentums, welches später nach Japan gelangte fanden ihren Platz im religiösen Gesamtverständnis der Japaner. Des weiteren wird nicht scharf zwischen Götter und Dämonen, gut und böse getrennt. Nach dem Kodex der Samurai steht die Absicht auch über der Handlung. Als Folge all dessen fehlt in den japanischen Geschichten in der Regel auch die Unterscheidung von Gut und Böse. Vielmehr steht der Protagonist und der Antagonist sich in einem Interessenskonflikt gegenüber. Die Interessen können sich in ihrer Essenz widersprechen und dennoch glaubhaft sein. 

Im Zentrum stehen die Motive der Charakteren. Ein Bildwechsel wird daher nicht unbedingt benutzt um den zeitlichen Verlauf zu markieren, sondern um einen Perspektivenwechsel zu ermöglichen. Für westliche Zuschauer gibt das den Eindruck einer Verlangsamung der Handlung. Dieser Fokus, zusammen mit der Eigenschaft, dass auch japanische Serien in der Regel abgeschlossen sind, geben den Charaktern der Geschichte die Möglichkeit dramatische Veränderungen zu durch gehen. In amerikanischen Produktionen fallen die Charakteren vergleichsweise flach und statisch mit wenig Möglichkeit zur Entwicklung auf Grund der episodenhaften Art aus.

Das ästhetische Prinzip von Wabi und Sabi\footnote{Definition, oder zumindest Andeutung} ist ebenfalls in den japanischen Filmen zu beobachtet. Auslassung ist genau so Teil eines Kunstwerks wie seine andern Bestandteile. Damit begründet sich auch das hohe Mass an Abstraktion zum Beispiel beim Charakterdesign.  

Eine weitere wichtige Rolle spielen symbolische Darstellungen. Gerade bei einer Analyse und einer damit verbunden Interpretation ist es wichtig sich aber bewusst zu sein, dass die japanische Kultur Grundlage der Interpretation sein muss. Im Gegensatz zu den Walt Disney Märchen wo die Standard Prinzessin blondes Haar trägt, ist die typische Haarfarbe für einen guten Charakter in japanischen Geschichten dunkel.~\footnote{Im Anime \textsc{Das wandelnde Schloss} welches später genauer betrachtet werden soll, trägt der Protagonist Hauro zunächst Blonde Haare. Doch als er endlich zu sich selbst und seinen Mut findet trägt er dunkle Haare.} 


\newpage
\section{Anwendung von kogn. Rel.}
Es lässt sich beides darin finden. Moderne Vorstellungen von Animismus und kognitive. Während das erste nur dabei hilf den Animismus zu Identifizieren, so dient das zweite der Begründung, warum es so erfolgreich ist.

\subsection{Das wandelnde Schloss}
Schablone vom Gebäude, Fixe Umgebung welche sich dennoch bewegt. Noch deutlicher wird das beim Rübenkopf, welcher in keiner Weise die Lebendigkeit des Schlosses zeigt (Keine Mimik, nichts bewegliches) aber dennoch in die gleiche Schublage gehört, da er sich zumindest selbständig bewegen kann.
Obwohl ihm vorigen Kapitel mehrheitlich auf das Schloss und nicht auf Calcifer eingegangen wurde, trotzdem noch eine Bemerkung: In der Szene, wo man sieht, wie die Lichter vom Himmel fallen, von denen auch Calcifer eines ist, verpuffen die meisten. Die springen vielleicht noch ein paar wenige Male vom Boden ab, bevor sie endgültig verglühen. Diese Lichter alleine reichen nicht aus, um unsere Aufmerksamkeit auf sich zu ziehen (trotz unheimlich schöner Animation), da ihnen der Bruch ihrer Ontologie fehlt. \emph{Farbige Lichter fallen vom Himmel und springen in Gestalt von Menschen ein paar Mal auf dem Boden, bevor sie verglühen.} Anders verhält sich das mit Calcifer, welcher nachdem ihm von Hauro das Herz gegeben wurde und später durch Sophie ein Leben ohne Herz ermöglicht wurde.

\subsection{Prinzessin Mononoke}
\subsubsection*{Tiergötter}
Tiere mit speziellen Attributen: Doppelter Schwanz, mehr Hauer, übernatürlich gross. Menschlicher Aspekt: Sie können (mit Ausgesuchten/speziellen) reden. Verhalten sich wie Menschen.

\subsubsection*{Der Waldgott und die Kodama}
Bruch Tier mit Menschengesicht. Sprich dafür nicht (im Vergleich mit den andern Tiergöttern) und agiert auch sonst nicht wie eine Person.

\section{Schluss}

\newpage
\begin{thebibliography}{9}

\bibitem{nieder06}
	Nieder, Julia. 
	(2006). 
	\emph{Die Filme von Hayao Miyazaki.}
	Marburg: Schüren-Verlag.

\bibitem{faulstich13}
	Faulstich, Werner.
	(2013).
	\emph{Grundkurs Filmanalyse.} 
	3. Aufl. 
	Paderborn: Wilhelm Fink Verlag.

\bibitem{boyer04}
	Boyer, Pascal.
	(2004).
	\emph{Und Mensch schuf Gott.}
	Stuttgart: Klett-Cotta.

\bibitem{harvey06}
	Harvey, Graham.
	(2006).
	\emph{Respecting the Living World.}
	New York: Columbia University Press.

\bibitem{thomas12}
	Thomas, Jolyon.
	(2012)
	\emph{Drawing on tradition. Manga, Anime, and Religion in Contemporary Japan.}
	Honolulu: University of Hawai'i Press.

\bibitem{miyazakiweb}
	Team Ghiblink. 
	(ohne Jahr). 
	\emph{The Hayao MIYZAKI Web}. 
	17. Aug. 2015. 
	\texttt{http://nausicaa.net/miyazaki/}

\bibitem{wandelndeSchloss}
	\textsc{Hauro no Ugoku Shiro / Das wandelnde Schloss}.
	Japan 2004. \\
	Drehbuch, Storyboard und Regie: Hayao Miyazaki.\\
	Produzent: Toshio Suzuki. \\
	Musik: Joe Hisaishi. \\
	Laufzeit: 119 Minuten. \\

\end{thebibliography}

\end{document}

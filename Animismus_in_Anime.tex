\documentclass[a4paper]{article}

%\usepackage[english]{babel}
\usepackage[utf8]{inputenc}
\usepackage{amsmath}
\usepackage{graphicx}
\usepackage[colorinlistoftodos]{todonotes}
\usepackage{enumitem, color, amssymb} % needed for roman numbering in enumerate enviroment
% deutsch
\usepackage[ngerman]{babel}
%\usepackage[ansinew]{inputenc}

\begin{document}
\begin{titlepage}
% Generiere Titel
	\begin{center}
	{\scshape\LARGE Institut für Religionswissenschaft Universität Bern \par}
	\vspace{1cm}
	{\huge\bfseries Animismus in Anime\par}
	{\large\bfseries Untersuchung zweier japanischen Animationsfilmen von Hayao Miyazaki auf animiste Elemente und deren Auswirkung in der internationalen Rezeption.\par}
	\vspace{2cm}
	{\Large\itshape Judith Fuog\par}
	\vfill
	\end{center}

% Titelseiten Text
\begin{minipage}{\textwidth}
	\begin{large}
	Art der Arbeit: Grosser religionswissenschaftlicher Essay\par
	Studiengang laut RSP: BA Min. (30 ECT) Science of Religion\par
	Fachsemester: 4.\par
	\hfill

	eingereicht bei: Prof. Dr. Jens Schlieter\par
	Abgabedatum: \today\par
	\hfill

	Mat.-Nr.: 09-926-809\par
	Anschrift: Sängergasse 25, CH-4054 Basel\par
	E-mail: judith.fuog@students.unibe.ch\par
	Tel.: 076 572 14 19\par
	\end{large}
\end{minipage}
\hfill
\end{titlepage}
\newpage
\tableofcontents
\newpage

%%%%%%%%%%%%%%%%%%%%%%%%%%%%%%%%%%%%%%%%%%%%%%%%%%%%%%%%%
% Text
\addcontentsline{toc}{section}{Einleitung}
\section*{Einleitung}

%!TEX root = Animismus_in_Anime.tex
Die Verbindung von Animismus und Anime ist nicht nur dem Wort nach gegeben. Sie darf jedoch auch nicht einfach impliziert werden. Ob ein Zusammenhang besteht und in welcher Weise dieser gegeben ist, soll im Nachfolgenden untersucht werden. Dazu werden Filme von Regisseur und Produzent Hayao Miyazaki auf animistische Elemente untersucht und dabei festgehalten werden, was sie als solche ausmacht.

Angestossen wurde das Interesse vor allem daran, dass ein Film wie \textsc{Chihiros Reise ins Zauberland}\footnote{\textsc{Sen to Chihiro no kamikakushi / Chihiros Reise ins Zauberland}. (2001). Japan: Ghibli Studio. Drehbuch, Storyboard und Regie: Hayao Miyazaki. Produzent: Toshio Suzuki.} international einen sehr grossen Anklang gefunden hat, obwohl sehr viele darin vorkommende Elemente spezifisch für die japanische Kultur sind. Es stellt sich die Frage wie es kommt, dass ein solcher Film dennoch von Personen ausserhalb dieses Kulturkreises verstanden werden kann. Wenn man sich etwas mit Miyazakis Filmen beschäftigt so kann man verschiedene immer wieder auftauchende Motive erkennen. Zu den bedeutensten gehört die Faszination des Fliegens und der Kampf zwischen der Natur und den Menschen. In dieser Arbeit spielt vorallem letzteres eine zentrale Rolle. Es soll im Zusammenhang von Kultur und Religion, noch etwas genauer gesagt im Zusammenhang mit Animismus untersucht werden. Der Animismus stellt gerade heute einen schwer greifbaren Begriff dar. Wie viele Themen und Begriffe, welche der Kolonialzeit entsprungen sind, ist eine saubere Definition oder Abgrenzung schwierig. Dennoch ist es naheliegend von Animismus zu sprechen, wenn Wolf- und Keilergötter ihren Wald schützen oder ein von einem Feuergeist animiertes Schloss durch die Welt zieht. Wir finden hier also Dinge, aber insbesondere die Natur selbst als beseelt. 

Die in der Wissenschaft gebräuchlichen Beispiele von Geistern, Hexen und anderen übernatürlichen Phänomenen, sind für die Ethnien, aus denen diese Beispiele stammen, real. Im Unterschied dazu nimmt der westliche Betrachter eine vom Glauben unabhängige, distanziert-analytische Position ein. Die Betrachter von Filmen hingegen bilden eine dritte Herangehensweise, weil sie sich mit den Charakteren identizifieren und die fiktive Umgebungswelt leben können. Filme schaffen somit einen Rahmen, um zumindest temporär in ein System von übernatürlichen und phantastischen Phänomenen abzutauchen, ohne seine rationale und faktische Weltsicht aufgeben zu müssen. Diese Kombination aus Distanz und Nähe gibt uns die Möglichkeit uns selbst zu studieren, obwohl sich die Vorstellung von Animismus auf die Untersuchung von Völker und Ethnien konzentriert, die sich dem westlichen Kolonialismus zum Trotz ihre Kultur und Religion erhalten konnten. Das Studium des Animismus ist somit fast zwangsläufig eines, das die Studierenden aus ihrer unmittelbaren Umwelt weg führt. 

Es mag nach dieser Argumentation etwas widersprüchlich scheinen, dass ausgerechnet japanische Animationsfilme Gegenstand der Untersuchung sind. Es gibt im Bereich der Animationsfilme genügend Alternativen\footnote{Um ein Beispiel zu nennen: \textsc{Song of the Sea} (2014) und \textsc{The Book of Kells} (2009) produziert von Cartoon Saloon, einem irischen Animations Studio.}, welche kulturell näher stehen. Doch am Ende ist die kulturelle Prägung des Rezeptionisten für diese Untersuchung entscheidend, und nicht die des Filmes. 

In den Filmen \textsc{Prinziessin Mononoke}\footnote{\textsc{Mononokehime / Prinzessin Mononoke}. (1997). Japan: Ghibli Studio. Drehbuch, Storyboard und Regie: Hayao Miyazaki. Produzent: Toshio Suzuki.} und \textsc{Das wandelnde Schloss}\footnote{\textsc{Hauro no Ugoku Shiro / Das wandelnde Schloss}. (2004). Japan: Ghibli Studio. Drehbuch, Storyboard und Regie: Hayao Miyazaki. Produzent: Toshio Suzuki.} haben wir einerseits ist das alte Japan, reich an Mythen und Legenden. Hier wandeln Götter und Geister unter den Sterblichen. Wenn auch die Welt angereichert ist mit phantastischen Wesen, so erkennen wir darin doch unsere eigene Vergangenheit in dieser Welt. Andrerseits haben wir eine Märchenwelt, ein gutes Jahrhundert jüngeres Europa, wo Magie und Zauberei zum Alltag gehören. Diese Welt ist uns zeitlich zwar näher, doch gerade der breite Gebrauch von Magie in einer Zeit, an die wir uns noch zu erinnern glauben, entfremdet sie für den Betrachter. In beide Welten ist das Übernatürliche Grund zur Furcht oder Faszination, jedoch ist niemand über das Vorhandensein dessen überrascht.

Um die Frage zu beantworten, warum diese Filme, welche für ein japanisches Publikum gemacht sind auch bei uns Anklang finden, werde ich mit Pascal Boyers Ansatz der kognitiven Religionswissenschaft arbeiten. Eine ganz andere Sicht bietet hingegen Harvey. Seine Interpretation von Animismus basiert auf einem gegenseitigen respektvollen Umgang, nicht nur zwischen Menschen sondern zwischen allem was belebt ist.

\medskip

Im ersten Kapitel soll dargelegt werden wie Animismus für die vorliegende Arbeit zu verstehen sei. Dazu wird zuerst eine kurzere Übersicht zum historischen Begriff des Animismus und seine Verwendung gegeben. Es folgt ein kurzer Abschnitt über den japanischen Animismus, den Shintoismus. Danach werden wir Graham Harves Werk \emph{Animism. Respecting the Living World} betrachten um eine moderne Interpretation des Animismus kennen zu lernen.\footnote{Harvey, Graham. (2006). \emph{Respecting the Living World.} New York: Columbia University Press.} Mit Pascal Boyers \emph{Und Mensch schuf Gott} als Vertreter der kognitiven Religionswissenschaft bekommen wir weitere Kriterien zur Untersuchung animistischer Elemente.\footnote{Boyer, Pascal. (2004).\emph{Und Mensch schuf Gott.} Stuttgart: Klett-Cotta.} Dem Teil über Animismus folgt im zweiten Kapitel eine Vertiefung über japanische Animationsfilme und ein biografischer Abriss über Hayao Miyazaki, sowie die Analyse der Filme \textsc{Prinzessin Mononoke} und \textsc{Das wandelnde Schloss}. In Kapitel drei folgt die Anwendung der Methoden nach Boyer und Harvey. Schlussendlich steht die Frage ob es diese animistischen Elemente sind, welche den Filmen von Miyazaki helfen, auch bei einem nicht japanischen Publikum erfolgreich zu sein.

\section{Animismus}
%!TEX root = Animismus_in_Anime.tex
Der Ausdruck \emph{Animismus}, von lateinisch \emph{anima} für Seele\footnote{Auch: Atem, Leben.}, wurde von Stahl erstmals verwendet und durch im Jahr 1871 von Edward Tylor in Primitive Culture eingeführt. Wie bei vielen Begriffen in der Religionswissenschaft, trägt auch dieser mehrere Bedeutungen und kann verschieden verwendet werden. In der Regel wird unter einer animistischen Religion eine schriftlose Religion verstanden. Früher wurden diese gerne als Naturreligionen oder als achaische oder primitive Religionen bezeichnet. In diesem Zusammenhang, aber nicht deckend, versteht man unter Animismus auch den Glauben an eine beseelte Umwelt. Somit ist der Mensch nicht das einzige beseelte Wesen, sondern auch Tiere und Naturobjekte können beseelt sein. Letztlich kann mit Animismus auch einfach der Glaube an Geister und Seelen verstanden werden.\footnote{RGG, 4. Aufl., 1 (2008). Animismus: 504-506.}

Es soll an dieser Stelle zunächst ein kurzer historischer Abriss des Animismus gegeben werden. Danach wird Graham Harveys Animismus \emph{Respecting the Living World} als Vertreter eines modernen Animismus vorgestellt. Schliesslich verlassen wir das ausdrückliche Gebiet des Animismus um eine ganz andere Perspektive auf Religion einzunehmen. Dazu wenden wir uns der kognitiven Religionswissenschaft zu. 

\subsection{Der alte Animismus}
Die nachfolgende Zusammenfassung basiert primär auf jener von Graham Harvey in \emph{Animism. Respecting the Living World}. 

Wie schon festgehalten, wird der Begriff des Animismus erstmals von Georg Ernst Stahl (1659-1734) verwendet. Er stellt die Theorie auf, dass es ein physikalisches Element gäbe, welches belebt. Je mehr Anima vorhanden ist, desto belebter ist ein Objekt. Während eine tote Person oder ein Stein keine Anima (mehr) aufweist, besitzt eine lebendige Person viel Anima. Auch Tiere und Pflanzen besitzen dem nach Anima, jedoch weniger als der Mensch.

James Frazer (1854-1914) stellt später die Theorie auf, dass die Wilden (savage) glauben, dass Pflanzen und Tiere gleichermassen beseelt seien wie die Menschen. Wenn nun die Wilden zu glauben beginnen, dass Pflanzen und Tiere nur temporär und durch eine andere Wesenheitbeseelt seien, entwickelt sich der Animismus in einer Folgestufe zum Polytheismus.

Edward Tylor (1932-1917) beschreibt in seinem Werk \emph{Primitive Culture} (Die Anfänge der Cultur) den Animismus als der Ursprung der Religion. Der Animismus würde im Laufe der Weiterentwicklung und Zivilisierung einer Kultur durch verschiedene andere Formen der Religion abgelöst werden. Nichtsdestotrotz würden sich auch in einer hochentwickelten und komplexen Religion noch Überreste der alten Religion in Form von Aberglaube finden.

Diese Religionstheorie Tylors kritisiert Robert R. Marett (1866-1943), weil Phänomene wie Ehrfurcht vor den Tieren, vor Blut oder Naturgewalten nicht berücksichtigt würde. Stattdessen führt Marett eine Dichotomie vom Alltäglichen und vom Ausseralltäglichen ein, wobei Letzteres durch Religion erklärt und verarbeitet würde. Das Ausseralltägliche teilt er weiter in die Begriffe Mana und Tabu ein. Mana beschreibt dabei die Begegnung mit einer übermenschlichen Macht, während Tabu für Furcht und Kontaktvermeidung aufgrund von Gefahr steht. Folglich ordnet er Religion dem Mana zu, während er das, was mit Tabu verbunden wird, als Magie bezeichnet.\footnote{Ruel, M. J. (2008). \glqq Marett, Robert Ranulph\grqq. \emph{International Encyclopedia of the Social Sciences}, in: Cengage Learning. (2015). \url{http://www.encyclopedia.com/doc/1G2-3045000765.html} 07. September 2015.} 

Ebenso wie Tylor, geht Emil Durkheim (1858-1917) von einer Ursprungsreligion aus. Im Unterschied dazu setzt er jedoch den Totemismus an die Stelle des Animismus. Er tut dies, weil er die sozialen Aspekte über die Erfahrungen des Individuums stellt.

Zusammenfassend kann gesagt werden, dass der \emph{alte} Animismus als Vorstufe für eine höher entwickelte Religion gesehen wurde. In der Geburtsstunde der Religionssoziologie und der Anthropologie war der Westen überzeugt, dass es eine lineare Entwicklung gebe, wobei der Westen auf der höchsten Stufe stehe. Weile diese Ansicht heutzutage als veraltet gilt, haben viele der früheren Werke über den Animismus ihre Bedeutung für die heutige Religionswissenschaft eingebüsst. Es gibt jedoch Versuche den Animismus neu zu definieren und ihm auf diese Weise eine neue Bedeutung in der Moderne zu geben. 

\subsubsection{Animismus und Shintoismus}
Shinto, oder auch Shintoismus bezeichnet die native japanische Religion. Ähnlich wie auch das Wort Bud\={o} setzt sich das Wort aus zwei japanischen Schriftzeichen zusammen, wobei das Zweite jeweils mit Weg übersetzt werden kann. Das erste Zeichen steht für \emph{Götter} oder \emph{Geister}, japanisch Kami oder ehrfürchtig Kami-sama. Dabei handelt es sich um übernatürliche Wesen, die in japanischen Mythen und Legenden vorkommen. Sie können auch natürlichen Phänomenen innewohnen oder Schutzpatron für eine jeweilige Region sein. Wird Kami als Terminus Technicus übernommen, kann Shinto als der \emph{Weg der Kami} übersetzt werden.\footnote{RGG, 4. Aufl., 7 (2004). Sintoismus: 1283-1286.}

Es wäre sicherlich interessant einen Abgleich zwischen dem Shintoismus und den animistischen Elemente, welche man in den Filmen findet, zu machen. Man könnte vergleichen wie nahe sich der historische Shintoismus und die moderne japanische Vorstellung sind. Das liegt jedoch nicht im Sinne dieser Arbeit, da hier der Animismus losgelöst von seiner Kultur gesucht wird. Deswegen dir der Shinto hier nur zur Vollständigkeit erwähnt, spielt aber für die weitere Untersuchung eine Nebensächliche Rolle.

\subsection{Harvey Graham: Ansätze für einen Modernen Animismus}
Der Animismus steckt heute insofern in einer Krise, als dass auf den \emph{alten} Animismus nicht einfach aufgebaut werden kann. Andrerseits sind die Phänomene des Animismus weiterhin interkulturell präsent. Die Phänomene an sich werden weiterhin von Ethnologen und Psychologen ernst genommen. Als Konsequenz werden sie jedoch als kognitiver Fehler, als Projektion, als Produkt einer überproduktiven Phantasie oder einer mangelnden Trennung von subjektiver und objektiver Welt eingeschätzt. 

Auf der anderen Seite wirkt die Moderne zu Gunsten des Animismus. Früher wurden fremde Kulturen belächelt. Man bezeichnete die Völker als primitiv, die an Naturgeister glaubten und diese anbeteten. Hundert Jahre später sehen wir unsere Existenzgrundlage bedroht, weil wir unsere Umwelt rücksichtslos ausgebeutet haben. Es ist daher verständlich, dass Haltungen welche die Natur in ein Gegenüber stellen und so eine respektvolle Interaktion ermöglichen, eine gewisse Sympathie erfahren. 

\subsubsection*{Graham Harvey}
Ein Beispiel dafür ist in Graham Harvey zu finden. Er sagt: 

\begin{quote}
	\glqq Animists are people who recognise that the world is full of persons, only some of whom are human, and that life is always lived in relationship with others. Animism is lived out in various ways that are all about learning to act respectfully towards and among other persons.\grqq ~(Harvey 2006: xi)
\end{quote}

Mit dieser Aussage beginnt Graham Harvey sein Buch \emph{Animism. Respecting the Living World}. Es ist bereits an dieser Stelle erkennbar, dass es Harvey in erster Linie darum geht eine bestimmte Lebenshaltung zu postulieren. Anhand zahlreicher Beispiele, welche er im Laufe seiner Forschung in Neuseeland, Australien, Hawaii, Neufundland, Niger, und Amerkia gesammelt hat, erklärt er, wie der Animismus zu verstehen sei. 

Harvey nimmt eine Unterscheidung zwischen \emph{altem} und \emph{neuem} Animismus vor. In der alten Vorstellung wird davon ausgegangen, dass Animisten Menschen sind, die nicht zwischen Objekt und Subjekt unterscheiden, sei es, weil sie es nicht können, sei es, weil sie es nicht wollen. Im Unterschied dazu suchen Neue Animisten Wege und Ansichten, wie sie mit andern Leuten richtig und respektvoll interagieren können. Zentral in Harveys Buch ist das Zusammenfassen von Menschen (humans) und Anderen-als-Menschen zu einer Übergruppe von Leuten (people). Es gibt also Leute, welche nicht Menschen sind. Dennoch kann und muss mit ihnen interagiert werden. Allerdings gibt es unter den Leuten auch hinterlistige und verschlagene Personen (Menschen und Andere-als-Menschen). Deshalb ist es wichtig allfällige Masken, Täuschungen und falsche Aussagen durchschauen zu können. Im Wissen etwa, dass es Leute gibt, die uns fressen möchten, ist es weise sowohl vorsichtig als auch konstruktiv im respektvollen Umgang mit den Anderen zu sein. 

Ich werde hier auf zwei seiner Beispiele eingehen. Das erste handelt von den Ojibwa, einem nordamerikanischen Indianerstamm. Harveys Überlegungen stützen sich dabei hauptsächlich auf Irving Hallowells Beobachtungen und Untersuchungen, wobei die Sprache im Zentrum steht\todo{Zitiere Hallowells Buch}. Als zweites soll Harveys Überlegungen zur Maori-Kunst dargestellt werden.

\subsubsection*{Die Sprache der Ojibwa}
Die Ojibwa geben uns ein Beispiel dafür, dass sich Animismus in der Grammatik der Sprache zeigen kann. So wie es im Deutschen (und verwandten Sprachen) eine Untescheidung zwischen männlich und weiblich gibt, unterscheidet die Ojibwe-Grammatik zwischen belebt (animated) und leblos (inanimated). Diese Unterscheidung ist keineswegs selbstredend. So wie wir „die Tasse“ oder „der Hund“ sagen, entspricht das grammatische Geschlecht nicht immer mit dem Geschlecht des Beschriebenen überein. In der Ojibwe-Sprache beschreibt die Grammatik auch die Steine als animiert. Doch als Antwort auf die Frage ob denn alle Steine leben würde, antwortete ein alter Ojibwe mit: „Nein. Aber ein paar schon“ (Harvey 06: 33). Aus diesem Beispiel geht hervor, dass der Animismus hier kein dogmatisches Glaubenssystem darstellt. Es ist möglich, dass ein Stein animiert ist, und doch lässt sich diese Aussage nicht auf alle Steine übertragen. Für diese Animisten ist also nicht grundsätzlich alles belebt. 

Eine weitere Anekdote erzählt von einem Stein, der durch einen weissen Händler ausgegraben wurde. Der Händler dachte, er gehöre zu einem zeremoniellen Pavillon. Also suchte er einen Ojibwa namens John auf. John beuge sich zum Stein und frage den Stein leise, ob er zu diesem Pavillon gehöre. Laut John antwortete der Stein, dass dem nicht so sei. Dieses Beispiel zeigt, dass mit dem Stein wie mit einer Person umgegangen wird. John spricht nicht \emph{zu} sondern \emph{mit} dem Stein (Harvey 06: 37). 

Weiter gibt es bei den Ojibwa auch Erzählungen von Steinen, die belebt sind und darüberhinaus anthropomorphe Merkmale besitzen. Dies sind zum Beispiel Steine, die so geformt sind, dass es aussieht, als ob sie einen Mund oder Augen haben. Gleichzeitig sind solche Merkmale nicht zwingend als Hinweis zur Beseeltheit des jeweiligen Steines zu verstehen. Das Aussehen kann trügen. Ein Stein gilt als animiert, wenn mit ihm gesprochen werden kann. Wenn man mit ihm, wie mit anderen Personen, interagieren kann. 

Ein nächstes und weitaus abstrakteres Beispiel findet sich bei den Saison\-geschichten (Seasonal Stories). Der Umgang mit diesen Geschichten entspricht dem respektvollen Umgang mit einer Person. Tatsächlich werden diese Geschichten Grossvater genannt und gelten entsprechend auch als ehrwürdig (Harvey 06: 42). Mit diesen Geschichten beschäftigt man sich nicht leichtfertig. Und auch wenn sie mitunter lustig sein können, so nimmt man sie doch ernst. Sie vermitteln Dinge von grosser Wichtigkeit, wenn man sich ihnen respektvoll annähert. 

\subsubsection*{Die Kunst der Maori}
Maori sind für ihre kunstvollen Schnitzereien von Pounamu Steinen\footnote{Sammelnbezeichnung der Maori für Nephrit-Jade und Bowenit. Im Englischen werden diese Steine schlicht \emph{greenstone} genannt.}, Knochen und Holz berühmt. Harvey möchte zeigen, dass diese Kunstwerke selbst (durch den Macher) beseelt sind.

Die Maori fühlen eine tiefe Verwandtschaft mit dem Ort an dem sie leben. Ein junger Mensch entwickelt sich nicht nur in Abhängigkeit seiner Familie und seines Clans, aber auch die Natur gehört zu seinen Vorvätern. Das Land wird als Quelle der Identität betrachtet. Es gehört und wird kontinuierlich geteilt von den Toten, den Lebenden und den Ungeborenen.\footnote{New Zealand Ministry of Justice. (ohne Jahr). \emph{Whenua}. 07. September 2015.

	\url{http://www.justice.govt.nz/publications/publications-archived/2001/he-hinatore-ki-te-ao-maori-a-glimpse-into-the-maori-world/part-1-traditional-maori-concepts/whenua}}
Es ist zum Beispiel Brauch, dass bei der Geburt eines Kindes die Plazenta vergraben wird. Somit ist das Neugeborene mit dem Ort verbunden.

Die Maori sehen in der Süsskartoffel nahe Verwandte, ohne deren Hilfe den Maori eine wichtige Nahrungsgrundlage fehlen würde. Ohne die Hilfe der Maori würde die Pflanze jedoch gar nicht erst wachsen und gedeihen können. Die Kartoffeln auszugraben und zu essen grenzt daher an Kannibalismus.\footnote{Kannibalismus ist unter Maori durchaus üblich. Dabei geht es in keiner Weise darum sich vom Menschenfleisch zu ernähren. Die Einverleibung fand von Freunden und Feinden statt.}

Ebenso stellt das Schnitzen von Knochen, welche ja in jedem Menschen vorhanden sind, keinen grösseren Eingriff dar, als das Fällen und Schnitzen von Bäumen und das Schnitzen von Holz. Eine Schnitzerei steht somit immer im Zusammenhang mit dem Nehmen von Leben. Die Überreste einer Schnitzerei werden jeweils zurückgegeben. Die kunstvolle Schnitzerei ist nicht dafür da, um davon abzulenken. Durch das Schnitzen findet eine Transformation statt, in der der Künstler das Potential das im Holz, Stein oder Knochen schlummert hervor bringt. Ein Pounamu Anhänger ist belebt und nicht einfach nur Schmuck oder Identität für den Träger. Er hat ein Geschlecht, einen Namen und verdient Respekt. 

\medskip
Die beiden aufgezeigten Kulturen in welchen Harvey von Animismus redet, zeigen, dass es sehr grosse Unterschiede darin gibt, wie Animisten mit der Welt um sie herum agieren. Dabei decken diese beiden Beispiele nur einen sehr kleinen Teil der Aspekte ab, welche Harvey dem Begriff Animismus zusammenfasst. Weitere Beispiele sollen bei der Filmanalyse an passender Stelle gezeigt werden. 





\subsection{Die Sicht der kognitiven Religionswissenschaft}
%!TEX root = Animismus_in_Anime.tex
In dieser Unterdisziplin der Religionswissenschaft werden Religion bzw. religiöse Phänomene aus der Perspektive der Kognitions- und Evolutionswissenschaft betrachtet. Es wird versucht zu erklären, weshalb religiöse Praktiken und Denkweisen universell verbreitet sind.\todo{wikipedia: \texttt{https://en.wikipedia.org/wiki/Cognitive\_science\_of\_religion} und \texttt{https://de.wikipedia.org/wiki/Kognitive\_Religionswissenschaft} 24.08.15}

Die kognitive Religionswissenschaft ist eine eher junge Disziplin, die sich erst gegen Ende des zwanzigsten Jahrhunderts etabliert hat. Zu ihren Begründern gehören unter anderen E. Thomas Lawson und Robert McCauley (\emph{Rethinking Religion: Connecting Cognition and Culture and Bringing Ritual to Mind: Psychological Foundations of Cultural Forms}), Pascal Boyer (\emph{Naturalness of Religous Ideas}) und Guthrie (\emph{Faces in the cloud}). An dieser Stelle soll Pascal Boyers \emph{Und Mensch schuf Gott} als Stellvertreter für die kognitiven Ansätze dienen um die Animationsfilme auf Animismus zu untersuchen.\todo{boyer}

Boyer erklärt in der Einleitung seines 2001 erschienenen Buches, dass Religion im Geiste des Menschen zu suchen sei. Denn jeder menschliche Geist habe das Zeug religiös zu sein. Es ist somit nicht Boyers Ziel zu beweisen, dass es Gott nicht gebe und nur ein Produkt unserer Fantasie sei, er möchte erklären können, warum religiöse Menschen glauben was sie glauben.

Bei der Frage nach dem Ursprung der Religion tauchen immer wieder ähnliche intuitive Begründungen auf: \emph{Die Religion bietet Erklärungen, Die Religion spendet Trost, Die Religion sichert die gesellschaftliche Ordnung, Die Religion ist eine kognitive Täuschung}\todo{Boyer 04: 14-15} usw. usf. Laut Boyer sind diese intuitiven Gewissheiten in ihrer Existenz zwar berechtigt, jedoch nicht dienlich, wenn es darum geht den Ursprung zu finden. Einen Ursprung im Sinne eines historischen Ereignis ist freilich eine Wunschvorstellung, welche aus dem Wunsch entspringt eine Ursache zu haben, aus der sich alle weiteren Phänomene ableiten lassen. Boyer zeigt in aller Ausführlichkeit, wie man in jeder dieser Vorstellung Widersprüche findet, oder dass sie schlicht nicht befriedigend sind. Beides macht sie als Ursprung ungeeignet. Für jedes Gebiet, das er abarbeitet, fügt er am Ende einen anderen Blickwinkel hinzu, in dem er aus kognitiver Sicht den Wert dieser intuitiven Annahme beschreibt.

Ein Beispiel für den Fehler im ersten Punkt (Religion bietet Erklärungen) ist das Donnergrollen. Boyer zufolge gibt es zu wenig her, anzunehmen, dass das Grollen die Reaktion von Geistern, Göttern oder Ahnen auf ein Fehlverhalten des Menschen sei. Der Aufwand für ein an und für sich simples Phänomen wäre unverhältnis gross. Denn um laute, grollende, dumpfe Geräusche bei Stürmen zu erklären, muss eine komplette imaginäre Welt mit übernatürlichen Mächten vorausgesetzt werden. Dies wirft an sich noch mehr Fragen auf: Woher sind diese Wesen gekommen? Wo sind sie? Warum sieht man sie nicht? Haben sie einen riesigen Mund um diese Geräusche zu erzeugen? Nur wenn ein solcher Glaube verbreitet ist, finden sich zu diesen Fragen auch Antworten. Diese Antworten sind aber teilweise so weit her geholt, dass es die Ausgangslage, die Geräusche eines Gewitters erklären zu wollen unwahrscheinlich macht. 

In seiner Erklärung geht Boyer von der Funktionsweise des Geistes im Allgemeinen aus, welche unabhängig von der Kultur überall gleich ist. Das scheint zu nächst eine Sackgasse, da kulturell grosse Unterschiede bezüglich religiöser Praktiken und Vorstellungen zu finden sind. Das geniale hier sei, argumentiert er, dass sich etwas so vielschichtiges wie Religion durch etwas erklären lasse, was überall gleich sei (d. h. das Gehirn). Es ist jedoch notwendig zunächst mehr darüber zu wissen, wie das Gehirn Informationen aufnimmt und verarbeitet.\todo{[S.11]} Denn die Arbeit, die ein Gehirn leistet, ist lange unterschätzt worden. Einerseits muss man von der verbreiteten Annahme wegkommen, es handle sich beim Geist um ein leeres Gefäss, welches beliebig mit Informationen (Erziehung, Bildung und persönlichen Erlebnissen) gefüllt werden könne. Anderseits gilt es von der Idee wegzukommen, der Geist könne wahllos mit Informationen gefüllt werden. Das Gehirn kann sich aus gutem Grund nicht alles Beliebige merken. Das Gehirn muss die relevanten Informationen aus der Umwelt identifizieren und auf eine spezifische Weise verarbeiten.\todo{[S. 12]} 

Boyer geht nun der Frage nach, auf welche Weise religiöse Konzepte überhaupt entstehen. Die Bezeichnung Mem als Kulturelement, also Vorstellungen, Werte, Geschichten und dergleichen, die die Menschen in ihrem Handeln beeinflussen und die weitergegeben werden\todo{\textsc{Boyer} 04: 50}, wurde vom Evolutionsbiologen Richard Dawkins erstmals vorgestellt. Ein Mem bezeichnet demnach einen Bewusstseinsinhalt\todo{wiki: \texttt{https://de.wikipedia.org/wiki/Mem} 24.08.15}, welcher durch Kommunikation in der Gesellschaft weitergegeben und somit vervielfältigt werden kann. Es ist das soziokulturelle Pendant zu den biologischen Genen in der Evolution. Sodann lassen sich die Meme ähnlich wie die Gene beschreiben: Information wird durch Kommunikation weiter gegeben (repliziert). Dadurch werden die Inhalte aber nicht einfach verbreitet, sondern auch leicht (oder schwerwiegend) abgeändert (mutiert).\footnote{Die Information ändert sich nicht erst durch deren Weitergabe. Etwas das wir erfahren wird bei verschiedenen Menschen bereits anders verarbeitet. Somit können zwei Personen genau das gleiche hören und eine andere Version des Inhalts in ihrem Bewusst sein haben.} Schliesslich werden nur jene die einprägsamen bzw. relevanten Meme tatsächlich weiter gegeben, sie werden selektiert.

Boyer beschreibt die Meme zwar als eine wunderbare Ausgangslage, er will ihnen aber nicht mehr als genau das zugestehen. Seine Kritik liegt darauf, dass es keine Replikation der eigentlichen Informationen gebe. Boyer zufolge werden Inhalte gerade nicht faktisch übergeben, sondern jeweils neu konstruiert. Zwei Menschen können zwei faktisch identische Aussagen machen, aber jeder hat die Information, welche er wiedergibt auf seine eigene Art rekonstruiert. Entsprechend stellt Boyer als nächstes seine Theorie zum Einfangen von Vorstellungen durch Schablonen vor.

\subsubsection{Schablonen, Vorstellungen und das Schlussfolgerungssystem}
Ein grosser Teil dessen, was wir wissen, musste uns niemand faktisch erzählen. Eine erstaunliche Eigenschaft unserer geistigen Fähigkeit ist es, durch die Kombination bereits existierenden Wissens und der Hinzugabe einer neuen Information zusätzliches Wissen zu generieren. Boyer zeigt dies auf, indem er das Kind als eine Person aufführt, deren Wissen fortlaufend erweitert wird. Wird einem Kind zum ersten Mal ein Seehund gezeigt, so hat es abgesehen vom Namen und dem äusseren Erscheinungsbild des Seehundes keine weiteren Informationen darüber. Dennoch wird das Kind erwarten, dass der Seehund isst, schläft und dass er sich fortplfanzt. Diese Informationen über das Seehund hat das Kind geschlussfolgert, indem es eine Annahme gemacht hat: Der Seehund ist ein Säugetier. Säugeteire essen, schlafen und pflanzen sich fort. Folglich bezeichnet Boyer das Säugetier als eine \emph{Schablone}. Mit dieser Säugetier-Schablone hat das Kind eine Seehund-Vorstellung gebildet.\todo{\textsc{Boyer 04: 59}} 

Über diesen Schablonen, die verschiedene Konzepte zusammenfassen und aus einem Informationsstück mehr Information schaffen, stehen die ontologischen Kategorien. Boyer zählt fünf auf: \textsc{Person, Tier, Pflanze, Naturobjekt, Werkzeug}. Die Schablonen, und im stärkerem Masse die ontologischen Kategorien, seien das, was über Kulturen hinweg universell gültig sei. Erst bei den konkreten Konzepten ergebe sich eine Varianz. Die Vorstellungen, die von Angehörigen einer gleichen Gruppe anhand einer Schablone hergestellt werden, sind sich in der Regel ähnlich. Die Vorstellungen einer anderen Gruppe kann davon jedoch stark abweichen, obwohl die gleiche Schablone benutzt wurde. 

Dieses System der Schablonen, welches mit Hilfe von Schlussfolgerungen zu Vorstellungen bzw. zu Konzepten führen, überträgt Boyer nun auf die Religion. Demnach gibt es Schablonen für religiöse Vorstellungen. Diese Schablonen werden universell geteilt, wenn auch die Konzepte regional stark variieren können. Darüber, dass übernatürliche Kräfte unsichtbar sein können, ist man sich weitgehend einig. Wenn es aber darum geht, was und warum übernatürliche Kräfte etwas tun, so gehen die Vorstellungen weit auseinander. 

Boyer betont, es müsse letztlich berücksichtigt werden, dass die kulturelle Varianz in der Regel geringer sei, als man allgemein annimmt. Beim Übermitteln findet durch die Schablonen ein Filtern der gegebenen Informationen statt, so dass daraus voraussagbare Strukturen gebaut werden.\todo{[65]}

\subsubsection{Beschaffenheit des Übernatürlichen}
Es folgt also die Suche nach dem mentalen Rezept für religiöse Vorstellungen. Mit einem Versuch verschiedener mehr oder weniger potenten religiösen Aussagen versucht Boyer dem Leser zu zeigen, dass man der Intuition folgend gewisse Aussagen über übernatürliche Wesen direkt ausschliessen kann, während man bei andern sofort glauben würde, dass es sich um eine existierende religiöse Vorstellung handelt. Folgende Kriterien für eine erfolgreiche (religiöse) Vorstellung kommen dabei heraus:\todo{Zusammenfassung: http://serendip.brynmawr.edu/exchange/node/1581 
Zusammenfassung 2: http://mason.gmu.edu/~rhanson/religion.html} 

Erstens habe jedes übernatürliche Konzept die Tendenz, eine seiner ontologischen Annahmen zu verletzen. Ein Geist gehört zur ontologischen Kategorie PERSON, doch dass er keinen physikalischen Körper hat bricht mit der Ontologie. Ein anderes Beispiel für diese Kontraintuitivität findet sich auch in der Natur, so etwa bei einer Raupe, welche nach der Metamorphose zu einem Schmetterling wird. Die Erwartung für ein TIER ist, dass es sich im Laufe des Wachstums nur durch Grösse und Masse verändert, jedoch nicht, dass es zu einem anderen TIER wird. Dabei ist es wichtig, dass tatsächlich gegen die ontologische Kategorie verstossen wird und es sich nicht nur um eine Merkwürdigkeit handelt. Eine PERSON, die ihre Hautfarbe ändert, ist dem zu Folge weniger erfolgreich, als eine PERSON, welche durch Wände gehen kann.\todo{fix Satz} 

Zweitens hat religiöses Denken die Tendenz auf Leute-Ähnliche übernatürliche Wesen zu fokussieren, welche Zugang zu sozial-relevanten Informationen haben. Der Austausch von Informationen ist für den Menschen kritisch, so Boyer. Wir sind darauf angewiesen, dass andere in der Gruppe Dinge wissen und uns diese übertragen können. Bei religiösen Vorstellungen wird in der Regel davon ausgegangen, dass ein Wesen die moralische Haltung eines Individuums oder auch der ganzen Gruppe teilt. Dieses Wesen weiss Bescheid darüber, wenn schlechte Dinge geschehen. Die Person erwartet folglich, dass dieses Wesen wertet und allenfalls böse wird und die Person für deren Verhalten bestraft. Eine solche Vorstellung macht diese Wesen zu wichtigen Subjekten für Gedanken und Diskussionen in einer Gruppe. 

Im dritten Punkt werden religiöse Rituale aus den Reinigungsritualen hergeleitet. Unser mentales System behandelt Krankheit mit Abscheu zum eigenen Schutz vor einer unsichtbaren Gefahr. So sollen auch Rituale in religiösen Vorstellungen vor allem zum Schutz vor unsichtbaren Gefahren und Mächten bestehen. 

Letztlich setzt Boyer noch den Fokus auf die Leiche und wie Menschen damit umgehen. Im Prinzip sind Leichen ein Spezialfall des ersten Punktes, wo es um die Verletzung der ontologischen Kategorie geht. Der Mensch, den wir sehen oder sogar gekannt haben, wird in unserem System der ontologischen Kategorie PERSON zugeordnet. Gleichzeitig verletzt die Leiche dieses Menschens jegliche Kriterien der PERSON-Kategorie und entspräche demnach der Kategorie NATUROBJEKT und sollte unsere Abscheu vor Krankheit wecken. Dieser Widerspruch macht die menschliche Leiche zum Prototyp für religiöse Objekte überall auf der Welt.

\section{Animes von Hayao Miyazaki}
%!TEX root = Animismus_in_Anime.tex
\newpage
\subsection{Hayao Miyazaki}
Hayao Miyazaki, einer der bekanntesten Animations Produzent Japans, wurde während dem 2. Weltkrieg im Januar 1941 unweit von Tokyo geboren. Sein Vater arbeitete für seinen Bruder in der Maschinenbau Firma Miyazaki Airplanes.~\footnote{Dieser Hintergrund wird gerne gebraucht um Miyazakis Faszination vom Fliegen zu begründen.} 

Miyazakis Interesse an der Animiation wurde sehr früh schon durch den Animationsfilm \textsc{Panda and the Magic Serpent} geweckt, welcher in seinen Jugendjahren veröffentlich wurde. Obwohl er zunächst Politik und Wirtschaft an einer renommierten Universität studierte, zog es ihn nach Abschluss seines Diplomes ins Animationsgeschäft. Als Hintergrundzeicher fand er bei der derzeit führendem Studio Toei-Animation eine Anstellung und machte sich schnell einen Namen. Nebenbei veröffentlichte er unter einem Pseudonym er eine eigene Manga-Serie\todo{welche?} und sammelte wo immer möglich Erfahrung in Storyentwicklung und Produktion. Toei-Animation schickte seine Animatoren gelegentlich auf Reisen um um Skizzenstudien der Landschaften oder Städte zu machen.~\footnote{So zum Beispiel reiste Miyazaki für die Produktion von \textsc{Alpine Girl Heidi} nach Europa.} 

Sein Freund Isao Takahata verfilmte 1972 Miyazakis erste Kurzgeschichte (\textsc{Adventures of Panda and Friends}). 1978 übernimmt Miyazaki dann erstmals die Inszinierung einer Anime-Serie (\textsc{Boy Conan}) und im darauf folgenden Jahr führte er zum ersten Mal Filmregie (\textsc{Schloss des Caliostro}). Von da an bekam Miyazaki immer öfters die Leitung für die Inszenierungen von Anime-Serien.

1982 begann Miyazaki mit dem Manga \textsc{Nausicaä aus dem Tal der Winde}. Die einzelnen Teile erschienen mit zahlreichen Unterbrechungen in einem monatlich erscheinendem Magazin. Erst 1994 fand der Manga einen Abschluss. Doch bereits 1983 begannen die Vorarbeiten für eine Verfilmung der Geschichte unter der Leitung von Miyazaki. Der Film erschien im derauf folgendem Jahr in den japanischen Kinos. 

Miyazaki machte sich im Jahr darauf mit ein paar Kollegen von Toei-Animation selbständig und gründete das Animations Studio Ghibli\footnote{Warmer Wüstenwind}. Bei der Position als Regisseur und als Produzent wechseln sich Miyazaki und Takahata ab. Das Studio war finanziell nicht abgesichert und riskierte zunächst mit jeder Produktion seinen Ruin. Etwas mehr Sicherheit gewann das Ghibli Studio nach der Veröffentlichung des Filmes \textsc{My Neighbor Totoro}, da insbesondere der Merchandising ein grosser Erfolg (auch heute noch) verbuchen kann. Mit den weiteren Produktionen stieg die Firma Ghibli zu den erfolgreichsten und bekanntesten in Japan auf.

Miyazaki wollte sich, mit der Fertigstellung von \textsc{Prinzessin Mononoke} in 1997 eigentlich in den Ruhestand setzen,\footnote{Zitat von Miyazaki: +/- zu anstrengend} begann aber dann mit den Vorbereitungen für \textsc{Chihiros Reise ins Zauberland} welcher 2001 in die japanischen Kinos kam. Während der Film nicht nur nationale sonder auch internationale Preise gewann\footnote{Darunter Oscar in der Kategorie Bester Animationsfilm}, arbeitete Miyazaki auch schon am nächsten Projekt. \textsc{Das wandelnde Schloss} welches 2004 in Japan veröffentlicht wurde, bescherte erneut einen Einnahmerekord dar in den japanischen Kinos, erlangte jedoch international nicht mehr ganz so viel Aufmerksamkeit wie die beiden Vorgänger Filme. 

Mit \textsc{Wie der Wind sich hebt} kündete Miyazaki erneut seinen Rücktritt an. An einer Abendkonferenz in September 2013 erklärte Miyazaki, dass er keine abendfüllenden Anime-Filme mehr machen werde.\footnote{\texttt{http://asienspiegel.ch/2014/08/grosse-ehre-fur-hayao-miyazaki/}} Die Zeit der traditionellen Animation, wo man noch mit Hand zeichne sei vorbei. Auch bei seinen späteren Filmen wurden Computergrafiken nur vereinzelnd eingesetzt. Überraschenderweise wagt der 74 jährige Japaner doch den Sprung ins neue Zeitalter der Animation. In seinem neusten Projekt arbeitet er an einem 10 minütigen Kurzanime, welcher auf der Kurzgeschichte \textsc{Boro, die Raupe}. Erstmal will er einen vollständig computer animierten Anime machen.\footnote{\texttt{http://asienspiegel.ch/2015/07/miyazaki-arbeitet-an-kurzanime/}}

Natur und so?\todo{wo und was soll noch dazu kommen?}

\subsection{Japanische Animationsfilme}
\subsubsection{Geschichte}
Die Bezeichnung \emph{Anime} für japanische Animationsfilme ist eine Fremdbezeichnung, welche sich erst nach dem Krieg durch die amerikanische Übernahme, gegen d\={o}ga, den japanischen Begriff für Animation, durchgesetzt hat. Die Symbiose zwischen Anime und Manga (Comic) ist spezielle in Japan. Es ist häufig so, dass ein erfolgreicher Manga verfilmt wird, oder aber dass zu einem Anime hinter ein Manga gezeichnet wird. Anders als im europäischen Raum zielen Manga und Anime auch auf ein viel breiteres Publikum ab. Zwar sind viele Geschichten für Kinder gedacht, aber auch Erwachsene werden als Zielpublikum ernst genommen. Daher finden wir in den japanischen Animationsfilmen in der Regel eine grössere Inhaltliche Palette als in den amerikanischen Produktionen. Durch das weite Spektrum des Zielpublikums finden sich auch Produktionen an allen Genres. Nebst den Geschichten welche typischerweise japanische Märchen, Mythen und Legenden oder Science Fiction thematisieren, finden sich auch Horror, Historie, Romanzen, Komödien und Erotik ihren Platz. Beim Inhalt zeigt sich, dass obwohl häufig Elemente aus der Japanischen Kultur eine zentrale Rolle spielen, nicht davor gescheut wird auch westliche Elemente zu integrieren. Dennoch sind japanische Animationsfilme üblicherweise für ein japanisches Publikum gedacht.

Gerade nach der Kriegszeit übten die amerikanische Filmindustrie einen grossen Einfluss auf. Einerseits wurde versucht dem erfolgreichen Beispiel zu folgen, anderseits bestand auch der Drang sich davon abzugrenzen und sich auf die eigene Kultur zu konzentrieren. Dem US-Beispiel folgend entstanden in den 1950er Jahre etliche Animiations Studios. Anders als die amerikanischen Studios wie Disney setzte die japanische Filmindustrie mehr auf Quantität als auf Qualität. Das führt dazu, dass typische Anime in der Regel einfacher, weniger hyperrealistisch gestaltet sind als die amerikanischen. Neben der Machart unterscheiden sich die japanischen Animationen von amerikanischen oder auch europäischen durch ihren kulturellen Hintergrund. 

Im Shinto finden wir einige der Erklärungen für Eigenart der japanischen Animations Filme. Der traditionelle Shinto ist eine unorganisierte, schriftlose Religion. Im Shinto wird das Universum zudem prinzipiell als mehrdeutig betrachtet. Das macht sie ideal dafür, verschiedene (auch widersprüchliche) Konzepte in sich aufzunehmen. So wurde zum Beispiel der Buddhismus in den Shinto integriert und auch Elemente des Christentums, welches später nach Japan gelangte fanden ihren Platz im religiösen Gesamtverständnis der Japaner. Des weiteren wird nicht scharf zwischen Götter und Dämonen, gut und böse getrennt. Nach dem Kodex der Samurai steht die Absicht auch über der Handlung. Als Folge all dessen fehlt in den japanischen Geschichten in der Regel auch die Unterscheidung von Gut und Böse. Vielmehr steht der Protagonist und der Antagonist sich in einem Interessenskonflikt gegenüber. Die Interessen können sich in ihrer Essenz widersprechen und dennoch glaubhaft sein. 

Im Zentrum stehen die Motive der Charakteren. Ein Bildwechsel wird daher nicht unbedingt benutzt um den zeitlichen Verlauf zu markieren, sondern um einen Perspektivenwechsel zu ermöglichen. Für westliche Zuschauer gibt das den Eindruck einer Verlangsamung der Handlung. Dieser Fokus, zusammen mit der Eigenschaft, dass auch japanische Serien in der Regel abgeschlossen sind, geben den Charaktern der Geschichte die Möglichkeit dramatische Veränderungen zu durch gehen. In amerikanischen Produktionen fallen die Charakteren vergleichsweise flach und statisch mit wenig Möglichkeit zur Entwicklung auf Grund der episodenhaften Art aus.

Das ästhetische Prinzip von Wabi und Sabi\footnote{Definition, oder zumindest Andeutung} ist ebenfalls in den japanischen Filmen zu beobachtet. Auslassung ist genau so Teil eines Kunstwerks wie seine andern Bestandteile. Damit begründet sich auch das hohe Mass an Abstraktion zum Beispiel beim Charakterdesign.  

Eine weitere wichtige Rolle spielen symbolische Darstellungen. Gerade bei einer Analyse und einer damit verbunden Interpretation ist es wichtig sich aber bewusst zu sein, dass die japanische Kultur Grundlage der Interpretation sein muss. Im Gegensatz zu den Walt Disney Märchen wo die Standard Prinzessin blondes Haar trägt, ist die typische Haarfarbe für einen guten Charakter in japanischen Geschichten dunkel.~\footnote{Im Anime \textsc{Das wandelnde Schloss} welches später genauer betrachtet werden soll, trägt der Protagonist Hauro zunächst Blonde Haare. Doch als er endlich zu sich selbst und seinen Mut findet trägt er dunkle Haare.} 

\section{Prinzessin Mononoke}
%!TEX root = Animismus_in_Anime.tex
\subsection{Filmhintergrund}
\subsubsection*{Veröffentlichung und Erfolg}
Mit \textsc{Prinzessin Mononoke} gelang Hayao Miyazaki erstmals einen internationalen Durchbruch. Der Film kam am 17. Juli 1997 in die japanischen Kinos. Mit 18.65 Millarden Yen (umrechnung?) spielte er in Japan mehr ein als Titanic (James Cameron). Er war der bisher erfolgreichste Film in Japan. Nachdem er zuerst in 1998 auf der 48. Berlinale das erste Mal in Deutschland vorgeführt wurde und \emph{irgendwelche Preise} gewonnen hat, kam er 1999 in den Vereinigten STaaten und Kanada und im Jahr darauf auch in Europa in die Kinos. Trotz den vielen Auszeichnungen welche der Film gewann, war der Film, sowie Ghibli Studio und Hayao Miyazaki vorwiegend unter Anime-Fankreisen bekannt. Es wurde ausserhalb von Japan auch nur wenig Werbung gemacht. Auf internationale Vermarktung wurde lange verzichtet, da Miyazaki und sein Team darüber entsetzt waren, wie starkt geschinitten \textsc{Nausicaä aus dem Tal der Winde} wurde. Nach dem riesigen Erflog an den japanischen Kassen zeigte das US-Studios Disney Interesse und sicherte sich die Verhandlungsposition der japanischen Filmemacher. Im Vertrag über die internationale Vermarktung der Ghibli-Filme welcher bald darauf geschlossen wurde konnte sich Miyazaki und sein Team das Recht sichern, über allfällige Schnittstellen selbst entscheiden zu können. 

\subsubsection*{Geschichtlicher Hintergrund und Miyazaki}
Erste Ideen für \textsc{Prinzessin Mononoke} hatte Miyazaki bereits 1970. Damals diente im als Plotvorlage das Märchen von der Schönen und dem Biest.\footnote{http://home.comcast.net/~rocksunner/miya\_e.html} Wegen dem leichtsinnigen Versprechen ihres Vaters, muss die Tochter des Fürsten ein Waldmonster [mononoke]\footnote{Geist/Monster/Gespenst -> http://nausicaa.net/miyazaki/mh/faq.html\#translation} heiraten. Doch Miyazaki kam nicht weiter mit der Geschichte und schob sie auf. 

\begin{quote} Actually, in the beginning I wanted to do a fantasy rather than a period drama set in Japan. However, when I said "Now let's do it", I didn't have the heart for it.\footnote{http://home.comcast.net/~rocksunner/miya\_e.html} 
\end{quote}

Er entschied sich gegen die anfänglich geplante Fantasy Umwelt und setzte die Geschichte im traditionellen feudalen Japan an. Doch in der Zeit, wo er sich mit den Hintergründen auseinander setzte, änderte sich auch das Herzstück der Geschichte. Miyazaki bekam den Wunsch einen tiefgründigeren, authentischeren Film zu machen. Somit fand er wieder zum Thema zurück, das er schon mit \textsc{Nausicaä aus dem Tal der Winde} im Fokus hatte: Das (konfliktreiche) zusammenleben einerseits von Mensch und Natur und anderseits aber auch von Mensch und Mensch. So blieb am Ende von der Ursprünglichen Geschichte nicht mehr übrig als der Name \emph{Mononoke}.\footnote{Das Märchen von der Prinzessin und dem Biest hat Miyazaki später in Form eines Bilderbuches veröffentlicht.} \todo{Nachweis}

Infos aus:\footnote{https://de.wikipedia.org/wiki/Prinzessin\_Mononoke\#Hintergr.C3.BCnde http://nausicaa.net/miyazaki/mh/filminfo.html http://home.comcast.net/~rocksunner/mono\_e.html}

Die Geschichte spielt in einem feudalen Japan zu einer Zeit, welche der Muromachi-Ära (1392-1573) ähnelt. Historisch wichtige Figuren bleiben im Hintergrund. Ein König/Shogun wird zwar erwähnt, jedoch sehen wir nur die Folgen der Interessentskonflikte der Mächtigen. Der Fokus der Geschichte liegt beim einfachen Menschen, insbesondere bei Aussenseiter der Gesellschaft, welche in einer Zeit der kulturellen Blüte, sozialen Umwälzungen und politischen Unruhe leben. 

\subsection{Zusammenfassung}
Ashitaka, der letzte Prinz eines in Harmonie mit der Natur lebenden Volks namens Emishi\footnote{Alte Bezeichnung für ein japanisches Urvolk. Nach der Heian-Zeit wurde das Volk Ezo genannt.}, muss seine Heimat verlassen weil er einen Dämon\footnote{Tatarigame = curse god (http://home.comcast.net/~rocksunner/mono\_e.html\#ashitaka)}, welcher das Dorf angegriffen hat tötete und so dessen Fluch auf sich zog. Ashitaka macht sich auf den Wald des Shishigamis\footnote{Bedeutung} zu finden. Von dort kommt nämlich der Dämon, welcher einst ein Keilergott war und erst durch den Schmerz und damit kommendem Hass in zu einem Dämon wurde. 

Bald findet sich Ashitaka zwischen verhärteten Fronten. Es herrscht Krieg zwischen den Waldgöttern und den Menschen welche in einer Erzschmiede arbeiten. Er trifft auf San, welche von den andern Mensch \emph{Prinzessin Mononoke} genannt wird, weil sie von der Wolfgöttin Moro aufgezogen wurde und sich wie ein Tier verhaltet. Ashitaka versucht zwischen San, welche für ihre Familie und ihren Lebensraum kämpft und den Bewohnern der Schmiede, welche den Waldabholzen um Erz zu gewinnen, was ebenfalls ihre Lebensgrundlage ist zu vermitteln. Erschwerend kommt hinzu, dass die Anführerin der Schmiedebewohner, Eboshi systematisch versucht die Waldgötter zu vernichten, um ihre Untertanen so zu schützen. Zudem hat sie dem Kaiser den Kopf des Shishigami versprochen. Sie erhofft sich so den Schutz für ihre Schmiede zu sichern, da die Gefahr besteht, dass sie vom selbigen Kaiser angegriffen werden könnte. 

Trotz Ashitakas Bemühungen kann dem Kampf nicht ausgewichen werden. Eboshi schiesst dem Waldgott Shishigami den Kopf ab und der Rumpf des Gottes droht alles zu zerstören. Durch das selbstlose Eingreiffen von San und Ashitaka kann das Verderbnis im letzten Moment gestoppt werden, doch vieles der alten Welt ist danach für immer verloren.

Obwohl man durch ästhetische Gestaltung einfach erraten kann, bei wem und wo Miyazakis Sympathie liegt, so ist es doch erstaunlich, dass der Geschichte jegliches Schwarz-Weiss denken fehlt. Jeder der kämpft hat seine Gründe. Ein durchgehendes Motiv ist der destruktiver Hass. Somit birgt das Filmende zwar Hoffnung in sich, jedoch bleibt die Problematik bestehen. Ashitaka und San wollen sich zwar weiterhin sehen, jedoch kann das Wolfmädchen den Mensch nicht vergeben und bleibt bei ihren Wolfbrüdern im Wald. Ashitaka, der zwar grossen Respekt vor der Natur und den Geistern zeigt kehrt aber dennoch in die Schmiede zurück um dort mit den anderen Menschen zu leben.  

\subsection{Figuren Analyse}
In Japans Altertum angesiedelt passt sich die belebte Natur gut ins Bild ein. In den Städten und Dörfen welche Ashitaka besucht auf seinem Weg zum Shishigami Wald haben die Leute praktisch keinen Kontakt zu den Naturgeistern. Zusehr sind sie mit den politischen Dingen beschäftigt. Als Ashitaka aber endlich im Reich des Shishigami ankommt ist das Verhältnis zwischen Mensch und Natur, oder besser gesagt zwischen Mensch und Waldgöttern ein anderes. Ein friedliches Nebeneinander scheint unmöglich.

\subsubsection*{Die Tiergötter: Moro, Nago und Okkoto}
Moro ist eine Wolfsgötting die zusammen mit ihren Söhnen und ihrer adoptierten Menschentochter San im Wald des Shishigamis lebt. Mit ihrer Grösse überragt sie alle Menschen. Sie ist weiss und hat einen doppelten Schwanz. Ihre Stimme ist tiefe und klingt nicht weiblich. Moro hasst die Menschen und Ebshi am meisten von allen. Für ihr Ziel, Eboshi zu töten riskiert Moro viel, nicht zu letzt ihr eigenes Leben und lebst im Tod, wo ihr Kopf vom Körper getrennt ist, schafft es Moro noch Eboshi den Arm ab zu beissen. Im Gegensatz zu andern Akteuren in der Geschichte (insbesondere den Wildschweinen) jedoch macht der Hass sie nicht blind. Und so stellt sie sich auch gegen den aufgebrachten Keilergott Okkoto um San zu retten.

\begin{quote}
Humans who attacked the forest threw a baby to me in order to escape my fangs. That was San...! She can't be human, neither can she fully be a wolf. She's my poor, ugly, loveable daughter.\footnote{http://home.comcast.net/~rocksunner/mono3e.html\#moro}
\end{quote}

Ein anderes Bild von den Tiergöttern bekommen wir durch die Keiler. Den Dämon, welcher Ashitaka am Anfang des Filmes bezwingt um sein Dorf zu schützen war einst ein mächtiger Keilergott und stammt auch aus dem Shishigami Wald. Als er zu den Emishi kommt hat der Schmerz und der Hass ihn bereit so wütend gemacht, dass er zu einem Dämon [Tatarigame]\footnote{Begriffserklärung} wurde. Ashitaka bittet den rasenden Gott Umkehr zu machen und sein Dorf zu verschonen und erst als er sich zwischen seinem Dorf und dem Dämon entscheiden muss, tötet er ihn. Die Dorfseherin verrichtet gleich darauf ein Versöhnungsritual in dem sie dem Dämon verspricht ein Schrein zu errichten und ihn bittet er möge nicht länger hassen. Doch eine Stimme erklingt aus dem Toten Keiler und verflucht alle Menschen (\glqq  Loathsome humans! You will know my wrath well for causing me pain\dots \grqq \footnote{http://home.comcast.net/~rocksunner/mono\_e.html\#ashitaka})

Im späteren Verlauf der Geschichte begegnet Ashitaka dem mächtigen Keilergott Okkoto, dem Herr aller Keiler. Okkoto ist weiss wie Moro und ihre Söhne, graue Stellen in seinem borstigem Fell und seine trüben Augen weisen jedoch darauf hin, dass er sehr alt ist. In seiner Grösse überragt der Keilergott sogar Moro und so wie sie zwei Schwänze hat, verfügt er über eine zweite Reihe von mächtigen Hauern. 

Okkoto erfährt durch Ashitaka von Nagos Schicksal und zeigt sich beschämt, dass einer aus seinem Klan ein so übles Schicksal passierte. Er zeigt jedoch weder Mitleid noch Vergeben gegenüber Ashitaka und teilt ihm mit, dass er ihn bei der nächsten Begegnung umbringen wird. Obwohl er einsichtiger scheint als die andern Keiler ist Okkoto so sturr, dass von seiner Weisheit am Ende nichts übrig bleibt. Von den Menschen durch Waldbrände und Lärm provoziert, rennen die Wildschweine unter Okkotos führung gerade in ein Massaker. Moro durchschaut die Absichten der Menschen und auch Okkoto vermutet was dahinter steckt, jedoch setzt er alles auf eine letzte Schlacht. 

\begin{quote}
Moro, look at my clan! Little by little they are becoming smaller and more stupid. If this keeps up, humans will be able to hunt us down like common meat\dots \footnote{http://home.comcast.net/~rocksunner/miya\_e.html} 
\end{quote}

Auch Moros Söhne sind nicht so prächtig und gross wie sie selbst. Zwar immer noch grösser als normale Wölfe und weiss, fehlen ein doppelter Schwanz oder sonst ein Merkmal, welches sie von normalen Wölfen unterscheidet. Aus Moros Handeln und dem was sie sagt geht hervor, dass sie keine Hoffnung für die Zukunft des Waldes und der darin lebenden Götter hat. 

Okkoto kehrt später schwer verletzt zurück, Menschen versteckt unter den Fellen der gefallenen Keiler umringen ihn und verletzten ihn weiter, doch Okkoto in seinem Wahn glaub seine Armee sei von den Toten auferstanden. Er ist bereits dabei zu einem Dämon zu werden, doch er führt die Jäger zum Zentrum des Waldes, zum Teich an welchem der Shishigami erscheint. 

\subsubsection*{Der Wald: Shishigami und die Kodama}
Als Ashitaka das erste Mal den Wald des Shishigamis betritt, begegnen ihm die Kodama. Kleine geisterhafte Geschöpfe dennen ein kindliches Gemüt inne wohnt weisen ihm den weg zum Herz des Waldes. Sie sind ein Zeichen dafür, dass der Wald gesund ist. Die verletzten Schmiedebewohner, welche Ashitaka aus dem Fluss gezogen hat und die er zu ihrem Dorf bringen möchte, fürchten sich vor den kleinen Wesen. Als Ashitaka jedoch sieht, dass sein Reittier auch in der Anwesenheit der Kodama ruhig bleibt, sieht er in ihnen keine Direkte gefahr. Trotzdem schliesst er die Möglichkeit nicht aus, dass sie ihn in die Irre führen könnten.

In der Mitte des Waldes findet sich ein seichter See. Hier wohnt der Waldgott Shishigami. Er hat die Erscheinung eines mächtigen Hirschens, jedoch hat er an der Stelle eines Tierkopfs ein menschliches Gesicht. Wenn sie Sonne sich senkt und es Nacht wird kommt der Shishigami zur Lichtung beim See und verwandelt sich in den Nachtwandler. In einer ansatzweise humanoiden Form wächst er in Grösse über den Wald hinaus und schreitet substanzlos durch den Wald. Mit den ersten Sonnenstrahlen kehr er zur Lichtung zurück und verwandelt sich wieder in seine Hirschform. 

Der Shishigami ist der Herr dieses Waldes, als solcher kann er Leben geben oder Leben nehmen. Besonders deutlich wird das bei einer Nahaufnahme, als er über die Lichtung schreitet. Mit jedem Auftreten wachsen Blumen und Pflanzen, es wuchert regelrecht. Doch in dem Moment, wo sich der Fuss wieder hebt verwelkt alles und zurück bleibt ein kleiner Fleck tote Erde. Nicht nur die Tiere und Götter des Waldes wissen von den seltsamen Fähigkeiten von Shishigami. Der Kaiser beauftragt Eboshi den Kopf des Shishigami zu bringen, da er glaub, der Kopf könne jede Wunde heilen. In der Tat rettet Shishigami Ashitaka das Leben, in dem er die Schusswunde heilt, welche Ashitaka bekommen hat, als er versuchte San von den Schmiedebewohnern zu beschützen. Doch zu Ashitakas Leid erkennt er, dass der Shishigami den Fluch des wütenden Keilerdämons nicht entfernt hat und dass ihm somit immer noch ein schmerzlicher Tod bevorsteht. Jedoch ist Shishigami nicht ein guter allesheilender Gott. Wer ihn aufsucht kann auf Heilung hoffen, muss aber auch den Tod erwartet. Moro nennt den Tatarigame feige, dass er sich dem Shishigami nicht gestellt hat. Der Waldgott hätte ihn heilen können, und wenn nicht, dann hätte er ihn getötet und Nago hätte nicht zu dem werden müssen was er am Ende bekam. Es scheint jedoch, dass Nagos Furcht vor dem Shishigami berechtig gewesen war. Okkoto stirbt unter Shishigamis Berührung.

Wo die Tiergötter gegen die Menschen und ihre Zerstörung des Waldes kämpfen, scheint der Waldgott selbst gerade zu gleichgültig. Die Wildschweine warten bis am Ende darauf, dass Shishigami ihnen hilft die Menschen zu verjagen. Doch der Shishigami tut nichts dergleichen. Selbst als ihn Eboshi anschiesst, passiert nicht mehr, als dass er einerseits für einen kurzen Moment im Wasser über welches er sonst läuft einsinkt, bevor er seinen Gang fortsetzt. Zweitens lässt er aus dem Geweher mit welchem Eboshi auf ihn zielt Pflanzen wachsen und gibt somit einmal mehr ein Bild von Leben und Tod. Als Eboshi es endlich schafft dem Waldgott den Kopf vom Rumpf zu schiessen, quillt das Innere des Waldgottes aus ihm heraus und zerstörrt alles auf seinem Weg. Im gleichen Moment fallen auch die Kodama von den Bäumen: der Wald stirbt. Im kopflosen Zustand bringt der Shishigami nur Zerstörrung und Tod. Erst als San und Ashitaka es schaffen den Kopf zurück zugeben, findet die Zerstörung ein Ende und in einer mächtigen Erschütterung wird aus der Zerstörrung neues Leben geschaffen. 

\section{Das wandelnde Schloss}
%!TEX root = Animismus_in_Anime.tex
%%%%%%%%%% HOWLS MOVING CASTLE %%%%%%%%%%%%%%%
\subsubsection{Filmhintergrund} 
\textsc{Das wandelnde Schloss} ist Miyazakis Adaption des Buches \glqq Sophie im Schloss des Zauberers \grqq von Diana Wynne Jones. Im Buch spielt die Geschichte wenigstens zum Teil in Wales. Also hat sich auch Miyazaki für ein europäisches Setting entschieden. So sieht man des Öfteren scharfkantige mit Schnee bedeckte Bergspitzen, raue, aber saftig grüne Alpenwiesen und wunderschöne Täler mit glasklaren Bächen und Seen. Als Vorlage dienten unter anderem die europäischen Städte Cardiff, Colmar, Heidelberg und Paris (\textsc{Nieder} 2006: 107). Obwohl es sich bei dem Film um eine Adaption handelt, weicht Miyazaki so stark von der Vorlage ab, das manche Dinge nur mit Hilfe des Buches verständlich zu sein scheinen. 
Die grösste Veränderung betrifft den Charakter der Hexe aus dem Ödland. Im Buch trägt sie eindeutig die Rolle der bösen Antagonistin, während im Film die Bürde des Gegenspielers auf verschiedene Charakter verteilt wird. Dies ist typisch für Miyazaki. In seinen Filmen gibt es keine klare Linie zwischen Gut und Böse. 

\subsubsection{Zusammenfassung} 
Der Zauberer Hauro zieht in seinem wandelnden Schloss umher. Es wird gemunkelt, er fresse die Herzen hübscher Mädchen. Sophie, die sich im Schatten ihrer hübschen Schwester und Mutter sieht, ist unzufrieden mit sich und ihrem Leben als Hutmacherin in dem Laden, den sie von ihrem verstorbenen Vater übernommen hat.

Eines Tages eilt ihr ein fremder Schönling zur Hilfe um von zwei übergriffigen Männern zu entkommen. Sophie verliebt sich in den jungen Mann, von dem sie aber vermutet, dass es sich um den Zauberer Hauro handelt. Diese kurze Begegnung reicht bereits auf, um die Aufmerksamkeit der Hexe aus dem Ödland auf sich zu ziehen. Diese hat scheinbar noch eine offene Rechnung mit Hauro. In der Folge wird Sophie durch einen Fluch in eine 80-jährige Greisin verwandelt. 

Auf der Suche nach etwas, was ihren Fluch brechen kann, findet sich Sophie bald darauf im wandelnden Schloss des Zauberers Hauro wieder. Sophie heuert kurzerhand als Hausdame und Putzfrau im Schloss an. Sie schliesst einen Handel mit Calcifer, dem Feuerdämon, welcher das Schloss steuert und bewegt: Er verspricht ihren Fluch zu brechen, wenn sie ihn von dem Packt mit Hauro befreit. In der Zwischenzeit ist ein offener Krieg zwischen den Nachbarländern ausgebrochen. Hauro, in den beiden Ländern unter verschiedenen Namen bekannt, soll auf beiden Seiten mitkämpfen. Erst mit Sophies Hilfe findet Hauro den Mut sich nicht mehr zu verstecken und stellt sich seiner Verantwortung. 

Es müssen aber erst noch viele Abenteuer bestanden werden, bis die beiden erkennen, dass sie einander lieben. Erst dann kann Sophie dem Zauberer sein flammendes Herz wieder zurück in seine Brust drücken und zugleich den Packt mit Calcifer lösen und ihren eigenen Fluch brechen. 

\subsubsection{Figurenanalyse} 
Auffallend in diesem Film ist, wie praktische alle Charakteren eine Metamorphose im Verlauf der Geschichte durchgehen. Aus dem eitlen blonden und gleichzeitig feigen Zauberer Hauro wird ein verantwortungsvoller liebender junger Mann. Die Hexe aus dem Ödland verwandelt sich zu einer tattrigen Frau. Eine Vogelscheuche mit einer Rübe als Kopf wird zum Prinzen. Und aus der alten Sophie wird wieder ein junges Mädchen.  

Obwohl viele der Charaktere interessante Untersuchungsgegenstände ergeben würden, beschränken wir uns hier auf das wandelnde Schloss und den Feuerdämon. Die Geschichte zwischen Sophie und dem Zauberer Hauro steht zwar inhaltlich im Zentrum des Filmes, für unser Interesse sind aber das Schloss und der Feuerdämon interessanter. Auf die lebendige Vogelscheuche wird nicht weiter eingegangen, da sie nur eine auslassbare Nebenhandlung darstellt. 

\subsubsection*{Calcifer und das wandelnde Schloss} 
Auch das titelgebende Schloss geht mehrere Schritte der Verwandlung durch. So ist es am Anfang eine riesige Maschinerie mit zahllosen Türmen, Röhren, Kammern und Öffnungen, deren Sinn und Zweck man nicht erraten kann. Es scheppert und klappert, pfeifft und knarrt mit jedem Schritt. Das Schloss läuft auf vier Vogelbeinen, die aus Metall hergestellt sind. Rostrot bis -braun dominiert das Konstrukt. Den Rumpf kann man in zwei Hauptteile unterteilen: Oben finden sich Schornsteine, Hausteile, Masten und schwere Kuppeln mit Guckrohren. Der untere Teil sieht aus wie ein Fisch mit Rübennase auf vier stelzigen Vogelbeinen. Ein langer Schlitz, der in zwei Gucklöchern endet, erzeugt die Illusion von Augen und Mund. Auf der Hinterseite des unteren Teiles ist eine senkrecht stehende Schwanzflosse angebracht. Die Last der oberen Teile schwankt bei jedem Schritt. Es ist vielmehr ein wandelndes Ungetüm als ein wandelndes Schloss. 

In dem Moment, in dem Sophie Calcifer aus dem Schloss trägt, verliert das Gebäude seine Integrität und sackt in sich zusammen. Sophie kehrt wieder mit Calcifer zurück und bittet ihn das Schloss erneut mobil zu machen. Erst nachdem sie ihm ihren Zopf opfert, verfügt Calcifer wieder über genügend Energie, um wenigstens einen Teil des Schlosses zu beleben. Was dabei heraus kommt ist eine viel kleinere und agilere Version des vorigen Baus. Dazu wurde viel unnützer Ballast abgeworfen, doch mangelt es nun auch an Komfort und Sicherheit. 
Im weiteren Verlauf der Geschichte erlischt Calcifer nahezu. Mit dem Leben, welches dem Feuerdämon entweicht, zerfällt auch das wandelnde Schloss. Nach einer Nacht, in der Calcifer zu einer kleine blauen Flamme reduziert wird, ist alles was vom Schloss noch übrig bleibt eine hölzerne Plattform, getragen von zwei Beinen. Das Wesen von Calcifer und dem wandelnden Schloss ist untrennbar miteinander verschmolzen. Calcifer gibt dem Schloss Leben, und das Schloss ist der Körper, in dem Calcifer wohnt. 

\section{Analyse}
%!TEX root = Animismus_in_Anime.tex
Für Völker und Ethnien, von welchen beliebte Beispiele von Geistern, Hexen und anderen übernatürlichen Phänomenen kommen, sind diese Phänomene real. So nimmt man als westlicher\todo{wie lässt sich das beschreiben?} Betrachter eine distanzierte analytische Position ein. Es wird versucht zu erklären ohne dass man das was man untersucht glaubt.\todo{formulieren!} In Filmen, wo sich der Betrachter mit den Charakteren identifiziert und in der fiktiven Umgebungswelt lebt, schaffen den Rahmen zumindest temporär in ein System von übernatürlichen und phantastischen Phänomenen abzutauchen, ohne seine rationale und faktische Weltsicht aufzugeben zu müssen. Die Vorstellung von Animismus konzentriert sich nach wie vor auf die Untersuchung von Gruppen und Ethnien, welche nicht von westlichen Kolonialismus überrannt wurden und sich ihre Kultur und somit auch Teile ihre Religion erhalten konnten. Das Studium von Animismus ist somit fast zwangsläufig eines, das einem fort führt aus seiner eigenen Welt.\todo{Umgebung/Kultur/Glaube/Weltanschauung} Doch durch die Analyse von Filmen (im weiteren Sinne auch von Literatur) gibt uns die Möglichkeit uns selbst zu studieren.

Es mag nach dieser Argumentation etwas widersprüchlich scheinen, dass ausgerechnet japanische Animationsfilme Gegenstand der Untersuchung sind. Es gibt im Bereich der Animationsfilme genügend Alternativen\footnote{Um ein Beispiel zu nennen: \textsc{Song of the Sea} (2014) und \textsc{The Book of Kells} (2009) produziert von Cartoon Saloon, einem irischen Animations Studio.}, welche kulturell näher wären. Doch am Ende ist die kulturelle Prägung des Rezeptionisten für diese Untersuchung entscheidend und nicht die des Filmes.

In den Filmen \emph{Prinziessin Mononoke} und \emph{Das wandelnde Schloss} haben wir einerseits ist da das alte, Mythen und Legenden reiche Japan, in dem Götter und Geister welche zwischen Sterblichen wandeln. Wenn auch die Welt angereichert ist mit phantastischen Wesen, so erkennen wir darin doch unserer eigene Vergangenheit in dieser Welt. Anderseits haben wir eine Märchenwelt, ein gutes Jahrhundert jüngeres Europa, wo Magie und Zauberei zum Alltag gehören. Diese Welt ist uns zeitlich zwar näher, doch gerade der breite Gebrauch von Magie in einer Zeit an die wir uns noch zu erinnern glauben, entfremdet sie für den Betrachter. In beide Welten ist das Übernatürliche Grund zur Furcht oder Faszination, jedoch ist niemand über das Vorhandensein vom diesem überrascht. \par

\medskip

In der folgenden Untersuchung werden beide Filme
In der folgenden Untersuchung der beiden Filme werden zwei Schwerpunkte gesetzt: Erstens wird untersucht wo und wie man Animismus im Sinne von Graham Harveys findet und zweitens sollen die Filme nach Elementen untersucht werden, welche Pascal Boyers Kriterien genügen.

\subsection{Mononoke}
\subsubsection{Konflikt als Resultat fehlenden Respekts}
Durch seinen historisch Hintergrund fügt sich die Welt des Films von sich aus in dem Shinto und somit in einen geschichtlichen Animismus ein. Die Welt vom Mononoke ist voll von verschiedenen \emph{Leuten}. Die meisten davon sind Menschen aber nicht wenige sind Nicht-Menschen-Leute. Es gibt verschiedene Interessensgruppen und entscheiden ist dabei nicht, ob es sich um Menschen oder Andere-als-Menschen handelt. Der Konflikt zwischen den Eisenschmiedebewohner und den Tieren des Waldes ist nicht grösser als der zwischen den Bewohnern der Eisenschmiede und den Samurai des Königs und die Affengötter werfen mit Steinen nach den Wolfsgöttern. Der Respektvolle Umgang zwischen den verschiedenen Menschen und Andere-als-Menschen Gruppe steht im Zentrum der Geschichte. Der rücksichtslose Umgang mit den andern ist der Grund der vielen Konflikte. Am Ende ist das Schlimmste zwar abgewandt, aber die Grundproblematik besteht und einzig ein gegenseitiger Respekt kann weiterhin den Frieden halten.

Miyazakis Botschaft in Mononoke ist ein Schuss ins Blaue für den Animismus den Harvey plädiert. Es ist hingegen schwieriger Harveys Beispiele für Animismus in Mononoke anzuwenden. Das Übernatürliche ist so real und fassbar in der Welt von Mononoke, dass es kaum Platz für das stille Belebte lässt wie zum Beispiel ein geschnitztes Anhänger der Maori oder ein Stein der Ojibwa. Eine Ausnahme dazu sind höchstens die Kodama, die kleine stummen Geister, welche Zeichen für einen gesunden Wald sind.\todo{mehr dazu?} 

\subsubsection{Die ontologische Kategorie TIER}
Boyers Ansatz um die Faszination der übernatürlichen zu Erklären bringt zunächst keine überraschende Resultate. Viel mehr haben wir mit den Tiergöttern einfache vorzeige Beispiele des ontologischen Verstosses der Kategorie TIER. Die Tiergötter des Waldes, die Wölfe, die Keiler und auch die Affen, handeln, sprechen und denken wie eine PERSON. Angereichert werden sie durch spezielle Merkmale, welche aber nur einen Verstoss der Schablone für ihre jeweilige Art darstellt. Die Tiergötter sind alle grösser als ihre natürlichen Verwandten, die Wölfin Moro besitzt zwei Schwänze und der Keiler Okkoto hat eine zusätzliche Reihe von Hauern. Jedoch sind das nicht die ausschlaggebenden Elemente. Das sehen wir am Beispiel von Ashitakas Reittier. Yakul hat ein kräftiges rot-oranges Fell welches cremig weiss ist an seiner Unterseite. Er hat den Körperbau eines Hirsches, ist aber kräftiger und grösser gebaut und er hat zwei lang Hörner anstelle eines Geweihs. Das Tier passt also zu keiner konkreten Vorstellung die wir haben, aber er passt zur Schablone welche eben Rehe, Gämsen und Elche einschliesst. Damit ist er höchsten besonders, aber er reiht sich nicht in die Vorstellungen religiöser Phänomene ein.

Nach ausführlichem Suchen könnte man der Wolfgöttin Moro, beziehungsweise dem Mädchen San einen weiteren Bruch zuschreiben, da die beiden in einer Mutter-Tochter Beziehung stehen und das über die ontologische Kategorie von TIER beziehungsweise PERSON drüber hinaus. Da es sich aber explizit um eine Adoption handelt, sollte dieser Punkt nicht für sich alleine stehen und in das PERSON sein von Moro eingeschlossen werden.

Interessanter und zugleich ungleich schwieriger für die Untersuchung mit Boyers Methoden ist der Waldgott Shishigami. Es ist nicht ganz einfach den Waldgott einzuordnen. Während dem Tag streift er mit einer Herde Rehen durch den Wald. Bei Nacht wechselt er die Form und kann ohne Widerstand durch den Wald laufen. Er ähnelt einem Hirsch, hat aber ein menschliches Gesicht. Er spricht nicht, aber mit seiner Berührung kann er Leben nehmen oder geben. Er läuft im Wald und ist zugleich der Geist des Waldes. Shishigami scheint voller Brüche und Verstosse in unseren Schablonen und Kategorien. Er bricht so viele Annahmen, dass es schwer ist ihn überhaupt einer ontologischen Kategorie zu zuordnen. Letztlich ist jedoch weniger sein (nicht ausfindbarer) Kategoriebruch das was ihn seiner Darstellung von einem religiösen Phänomen ausmacht. Wichtiger ist die Annahme, dass der Shishigami Dinge \emph{weiss}. Als Rezeption und als auch die Charakter im Film fragt man sich, wie der Waldgott entscheidet, wen er tötet und wen er heilt. Es könnte pure Willkür sein, doch der Gedanke liegt näher, dass der Waldgott über \emph{Informationen} verfügt. Wie Boyer das beschreibt zeigt sich dann auch, dass dieses Wissen eine Wichtige Rolle im sozialen Umfeld spielt. Als der Keilergott halb Wahnsinnig zum Waldgott stürmt, geht er davon aus, dass Shishigami \emph{weiss}, dass er und seine Krieger tapfer gekämpft haben und den Wald verteidigen.

Im Ritual, welches die Seherin, aus dem Dorf von dem Ashitaka stammt, soll der böse Geist des gefallenen Dämon besänftigt und das Dorf somit geschützt werden. Somit entspricht dieses Ritual Boyer Beschreibung von religiösen Ritualen, da es im Kern darum geht sich präventiv vor etwas zu schützen. Um den unheimlichen Eindruck zu verstärken den man sowie so schon vor Leichen hat, kommt hier hinzu, dass der Dämon obwohl er schon gestorben ist noch einen Fluch für alle hörbar spricht.\todo{Besseren Abschluss/übergang für nächstes Kapitel?}

\subsection{Wandelndes Schloss}
\subsubsection{Alles lebt}
Aus Sicht des alten Animismus ist es etwas schwieriger in diesem Film anzusetzen als es bei Mononoke war. Abgesehen von der Magie gleicht die Welt in der die Geschichte spielt unserer Welt vor nicht all zu langer Zeit. Eine Zeit in welcher man anderen Kulturen nach gesagt hat, dass sie \emph{primitiv} sind, weil sie an eine beseelte Welt glauben. Natürlich lässt sich unsere Vergangenheit nicht mit der Märchenwelt aus dem Film gleichsetzen, aber es soll gezeigt sein, dass der Animismus hier anders zu finden ist als in Mononoke.

Calcifer redet und interagiert mit den anderen Bewohnern des Schlosses. Das macht ihn, wenn man Harveys Ansatz folgt, eindeutig zu einer Person. Um es für den Betrachter einfacher zu machen, das Feuer als Person zu sehen bekommt Calcifer zwei grosse Glupschaugen und ein Mund von variabler Grösse, in das er sich gerne Holzstücke reinstopft. Die Augen sind in der Regel rund, weisse Kreise mit schwarzen Pupillen. Doch mit dem flackern seines ganzen Körpers verändert sich die Form der Augen zwischen durch unmerklich, so dass sie schlitzförmiger werden und einen gefährlichen (dämonischen) Eindruck machen. Seine Flammenform, welche keine feste Umrisslinie hat flackert beständig und von Zeit zu Zeit lösen sich kleine Flämmchen von ihm. Calcifer gehört also sicherlich zu jenen Anders-als-Menschen mit denen mal alleine der Vorsicht wegen mit Respekt behandeln sollte. Obwohl das Feuer freundlich sein kann, schimmert immer wieder seine dämonische und unberechenbare Seite hindurch.

Aus Sicht von Harvey macht es wenig Sinn zwischen dem Feuer Calcifer und dem wandelnden Schloss zu unterscheiden, da beides das gleiche ist.

Sophie zeigt (fast) allen gegenüber einen respektvollen, hilfsbereiten Umgang. Sei dies nun Hauro, Markl (Hauros junger Assisten) oder der Hexe aus dem Ödland\todo{es gibt hier verschiedene Situationen.} als Menschen oder dem Rübenkopf und Calcifer als Andere-als-Menschen. Was einem nur klar wird, wenn man auch die Buchvorlage von Jones gelesen hat, ist, dass Sophie selbst eine magische Fähigkeit hat Dinge zu \emph{beleben}. Es ist daher also nicht wunderlich, dass Sophie vorsichtshalber mit allem einen anständigen Umgang pflegt. 

\subsubsection{Ist ein Haus ein WERKZEUG?}
Ist ein Haus ein Werkzeug? Im ersten Moment scheint die Idee etwas abwegig, doch das liegt wohl daran dass das gleiche System welches Boyer beschreibt bei der Reflextion des Wortes Werkzeug die Schablone \emph{Handwerkzeuge} nimmt. Ein Schraubenzieher, ein Hammer, vielleicht auch ein Handmixer - das alles passt zur Schablone der Handwerkzeuge. Erst mit einem Schritt zurück (oder mit Hilfe des Ausschlusverfahren), kommt man zum Schluss, dass auch ein Haus ein Werkzeug ist. Ein Werkzeug ist etwas (in der Regel) Menschengemachtes das eine Funktion oder einen Zweck erfüllen soll. Mit dieser Beschreibung können wir also getrost sagen, dass ein Haus zur ontologischen Kategorie WERKZEUG gehört. Für das wandelnde Schloss erwarten wir nach Boyer jetzt einen Bruch, welche diese Kategorie verletzt. Eine weitere Eigenschaft von Werkzeugen ist auch, dass die (so fern sie funktionieren) die Absicht eines Menschen erfüllen. Werkzeuge haben keine Selbstbestimmung. Und genau darüber verfügt das wandelnde Schloss. Es lässt sich argumentieren, dass das Schloss vom Feuerdämon Calcifer belebt und gesteuert wird. Die Verletzung der ontologischen Kategorie bleibt jedoch die gleiche mit dem Unterschied, dass es statt dem WERKZEUG zu dem das Haus gehört nun dem NATUR OBJEKT ist. In beiden Fällen wird der Eindruck dadurch verstärkt, dass Eigenschaften einer anderen ontologischen Kategorie in das Aussehen einfliessen. 

\addcontentsline{toc}{section}{Schlusswort}
%!TEX root = Animismus_in_Anime.tex
Mag Anime und Miyazaki -> vielleicht doch besser ins Vorwort

Die Seherin vollführt ein Ritual um sich und ihr Dorf vor dem Zorn des Gefallen Gottes zu schützen. Diese beiden helfen aber nur ein religiöses Phänomen zu identifizieren, jedoch nicht umbedingt zu erklären, warum das auch für uns westler so interessant ist. Oder ist es gerade, dass das Ritual anders ist, aber die Notwendigkeit (Schablone davon) in uns allen vorhanden und wir es deswegen verstehen? Letztlich noch kurz die falschen Wildschweine: Unheimlich auch für den Betrachter. Umgekehrt von den Tiergöttern: PERSON die aussieht wie TIER.

\addcontentsline{toc}{section}{Literaturverzeichnis}
\addcontentsline{toc}{section}{Filmografie}
%%%%%%%%%%%%%%%%%%%%%%%%%%%%%%%%%%%%%%%%%%%%%%%%%%%%%%%%%
% Bibliographie
\newpage
\begin{thebibliography}{9}

\bibitem{nieder06}
	Nieder, Julia. 
	(2006). 
	\emph{Die Filme von Hayao Miyazaki.}
	Marburg: Schüren-Verlag.

\bibitem{faulstich13}
	Faulstich, Werner.
	(2013).
	\emph{Grundkurs Filmanalyse.} 
	3. Aufl. 
	Paderborn: Wilhelm Fink Verlag.

\bibitem{boyer04}
	Boyer, Pascal.
	(2004).
	\emph{Und Mensch schuf Gott.}
	Stuttgart: Klett-Cotta.

\bibitem{harvey06}
	Harvey, Graham.
	(2006).
	\emph{Respecting the Living World.}
	New York: Columbia University Press.

\bibitem{thomas12}
	Thomas, Jolyon.
	(2012)
	\emph{Drawing on tradition. Manga, Anime, and Religion in Contemporary Japan.}
	Honolulu: University of Hawai'i Press.

\bibitem{miyazakiweb}
	Team Ghiblink. 
	(ohne Jahr). 
	\emph{The Hayao MIYZAKI Web}. 
	17. Aug. 2015. 
	\texttt{http://nausicaa.net/miyazaki/}

\bibitem{wandelndeSchloss}
	\textsc{Hauro no Ugoku Shiro / Das wandelnde Schloss}.
	Japan 2004. \\
	Drehbuch, Storyboard und Regie: Hayao Miyazaki.\\
	Produzent: Toshio Suzuki. \\
	Musik: Joe Hisaishi. \\
	Laufzeit: 119 Minuten. \\

\end{thebibliography}

\end{document}

%!TEX root = Animismus_in_Anime.tex
\subsection{Filmhintergrund}
\subsubsection*{Veröffentlichung und Erfolg}
Mit \textsc{Prinzessin Mononoke} gelang Hayao Miyazaki erstmals einen internationalen Durchbruch. Der Film kam am 17. Juli 1997 in die japanischen Kinos. Mit 18.65 Millarden Yen (umrechnung?) spielte er in Japan mehr ein als Titanic (James Cameron). Er war der bisher erfolgreichste Film in Japan. Nachdem er zuerst in 1998 auf der 48. Berlinale das erste Mal in Deutschland vorgeführt wurde und \emph{irgendwelche Preise} gewonnen hat, kam er 1999 in den Vereinigten STaaten und Kanada und im Jahr darauf auch in Europa in die Kinos. Trotz den vielen Auszeichnungen welche der Film gewann, war der Film, sowie Ghibli Studio und Hayao Miyazaki vorwiegend unter Anime-Fankreisen bekannt. Es wurde ausserhalb von Japan auch nur wenig Werbung gemacht. Auf internationale Vermarktung wurde lange verzichtet, da Miyazaki und sein Team darüber entsetzt waren, wie starkt geschinitten \textsc{Nausicaä aus dem Tal der Winde} wurde. Nach dem riesigen Erflog an den japanischen Kassen zeigte das US-Studios Disney Interesse und sicherte sich die Verhandlungsposition der japanischen Filmemacher. Im Vertrag über die internationale Vermarktung der Ghibli-Filme welcher bald darauf geschlossen wurde konnte sich Miyazaki und sein Team das Recht sichern, über allfällige Schnittstellen selbst entscheiden zu können. 

\subsubsection*{Geschichtlicher Hintergrund und Miyazaki}
Erste Ideen für \textsc{Prinzessin Mononoke} hatte Miyazaki bereits 1970. Damals diente im als Plotvorlage das Märchen von der Schönen und dem Biest.\footnote{http://home.comcast.net/~rocksunner/miya\_e.html} Wegen dem leichtsinnigen Versprechen ihres Vaters, muss die Tochter des Fürsten ein Waldmonster [mononoke]\footnote{Geist/Monster/Gespenst -> http://nausicaa.net/miyazaki/mh/faq.html\#translation} heiraten. Doch Miyazaki kam nicht weiter mit der Geschichte und schob sie auf. 

\begin{quote} Actually, in the beginning I wanted to do a fantasy rather than a period drama set in Japan. However, when I said "Now let's do it", I didn't have the heart for it.\footnote{http://home.comcast.net/~rocksunner/miya\_e.html} 
\end{quote}

Er entschied sich gegen die anfänglich geplante Fantasy Umwelt und setzte die Geschichte im traditionellen feudalen Japan an. Doch in der Zeit, wo er sich mit den Hintergründen auseinander setzte, änderte sich auch das Herzstück der Geschichte. Miyazaki bekam den Wunsch einen tiefgründigeren, authentischeren Film zu machen. Somit fand er wieder zum Thema zurück, das er schon mit \textsc{Nausicaä aus dem Tal der Winde} im Fokus hatte: Das (konfliktreiche) zusammenleben einerseits von Mensch und Natur und anderseits aber auch von Mensch und Mensch. So blieb am Ende von der Ursprünglichen Geschichte nicht mehr übrig als der Name \emph{Mononoke}.\footnote{Das Märchen von der Prinzessin und dem Biest hat Miyazaki später in Form eines Bilderbuches veröffentlicht.} \todo{Nachweis}

Infos aus:\footnote{https://de.wikipedia.org/wiki/Prinzessin\_Mononoke\#Hintergr.C3.BCnde http://nausicaa.net/miyazaki/mh/filminfo.html http://home.comcast.net/~rocksunner/mono\_e.html}

Die Geschichte spielt in einem feudalen Japan zu einer Zeit, welche der Muromachi-Ära (1392-1573) ähnelt. Historisch wichtige Figuren bleiben im Hintergrund. Ein König/Shogun wird zwar erwähnt, jedoch sehen wir nur die Folgen der Interessentskonflikte der Mächtigen. Der Fokus der Geschichte liegt beim einfachen Menschen, insbesondere bei Aussenseiter der Gesellschaft, welche in einer Zeit der kulturellen Blüte, sozialen Umwälzungen und politischen Unruhe leben. 

\subsection{Zusammenfassung}
Ashitaka, der letzte Prinz eines in Harmonie mit der Natur lebenden Volks namens Emishi\footnote{Alte Bezeichnung für ein japanisches Urvolk. Nach der Heian-Zeit wurde das Volk Ezo genannt.}, muss seine Heimat verlassen weil er einen Dämon\footnote{Tatarigame = curse god (http://home.comcast.net/~rocksunner/mono\_e.html\#ashitaka)}, welcher das Dorf angegriffen hat tötete und so dessen Fluch auf sich zog. Ashitaka macht sich auf den Wald des Shishigamis\footnote{Bedeutung} zu finden. Von dort kommt nämlich der Dämon, welcher einst ein Keilergott war und erst durch den Schmerz und damit kommendem Hass in zu einem Dämon wurde. 

Bald findet sich Ashitaka zwischen verhärteten Fronten. Es herrscht Krieg zwischen den Waldgöttern und den Menschen welche in einer Erzschmiede arbeiten. Er trifft auf San, welche von den andern Mensch \emph{Prinzessin Mononoke} genannt wird, weil sie von der Wolfgöttin Moro aufgezogen wurde und sich wie ein Tier verhaltet. Ashitaka versucht zwischen San, welche für ihre Familie und ihren Lebensraum kämpft und den Bewohnern der Schmiede, welche den Waldabholzen um Erz zu gewinnen, was ebenfalls ihre Lebensgrundlage ist zu vermitteln. Erschwerend kommt hinzu, dass die Anführerin der Schmiedebewohner, Eboshi systematisch versucht die Waldgötter zu vernichten, um ihre Untertanen so zu schützen. Zudem hat sie dem Kaiser den Kopf des Shishigami versprochen. Sie erhofft sich so den Schutz für ihre Schmiede zu sichern, da die Gefahr besteht, dass sie vom selbigen Kaiser angegriffen werden könnte. 

Trotz Ashitakas Bemühungen kann dem Kampf nicht ausgewichen werden. Eboshi schiesst dem Waldgott Shishigami den Kopf ab und der Rumpf des Gottes droht alles zu zerstören. Durch das selbstlose Eingreiffen von San und Ashitaka kann das Verderbnis im letzten Moment gestoppt werden, doch vieles der alten Welt ist danach für immer verloren.

Obwohl man durch ästhetische Gestaltung einfach erraten kann, bei wem und wo Miyazakis Sympathie liegt, so ist es doch erstaunlich, dass der Geschichte jegliches Schwarz-Weiss denken fehlt. Jeder der kämpft hat seine Gründe. Ein durchgehendes Motiv ist der destruktiver Hass. Somit birgt das Filmende zwar Hoffnung in sich, jedoch bleibt die Problematik bestehen. Ashitaka und San wollen sich zwar weiterhin sehen, jedoch kann das Wolfmädchen den Mensch nicht vergeben und bleibt bei ihren Wolfbrüdern im Wald. Ashitaka, der zwar grossen Respekt vor der Natur und den Geistern zeigt kehrt aber dennoch in die Schmiede zurück um dort mit den anderen Menschen zu leben.  

\subsection{Figuren Analyse}
In Japans Altertum angesiedelt passt sich die belebte Natur gut ins Bild ein. In den Städten und Dörfen welche Ashitaka besucht auf seinem Weg zum Shishigami Wald haben die Leute praktisch keinen Kontakt zu den Naturgeistern. Zusehr sind sie mit den politischen Dingen beschäftigt. Als Ashitaka aber endlich im Reich des Shishigami ankommt ist das Verhältnis zwischen Mensch und Natur, oder besser gesagt zwischen Mensch und Waldgöttern ein anderes. Ein friedliches Nebeneinander scheint unmöglich.

\subsubsection*{Die Tiergötter: Moro, Nago und Okkoto}
Moro ist eine Wolfsgötting die zusammen mit ihren Söhnen und ihrer adoptierten Menschentochter San im Wald des Shishigamis lebt. Mit ihrer Grösse überragt sie alle Menschen. Sie ist weiss und hat einen doppelten Schwanz. Ihre Stimme ist tiefe und klingt nicht weiblich. Moro hasst die Menschen und Ebshi am meisten von allen. Für ihr Ziel, Eboshi zu töten riskiert Moro viel, nicht zu letzt ihr eigenes Leben und lebst im Tod, wo ihr Kopf vom Körper getrennt ist, schafft es Moro noch Eboshi den Arm ab zu beissen. Im Gegensatz zu andern Akteuren in der Geschichte (insbesondere den Wildschweinen) jedoch macht der Hass sie nicht blind. Und so stellt sie sich auch gegen den aufgebrachten Keilergott Okkoto um San zu retten.

\begin{quote}
Humans who attacked the forest threw a baby to me in order to escape my fangs. That was San...! She can't be human, neither can she fully be a wolf. She's my poor, ugly, loveable daughter.\footnote{http://home.comcast.net/~rocksunner/mono3e.html\#moro}
\end{quote}

Ein anderes Bild von den Tiergöttern bekommen wir durch die Keiler. Den Dämon, welcher Ashitaka am Anfang des Filmes bezwingt um sein Dorf zu schützen war einst ein mächtiger Keilergott und stammt auch aus dem Shishigami Wald. Als er zu den Emishi kommt hat der Schmerz und der Hass ihn bereit so wütend gemacht, dass er zu einem Dämon [Tatarigame]\footnote{Begriffserklärung} wurde. Ashitaka bittet den rasenden Gott Umkehr zu machen und sein Dorf zu verschonen und erst als er sich zwischen seinem Dorf und dem Dämon entscheiden muss, tötet er ihn. Die Dorfseherin verrichtet gleich darauf ein Versöhnungsritual in dem sie dem Dämon verspricht ein Schrein zu errichten und ihn bittet er möge nicht länger hassen. Doch eine Stimme erklingt aus dem Toten Keiler und verflucht alle Menschen (\glqq  Loathsome humans! You will know my wrath well for causing me pain\dots \grqq \footnote{http://home.comcast.net/~rocksunner/mono\_e.html\#ashitaka})

Im späteren Verlauf der Geschichte begegnet Ashitaka dem mächtigen Keilergott Okkoto, dem Herr aller Keiler. Okkoto ist weiss wie Moro und ihre Söhne, graue Stellen in seinem borstigem Fell und seine trüben Augen weisen jedoch darauf hin, dass er sehr alt ist. In seiner Grösse überragt der Keilergott sogar Moro und so wie sie zwei Schwänze hat, verfügt er über eine zweite Reihe von mächtigen Hauern. 

Okkoto erfährt durch Ashitaka von Nagos Schicksal und zeigt sich beschämt, dass einer aus seinem Klan ein so übles Schicksal passierte. Er zeigt jedoch weder Mitleid noch Vergeben gegenüber Ashitaka und teilt ihm mit, dass er ihn bei der nächsten Begegnung umbringen wird. Obwohl er einsichtiger scheint als die andern Keiler ist Okkoto so sturr, dass von seiner Weisheit am Ende nichts übrig bleibt. Von den Menschen durch Waldbrände und Lärm provoziert, rennen die Wildschweine unter Okkotos führung gerade in ein Massaker. Moro durchschaut die Absichten der Menschen und auch Okkoto vermutet was dahinter steckt, jedoch setzt er alles auf eine letzte Schlacht. 

\begin{quote}
Moro, look at my clan! Little by little they are becoming smaller and more stupid. If this keeps up, humans will be able to hunt us down like common meat\dots \footnote{http://home.comcast.net/~rocksunner/miya\_e.html} 
\end{quote}

Auch Moros Söhne sind nicht so prächtig und gross wie sie selbst. Zwar immer noch grösser als normale Wölfe und weiss, fehlen ein doppelter Schwanz oder sonst ein Merkmal, welches sie von normalen Wölfen unterscheidet. Aus Moros Handeln und dem was sie sagt geht hervor, dass sie keine Hoffnung für die Zukunft des Waldes und der darin lebenden Götter hat. 

Okkoto kehrt später schwer verletzt zurück, Menschen versteckt unter den Fellen der gefallenen Keiler umringen ihn und verletzten ihn weiter, doch Okkoto in seinem Wahn glaub seine Armee sei von den Toten auferstanden. Er ist bereits dabei zu einem Dämon zu werden, doch er führt die Jäger zum Zentrum des Waldes, zum Teich an welchem der Shishigami erscheint. 

\subsubsection*{Der Wald: Shishigami und die Kodama}
Als Ashitaka das erste Mal den Wald des Shishigamis betritt, begegnen ihm die Kodama. Kleine geisterhafte Geschöpfe dennen ein kindliches Gemüt inne wohnt weisen ihm den weg zum Herz des Waldes. Sie sind ein Zeichen dafür, dass der Wald gesund ist. Die verletzten Schmiedebewohner, welche Ashitaka aus dem Fluss gezogen hat und die er zu ihrem Dorf bringen möchte, fürchten sich vor den kleinen Wesen. Als Ashitaka jedoch sieht, dass sein Reittier auch in der Anwesenheit der Kodama ruhig bleibt, sieht er in ihnen keine Direkte gefahr. Trotzdem schliesst er die Möglichkeit nicht aus, dass sie ihn in die Irre führen könnten.

In der Mitte des Waldes findet sich ein seichter See. Hier wohnt der Waldgott Shishigami. Er hat die Erscheinung eines mächtigen Hirschens, jedoch hat er an der Stelle eines Tierkopfs ein menschliches Gesicht. Wenn sie Sonne sich senkt und es Nacht wird kommt der Shishigami zur Lichtung beim See und verwandelt sich in den Nachtwandler. In einer ansatzweise humanoiden Form wächst er in Grösse über den Wald hinaus und schreitet substanzlos durch den Wald. Mit den ersten Sonnenstrahlen kehr er zur Lichtung zurück und verwandelt sich wieder in seine Hirschform. 

Der Shishigami ist der Herr dieses Waldes, als solcher kann er Leben geben oder Leben nehmen. Besonders deutlich wird das bei einer Nahaufnahme, als er über die Lichtung schreitet. Mit jedem Auftreten wachsen Blumen und Pflanzen, es wuchert regelrecht. Doch in dem Moment, wo sich der Fuss wieder hebt verwelkt alles und zurück bleibt ein kleiner Fleck tote Erde. Nicht nur die Tiere und Götter des Waldes wissen von den seltsamen Fähigkeiten von Shishigami. Der Kaiser beauftragt Eboshi den Kopf des Shishigami zu bringen, da er glaub, der Kopf könne jede Wunde heilen. In der Tat rettet Shishigami Ashitaka das Leben, in dem er die Schusswunde heilt, welche Ashitaka bekommen hat, als er versuchte San von den Schmiedebewohnern zu beschützen. Doch zu Ashitakas Leid erkennt er, dass der Shishigami den Fluch des wütenden Keilerdämons nicht entfernt hat und dass ihm somit immer noch ein schmerzlicher Tod bevorsteht. Jedoch ist Shishigami nicht ein guter allesheilender Gott. Wer ihn aufsucht kann auf Heilung hoffen, muss aber auch den Tod erwartet. Moro nennt den Tatarigame feige, dass er sich dem Shishigami nicht gestellt hat. Der Waldgott hätte ihn heilen können, und wenn nicht, dann hätte er ihn getötet und Nago hätte nicht zu dem werden müssen was er am Ende bekam. Es scheint jedoch, dass Nagos Furcht vor dem Shishigami berechtig gewesen war. Okkoto stirbt unter Shishigamis Berührung.

Wo die Tiergötter gegen die Menschen und ihre Zerstörung des Waldes kämpfen, scheint der Waldgott selbst gerade zu gleichgültig. Die Wildschweine warten bis am Ende darauf, dass Shishigami ihnen hilft die Menschen zu verjagen. Doch der Shishigami tut nichts dergleichen. Selbst als ihn Eboshi anschiesst, passiert nicht mehr, als dass er einerseits für einen kurzen Moment im Wasser über welches er sonst läuft einsinkt, bevor er seinen Gang fortsetzt. Zweitens lässt er aus dem Geweher mit welchem Eboshi auf ihn zielt Pflanzen wachsen und gibt somit einmal mehr ein Bild von Leben und Tod. Als Eboshi es endlich schafft dem Waldgott den Kopf vom Rumpf zu schiessen, quillt das Innere des Waldgottes aus ihm heraus und zerstörrt alles auf seinem Weg. Im gleichen Moment fallen auch die Kodama von den Bäumen: der Wald stirbt. Im kopflosen Zustand bringt der Shishigami nur Zerstörrung und Tod. Erst als San und Ashitaka es schaffen den Kopf zurück zugeben, findet die Zerstörung ein Ende und in einer mächtigen Erschütterung wird aus der Zerstörrung neues Leben geschaffen. 
%!TEX root = Animismus_in_Anime.tex
\subsubsection*{Veröffentlichung und Erfolg} 
Mit \textsc{Prinzessin Mononoke} gelingt Hayao Miyazaki der internationale Durchbruch. Der Film kommt am 17. Juli 1997 in die japanischen Kinos. Mit 18,65 Millarden Yen (das entspricht damals 242,45 Mio CHF) spielt er in Japan mehr ein als \emph{Titanic} von James Cameron. Damit ist Prinzessin Mononoke der bisher erfolgreichste Film in Japan. Nachdem er 1998 auf der 48. Berlinale das erste Mal in Deutschland vorgeführt wird und \todo{zahlreiche Preise} gewinnt, kommt er 1999 in den Vereinigten Staaten und Kanada und im Jahr darauf auch in Europa in die Kinos. Trotz der vielen Auszeichnungen, die der Film gewonnen hat, blieb der Film, sowie das Ghibli Studio und Hayao Miyazaki, vorwiegend nur unter Anime-Fankreisen bekannt. Dies hängt sicherlich damit zusammen, dass ausserhalb von Japan kaum Werbung gemacht wurde. Auf internationale Vermarktung wurde lange verzichtet, da Miyazaki und sein Team darüber entsetzt waren, wie starkt \textsc{Nausicaä aus dem Tal der Winde} im Ausland geschnitten wurde. 
Nach dem riesigen Erflog an den japanischen Kassen zeigte das US-Studios Disney Interesse und sicherte sich die Verhandlungsposition der japanischen Filmemacher. Im Vertrag über die internationale Vermarktung der Ghibli-Filme, welche bald darauf geschlossen wurde, konnte sich Miyazaki und sein Team das Recht sichern, über allfällige Schnittstellen selbst entscheiden zu dürfen. 

\subsubsection*{Geschichtlicher Hintergrund und Miyazaki} 
Erste Ideen für \textsc{Prinzessin Mononoke} hatte Miyazaki bereits 1970. Damals diente ihm als Plotvorlage das Märchen von der Schönen und dem Biest.\todo{http://home.comcast.net/~rocksunner/miya\_e.html} Wegen dem leichtsinnigen Versprechen ihres Vaters, muss die Tochter des Fürsten ein Waldmonster [mononoke]\todo{Geist/Monster/Gespenst -> http://nausicaa.net/miyazaki/mh/faq.html\#translation} heiraten. Weil Miyazaki mit der Geschichte nicht weiter kam, schob er ihre Bearbeitung hinaus. 

\begin{quote} \glqq Actually, in the beginning I wanted to do a fantasy rather than a period drama set in Japan. However, when I said "Now let's do it", I didn't have the heart for it.\grqq \todo{http://home.comcast.net/~rocksunner/miya\_e.html} 
\end{quote}

Er entscheidet sich schliesslich gegen die anfänglich geplante Fantasy-Umwelt und setzt die Geschichte im traditionellen feudalen Japan an. Doch in der Zeit, in der er sich mit den Hintergründen auseinander setzt, ändert er auch das Herzstück der Geschichte. In Miyazaki erwächst der Wunsch einen tiefgründigeren, authentischeren Film zu machen. Dadurch findet er wieder zum Thema zurück, das er schon mit \textsc{Nausicaä aus dem Tal der Winde} im Fokus hat: Das (konfliktreiche) zusammenleben von Mensch und Natur einerseits, und von Mensch und Mensch andrerseits. So bleibt am Ende von der ursprünglichen Geschichte nicht mehr übrig als der Name \emph{Mononoke}.\todo{Das Märchen von der Prinzessin und dem Biest hat Miyazaki später in Form eines Bilderbuches veröffentlicht.} 

Infos aus:\todo{https://de.wikipedia.org/wiki/Prinzessin\_Mononoke\#Hintergr.C3.BCnde http://nausicaa.net/miyazaki/mh/filminfo.html http://home.comcast.net/~rocksunner/mono\_e.html}

\subsection{Zusammenfassung}
Die Geschichte spielt im feudalen Japan zu einer Zeit, die der Muromachi-Ära (1392-1573) ähnelt. Historisch wichtige Figuren bleiben im Hintergrund. Ein König bzw. Shogun wird zwar erwähnt, jedoch bekommen die Zuschauer nur die Folgen der Interessenskonflikte der Mächtigen zu sehen. Nicht bei ihnen liegt der Fokus der Geschichte, sondern bei den einfachen Menschen und mehr noch bei den Aussenseitern der Gesellschaft. Sie leben in einer Zeit der kulturellen Blüte, sozialer Umwälzungen und politischer Unruhe. 

Ashitaka, der letzte Prinz eines in Harmonie mit der Natur lebenden Volkes namens Emishi\todo{Alte Bezeichnung für ein japanisches Urvolk. Nach der Heian-Zeit wurde das Volk Ezo genannt.}, muss seine Heimat verlassen, weil er einen Dämon\todo{Tatarigame = curse god (http://home.comcast.net/~rocksunner/mono\_e.html\#ashitaka)}, der das Dorf angegriffen hat, tötete und so dessen Fluch auf sich zog. Ashitaka macht sich auf, den Wald des Shishigamis\todo{Bedeutung} zu finden. Denn von dort kommt der Dämon, welcher einst ein Keilergott war und erst durch den Schmerz und den daraus erwachsenen Hass zu einem Dämon wurde. 

Bald findet sich Ashitaka zwischen verhärteten Fronten. Es herrscht Krieg zwischen den Waldgöttern und den Menschen, welche in einer Erzschmiede arbeiten. Ashitaka trifft auf San, die von den anderen Mensch \emph{Prinzessin Mononoke} genannt wird, weil sie von der Wolfgöttin Moro aufgezogen wurde und sich wie ein Tier verhält. Ashitaka versucht zwischen San, die für ihre Familie und ihren Lebensraum kämpft, und den Bewohnern der Schmiede, welche den Wald abholzen um Erz zu gewinnen, was ihre Lebensgrundlage ist, zu vermitteln. Erschwerend kommt hinzu, dass die Anführerin der Schmiedebewohner, Eboshi, systematisch versucht die Waldgötter zu vernichten, um ihre Untertanen vor ihnen zu schützen. Darüberhinaus hat sie dem Kaiser den Kopf des Shishigami versprochen. Sie erhofft sich so den Schutz für ihre Schmiede zu sichern, da die Gefahr besteht, dass sie vom selbigen Kaiser angegriffen werden könnte. 

Trotz Ashitakas Bemühungen kann dem bewaffneten Konflikt nicht ausgewichen werden. Eboshi schiesst dem Waldgott Shishigami den Kopf ab. Als Folge dessen droht der nun kopflos tätige Rumpf des Gottes alles zu zerstören. Durch das selbstlose Eingreiffen von San und Ashitaka kann das Verderbnis im letzten Moment gestoppt werden. Doch vieles der alten Welt ist danach für immer verloren.

Obwohl man durch ästhetische Gestaltung einfach erraten kann, bei wem und wo Miyazakis Sympathie liegt, so ist es doch erstaunlich, dass der Geschichte jegliches Schwarz-Weiss denken fehlt. Jeder, der kämpft, hat seine Gründe. Ein durchgehendes Motiv ist der destruktive Hass. Somit birgt das Film-Ende zwar Hoffnung in sich, die Problematik bleibt jedoch bestehen. 
Obwohl Ashitaka und San sich weiterhin sehen wollen, kann das Wolfmädchen den Mensch nicht vergeben und bleibt bei ihren Wolfbrüdern im Wald. Ashitaka, der zwar grossen Respekt vor der Natur und den Geistern zeigt, kehrt dennoch in die Schmiede zurück, um dort mit den Menschen zu leben.  

\subsection{Figuren Analyse}
In Japans Altertum angesiedelt passt sich die belebte Natur gut ins Bild ein. In jenen Städten und Dörfern, die Ashitaka auf seinem Weg zum Shishigami Wald besucht, haben die Leute so gut wie keinen Kontakt zu den Naturgeistern. Zu sehr sind sie mit den politischen Problemen beschäftigt. Als Ashitaka endlich im Reich des Shishigami ankommt, ist das Verhältnis zwischen Mensch und Natur, oder vielmehr zwischen Mensch und Waldgöttern ein anderes. Ein friedliches Nebeneinander scheint unmöglich. 

\subsubsection*{Die Tiergötter: Moro, Nago und Okkoto}
Moro ist eine Wolfsgöttin die zusammen mit ihren Söhnen und ihrer adoptierten Menschentochter San im Wald des Shishigami lebt. Mit ihrer Grösse überragt sie alle Menschen. Sie ist weiss und hat einen doppelten Schwanz. Ihre Stimme ist tief und klingt nicht weiblich. Moro hasst die Menschen; Eboshi am meisten von allen. Für ihr Ziel, Eboshi zu töten, riskiert Moro viel, nicht zu letzt ihr eigenes Leben. Selbst im Tod, als ihr Kopf vom Körper getrennt ist, schafft es Moro noch Eboshi den Arm ab zu beissen. Im Gegensatz zu anderen Akteuren in der Geschichte (insbesondere den Wildschweinen) jedoch macht der Hass sie nicht blind. Und so stellt sie sich auch gegen den aufgebrachten Keilergott Okkoto um San zu retten.

\begin{quote}
\glqq Humans who attacked the forest threw a baby to me in order to escape my fangs. That was San...! She can't be human, neither can she fully be a wolf. She's my poor, ugly, loveable daughter.\grqq \todo{http://home.comcast.net/~rocksunner/mono3e.html\#moro}
\end{quote}

Ein anderes Bild von den Tiergöttern bekommen wir durch die Keiler. Der Dämon, den Ashitaka am Anfang des Filmes bezwingt um sein Dorf zu schützen, war einst ein mächtiger Keilergott. Auch er stammt aus dem Shishigami-Wald. Als er zu den Emishi kommt, hat der Schmerz und der Hass ihn bereits so wütend gemacht, dass er zu einem Dämon [Tatarigame]\todo{Begriffserklärung} wurde. Ashitaka bittet den rasenden Gott Umkehr zu machen und sein Dorf zu verschonen. Erst als sich Ashitaka zwischen seinem Dorf und dem Dämon entscheiden muss, tötet er ihn. Die Dorfseherin verrichtet gleich darauf ein Versöhnungsritual, in dessen Verlauf sie dem Dämon verspricht einen Schrein zu errichten und indem sie ihn bittet er möge nicht länger hassen. Doch eine Stimme erklingt aus dem Toten Keiler und verflucht alle Menschen (\glqq  Loathsome humans! You will know my wrath well for causing me pain\dots \grqq \todo{http://home.comcast.net/~rocksunner/mono\_e.html\#ashitaka}) 

Im späteren Verlauf der Geschichte begegnet Ashitaka dem mächtigen Gott Okkoto, dem Herr aller Keiler. Okkoto ist weiss wie Moro und ihre Söhne. Graue Stellen in seinem borstigem Fell und seine trüben Augen weisen jedoch darauf hin, dass er sehr alt ist. In seiner Grösse überragt der Keilergott sogar Moro. So, wie sie zwei Schwänze hat, verfügt er über eine zweite Reihe von mächtigen Hauern.  

Von Ashitaka erfährt Okkoto von Nagos Schicksal. Er zeigt sich beschämt, dass einer aus seinem Klan ein so übles Schicksal wiederfahren ist. Er zeigt jedoch weder Mitleid noch Vergeben gegenüber Ashitaka und teilt ihm mit, dass er ihn bei der nächsten Begegnung umbringen werde. Obwohl er einsichtiger als die andern Keiler zu sein scheint, ist Okkoto so sturr, dass von seiner Weisheit am Ende nichts übrig bleibt. Von den Menschen durch Waldbrände und Lärm provoziert, rennen die Wildschweine unter Okkotos führung geradewegs in ein Massaker. Moro durchschaut die Absichten der Menschen und auch Okkoto vermutet was dahinter steckt, trotzdem setzt er alles auf eine letzte Schlacht.  

\begin{quote}
\glqq Moro, look at my clan! Little by little they are becoming smaller and more stupid. If this keeps up, humans will be able to hunt us down like common meat\dots\grqq \todo{http://home.comcast.net/~rocksunner/miya\_e.html} 
\end{quote}

Auch Moros Söhne sind nicht so prächtig und gross wie sie selbst. Zwar sind sie immer noch grösser als normale Wölfe, und weiss, es fehlen aber ein doppelter Schwanz und sonstige Merkmale, die sie von normalen Wölfen unterscheidet. Aus Moros Handeln und dem was sie sagt geht hervor, dass sie keine Hoffnung für die Zukunft des Waldes und der darin lebenden Götter sieht. 

Okkoto kehrt später schwer verletzt zurück. Menschen, versteckt unter den Fellen der gefallenen Keiler, umringen ihn und verletzten ihn weiter. Dem zuwider glaubt Okkoto im Wahn seine Armee sei von den Toten auferstanden. Obgleich er bereits dabei ist zu einem Dämon zu werden, führt er die Jäger noch zum Zentrum des Waldes, zum Teich, an welchem der Shishigami erscheint.  

\subsubsection*{Der Wald: Shishigami und die Kodama} 
Als Ashitaka das erste Mal den Wald des Shishigami betritt, begegnen ihm die Kodama, kleine geisterhafte Geschöpfe, denen ein kindliches Gemüt inne wohnt. Sie sind eigentlich ein Zeichen dafür, dass der Wald gesund ist. Sie sind sozusagen seine weissen Blutkörperchen. In diesem Sinne weisen sie ihm auch den weg zum Herz des Waldes. Die verletzten Schmiedebewohner, welche Ashitaka aus dem Fluss gezogen hat und die er zu ihrem Dorf bringen möchte, fürchten sich vor den kleinen Wesen. Als Ashitaka jedoch sieht, dass sein Reittier auch in der Anwesenheit der Kodama ruhig bleibt, sieht er in ihnen keine direkte Gefahr. Nichtsdestotrotz schliesst er die Möglichkeit nicht aus, dass sie ihn in die Irre führen könnten. 

In der Mitte des Waldes findet sich ein seichter See. Hier wohnt der Waldgott Shishigami. Er hat die Erscheinung eines mächtigen Hirsches. An der Stelle eines Tierkopfs hat er aber menschliches Gesicht. Wenn die Sonne sich senkt und es Nacht wird kommt der Shishigami zur Lichtung beim See und verwandelt sich in den Nachtwandler. In einer ansatzweise humanoiden Form wächst er in der Grösse über den Wald hinaus und schreitet substanzlos durch den Wald. Mit den ersten Sonnenstrahlen kehrt er zur Lichtung zurück und verwandelt sich wieder in seine Hirschform. 

Der Shishigami ist der Herr dieses Waldes, als solcher kann er Leben geben oder Leben nehmen. Besonders deutlich wird das bei einer Nahaufnahme, als er über die Lichtung schreitet. Mit jedem Auftreten wachsen Blumen und Pflanzen, es wuchert regelrecht. Doch in dem Moment, wo sich der Fuss wieder hebt verwelkt alles und zurück bleibt ein kleiner Fleck toter Erde. Nicht nur die Tiere und Götter des Waldes wissen von den seltsamen Fähigkeiten von Shishigami. Der Kaiser beauftragt Eboshi den Kopf des Shishigami zu bringen, da er glaubt, der Kopf könne jede Wunde heilen. In der Tat rettet Shishigami Ashitaka das Leben, in dem er die Schusswunde heilt, die Ashitaka abbekommen hat, als er versuchte San von den Schmiedebewohnern zu beschützen. Doch zu Ashitakas Leid erkennt er, dass der Shishigami den Fluch des wütenden Keilerdämons nicht entfernt hat, und dass ihm somit immer noch ein schmerzlicher Tod bevorsteht. Shishigami ist kein allgültiger und alles heilender Gott. Wer ihn aufsucht, kann auf Heilung hoffen, muss aber auch den Tod erwarten. Moro nennt den Tatarigame feige, dass er sich dem Shishigami nicht gestellt hat. Der Waldgott hätte ihn heilen können, und wenn nicht, dann hätte er ihn getötet und Nago hätte nicht zu dem werden müssen, was er am Ende war. Es scheint jedoch, dass Nagos Furcht vor dem Shishigami berechtig gewesen war. Okkoto stirbt unter Shishigamis Berührung. 

Wo die Tiergötter gegen die Menschen und ihre Zerstörung des Waldes kämpfen, scheint der Waldgott selbst geradezu gleichgültig. Die Wildschweine warten bis am Ende darauf, dass Shishigami ihnen hilft die Menschen zu verjagen. Doch der Shishigami tut nichts dergleichen. Selbst als ihn Eboshi anschiesst, passiert nicht mehr, als dass er einerseits für einen kurzen Moment im Wasser, über welches er sonst läuft, einsinkt, bevor er seinen Gang fortsetzt. Zweitens lässt er aus dem Gewehr, mit welchem Eboshi auf ihn zielt, Pflanzen wachsen. Einmal mehr gibt er ein Bild von Leben und Tod. Als Eboshi es endlich schafft dem Waldgott den Kopf vom Rumpf zu schiessen, quillt das Innere des Waldgottes aus ihm heraus und zerstört alles auf seinem Weg. Im gleichen Moment fallen auch die Kodama von den Bäumen: der Wald stirbt. Im kopflosen Zustand bringt der Shishigami nur Zerstörung und Tod hervor. Erst als San und Ashitaka es fertig bringen den Kopf zurück zugeben, findet die Vernichtung ein Ende und in einer mächtigen Erschütterung wird aus der Zerstörrung neues Leben geschaffen. 
%!TEX root = Animismus_in_Anime.tex
Dieses Kapitel stützt sich im Wesentlichen auf das Buch \emph{Die Filme von Hayao Miyazaki} von Julia Nieder.\footnote{Nieder, Julia. (2006). \emph{Die Filme von Hayao Miyazaki.}	Marburg: Schüren-Verlag.}

\subsection{Japanische Animationsfilme}
\subsubsection{Geschichte}
Die Bezeichnung \emph{Anime} für japanische Animationsfilme ist eine Fremdbezeichnung, welche sich erst nach dem Zweiten Weltkrieg durch die amerikanische Übernahme, gegen \emph{d\={o}ga}, den japanischen Begriff für Animation, durchgesetzt hat. Die Symbiose zwischen Anime und Manga (Comic) ist in Japan eine Besondere. Es kommt oft vor, dass ein erfolgreicher Manga verfilmt wird, oder aber dass zu einem bestehenden Anime ein Manga gezeichnet wird. 
Anders als im europäischen und amerikanischen Raum zielen Manga und Anime auf ein viel breiteres Publikum ab. Zwar sind viele Geschichten für Kinder gedacht, aber auch Erwachsene werden als Zielgruppe ernst genommen. Daher findet sich in den japanischen Animationsfilmen in der Regel eine grössere inhaltliche Palette, als etwa in amerikanischen Produktionen. Weil das Spektrum des Zielpublikums so gross ist, gibt es Produktionen in allen Genres. Nebst den Geschichten welche typischerweise japanische Märchen, Mythen und Legenden oder Science Fiction thematisieren, finden auch Horror, Historie, Romanzen, Komödien und Erotik ihren Platz. Beim Inhalt zeigt sich, dass, obwohl häufig Elemente aus der japa\-nischen Kultur eine zentrale Rolle spielen, nicht davor zurückgescheut wird auch westliche Elemente zu integrieren. Dennoch sind japanische Animationsfilme primär für ein japa\-nisches Publikum gedacht. 

Gerade nach der Kriegszeit übt die amerikanische Filmindustrie einen grossen Einfluss aus. Einerseits wird versucht dem erfolgreichen Beispiel zu folgen, andrer\-seits besteht der Drang sich davon abzugrenzen und sich auf die eigene Kultur zu konzentrieren. Dem US-Beispiel folgend entstehen in den 1950er Jahren etliche Animiationsstudios. Anders als die amerikanischen Studios wie Walt Disney's setzt die japanische Filmindustrie mehr auf Quantität als auf Qualität. Das führt dazu, dass typische Anime in der Regel einfacher, weniger hyperrealistisch gestaltet sind, als die amerikanischen. Neben der Machart unterscheiden sich die japanischen Animationen von amerikanischen oder auch europäischen durch ihren kulturellen Hintergrund. 

\subsubsection{Eigenschaften der japanischen Animationsfilme}
Im Shinto finden sich einige der Erklärungen für Eigenarten der japanischen Animationsfilme. Der traditionelle Shinto ist eine unorganisierte, schriftlose Religion. Im Shinto wird das Universum prinzipiell als mehrdeutig betrachtet. Das macht sie ideal dafür, verschiedene (auch widersprüchliche) Konzepte in sich aufzunehmen. So wird zum Beispiel der Buddhismus ebenso in den Shinto integriert, wie auch Elemente des Christentums. Auf diese Weise fanden sie ihren Platz im religiösen Gesamtverständnis der Japaner. Des Weiteren wird nicht scharf zwischen Götter und Dämonen oder zwischen Gut und Böse unterschieden. Nach dem Kodex der Samurai, der japanischen Krieger, \mbox{steht} die Absicht über der Handlung. Als Folge dessen fehlt es in den japanischen Geschichten in der Regel auch an der dualistischen Unterscheidung von Gut und Böse. Der Protagonist und der Antagonist stehen sich vielmehr in einem Interessenskonflikt gegenüber und somit stehen Motive der Charakteren im Mittelpunkt und weniger ihre Handlungen. Ein Bildwechsel, von einer Person zur anderen, wird daher nicht unbedingt benutzt um den zeitlichen Verlauf zu markieren, sondern um einen Perspektivenwechsel zu ermöglichen. Für westliche Zuschauer gibt das den Eindruck einer Verlangsamung der Handlung. Dieser Fokus, zusammen mit der Eigenschaft, dass auch japanische Serien in der Regel abgeschlossen sind, geben den Charakteren der Geschichte die Möglichkeit, dramatische Veränderungen zu durchlaufen. In amerikanischen Produktionen fallen die Charakteren vergleichsweise flach und statisch mit wenig Möglichkeit zur Entwicklung aufgrund der episodenhaften Art aus. 

Das ästhetische Prinzip von Wabi und Sabi\footnote{Das japanische Konzept ist eng mit dem Zen-Buddhismus verbunden. Im Zentrum steht die Art und Weise wie Dinge wahrgenommen werden. Dazu gehört die Betrachtung scheinbar unbedeutender Details, als auch die Wahrnehmung dessen, was nicht da ist.} ist ebenfalls in den japanischen Filmen zu beobachten. Auslassung ist genau so Teil eines Kunstwerks wie seine übrigen Bestandteile. Damit begründet sich das hohe Mass an Abstraktion zum Beispiel beim Charakterdesign.
 
Eine wichtige Rolle spielen ausserdem symbolische Darstellungen. Die Grundlage der Interpretation japanischer Filme muss die japanische Kultur bilden. Während im typischen Walt-Disney-Märchen die Prinzessin blondes Haar trägt, ist die typische Haarfarbe für einen guten Charakter in japanischen Geschichten dunkel.~\footnote{Im Anime \textsc{Das wandelnde Schloss}, welcher später genauer betrachtet werden soll, trägt der Protagonist Hauro zunächst blonde Haare. Doch als er endlich zu sich selbst und seinem Mut findet, sind seine Haare schwarz.}

Die Bezugnahme auf die japanische Kultur kann insofern umgangen werden, als dass Pascal Boyer mit seinem kognitionswissenschaftlichen Ansatz eine kulturübergreifende Herangehensweise ermöglicht.

\subsection{Hayao Miyazaki}
Hayao Miyazaki, einer der bekanntesten Produzenten animierter Filme in Japan, wird während dem 2. Weltkrieg im Januar 1941 unweit von Tokyo geboren. Sein Vater arbeitet, bei seinem Bruder, im Maschinenbauunternehmen Miyazaki Airplanes.\footnote{Dieser Hintergrund wird gerne gebraucht um Miyazakis Faszination vom Fliegen zu begründen.} 

Miyazakis Interesse an der Animiation wird schon sehr früh durch den Animationsfilm \textsc{Panda and the Magic Serpent} geweckt, welcher in seinen Jugendjahren veröffentlich wird. Obwohl Miyazaki zunächst Politik und Wirtschaft studiert, zieht es ihn nach Abschluss seines Diploms ins Animationsgeschäft. Als Hintergrundzeicher findet er bei dem seinerzeit führenden Studio Toei-Animation eine Anstellung und macht sich binnen Kürze einen Namen. Nebenbei veröffentlicht er unter einem Pseudonym eine eigene Manga-Serie und sammelt wo immer möglich Erfahrung in Storyentwicklung und Produktion. Um Skizzenstudien der Landschaften und Städte zu machen, wird er wie alle Ani\-matoren von Toei-Animation auf Reisen geschickt. So reist Miyazaki für die Produktion von \textsc{Alpine Girl Heidi} nach Europa.

Sein Freund Isao Takahata (*1935) verfilmt 1972 Miyazakis erste Kurz\-geschichte \textsc{Adventures of Panda and Friends} und 1978 übernimmt Miyazaki erstmals die Inszenierung einer Anime-Serie. Im darauf folgenden Jahr führt er zum ersten Mal Filmregie. Von da an bekommt Miyazaki immer öfters die Leitung für die Inszenierungen von Anime-Serien. 1982 beginnt Miyazaki mit dem Manga \textsc{Nausicaä aus dem Tal der Winde}. Die einzelnen Teile erscheinen mit zahlreichen Unterbrechungen in einem monatlich erscheinenden Magazin. Erst 1994 findet der Manga einen Abschluss. Bereits 1983 beginnen die Vorarbeiten für eine Verfilmung der Geschichte unter der Leitung von Miyazaki. Der Film erscheint im darauf folgenden Jahr in den japanischen Kinos. 

1984 macht sich Miyazaki mit ein paar Kollegen von Toei-Animation selbständig und gründet das Animations Studio Ghibli. Bei der Position als Regisseur und als Produzent wechseln sich Miyazaki und Takahata ab. Das Studio ist zunächst finanziell nicht abgesichert und riskiert mit jeder Produktion seinen Ruin. Etwas mehr Sicherheit gewinnt das Ghibli Studio nach der Veröffentlichung des Filmes \textsc{My Neighbor Totoro}, da insbesondere das Merchandising (bis auf den heutigen Tag) grossen Erfolg verbuchen kann. Mit den weiteren Produktionen steigt die Firma Ghibli zu den erfolgreichsten und be\-kanntesten in Japan auf. 

Mit der Fertigstellung von \textsc{Prinzessin Mononoke} will sich Miyazaki 1997 eigentlich in den Ruhestand setzen. Dennoch beginnt er mit den Vorberei\-tungen für \textsc{Chihiros Reise ins Zauberland}, welcher 2001 in die japanischen Kinos kommt. Während der Film nicht nur nationale sondern auch internationale Preise gewinnt\footnote{Darunter ein Oscar in der Kategorie: Bester Animationsfilm.}, arbeitet Miyazaki auch schon am nächsten Projekt: \textsc{Das wandelnde Schloss}. Dieses wird 2004 in Japan veröffentlicht und beschert erneut einen Einnahmerekord in den japanischen Kinos. International hingegen erlangt es nicht mehr ganz so viel Aufmerksamkeit wie die beiden Vorgängerfilme. 

Mit \textsc{Wie der Wind sich hebt} kündet Miyazaki erneut seinen Rücktritt an. An einer Abendkonferenz in September 2013 erklärt er, dass er keine abend\-füllenden Anime-Filme mehr machen werde.\footnote{Knüsel, Jan. (2015). \emph{Asienspiegel. News aus Japan.}\\ \url{http://asienspiegel.ch/2014/08/grosse-ehre-fur-hayao-miyazaki/} 10. September 2015.} Die Zeit der traditionellen Animation, für die man noch von Hand zeichne, sei vorbei.\footnote{Mit der Produktion von \textsc{Prinzessin Mononoke} werden erstmals einzelne Elemente der Produktion digital umgesetzt. Dabei wird jedoch streng kontrolliert, dass die Effekte nicht die feinen, handgezeichneten Bilder stören.} Überraschenderweise wagt der 74 jährige Japaner nun doch den Sprung ins neue Zeitalter der Animation. In seinem neusten Projekt arbeitet er an einem 10-minütigen Kurz-Anime, der auf der Kurzgeschichte \textsc{Boro, die Raupe} basiert. Erstmals will er einen vollständig computer animierten Anime machen.\footnote{Knüsel, Jan. (2015). \emph{Asienspiegel. News aus Japan.}\\ \url{http://asienspiegel.ch/2015/07/miyazaki-arbeitet-an-kurzanime} 10. September 2015.}
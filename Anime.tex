%!TEX root = Animismus_in_Anime.tex
\newpage
\subsection{Hayao Miyazaki}
Hayao Miyazaki, einer der bekanntesten Animations Produzent Japans, wurde während dem 2. Weltkrieg im Januar 1941 unweit von Tokyo geboren. Sein Vater arbeitete für seinen Bruder in der Maschinenbau Firma Miyazaki Airplanes.~\footnote{Dieser Hintergrund wird gerne gebraucht um Miyazakis Faszination vom Fliegen zu begründen.} 

Miyazakis Interesse an der Animiation wurde sehr früh schon durch den Animationsfilm \textsc{Panda and the Magic Serpent} geweckt, welcher in seinen Jugendjahren veröffentlich wurde. Obwohl er zunächst Politik und Wirtschaft an einer renommierten Universität studierte, zog es ihn nach Abschluss seines Diplomes ins Animationsgeschäft. Als Hintergrundzeicher fand er bei der derzeit führendem Studio Toei-Animation eine Anstellung und machte sich schnell einen Namen. Nebenbei veröffentlichte er unter einem Pseudonym er eine eigene Manga-Serie\todo{welche?} und sammelte wo immer möglich Erfahrung in Storyentwicklung und Produktion. Toei-Animation schickte seine Animatoren gelegentlich auf Reisen um um Skizzenstudien der Landschaften oder Städte zu machen.~\footnote{So zum Beispiel reiste Miyazaki für die Produktion von \textsc{Alpine Girl Heidi} nach Europa.} 

Sein Freund Isao Takahata verfilmte 1972 Miyazakis erste Kurzgeschichte (\textsc{Adventures of Panda and Friends}). 1978 übernimmt Miyazaki dann erstmals die Inszinierung einer Anime-Serie (\textsc{Boy Conan}) und im darauf folgenden Jahr führte er zum ersten Mal Filmregie (\textsc{Schloss des Caliostro}). Von da an bekam Miyazaki immer öfters die Leitung für die Inszenierungen von Anime-Serien.

1982 begann Miyazaki mit dem Manga \textsc{Nausicaä aus dem Tal der Winde}. Die einzelnen Teile erschienen mit zahlreichen Unterbrechungen in einem monatlich erscheinendem Magazin. Erst 1994 fand der Manga einen Abschluss. Doch bereits 1983 begannen die Vorarbeiten für eine Verfilmung der Geschichte unter der Leitung von Miyazaki. Der Film erschien im derauf folgendem Jahr in den japanischen Kinos. 

Miyazaki machte sich im Jahr darauf mit ein paar Kollegen von Toei-Animation selbständig und gründete das Animations Studio Ghibli\footnote{Warmer Wüstenwind}. Bei der Position als Regisseur und als Produzent wechseln sich Miyazaki und Takahata ab. Das Studio war finanziell nicht abgesichert und riskierte zunächst mit jeder Produktion seinen Ruin. Etwas mehr Sicherheit gewann das Ghibli Studio nach der Veröffentlichung des Filmes \textsc{My Neighbor Totoro}, da insbesondere der Merchandising ein grosser Erfolg (auch heute noch) verbuchen kann. Mit den weiteren Produktionen stieg die Firma Ghibli zu den erfolgreichsten und bekanntesten in Japan auf.

Miyazaki wollte sich, mit der Fertigstellung von \textsc{Prinzessin Mononoke} in 1997 eigentlich in den Ruhestand setzen,\footnote{Zitat von Miyazaki: +/- zu anstrengend} begann aber dann mit den Vorbereitungen für \textsc{Chihiros Reise ins Zauberland} welcher 2001 in die japanischen Kinos kam. Während der Film nicht nur nationale sonder auch internationale Preise gewann\footnote{Darunter Oscar in der Kategorie Bester Animationsfilm}, arbeitete Miyazaki auch schon am nächsten Projekt. \textsc{Das wandelnde Schloss} welches 2004 in Japan veröffentlicht wurde, bescherte erneut einen Einnahmerekord dar in den japanischen Kinos, erlangte jedoch international nicht mehr ganz so viel Aufmerksamkeit wie die beiden Vorgänger Filme. 

Mit \textsc{Wie der Wind sich hebt} kündete Miyazaki erneut seinen Rücktritt an. An einer Abendkonferenz in September 2013 erklärte Miyazaki, dass er keine abendfüllenden Anime-Filme mehr machen werde.\footnote{\texttt{http://asienspiegel.ch/2014/08/grosse-ehre-fur-hayao-miyazaki/}} Die Zeit der traditionellen Animation, wo man noch mit Hand zeichne sei vorbei. Auch bei seinen späteren Filmen wurden Computergrafiken nur vereinzelnd eingesetzt. Überraschenderweise wagt der 74 jährige Japaner doch den Sprung ins neue Zeitalter der Animation. In seinem neusten Projekt arbeitet er an einem 10 minütigen Kurzanime, welcher auf der Kurzgeschichte \textsc{Boro, die Raupe}. Erstmal will er einen vollständig computer animierten Anime machen.\footnote{\texttt{http://asienspiegel.ch/2015/07/miyazaki-arbeitet-an-kurzanime/}}

Natur und so?\todo{wo und was soll noch dazu kommen?}

\subsection{Japanische Animationsfilme}
\subsubsection{Geschichte}
Die Bezeichnung \emph{Anime} für japanische Animationsfilme ist eine Fremdbezeichnung, welche sich erst nach dem Krieg durch die amerikanische Übernahme, gegen d\={o}ga, den japanischen Begriff für Animation, durchgesetzt hat. Die Symbiose zwischen Anime und Manga (Comic) ist spezielle in Japan. Es ist häufig so, dass ein erfolgreicher Manga verfilmt wird, oder aber dass zu einem Anime hinter ein Manga gezeichnet wird. Anders als im europäischen Raum zielen Manga und Anime auch auf ein viel breiteres Publikum ab. Zwar sind viele Geschichten für Kinder gedacht, aber auch Erwachsene werden als Zielpublikum ernst genommen. Daher finden wir in den japanischen Animationsfilmen in der Regel eine grössere Inhaltliche Palette als in den amerikanischen Produktionen. Durch das weite Spektrum des Zielpublikums finden sich auch Produktionen an allen Genres. Nebst den Geschichten welche typischerweise japanische Märchen, Mythen und Legenden oder Science Fiction thematisieren, finden sich auch Horror, Historie, Romanzen, Komödien und Erotik ihren Platz. Beim Inhalt zeigt sich, dass obwohl häufig Elemente aus der Japanischen Kultur eine zentrale Rolle spielen, nicht davor gescheut wird auch westliche Elemente zu integrieren. Dennoch sind japanische Animationsfilme üblicherweise für ein japanisches Publikum gedacht.

Gerade nach der Kriegszeit übten die amerikanische Filmindustrie einen grossen Einfluss auf. Einerseits wurde versucht dem erfolgreichen Beispiel zu folgen, anderseits bestand auch der Drang sich davon abzugrenzen und sich auf die eigene Kultur zu konzentrieren. Dem US-Beispiel folgend entstanden in den 1950er Jahre etliche Animiations Studios. Anders als die amerikanischen Studios wie Disney setzte die japanische Filmindustrie mehr auf Quantität als auf Qualität. Das führt dazu, dass typische Anime in der Regel einfacher, weniger hyperrealistisch gestaltet sind als die amerikanischen. Neben der Machart unterscheiden sich die japanischen Animationen von amerikanischen oder auch europäischen durch ihren kulturellen Hintergrund. 

Im Shinto finden wir einige der Erklärungen für Eigenart der japanischen Animations Filme. Der traditionelle Shinto ist eine unorganisierte, schriftlose Religion. Im Shinto wird das Universum zudem prinzipiell als mehrdeutig betrachtet. Das macht sie ideal dafür, verschiedene (auch widersprüchliche) Konzepte in sich aufzunehmen. So wurde zum Beispiel der Buddhismus in den Shinto integriert und auch Elemente des Christentums, welches später nach Japan gelangte fanden ihren Platz im religiösen Gesamtverständnis der Japaner. Des weiteren wird nicht scharf zwischen Götter und Dämonen, gut und böse getrennt. Nach dem Kodex der Samurai steht die Absicht auch über der Handlung. Als Folge all dessen fehlt in den japanischen Geschichten in der Regel auch die Unterscheidung von Gut und Böse. Vielmehr steht der Protagonist und der Antagonist sich in einem Interessenskonflikt gegenüber. Die Interessen können sich in ihrer Essenz widersprechen und dennoch glaubhaft sein. 

Im Zentrum stehen die Motive der Charakteren. Ein Bildwechsel wird daher nicht unbedingt benutzt um den zeitlichen Verlauf zu markieren, sondern um einen Perspektivenwechsel zu ermöglichen. Für westliche Zuschauer gibt das den Eindruck einer Verlangsamung der Handlung. Dieser Fokus, zusammen mit der Eigenschaft, dass auch japanische Serien in der Regel abgeschlossen sind, geben den Charaktern der Geschichte die Möglichkeit dramatische Veränderungen zu durch gehen. In amerikanischen Produktionen fallen die Charakteren vergleichsweise flach und statisch mit wenig Möglichkeit zur Entwicklung auf Grund der episodenhaften Art aus.

Das ästhetische Prinzip von Wabi und Sabi\footnote{Definition, oder zumindest Andeutung} ist ebenfalls in den japanischen Filmen zu beobachtet. Auslassung ist genau so Teil eines Kunstwerks wie seine andern Bestandteile. Damit begründet sich auch das hohe Mass an Abstraktion zum Beispiel beim Charakterdesign.  

Eine weitere wichtige Rolle spielen symbolische Darstellungen. Gerade bei einer Analyse und einer damit verbunden Interpretation ist es wichtig sich aber bewusst zu sein, dass die japanische Kultur Grundlage der Interpretation sein muss. Im Gegensatz zu den Walt Disney Märchen wo die Standard Prinzessin blondes Haar trägt, ist die typische Haarfarbe für einen guten Charakter in japanischen Geschichten dunkel.~\footnote{Im Anime \textsc{Das wandelnde Schloss} welches später genauer betrachtet werden soll, trägt der Protagonist Hauro zunächst Blonde Haare. Doch als er endlich zu sich selbst und seinen Mut findet trägt er dunkle Haare.} 
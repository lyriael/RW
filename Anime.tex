%!TEX root = Animismus_in_Anime.tex
\newpage
\section{Ausgewählte Anime}
\subsection{Japanische Animationsfilme}
Die Bezeichnung \emph{Anime} für japanische Animationsfilme ist eine Fremdbezeichnung, welche sich erst nach dem Krieg durch die amerikanische Übernahme, gegen d\={o}ga, den japanischen Begriff für Animation, durchgesetzt hat. Die Symbiose zwischen Anime und Manga (Comic) ist spezielle in Japan. Es ist häuffig so, dass ein erfolgreicher Manga verfilmt wird, oder aber dass zu einem Anime hinter ein Manga gezeichnet wird. Anders als im europäischen Raum zielen Manga und Anime auch auf ein viel breiteres Publikum ab. Zwar sind viele Geschichten für Kinder gedacht, aber auch Erwachsene werden als Zielpublikum ernst genommen. Daher finden wir in den japanischen Animationsfilmen in der Regel eine grössere Inhaltliche Palette als in den amerikanischen Produktionen. Durch das weite Spektrum des Zielpublikums finden sich auch Produktionen an allen Genres. Nebst den Geschichten welche typischerweise japanische Märchen, Mythen und Legenden oder Science Fiction thematisieren, finden sich auch Horror, Historie, Romanzen, Komödien und Erotik ihren Platz. Beim Inhalt zeigt sich, dass obwohl häufig Elemente aus der Japanischen Kultur eine zentrale Rolle spielen, nicht davor gescheut wird auch westliche Elemente zu integrieren. Dennoch sind japanische Animationsfilme üblicherweise für ein japanisches Publikum gedacht, das hat auch damit zu tun, dass sich Japan kulturell in der Vorkriegszeit und auch während dem Krieg von Amerika abgrenzen wollte. 

Gerade nach der Kriegszeit übten die amerikanische Filmindustrie einen grossen Einfluss auf. Dem US-Beispiel folgend enstanden in den 1950er Jahre etliche Animiations Studios. Anders als die amerikanischen Studios wie Disney setzte die japanische Filmindustrie mehr auf Quantität als auf Qualität. Das führt dazu, dass typische Anime in der Regel einfacher, weniger hyperrealistisch gestaltet sind als die amerikanischen. Neben der Machart unterscheiden sich die japanischen Animationen von amerikanischen oder auch europäischen durch ihren kulturellen Hintergrund. Hierfür ein paar Beispiele:

\begin{itemize}
	\item Leere
	\item Richtung
	\item Mehrfach Interpretation
\end{itemize}

Natürlich sind solche Elemente nicht gänzlich unbekannt aus amerikanischen oder europäischen Animationsfilmen (Beispiele??), man findet diese Konzepte aber sehr viel häufiger in japanischen Produktionen. Gerade bei einer Analyse und einer damit verbunden Interpretation ist es wichtig sich dessen bewusst zu sein. Beispielsweise muss man auch vorsichtig bei der Interpretation von Symbolen und Farben sein. Im Gegensatz zu den Walt Disney Märchen wo die Prinzessin idealerweise blondes Haar trägt, ist die typische Haarfarb für einen guten Charakter dunkel.~\footnote{Im Anime \textsc{Das wandelnde Schloss} welches später genauer betrachtet werden soll, trägt der Protagonist Hauro zunächst Blonde Haare. Doch als er endlich zu sich selbst und seinen Mut findet trägt er dunkle Haare.} 

\subsection{Hayao Miyazaki}

Hayao Miyazaki, einer der bekanntesten Animations Produzent Japans, wurde während dem 2. Weltkrieg im Januar 1941 unweit von Tokyo geboren. Sein Vater arbeitete für seinen Bruder in der Maschinenbau Firma Miyazaki Airplanes.~\footnote{Dieser Hintergrund wird gerne gebraucht um Miyazakis Faszination vom Fliegen zu begründen.} Miyazakis Interesse an der Animiation wurde sehr früh schon durch den Animationsfilm \textsc{Panda and the Magic Serpent} geweckt, welcher in seinen Jugendjahren veröffentlich wurde. Obwohl er zunächst Politik und Wirtschaft an einer renommierten Universität studierte, zog es ihn nach Abschluss seines Diplomes doch ins Animationsgeschäft. Als Hintergrundzeicher machte er sich bei der derzeit führendem Studio Toei-Animation schnell einen Namen. Unter einem Pseudonym veröffentlichte er eine eigene Manga-Serie und sammelte nebenbei wo immer möglich Erfahrung in Storyentwicklung und Produktion. Von Toei-Animation wurde er auch öfter auf Reisen geschickt um Skizzenstudien der Landschaften oder Städte zu machen.~\footnote{So zum Beispiel für die Produktion \textsc{Alpine Girl Heidi}.} Mit
\begin{itemize}
	\item Erfolg bei Tohei
	\item Nausicaa
	\item Ghibli
	\item Mononoke
	\item Spirited away
	\item Pansionierung
\end{itemize} 

%%%%%%%%%% HOWLS MOVING CASTLE %%%%%%%%%%%%%%%
\subsection{Das wandelnde Schloss}

\subsubsection{Filmhintergrund}
\textsc{Das wandelnde Schloss} ist Miyazakis Adaption des Buches \glqq Sophie im Schloss des Zauberers \grqq von Diana Wynne Jones. Im Buch spielt die Geschichte zumindest zum Teil in Wales, also hat sich auch Miyazaki für ein europäisches Setting entschieden. So sieht man des öfteren scharfkantige Bergspitzen welche mit Schnee bedeckt sind, raue aber saftig grüne Alpenwiesen und wunderschöne Täler mit glasklaren Bächen und Seen. Als Vorlage dienten unter anderem die europäischen Städte Cardiff, Colmar, Heidelberg und Paris.~\footnote{\textsc{Nieder} 2006: 107.} Obwohl es sich bei dem Film um eine Adaption handelt weicht Miyazaki so stark von der Vorlage ab, das man manche Dinge nur mit Hilfe des Buches zu verstehen glaubt. Die grösste Veränderung betrifft den Charakter der Hexe aus dem Ödland. Im Buch trägt sie eindeutig die Rolle der bösen Antagonistin, während im Film die Bürde des Gegenspielers auf verschiedene Charakter verteilt wird. Dies ist typisch für Miyazaki, denn in seinen Filmen findet keine klare Linie zwischen Gut und Böse. 

\subsubsection{Zusammenfassung}
Der Zauberer Hauro zieht in seinem wandelnden Schloss umher und es wird gemunkelt, dass er die Herzen hübscher Mädchen frisst. Sophie ist unzufrieden mit sich und ihrem Leben als Hutmacherin. Den Laden hat sie von ihrem verstorbenen Vater übernommen und sie sieht sich im Schatten ihrer hübschen Schwester und Mutter stehen. 

Eines Tages eilt ihr ein fremder Schönling von zwei über griffigen Männern zu entkommen. Sophie verliebt sich in den jungen Mann, von dem sie aber vermutet, dass es sich um den Zauberer Hauro handelt. Diese kurze Begegnung reicht bereits auf, dass sie die Aufmerksamkeit der Hexe aus dem Ödland auf sich zieht, welche scheinbar eine offene Rechnung mit Hauro hat. Sophie wird durch einen Fluch in eine 80-jährige Greisin verwandelt. 

Auf der Suche nach etwas, was ihren Fluch brechen kann findet sich Sophie bald darauf im wandelnden Schloss des Zauberers Hauro wieder. Sophie heuert kurzerhand als Hausdame und Putzfrau im Schloss an. Sie schliesst einen Handel mit Calcifer, dem Feuerdämon, welcher das Schloss steuert und bewegt: Er verspricht ihren Flucht zu brechen, wenn sie ihn von dem Packt mit Hauro befreit. In der Zwischenzeit ist ein offener Krieg zwischen den Nachbarländern ausgebrochen und Hauro soll auf beiden Seiten mitkämpfen. Erst mit Sophies Hilfe kann Hauro seine Feigheit ablegen und übernimmt Verantwortung. 

Es müssen dann noch viele Abenteuer bestanden werden, bis die beiden erkennen, dass sie sich lieben. Erst dann kann Sophie dem Zauberer sein flammendes Herz wieder zurück in seine Brust drücken und somit den Packt mit Calcifer lösen und ihren eigenen Fluch brechen. 

\subsubsection{Das Schloss und der Feuerdämon}
\subsubsection*{Verlauf}
Das titelgebende Schloss, welches durch den Feuerdämon Calcifer belebt wird, ist trotz seines schäbigen und schrottreifen Aussehen von einer Lebendigkeit erfüllt, welche einen staunen lässt. Wie praktisch alle Charakteren in diesem Film geht auch das Schloss eine Metamorphose durch. Am Anfang ist es eine riesige Maschinerie mit zahllosen Türmen, Röhren, Kammern und Öffnungen deren Sinn und Zweck man nicht einmal erraten kann. Es scheppert und kleppert, peifft und knarrt mit jedem Schritt. Es ist vielmehr ein wandelndes Ungetüm als ein wandelndes Schloss.

In dem Moment, wo Sophie Calcifer aus dem Schloss trägt, verliert das Schloss seine Integrität und sackt in sich zusammen. Wenn Sophie dann wieder mit Calcifer hinein geht und ihn bittet das Schloss wieder mobil zumachen benötigt er etwas von ihr. Sie gibt ihm ihr Zopf. Das gibt Calcifer wieder genügen Energie ein Teil des Schlosses zu beleben. Was dabei aber heraus kommt ist eine viel kleinere und agilere Version. Viel unnützer Balast wurde abgeworfen, aber dafür mangelt es jetzt auch an Komfort und Sicherheit. 

Im weiteren Verlauf der Geschichte erlischt Calcifer nahezu und mit dem Leben, das aus dem Feuerteufel geht, zerfällt auch das wandelnde Schloss. Nach einer Nacht, wo Calcifer zu einer kleine blauen Flamme reduziert wurde, ist alles was vom Schloss noch übrig bleibt eine hölzerne Plattform, getragen von zwei Beinen. 

\subsubsection*{Figurenanalyse}
Das Wesen von Calcifer und dem wandelnden Schloss ist also untrennbar miteinander verschmolzen. Calcifer redet und interagiert mit den anderen Bewohnern des Schlosses. Das macht ihn, wenn man Harveys Ansatz folgt eindeutig zu einer Person. Um es für den Betrachter einfacher zu machen, das Feuer als Person zu sehen bekommt Calcifer zwei grosse Glupschaugen und ein Mund von variabler Grösse, in das er sich gerne Holzstücke reinstopft. Die Augen sind in der Regel rund, weisse Kreise mit schwarzen Pupillen. Doch mit dem flackern seines ganzen Körpers verändert sich die Form der Augen zwischen durch unmerklich, so dass sie schlitzförmiger werden und einen gefährlichen (dämonischen) Eindruck machen. Seine Flammenform, welche keine feste Umrisslinie hat flackert beständig und von Zeit zu Zeit lösen sich kleine Flämmchen von ihm. 

Es stellt sich natürlich die Frage, inwiefern sich aus Calcifers Beseeltheit auch die des Schlosses schliessen lässt, da das Schloss von Calcifer gebaut und gesteuert wird, ist es im Prinzip Spiegel von Calcifer Zustand. Wir wissen, dass Calcifier das Schloss belebt - animiert. Doch wie wird dieser Eindruck, dass eine Konstruktion belebt ist an den Rezeptionist vermittelt?

Das Schloss läuft (am Anfang) auf vier Vogelbeinen welche aus Metall gemacht sind. Ein Rostrot bis -braun dominiert das Konstrukt. Den Rumpf kann man in zwei Hauptteile unterteilen: Oben finden sich Schornsteine, Hausteile, Masten und schwere Kuppeln mit Guckrohren. Der untere Teil sieht aus wie ein Fisch mit Rübennase auf vier stelzigen Vogelbeinen. Ein langer Schlitz, welcher in zwei Gucklöchern endet gibt die Illusion von Augen und Mund. Auf der Hinterseite des unteren Teiles ist eine senkrecht stehende Schwanzflosse. Der Eingang des Schlosses ist auf der Rückseite, wo man bei einem Tier den *arsch* erwarten würde. Die Last der oberen Teile schwankt bei jedem Schritt.
Abgesehen von den äusseren Merkmale welche dem Schloss tierähnlicher machen kommen seine Bewegungen und seine Reaktionen. Durch das individuelle Bewegen und langsame Vergrössern und Verkleinern einzelner Teile gibt dem Schloss etwas organisches. 

Obwohl Calcifer das Schloss gebaut hat, steuert und auch seine Emotionen durch das Schloss geäussert werden, so es für den Rezipienten doch ein eigenes Wesen.

\subsection{Prinzessin Mononoke}
\subsubsection{Filmhintergrund}
\subsubsection*{Veröffentlichung und Erfolg}
Mit \textsc{Prinzessin Mononoke} gelang Hayao Miyazaki erstmals einen internationalen Durchbruch. Der Film kam am 17. Juli 1997 in die japanischen Kinos. Mit 18.65 Millarden Yen (umrechnung?) spielte er in Japan mehr ein als Titanic (James Cameron). Er war der bisher erfolgreichste Film in Japan. Nachdem er zuerst in 1998 auf der 48. Berlinale das erste Mal in Deutschland vorgeführt wurde und \emph{irgendwelche Preise} gewonnen hat, kam er 1999 in den Vereinigten STaaten und Kanada und im Jahr darauf auch in Europa in die Kinos. Trotz den vielen Auszeichnungen welche der Film gewann, war der Film, sowie Ghibli Studio und Hayao Miyazaki vorwiegend unter Anime-Fankreisen bekannt. Es wurde ausserhalb von Japan auch nur wenig Werbung gemacht. Auf internationale Vermarktung wurde lange verzichtet, da Miyazaki und sein Team darüber entsetzt waren, wie starkt geschinitten \textsc{Nausicaä aus dem Tal der Winde} wurde. Nach dem riesigen Erflog an den japanischen Kassen zeigte das US-Studios Disney Interesse und sicherte sich die Verhandlungsposition der japanischen Filmemacher. Im Vertrag über die internationale Vermarktung der Ghibli-Filme welcher bald darauf geschlossen wurde konnte sich Miyazaki und sein Team das Recht sichern, über allfällige Schnittstellen selbst entscheiden zu können. 

\subsubsection*{Geschichtlicher Hintergrund und Miyazaki}
Erste Ideen für \textsc{Prinzessin Mononoke} hatte Miyazaki bereits 1970. Damals diente im als Plotvorlage das Märchen von der Schönen und dem Biest.\footnote{http://home.comcast.net/~rocksunner/miya\_e.html} Wegen dem leichtsinnigen Versprechen ihres Vaters, muss die Tochter des Fürsten ein Waldmonster [mononoke]\footnote{Geist/Monster/Gespenst -> http://nausicaa.net/miyazaki/mh/faq.html\#translation} heiraten. Doch Miyazaki kam nicht weiter mit der Geschichte und schob sie auf. 

\begin{quote} Actually, in the beginning I wanted to do a fantasy rather than a period drama set in Japan. However, when I said "Now let's do it", I didn't have the heart for it.\footnote{http://home.comcast.net/~rocksunner/miya\_e.html} 
\end{quote}

Er entschied sich gegen die anfänglich geplante Fantasy Umwelt und setzte die Geschichte im traditionellen feudalen Japan an. Doch in der Zeit, wo er sich mit den Hintergründen auseinander setzte, änderte sich auch das Herzstück der Geschichte. Miyazaki bekam den Wunsch einen tiefgründigeren, authentischeren Film zu machen. Somit fand er wieder zum Thema zurück, das er schon mit \textsc{Nausicaä aus dem Tal der Winde} im Fokus hatte: Das (konfliktreiche) zusammenleben einerseits von Mensch und Natur und anderseits aber auch von Mensch und Mensch. So blieb am Ende von der Ursprünglichen Geschichte nicht mehr übrig als der Name \emph{Mononoke}.\footnote{Das Märchen von der Prinzessin und dem Biest hat Miyazaki später in Form eines Bilderbuches veröffentlicht.} \todo{Nachweis}

Infos aus:\footnote{https://de.wikipedia.org/wiki/Prinzessin\_Mononoke\#Hintergr.C3.BCnde http://nausicaa.net/miyazaki/mh/filminfo.html http://home.comcast.net/~rocksunner/mono\_e.html}
\subsubsection{Zusammenfassung}
Ashitaka, der letzte Prinz eines in Harmonie mit der Natur lebenden Volks namens Emishi\footnote{Alte Bezeichnung für ein japanisches Urvolk. Nach der Heian-Zeit wurde das Volk Ezo genannt.}, muss seine Heimat verlassen weil er einen Dämon\footnote{Tatarigame = curse god (http://home.comcast.net/~rocksunner/mono\_e.html\#ashitaka)}, welcher das Dorf angegriffen hat tötete und so dessen Fluch auf sich zog. Ashitaka macht sich auf den Wald des Shishigamis\footnote{Bedeutung} zu finden. Von dort kommt nämlich der Dämon, welcher einst ein Keilergott war und erst durch den Schmerz und damit kommendem Hass in zu einem Dämon wurde. 

Bald findet sich Ashitaka zwischen verhärteten Fronten. Es herrscht Krieg zwischen den Waldgöttern und den Menschen welche in einer Erzschmiede arbeiten. Er trifft auf San, welche von den andern Mensch \emph{Prinzessin Mononoke} genannt wird, weil sie von der Wolfgöttin Moro aufgezogen wurde und sich wie ein Tier verhaltet. Ashitaka versucht zwischen San, welche für ihre Familie und ihren Lebensraum kämpft und den Bewohnern der Schmiede, welche den Waldabholzen um Erz zu gewinnen, was ebenfalls ihre Lebensgrundlage ist zu vermitteln. Erschwerend kommt hinzu, dass die Anführerin der Schmiedebewohner, Eboshi systematisch versucht die Waldgötter zu vernichten, um ihre Untertanen so zu schützen. Zudem hat sie dem Kaiser den Kopf des Shishigami versprochen. Sie erhofft sich so den Schutz für ihre Schmiede zu sichern, da die Gefahr besteht, dass sie vom selbigen Kaiser angegriffen werden könnte. 

Obwohl man durch ästhetische Gestaltung einfach erraten kann, bei wem und wo Miyazakis Sympathie liegt, so ist es doch erstaunlich, dass der Geschichte jegliches Schwarz-Weiss denken fehlt. Jeder der kämpft hat seine Gründe. Ein durchgehendes Motiv ist der destruktiver Hass. Somit birgt das Filmende zwar Hoffnung in sich, jedoch bleibt die Problematik bestehen. Ashitaka und San wollen sich zwar weiterhin sehen, jedoch kann das Wolfmädchen den Mensch nicht vergeben und bleibt bei ihren Wolfbrüdern im Wald. Ashitaka, der zwar grossen Respekt vor der Natur und den Geistern zeigt kehrt aber dennoch in die Schmiede zurück um dort mit den anderen Menschen zu leben.  

Die Geschichte spielt in einem feudalen Japan zu einer Zeit, welche der Muromachi-Ära (1392-1573) ähnelt. Historisch wichtige Figuren bleiben im Hintergrund. Ein König/Shogun wird zwar erwähnt, jedoch sehen wir nur die Folgen der Interessentskonflikte der Mächtigen. Der Fokus der Geschichte liegt beim einfachen Menschen, insbesondere bei Aussenseiter der Gesellschaft, welche in einer Zeit der kulturellen Blüte, sozialen Umwälzungen und politischen Unruhe leben. 

\subsubsection{Tiergötter, Waldgott und Waldgeister}
In Japans Altertum angesiedelt passen sich die belebte Natur gut ins Bild ein. In den Städten und Dörfen welche Ashitaka besucht auf seinem Weg zum Shishigami Wald haben die Leute praktisch keinen Kontakt zu den Naturgeistern. Zusehr sind sie mit den politischen Dingen beschäftigt. Als Ashitaka aber endlich im Reich des Shishigami ankommt ist das Verhältnis zwischen Mensch und Natur, oder besser gesagt zwischen Mensch und Waldgöttern ein anderes. Ein friedliches Nebeneinander scheint unmöglich.

\subsubsection*{Die Tiergötter: Moro, Nago und Okkoto}
Moro ist eine Wolfsgötting die zusammen mit ihren Söhnen im Wald des Shishigamis lebt. Mit ihrer Grösse überragt sie alle Menschen. Sie hat eine tiefe Stimme, welche nicht unbedingt weiblich klingt. Moro hasst die Menschen und Ebshi am meisten von allen. Für ihr Ziel, Eboshi zu töten riskiert Moro viel, nicht zu letzt ihr eigenes Leben und lebst im Tod, wo ihr Kopf vom Körper getrennt ist, schafft es Moro noch Eboshi den Arm ab zu beissen. Im Gegensatz zu andern Akteuren in der Geschichte (insbesondere den Wildschweinen) macht der Hass sie nicht blind. Und so stellt sie sich auch gegen den aufgebrachten Keilergott Okkoto um San zu retten.

\begin{quote}
Humans who attacked the forest threw a baby to me in order to escape my fangs. That was San...! She can't be human, neither can she fully be a wolf. She's my poor, ugly, loveable daughter.\footnote{http://home.comcast.net/~rocksunner/mono3e.html\#moro}
\end{quote}

Ein anderes Bild von den Tiergöttern bekommen wir durch die Keiler. Den Dämon, welcher Ashitaka am Anfang des Filmes bezwingt um sein Dorf zu schützen war einst ein mächtiger Keilergott auch aus dem Shishigami Wald. Als er zu den Emishi kommt hat der Schmerz und der Hass ihn bereit so wütend gemacht, dass er zu einem Dämon [Tatarigame] wurde. Ashitaka bittet den rasenden Gott Umkehr zu machen und sein Dorf zu verschonen und erst als er sich zwischen seinem Dorf und dem Dämon entscheiden muss, tötet er ihn. Die Dorfseherin verrichtet gleich darauf ein Versöhnungsritual in dem sie dem Dämon verspricht ein Schrein zu errichten und ihn bittet er möge nicht länger hassen. Doch eine Stimme erklingt aus dem Toten Keiler und verflucht alle Menschen (\glqq  Loathsome humans! You will know my wrath well for causing me pain\dots \grqq \footnote{http://home.comcast.net/~rocksunner/mono\_e.html\#ashitaka})

Im späteren Verlauf der Geschichte begegnet Ashitaka dem mächtigen Keilergott Okkoto. Okkoto ist weiss wie Moro und ihre Söhne, graue Stellen in seinem borstigem Fell weisen jedoch darauf hin, dass er alt ist. So sind auch seine Augen glasig und San stellt fest, dass er blind ist. In seiner Grösse überragt der Keilergott Moro und so wie sie zwei Schwänze hat, verfügt er über eine zweite Reihe von mächtigen Hauern. 

Otokko erfährt durch Ashitaka von Nagos Schicksal und zeigt sich beschämt, dass einer aus seinem Klan ein so übles Schicksal passierte. Doch entgegen seiner anfänglich gezeigten Weisheit sind es die Wildschweine, welche von den Menschen provoziert in ein Massaker rennen. Moro sieht dies voraus, und auch Okkoto vermutet es, jedoch setzt er alles auf eine letzte Schlacht. 

\begin{quote}
Moro, look at my clan! Little by little they are becoming smaller and more stupid. If this keeps up, humans will be able to hunt us down like common meat\dots \footnote{http://home.comcast.net/~rocksunner/miya\_e.html} 
\end{quote}

Moro macht selbst keine Aussage dieser Art, jedoch geht aus ihrem Handeln und dem was sie sagt hervor, dass auch sie keine Hoffnung für die Zukunft hat. Auch ihre Nachkommen, die beiden Wolfswürder von San sind kleiner und schmächtiger als sie, selbst wenn sie noch etwas grösser sind als normale Wölfe und weiss, so haben sie keine äusseren Merkmale mehr, welche sie von normalen Wölfen unterscheidet, im Gegensatz zu Moro welche zwei Schwänze hat.

Okkoto kehrt schwer verletzt zurück, Menschen unter versteckt unter den Fellen der gefallenen Keiler umringen ihn und verletzten ihn weiter, doch Okkoto in seinem Wahn glaub seine Armee sei von den Toten auferstanden und führt sie schilesslich zum Zentrum des Waldes, zum Teich an welchem der Shishigami erscheint.

\subsubsection*{Der Wald: Shishigami und die Kodama}
Als Ashitaka das erste Mal den Wald des Shishigamis betritt, begegnen ihm die Kodama. Kleine geisterhafte Geschöpfe dennen ein kindliches Gemüt inne wohnt weisen ihm den weg zum Herz des Waldes. Sie sind ein Zeichen dafür, dass der Wald gesund ist, erklärt ???. Die verletzten Schmiedebewohner, welche Ashitaka aus dem Fluss gezogen hat und die er zu ihrem Dorf bringen möchte, fürchten sich vor den kleinen Wesen. Als Ashitaka jedoch sieht, dass sein Reittier auch in der Anwesenheit der Kodama ruhig bleibt, sieht er in ihnen keine Direkte gefahr. Trotzdem schliesst er die Möglichkeit nicht aus, dass sie ihn in die Irre führen könnten.

In der Mitte des Waldes findet sich ein seichter See. Hier wohnt der Waldgott Shishigami. Er hat die Erscheinung eines mächtigen Hirschens, jedoch hat er an der Stelle eines Tierkopfs ein menschliches Gesicht. Wenn sie Sonne sich senkt und es Nacht wird kommt der Shishigami zur Lichtung beim See und verwandelt sich in den Nachtwandler. In einer ansatzweise humanoiden Form wächst er in Grösse über den Wald hinaus und schreitet substanzlos durch den Wald. Mit den ersten Sonnenstrahlen kehr er zur Lichtung zurück und verwandelt sich wieder in seine Hirschform. 

Der Shishigami ist der Herr dieses Waldes, als solcher kann er Leben geben oder Leben nehmen. Besonders deutlich wird das bei einer Nahaufnahme, als er über die Lichtung schreitet. Mit jedem Auftreten wachsen Blumen und Pflanzen, es wuchert regelrecht. Doch in dem Moment, wo sich der Fuss wieder hebt verwelkt alles und zurück bleibt ein kleiner Fleck tote Erde. Nicht nur die Tiere und Götter des Waldes wissen von den seltsamen Fähigkeiten von Shishigami. Der Kaiser beauftragt Eboshi den Kopf des Shishigami zu bringen, da er glaub, der Kopf könne jede Wunde heilen. In der Tat rettet Shishigami Ashitaka das Leben, in dem er die Schusswunde heilt, welche Ashitaka bekommen hat, als er versuchte San von den Schmiedebewohnern zu beschützen. Doch zu Ashitakas Leid erkennt er, dass der Shishigami den Fluch des wütenden Keilerdämons nicht entfernt hat und dass ihm somit immer noch ein schmerzlicher Tod bevorsteht. Jedoch ist Shishigami nicht ein guter allesheilender Gott. Wer ihn aufsucht kann auf Heilung hoffen, muss aber auch den Tod erwartet. Moro nennt den Tatarigame feige, dass er sich dem Shishigami nicht gestellt hat. Der Waldgott hätte ihn heilen können, und wenn nicht, dann hätte er ihn getötet und Nago hätte nicht zu dem werden müssen was er am Ende bekam. Es scheint jedoch, dass Nagos Furcht vor dem Shishigami berechtig gewesen war. Okkoto stirbt unter Shishigamis Berührung.

Wo die Tiergötter gegen die Menschen und ihre Zerstörung des Waldes kämpfen, scheint der Waldgott selbst gerade zu gleichgültig. Die Wildschweine warten bis am Ende darauf, dass Shishigami ihnen hilft die Menschen zu verjagen. Doch der Shishigami tut nichts dergleichen. Selbst als ihn Eboshi anschiesst, passiert nicht mehr, als dass er einerseits für einen kurzen Moment im Wasser über welches er sonst läuft einsinkt, bevor er seinen Gang fortsetzt. Zweitens lässt er aus dem Geweher mit welchem Eboshi auf ihn zielt Pflanzen wachsen und gibt somit einmal mehr ein Bild von Leben und Tod. Als Eboshi es endlich schafft dem Waldgott den Kopf vom Rumpf zu schiessen, quillt das Innere des Waldgottes aus ihm heraus und zerstörrt alles auf seinem Weg. Im gleichen Moment fallen auch die Kodama von den Bäumen: der Wald stirbt. Im kopflosen Zustand bringt der Shishigami nur Zerstörrung und Tod. Erst als San und Ashitaka es schaffen den Kopf zurück zugeben, findet die Zerstörung ein Ende und in einer mächtigen Erschütterung wird aus der Zerstörrung neues Leben geschaffen. 
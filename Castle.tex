%!TEX root = Animismus_in_Anime.tex
%%%%%%%%%% HOWLS MOVING CASTLE %%%%%%%%%%%%%%%
\subsection{Filmhintergrund}
\textsc{Das wandelnde Schloss} ist Miyazakis Adaption des Buches \glqq Sophie im Schloss des Zauberers \grqq von Diana Wynne Jones. Im Buch spielt die Geschichte zumindest zum Teil in Wales, also hat sich auch Miyazaki für ein europäisches Setting entschieden. So sieht man des öfteren scharfkantige Bergspitzen welche mit Schnee bedeckt sind, raue aber saftig grüne Alpenwiesen und wunderschöne Täler mit glasklaren Bächen und Seen. Als Vorlage dienten unter anderem die europäischen Städte Cardiff, Colmar, Heidelberg und Paris.~\footnote{\textsc{Nieder} 2006: 107.} Obwohl es sich bei dem Film um eine Adaption handelt weicht Miyazaki so stark von der Vorlage ab, das man manche Dinge nur mit Hilfe des Buches zu verstehen glaubt. Die grösste Veränderung betrifft den Charakter der Hexe aus dem Ödland. Im Buch trägt sie eindeutig die Rolle der bösen Antagonistin, während im Film die Bürde des Gegenspielers auf verschiedene Charakter verteilt wird. Dies ist typisch für Miyazaki, denn in seinen Filmen findet keine klare Linie zwischen Gut und Böse. 

\subsection{Zusammenfassung}
Der Zauberer Hauro zieht in seinem wandelnden Schloss umher und es wird gemunkelt, dass er die Herzen hübscher Mädchen frisst. Sophie ist unzufrieden mit sich und ihrem Leben als Hutmacherin. Den Laden hat sie von ihrem verstorbenen Vater übernommen und sie sieht sich im Schatten ihrer hübschen Schwester und Mutter stehen. 

Eines Tages eilt ihr ein fremder Schönling von zwei über griffigen Männern zu entkommen. Sophie verliebt sich in den jungen Mann, von dem sie aber vermutet, dass es sich um den Zauberer Hauro handelt. Diese kurze Begegnung reicht bereits auf, dass sie die Aufmerksamkeit der Hexe aus dem Ödland auf sich zieht, welche scheinbar eine offene Rechnung mit Hauro hat. Sophie wird durch einen Fluch in eine 80-jährige Greisin verwandelt. 

Auf der Suche nach etwas, was ihren Fluch brechen kann findet sich Sophie bald darauf im wandelnden Schloss des Zauberers Hauro wieder. Sophie heuert kurzerhand als Hausdame und Putzfrau im Schloss an. Sie schliesst einen Handel mit Calcifer, dem Feuerdämon, welcher das Schloss steuert und bewegt: Er verspricht ihren Flucht zu brechen, wenn sie ihn von dem Packt mit Hauro befreit. In der Zwischenzeit ist ein offener Krieg zwischen den Nachbarländern ausgebrochen und Hauro soll auf beiden Seiten mitkämpfen. Erst mit Sophies Hilfe kann Hauro seine Feigheit ablegen und übernimmt Verantwortung. 

Es müssen dann noch viele Abenteuer bestanden werden, bis die beiden erkennen, dass sie sich lieben. Erst dann kann Sophie dem Zauberer sein flammendes Herz wieder zurück in seine Brust drücken und somit den Packt mit Calcifer lösen und ihren eigenen Fluch brechen. 

\subsubsection{Das Schloss und der Feuerdämon}
\subsubsection*{Verlauf}
Das titelgebende Schloss, welches durch den Feuerdämon Calcifer belebt wird, ist trotz seines schäbigen und schrottreifen Aussehen von einer Lebendigkeit erfüllt, welche einen staunen lässt. Wie praktisch alle Charakteren in diesem Film geht auch das Schloss eine Metamorphose durch. Am Anfang ist es eine riesige Maschinerie mit zahllosen Türmen, Röhren, Kammern und Öffnungen deren Sinn und Zweck man nicht einmal erraten kann. Es scheppert und kleppert, peifft und knarrt mit jedem Schritt. Es ist vielmehr ein wandelndes Ungetüm als ein wandelndes Schloss.

In dem Moment, wo Sophie Calcifer aus dem Schloss trägt, verliert das Schloss seine Integrität und sackt in sich zusammen. Wenn Sophie dann wieder mit Calcifer hinein geht und ihn bittet das Schloss wieder mobil zumachen benötigt er etwas von ihr. Sie gibt ihm ihr Zopf. Das gibt Calcifer wieder genügen Energie ein Teil des Schlosses zu beleben. Was dabei aber heraus kommt ist eine viel kleinere und agilere Version. Viel unnützer Balast wurde abgeworfen, aber dafür mangelt es jetzt auch an Komfort und Sicherheit. 

Im weiteren Verlauf der Geschichte erlischt Calcifer nahezu und mit dem Leben, das aus dem Feuerteufel geht, zerfällt auch das wandelnde Schloss. Nach einer Nacht, wo Calcifer zu einer kleine blauen Flamme reduziert wurde, ist alles was vom Schloss noch übrig bleibt eine hölzerne Plattform, getragen von zwei Beinen. 

\subsection*{Figurenanalyse}
Das Wesen von Calcifer und dem wandelnden Schloss ist also untrennbar miteinander verschmolzen. Calcifer redet und interagiert mit den anderen Bewohnern des Schlosses. Das macht ihn, wenn man Harveys Ansatz folgt eindeutig zu einer Person. Um es für den Betrachter einfacher zu machen, das Feuer als Person zu sehen bekommt Calcifer zwei grosse Glupschaugen und ein Mund von variabler Grösse, in das er sich gerne Holzstücke reinstopft. Die Augen sind in der Regel rund, weisse Kreise mit schwarzen Pupillen. Doch mit dem flackern seines ganzen Körpers verändert sich die Form der Augen zwischen durch unmerklich, so dass sie schlitzförmiger werden und einen gefährlichen (dämonischen) Eindruck machen. Seine Flammenform, welche keine feste Umrisslinie hat flackert beständig und von Zeit zu Zeit lösen sich kleine Flämmchen von ihm. 

Es stellt sich natürlich die Frage, inwiefern sich aus Calcifers Beseeltheit auch die des Schlosses schliessen lässt, da das Schloss von Calcifer gebaut und gesteuert wird, ist es im Prinzip Spiegel von Calcifer Zustand. Wir wissen, dass Calcifier das Schloss belebt - animiert. Doch wie wird dieser Eindruck, dass eine Konstruktion belebt ist an den Rezeptionist vermittelt?

Das Schloss läuft (am Anfang) auf vier Vogelbeinen welche aus Metall gemacht sind. Ein Rostrot bis -braun dominiert das Konstrukt. Den Rumpf kann man in zwei Hauptteile unterteilen: Oben finden sich Schornsteine, Hausteile, Masten und schwere Kuppeln mit Guckrohren. Der untere Teil sieht aus wie ein Fisch mit Rübennase auf vier stelzigen Vogelbeinen. Ein langer Schlitz, welcher in zwei Gucklöchern endet gibt die Illusion von Augen und Mund. Auf der Hinterseite des unteren Teiles ist eine senkrecht stehende Schwanzflosse. Der Eingang des Schlosses ist auf der Rückseite, wo man bei einem Tier den *arsch* erwarten würde. Die Last der oberen Teile schwankt bei jedem Schritt.
Abgesehen von den äusseren Merkmale welche dem Schloss tierähnlicher machen kommen seine Bewegungen und seine Reaktionen. Durch das individuelle Bewegen und langsame Vergrössern und Verkleinern einzelner Teile gibt dem Schloss etwas organisches. 

Obwohl Calcifer das Schloss gebaut hat, steuert und auch seine Emotionen durch das Schloss geäussert werden, so es für den Rezipienten doch ein eigenes Wesen.

\subsection{Kognitive Sicht}
Schablone vom Gebäude, Fixe Umgebung welche sich dennoch bewegt. Noch deutlicher wird das beim Rübenkopf, welcher in keiner Weise die Lebendigkeit des Schlosses zeigt (Keine Mimik, nichts bewegliches) aber dennoch in die gleiche Schublage gehört, da er sich zumindest selbständig bewegen kann.
Obwohl ihm vorigen Kapitel mehrheitlich auf das Schloss und nicht auf Calcifer eingegangen wurde, trotzdem noch eine Bemerkung: In der Szene, wo man sieht, wie die Lichter vom Himmel fallen, von denen auch Calcifer eines ist, verpuffen die meisten. Die springen vielleicht noch ein paar wenige Male vom Boden ab, bevor sie endgültig verglühen. Diese Lichter alleine reichen nicht aus, um unsere Aufmerksamkeit auf sich zu ziehen (trotz unheimlich schöner Animation), da ihnen der Bruch ihrer Ontologie fehlt. \emph{Farbige Lichter fallen vom Himmel und springen in Gestalt von Menschen ein paar Mal auf dem Boden, bevor sie verglühen.} Anders verhält sich das mit Calcifer, welcher nachdem ihm von Hauro das Herz gegeben wurde und später durch Sophie ein Leben ohne Herz ermöglicht wurde.
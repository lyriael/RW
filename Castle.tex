%!TEX root = Animismus_in_Anime.tex
%%%%%%%%%% HOWLS MOVING CASTLE %%%%%%%%%%%%%%%
\subsubsection{Filmhintergrund} 
\textsc{Das wandelnde Schloss} ist Miyazakis Adaption des Buches \glqq Sophie im Schloss des Zauberers\grqq~ von Diana Wynne Jones. Im Buch spielt die Geschichte wenigstens zum Teil in Wales. Also hat sich auch Miyazaki für ein europäisches Setting entschieden. So sieht man des Öfteren scharfkantige mit Schnee be\-deckte Bergspitzen, raue, aber saftig grüne Alpenwiesen und wunderschöne Täler mit glasklaren Bächen und Seen. Als Vorlage dienten unter anderem die europäischen Städte: Cardiff, Colmar, Heidelberg und Paris.\footnote{\textsc{Nieder} 2006: 107.} Obwohl es sich bei dem Film um eine Adaption handelt, weicht Miyazaki so stark von der Vorlage ab, das manche Dinge nur mit Hilfe des Buches verständlich zu sein scheinen. 
Die grösste Veränderung betrifft den Charakter der Hexe aus dem Ödland. Im Buch trägt sie eindeutig die Rolle der bösen Antagonistin, während im Film die Bürde des Gegenspielers auf verschiedene Charakter verteilt wird. Dies ist typisch für Miyazaki. In seinen Filmen gibt es keine klare Linie zwischen Gut und Böse. 

\subsubsection{Zusammenfassung} 
Der Zauberer Hauro zieht in seinem wandelnden Schloss umher. Es wird ge\-munkelt, er fresse die Herzen hübscher Mädchen. Sophie, die sich im Schatten ihrer hübschen Schwester und Mutter sieht, ist unzufrieden mit sich und ihrem Leben als Hutmacherin in dem Laden, den sie von ihrem verstorbenen Vater übernommen hat.

Eines Tages eilt ihr ein fremder Schönling zu Hilfe, um zwei übergriffigen Männern zu entkommen. Sophie verliebt sich in den jungen Mann, von dem sie aber vermutet, dass es sich um den Zauberer Hauro handelt. Diese kurze Begegnung reicht bereits aus, um die Aufmerksamkeit der Hexe aus dem Ödland auf sich zu ziehen. Diese hat scheinbar noch eine offene Rechnung mit Hauro. In der Folge wird Sophie durch einen Fluch in eine 80-jährige Greisin verwandelt. 

Auf der Suche nach etwas, was ihren Fluch brechen kann, findet sich Sophie bald darauf im wandelnden Schloss des Zauberers Hauro wieder. Sophie heuert kurzerhand als Hausdame und Putzfrau im Schloss an. Sie schliesst einen Handel mit Calcifer, dem Feuerdämon, welcher das Schloss steuert und bewegt: Er verspricht ihren Fluch zu brechen, wenn sie ihn von dem Packt mit Hauro befreit. In der Zwischenzeit ist ein offener Krieg zwischen den Nachbarländern ausgebrochen. Hauro, in den beiden Ländern unter verschiedenen Namen bekannt, soll auf beiden Seiten mitkämpfen. Erst mit Sophies Hilfe findet Hauro den Mut sich nicht mehr zu verstecken; Er stellt sich seiner Verantwortung. 

Es müssen aber erst noch viele Abenteuer bestanden werden, bis die beiden erkennen, dass sie einander lieben. Erst dann kann Sophie dem Zauberer sein flammendes Herz wieder zurück in seine Brust drücken und zugleich den Packt mit Calcifer lösen und ihren eigenen Fluch brechen. 

\subsubsection{Figurenanalyse} 
Auffallend in diesem Film ist, dass praktisch alle Charakteren eine Metamorphose im Verlauf der Geschichte durchgehen. Aus dem eitlen blonden und gleich\-zeitig feigen Zauberer Hauro wird ein verantwortungsvoller liebender junger Mann. Die Hexe aus dem Ödland verwandelt sich zu einer tattrigen Frau. Eine Vogelscheuche mit einer Rübe als Kopf wird zum Prinzen. Und aus der alten Sophie wird wieder ein junges Mädchen.  

Obwohl viele der Charaktere interessante Untersuchungsgegenstände ergeben würden, beschränken wir uns hier auf das wandelnde Schloss und den Feuerdämon. Die Geschichte zwischen Sophie und dem Zauberer Hauro steht zwar inhaltlich im Zentrum des Filmes, für unser Interesse sind aber das Schloss und der Feuerdämon ergiebiger. Auf die lebendige Vogelscheuche wird nicht weiter eingegangen, da sie nur eine auslassbare Nebenhandlung darstellt. 

\subsubsection*{Calcifer und das wandelnde Schloss} 
Auch das titelgebende Schloss durchläuft mehrere Schritte der Verwandlung. So ist es am Anfang eine riesige Maschinerie mit zahllosen Türmen, Röhren, Kammern und Öffnungen, deren Sinn und Zweck man nicht erraten kann. Es scheppert und klappert, pfeifft und knarrt mit jedem Schritt. Das Schloss läuft auf vier Vogelbeinen, die aus Metall hergestellt sind. Rostrot bis -braun dominiert das Konstrukt. Den Rumpf kann man in zwei Hauptteile unterteilen: Oben finden sich Schornsteine, Hausteile, Masten und schwere Kuppeln mit Guckrohren. Der untere Teil sieht aus wie ein Fisch mit Rübennase auf vier stelzigen Vogelbeinen. Ein langer Schlitz, der in zwei Gucklöchern endet, erzeugt die Illusion von Augen und Mund. Auf der Hinterseite des unteren Teiles ist eine senkrecht stehende Schwanzflosse angebracht. Die Last der oberen Teile schwankt bei jedem Schritt. Es ist vielmehr ein wandelndes Ungetüm als ein wandelndes Schloss. 

In dem Moment, in dem Sophie Calcifer aus dem Schloss trägt, verliert das Gebäude seine Integrität und sackt in sich zusammen. Sophie kehrt wieder mit Calcifer zurück und bittet ihn das Schloss erneut mobil zu machen. Erst nachdem sie ihm ihren Zopf opfert, verfügt Calcifer wieder über genügend Energie, um wenigstens einen Teil des Schlosses zu beleben. Was dabei heraus kommt ist eine viel kleinere und agilere Version des vorigen Baus. Dazu wurde viel unnützer Ballast abgeworfen, doch mangelt es nun auch an Komfort und Sicherheit. 
Im weiteren Verlauf der Geschichte erlischt Calcifer nahezu. Mit dem Leben, welches dem Feuerdämon entweicht, zerfällt auch das wandelnde Schloss. Nach einer Nacht, in der Calcifer zu einer kleine blauen Flamme reduziert wird, ist alles, was vom Schloss noch übrig bleibt eine hölzerne Plattform, getragen von zwei Beinen. 

Das Wesen von Calcifer und dem wandelnden Schloss ist untrennbar miteinander verschmolzen. Calcifer gibt dem Schloss Leben, und das Schloss ist der Körper, in dem Calcifer wohnt. 
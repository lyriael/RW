%!TEX root = Animismus_in_Anime.tex
\section{Animismus in der Moderne}
Das lateinische Wort \emph{anima}\footnote{Das kann man auch anders übersetzten!! wiki sagt: Wind, Hauch} für Seele lässt den Animismus wurde von Stahl erstmals eingeführt und durch Edward Tylor in \emph{Primitive Culture} in 1871 eingeführt. Wie bei vielen Begriffen in der Religionswissenschaft, trägt der Begriff mehrere Bedeutung und kann verschieden verwendet werden. In der Regel wird bei einer animistischen Religion von einer schriftlosen Religion aus. Früher wurden die gerne als Natur-, achaische oder primitive Religionen bezeichnet. In diesem Zusammenhang, aber nicht deckend, versteht man unter Animismus auch den Glauben an eine beseelte Umwelt. Somit ist der Mensch nicht das einzige beseelte Wesen, sonder auch Tiere und Naturobjekte können beseelt sein. Letztlich kann mit Animismus auch einfach der Glaube an Geister und Seelen verstanden werden.~\footnote{RGG 1: 504} 
Religionssoziologie\todo{sollte vielleicht erwähnt werden?}

Es soll an dieser Stelle zunächst ein kurzer Historischer Abriss des Animismus gegeben werden. Danach wird Graham Harveys Animismus (\emph{Respecting the Living World}) als Vertreter eines modernen Animismus vorgestellt. Schliesslich verlassen wir das ausdrückliche Gebiet des Animismus um eine ganz andere Perspektive auf Religion zu haben und wenden uns noch der kognitiven Religionswissenschaft zu. 

\subsection{Der alte Animismus}

Der erste, welcher den Begriff Animismus verwendet hat war Georg Ernst Stahl gewesen. Er stellte die Theorie auf, dass es ein physikalisches Element gäbe, welches belebt. Eine lebendige Person hat demensprechend viel anima, während eine tote Person, oder ein Stein kein anima (mehr) hat. Dabei gibt es eine Abstufung, so dass Tiere und Pflanzen auch anima besitzen, jedoch weniger als der Mensch.~\footnote{\textsc{Harvey 06: 3-4}}

James Frazer (1854-1914) stellt die Theorie auf, dass die Wilden (savage) Pflanzen und Tiere genau so beseelt glaubten wie die Menschen. Der Animismus werde dann zum Polytheismus, wenn dann die Wilden beginnen zu glauben, dass Pflanzen und Tiere nur temporär durch eine andere Wesenheit beseelt seien.~\footnote{\textsc{Harvey 06: 3-4}} 

Edward Tylor (1932-1917) beschreibt in seinem Werk \glqq Primitive Culture\grqq (Die Anfänge der Cultur) den Animismus als der Ursprung der Religion. Der Animismus würde dann, im Laufe der Weiterentwicklung und Zivilisierung einer Kultur durch verschiedene andere Formen der Religion abgelöst werden. Doch auch in einer hochentwickelten und komplexen Religion würden sich noch Überreste der alten Religion in Form von Aberglaube finden.

Robert R. Marett (1866-1943) kritisierte Tylors Religionstheorie welche den Animismus als Ursprungsreligion setzten weil diese Phänomene wie Ehrfurcht vor Tieren, Blut oder Naturgewalten nicht berücksichtigten.Marett führt die Dichotomie vom Alltäglichen und vom Ausseraltäglichen ein, wobei letzteres durch Religion erklärt und verarbeitet würde. Das Ausseralltägliche teilt er weiter in die Begriffe Mana und Tabu. Mana beschreibe die Begegnung mit einer übermenschlichen Macht, während Tabu für Furcht und Kontaktvermeidung aufgrund von Gefahr steht. Er ordnet dann Religion dem Mana an, während er das was mit Tabu verbunden wird als Magie bezeichnet.\footnote{\texttt{https://de.wikipedia.org/wiki/Robert\_Ranulph\_Marett} 25.08.15} 

Emil Durkheim (1858-1917) setzt Anstelle des Animismus den Totmismus als Ursprungsreligion. Seiner Meinung nach ständen soziale Aspekte über den Erfahrungen eines Individuums.

Zusammenfassend kann man sagen, dass der \glqq alte\grqq Animismus als Vorstufe für eine bessere Religion gesehen wurde. In der Geburtsstunde der Religionssoziologie und der Anthropologie war der Westen überzeugt, dass es eine lineare Entwicklung gibt, wobei der Westen ganz oben an dieser Skala steht. Diese Ansicht gilt heute als veraltet und somit haben auch frühere Werke über Animismus ihre Bedeutung in der heutigen Religionswissenschaft eingebüsst. Es gibt aber auch Versuche den Animismus neu zu definieren, - ihm eine Bedeutung in der Moderne zu geben. 

\subsection{Ansätze für einen Modernen Animismus}
\subsubsection*{Moderner Animismus}
Der Animismus steckt heute in so fern in einer Krise, da auf den \glqq alten\grqq Animismus nicht weiter aufgebaut werden kann. Anderseits sind die Phänomene des Animismus weiterhin interkulturell präsent. Die Phänomene sind weiter hin von Ethnologen und Psychologen ernst genommen worden. Als Konsequenz werden sie als kognitiver Fehler\footnote{In diesem Zusammenhang siehe nächstes Kapitel.}, als Projektion, als Produktion einer überproduktiven Phantasie oder einer mangelnden Trennung von subjektiver und objektiver Welt eingeschätzt. 

Auf der anderen Seite wirkt die Moderne zu Gunsten des Animismus. Früher wurden Kulturen belächelt und man bezeichnete sie als primitiv, wenn sie an Naturgeister glauben und diese anbeteten. Hundert Jahre später sehen wir unsere Existenzgrundlage bedroht, weil wir unseren Umgebung rücksichtslos ausgenommen haben. Es ist daher verständlich, dass Haltungen welche die Natur in ein Gegenüber stellt, mit dem man (respektvoll) interagieren kann eine gewisse Sympathie erfährt.

\subsubsection*{Graham Harvey}
\begin{quote}
	Animists are people who recognise that the world is full of persons, only some of whom are human, and that life is always lived in relationship with others. Animism is lived out in various ways that are all about learning to act respectfully towards and among other persons.\cite{harvey06}
\end{quote}

Mit dieser Aussage beginnt Graham Harvey sein Buch \emph{Animism. Respecting the Living World.}\footnote{\textsc{harvey06}}. Es ist hier bereits erkennbar, dass es Harvey in erster Linie darum geht eine Lebenshaltung zu postulieren. Er erklärt anhand vieler Beispiel welche er im Laufe seiner Forschung bei den *** gemacht wie dieser Animismus zu verstehen ist. Aber wichtiger scheint es zu sein, dass sein Anliegen, eine Aufforderung zu mehr Respekt auf dem Gegenüber was wir nicht (er)kennen zu zollen, beim Leser ankommt. So redet er auch weniger vom Animismus, als von Animisten.

Harvey macht ebenfalls eine Unterscheidung vom \glqq alten\grqq und \glqq neuen\grqq Animismus. In der alten Vorstellung ging man davon aus, dass Animisten Menschen sind, welche nicht zwischen Objekten und Subjektion unterscheiden konnten oder wollten. Neue Animisten hingegen suchen Wege und Ansichten wie sie mit andern Leuten richtig und respektvoll interagieren können. Zentral in Harvey Buch ist das Zusammenfassen von Menschen (humans) und Andere-als-Menschen zu einer Übergruppe von Leuten (people). Es gibt also Leute, welche nicht Menschen sind. Es handelt sich aber dennoch um Leute mit denen man interagieren kann (oder muss). Unter den Leuten gibt es hinterlistige und verschlagene Personen (Menschen und Andere-als-Menschen) und es ist wichtig allfällige Masken, Täuschungen und falsche Aussagen zu durchschauen zu können. Im Wissen, dass es Leute gibt, welche uns gerne essen möchten, ist es weise sowohl vorsichtig als auch konstruktiv im respektvollen Umgang mit andern zu sein.

Ich werde hier auf zwei seiner Beispiele eingehen. Das erste handelnd von den Ojibwe, einem Nordamerikanischen Indianerstamm. Seine Überlegungen stützen sich hauptsächlich auf Irving Hallowells Beobachtungen und Untersuchungen wobei die Sprache im Zentrum steht.~\footnote{Zitiere Hallowells Buch} Als zweites soll Harveys ÜBerlegungen zu Maori Kunst dargestellt werden. Nebst seinen persönlichen Erfahrungen\todo{Stimmt das?} benutzt er XX\todo{wer bitte?} als primäre Quelle.

\subsubsection*{Die Sprache der Ojibwa}
Die Ojibwa geben uns ein Beispiel dafür, dass sich Animismus in der Grammatik der Sprache zeigen kann. So wie es im Deutschen (und ähnlichen Sprachen) eine Untescheidung zwischen männlich und weiblich gibt, unterscheidet die ojibwe Grammatik zwischen belebt (animated) und leblos (inanimated). Die Unterscheidung ist aber nicht selbstredend. So wie wir bei uns \glqq die Tasse\grqq oder \glqq der Hund\grqq sagen, ist das nicht immer eine eindeutige Aussage über das Geschlecht des Beschriebenen. In der ojibwe Sprache geschreibt die Grammatik die Steine als animiert. Doch als Antwort auf die Frage ob denn alle Steine leben würde, antwortete ein alter Ojibwe mit: \glqq Nein. Aber ein paar schon.\grqq\footnote{\textsc{Harvey 06: 33}}. Aus diesem Beispiel geht hervor, dass der Animismus hier kein dogmatisches Glaubenssystem ist. Es ist möglich, dass ein Stein animiert ist, jedoch lässt sich diese Aussage nicht auf alle Steine übertragen. Für diese Animisten ist also nicht grundsätzlich alles belebt.

Eine weitere Anekdote erzählt von einem Stein, der durch einem weissen Händler ausgegraben wurden. Der Händler dachte dass er zu einer zeremoniellen Pavillon gehöre, also suchte er einen Ojibwa namens John auf. John beuge sich zum Stein und frage den Stein leise, ob er zu diesem Pavillon gehöre. Laut John antwortete der Stein, dass dem nicht so sei. Das wichtige was wir hier sehen ist, dass mit dem Stein wie mit einer Person umgegangen wurde. John sprach nicht \emph{zu} sondern \emph{mit} dem Stein.\footnote{\textsc{Harvey 06: 37}} 

Es gibt auch Erzählungen bei den Ojibwa von Steinen welche belebt sind und anthropomorphe Merkmale besitzen. Zum Beispiele Steine die so geformt sind, dass es aussieht als ob sie einen Mund, oder Augen haben. Solche Merkmale werden aber nicht zwingend als Hinweis zur Beseeltheit des Steines gelesen. Das Aussehen kann trügen. Ein Stein gilt als animiert, wenn mit ihm gesprochen werden kann. Wenn man mit ihm wie mit andern Personen interagieren kann.

Ein anderes und weitaus abstrakteres Beispiel findet sich bei den Saisongeschichten (Seasonal Stories). Der Umgang mit diesen Geschichten entspricht dem respektvollen Umgang mit einer Person. Tatsächlich werden diese Geschichten Grossvater genannt und sind entsprechend auch ehrwürdig.\footnote{\textsc{Harvey 06: 42}} Man beschäftigt sich nicht leichtfertig mit diesen Geschichten, und auch wenn sie auch lustig sein können, so nimmt man sie doch ernst. Sie vermitteln Dinge grosser Wichtigkeit, wenn man sich ihnen respektvoll annähert.

\subsubsection*{Die Kunst der Maori}
Maori sind für ihre kunstvollen Schnitzereien von Pounamu Steinen\footnote{Sammelnbezeichnung der Maori für Nephrit-Jade und Bowenit. Im Englischen werden diese Steine schlicht \emph{greenstone} genannt.}, Knochen und Holz berühmt. Harvey möchte zeigen, dass diese Kunstwerke selbst (durch den Macher) beseelt sind. 

Die Maori fühlen eine tiefe Verwandtschaft mit dem Ort an dem sie leben. Ein junger Mensch entwickelt sich in Abhängigkeit seiner Familie und seines Clans, aber auch die Natur gehört zu seinen Vorvätern. Das Land wird als Quelle der Identität betrachtet. Es gehört und wird kontinuierlich geteilt von den Toten, den Lebenden und den Ungeborenen.\footnote{\texttt{http://www.justice.govt.nz/publications/publications-archived/2001/he-hinatore-ki-te-ao-maori-a-glimpse-into-the-maori-world/part-1-traditional-maori-concepts/whenua}} Es ist zum Beispiel Brauch, dass bei der Geburt eines Kindes die Plazenta vergraben wird. Somit ist das Neugeborene mit dem Ort verbunden.

Die Maori sehen in der Süsskartoffel nahe Verwandte, ohne deren Hilfe den Maori eine wichtige Nahrungsgrundlage fehlen würde. Ohne die Hilfe der MAori würde die Pflanze jedoch gar nicht erst wachsen und gedeien können. Die Kartoffeln aus zugraben und zu essen grenzt daher an Kannibalismus.\footnote{Kannibalismus ist unter Maori durchaus üblich. Dabei geht es in keiner Weise darum sich vom Menschenfleisch zu ernähren. Die Einverleibung fand von Freunden und Feinden statt.}

Das Schnitzen von Knochen, welche in jedem Menschen vorhanden sind, stellt keinen grösseren Eingriff dar, als das Fällen und Schnitzen von Bäumen und das Schnitzen von Holz. Eine Schnitzerei steht somit immer im Zusammenhang dem Nehmen von Leben. Die Überreste einer Schnitzerei werden jeweils zurückgegeben. Die kunstvolle Schnitzerei ist nicht dafür da, um davon abzulenken. Durch das Schnitzen findet eine Transformation statt, in der der Künstler das Potential das im Holz, Stein oder Knochen schlummert hervor bringt. Ein Pounamu Anhänger ist belebt und nicht einfach nur Schmuck oder Identität für den Träger. Er hat ein Geschlecht, einen Namen und verdient Respekt. 

Ein Wharenui, das Gemeinschaftshaus einer Maori Gruppe, gilt als belebt. Durch die kunstvollen Schnitzereien wird der Vorfahre enthüllt und präsentiert. Das Haus \emph{isst} jene, welche eintreten und transformiert das \emph{tapu}-Neuheit in eine \emph{noa}-Normalität.\footnote{Tapu und Noa?} Der Gast ist somit nicht zum Einheimischen geworden, aber statt der Befindlichkeit findet man eine Normalität, selbst wenn diese nicht alltäglich ist. 

\smallskip
Die beiden aufgezeigten Kulturen in welchen Harvey von Animismus redet zeigen, dass es sehr grosse Unterschiede darin gibt, wie Animisten mit der Welt um sie herum agieren. Dabei decken diese beiden Beispiele nur einen sehr kleinen Teil der Aspekte ab, welche Harvey dem Begriff Animismus zusammenfasst. Weitere Beispiele sollen bei der Filmanalyse an passender Stelle gezeigt werden.

\subsection{Die Sicht der kognitiven Religionswissenschaft}
In dieser Underdisziplin der Religionswissenschaft wird Religion oder religiöse Phänomene aus der Perspektive der Kognitions- und Evolutionswissenschaft betrachtet. Es wird versucht zu erklären, weshalb religiöse Praktiken und und Denkweisen universell verbreitet sind.~\footnote{wikipedia: \texttt{https://en.wikipedia.org/wiki/Cognitive\_science\_of\_religion} und \texttt{https://de.wikipedia.org/wiki/Kognitive\_Religionswissenschaft} 24.08.15}

Die kognitive Religionswissenschaft ist eine eher junge Disziplin, welche sich erst Ende des zwanzigsten Jahrhunderts etabliert hat. Zu ihren Begründern gehören unter anderen E. Thomas Lawson und Robert McCauley (\emph{Rethinking Religion: Connecting Cognition and Culture and Bringing Ritual to Mind: Psychological Foundations of Cultural Forms}), Pascal Boyer (\emph{Naturalness of Religous Ideas}) und Guthrie (\emph{Faces in the cloud}).

An dieser Stelle soll Pascal Boyers \emph{Und Mensch schuf Gott}~\footnote{\cite{boyer04}} als Stellvertreter für andere kognitive Ansätze dienen um die Animationsfilme auf Animismus zu untersuchen. 

Im Geiste des Menschen ist zu suchen. Jeder menschlicher Geist hat das Zeug religiös zu sein. Boyers Theorien stützen sich auf Funktionieren des Geistes im Allgemeinen, unabhänig von Kultur. Das scheint zu nächst im Widerspruch, da es ja gerade kulturell grosse Unterschiede bezüglich religiöser Praktiken und Vorstellungen gibt. Das geniale hier sei, dass sich etwas so \emph{vielschichtiges} wie Religion durch etwas erklären lässt, was überall gleich ist (das Gehirn). Es ist jedoch notwendig zunächst mehr darüber zu wissen, wie das Gehirn Informationen aufnimmt und verarbeitet.[S.11]

Die Arbeit die ein Gehirn leistet ist lange unterschätzt worden. Einerseits muss man von der verbreiteten Annahme wegkommen, dass es sich beim Geist um ein leeres Gefäss handelt welches beliebig mit Informationen (Erziehung, Bildung und persönliche Erlebnissen) gefüllt werden kann. Anderseits auf von der Idee, dass der Geist mit wahllosen Informationen abgefüllt werden kann. Wir können uns bei weitem nicht alles merken und das ist auch gut so. Es braucht also etwas im Hirn, das relevante Informationen aus der Umwelt identifiziert und auf eine spezifische Weise zu verarbeiten mag. [S. 12]

Die relevanten Informationen werden nicht mit den Genen weiter gegeben. Aber das System, welches die Arbeit welche dahinter liegt verrichtet schon. Denn wenn man ein normale menschliches Gehirn besitzt kann daraus noch nicht geschlossen werden, dass dieser Mensch auch Religion hat. Es bedeutet lediglich, dass sich dieser Mensch Religion zu eigen machen kann. Daher ist die wichtige Frage: Wie muss der Nährboden für Informationen aussehen, damit sie als relevant gelten und erfolgreich verarbeitet werden. 

Bei der Frage nach dem Ursprung der Religion tauchen immer wieder ähnliche intuitive Begründungen auf: \emph{Die Religion bietet Erklärungen, Die Religion spendet Trost, Die Religion sichert die gesellschaftliche Ordnung, Die Religion ist eine kognitive Täuschung}~\footnote{\textsc{Boyer 04: 14-15}}. Laut Boyer sind diese intuitiven Gewissheiten in ihrer Existenz zwar berechtigt, jedoch nicht dienlich dabei den Ursprung zu finden. Einen Ursprung im Sinne eines historischen Ereignis ist eine Wunschvorstellung, welche aus dem Wunsch entspringt eine Ursache zu haben, aus der sich alle weiteren Phänomene ableiten lassen würden. Ausführlich zeigt Boyer, wie man in jeder dieser Vorstellung Widersprüche findet oder sie schlicht nicht befriedigend sind, welche sie als Ursprung ungeeignet machen. Doch für jedes Gebiet, das er abarbeitet fügt er am Ende \glqq[einen] andere[n] Blickwinkel\grqq  hinzu, in dem er aus kognitiver Sicht den Wert dieser intuitiven Annahme beschreibt:

\begin{itemize}
	\item Das Erkenntnissystem des Gehirns produziert Erklärungen, oft ohne dass wir uns dessen gewahr sind. - Religion als Erklärung.
	\item Emotionale Programme sind für uns lebenswichtig~\footnote{Hier das Beispiel der Angst vor einem Raubtier. [S.34]} und somit ein Aspekt unseres entwicklungsgeschichtlichen Erbes. - Emotionen in der Religion.
	\item Die Untersuchung des sozialen Bewusstseins (soziale Intelligenz) kann Antworten auf die Frage nach den Erwartungen an das gesellschaftliche Leben und Moral geben. - Religion, Moral und Gesellschaft.
	\item Bei all den Übernatürlichen Informationen welcher der Geist bekommt, werden nur manche als plausibel erachtet und so angeeignet. - Religion und Denken.
\end{itemize}

Boyer ist sehr grosszügig mit einleuchtenden Beispielen, jedoch ist nicht immer ganz klar, wie viel von dem was er sagt wissenschaftlich erwiesen ist, und wie viel davon Spekulation ist. 

Jedenfalls führt er nun weitere Konzepte ein, welche für seinen Ansatz unausschliessbar sind. Er zählt diese als den Inhalt eines Werkzeugkasten auf. Für die weitere Arbeit werden diese Werkzeuge eine zentrale Rolle spielen, da diese die Methode zur Analyse liefern aus Sicht der Kognitiven Religionswissenschaft.

\subsubsection*{Meme}
Die Bezeichnung Meme als Kulturelemente, also Vorstellungen, Werte, Geschichten und dergleiche, welche Menschen in ihrem Handeln beeinflusst und weitergegeben werden,~\footnote{\textsc{Boyer} 04: 50} wurde vom Evolutionsbiologen Richard Dawkins erstmals vorgestellt. Ein Meme bezeichnet demnach einen Bewusstseinsinhalt\footnote{wiki: \texttt{https://de.wikipedia.org/wiki/Mem} 24.08.15}, welcher durch Kommunikation in der Gesellschaft weitergegeben und somit vervielfältigt werden kann. Es ist das soziokulturelle Pendant zu den biologischen Genen in der Evolutions. So dann lassen sich die Meme auch ähnlich wie die Gene beschreiben: Information wird durch Kommunikation weiter gegeben (replizieren). Dadurch werden die Inhalte aber nicht einfach verbreitet, sonder auch leicht (oder schwerwiegend) abgeändert (mutieren).~\footnote{Die Information ändert sich nicht erst durch deren Weitergabe. Etwas das wir erfahren wird bei verschiedenen Menschen bereits anders verarbeitet. Somit können zwei Personen genau das gleiche hören und eine andere Version des Inhalts in ihrem Bewusst sein haben.} Schliesslich werden nur jene Memes tatsächlich weiter gegeben oder überhaupt erst erinnert, welche einprägsam sind (selektieren). Die Frage wäre nun, wo oder was entscheidet welche Memes weitergegeben werden und welche durch das Raster fallen, weil sie nicht relevant sind? 

Boyer beschreibt die Meme als wunderbare Ausgangslage, will ihnen aber nicht mehr als genau das zugestehen. Seine Kritik liegt darauf, dass es keine Replikation der eigentlichen Informationen gibt. Inhalte werden nicht faktisch übergeben, sondern jeweils neu konstruiert. Zwei Menschen können zwei faktisch identische Aussagen machen, aber jeder hat die Information, welche er wiedergibt auf seine eigene Art rekonstruiert. Entsprechend stellt Boyer als nächstes seine Theorie zum einfangen von Vorstellungen durch Schablonen vor. 

\subsubsection*{Schablonen und Vorstellungen}
Ein grosser Teil dessen, was wir wissen musste uns niemand faktisch erzählen. Eine erstaunliche Eigenschaft unserer unserer geistigen Fähigkeit ist es durch die Kombination bereits existierendem Wissen und der Hinzugabe einer neuen Information mehr Wissen zu generieren. Boyer zeigt dies anhand eines Beispiels mit einem Kind auf, mit dem Kind als eine Person, dessen Wissen erweitert wird. Zeigen wir einen Kind zum ersten Mal ein Walross, so hat es keine weiteren Informationen darüber als den Namen und seine äussere Erscheinung. Dennoch wird das Kin erwartet, dass das Walross isst, schläft und dass es Kinder haben kann. Diese Information über das Walross hat das Kind geschlussfogert, indem es eine Annahme gemacht hat: Das Walross ist ein TIER. Tiere essen, schlafen und bekommen Kinder. Boyer bezeichnet TIER als eine \emph{Schablone}. Mit dieser TIER-\emph{Schablone} hat das Kind eine Walross-Vorstellung gebildet.\footnote{\textsc{Boyer 04: 59}}   

Boyer stellt es zur Grundannahme, dass es deutlich weniger Schablonen\footnote{Beispiele die er nennt und auch öfters gebraucht: TIER, WERKZEUG, UNREINE SUBSTANZ, NATUROBJEKT, PERSON, PFLANZE. Bei diesen Schablonen handelt es sich vorwiegend um konkrete Dinge. Boyer erwähnt aber auch GESICHT als eine Schablone im abstrakten Sinne (\emph{das Gesicht verlieren}). \textsc{Boyer 04: 61}} als Vorstellungen gibt und dass diese Schablonen universell sind, im Gegensatz zu den Vorstellungen. 

\subsubsection*{epidemiologisches Modell}
Als weiteres Element in seinem Werkzeugkasten stellt Boyer die Kulturepidemie~\footnote{\textsc{Boyer 04: 62ff}} vor. Religiöse Vorstellungen und Phänomene betreffen in der Regel eine beliebig grosse Gruppe von Menschen.
\todo{Haaaa?}
Man kann also Religion als eine besondere Form der mentalen Epidemie erklären. Durch die Ausbreitung der Epidemie formen Menschen (auf Basis von unterschiedlichen Informationen) ähnlich strukturierte Formen religiöser Vorstellungen und Normen.\footnote{\textsc{Boyer 04: 64}} Hier setzt nun das Konzept der Schablonen und Vorstellungen an. Die Vorstellungen welche von Angehörigen einer gleichen Gruppe anhand einer Schablone hergestellt werden sind sich in der Regel ähnlich. Die Vorstellungen einer anderen Gruppe kann stark davon abweichen, trotzdem die gleiche Schablone benutzt wurde. So lasse sich das Beispiel mit den Tier Vorstellungen auch auf religiöse Vorstellungen übertragen. Demnach gibt es eine Schablone für religiöse Vorstellungen. Wie bei der Tierschablone können religiöse Vorstellungen übereinstimmen (einigermassen ähnlich strukturiert sein), obwohl die Information auf deren sie aufbaut von Mensch zu Mensch verschieden ist. Letztlich muss berücksichtig werden, dass die kulturelle Varianz in der Regel geringer ist, als man annimmt. Beim Übermitteln findet durch die Schablonen ein Filtern der gegebenen Informationen statt, so dass daraus voraussagbare Strukturen gebaut werden.[65]

\subsubsection*{Beschaffenheit des übernatürlichen}
Es folgt also die Suche nach dem mentalen Rezept für religiöse Vorstellungen. Mit einem Versuch verschiedener mehr oder weniger potenten religiösen Aussagen versucht Boyer dem Leser zu zeigen, dass man der Intuition folgend gewisse Aussagen direkt ausschliessen kann, während andere absolut denkbar wären als religiöse Vorstellung. 

Folgende Anleitung kristallisiert sich mit der Zeit heraus: In einer Aussage wird ein Vertreter einer ontologischen Kategorie\footnote{Whats that precious?} gewählt und mit einem Merkmal/Bemerkung behaftet, welche kontraintuitive bezüglich der ontologischen Kategorie ist. Ein Beispiel von Kontraintuitivität ist zum Beispiel die Raupe, welche nach der Metamorphose zu einem Schmetterling wird. Die Erwartung für ein TIER ist, dass es sich im Laufe des Wachstums nur durch grösse und Masse verändert, jedoch nicht, dass es zu einem anderen TIER wird. Als fiktives Beispiel einer religiösen Aussage nennt Boyer hier. 
\begin{quote}Manche Ebenholzbäume behalten Gespräche in Erinnerung, die Menschen in ihrem Schatten geführt haben.\end{quote} 

Hier ist der Ebenholzbaum ein Vertreter der onologischen Kategorie PFLANZE und hat das kontraintuitiv Merkmal eine geistige Präsenz zu haben. Dabei ist es wichtig, dass tatsächlich gegen die ontologische Kategorie verstossen wird und nicht nur einfach eine Merkwürdigkeit. Eine PERSON welche ihre Hautfarbe ändert, ist dem zu Folge weniger erfolgreich, als eine PERSON, welche durch Wände gehen kann. 

Diese Verstösse bilden den Kern der religiösen Aussage. Es ist aber auch üblich, den ontologischen Kategorienverstoss mit weiteren Verstössen auszuschmücken, welche aber nicht mehr kontraintuitiv bezüglich der Ontologischen Kategorie sind.

\begin{itemize}
	\item Nur eine Stufe der Ontologie brechen. (PERSON zu TIER, NATUR OBJEKT zu TIER)
	\item Wichtigkeit der Information. INformation die von der Gruppe von Menschen verwaltet wird. Wichtigkeit der Kommunikation. Wichtigkeit andere Menschen zu verstehen.
\end{itemize}
\subsubsection*{komplexität des Hirns: was damit gemacht wird}

TODO\todo{Fertig machen -> Google Drive: Boyer Methodik}

\subsection{Shinto: Animismus in Japan?}
TODO
Traditioneller Glaube von Japan ist ein Geister, Vorfahre und Naturglaube bla.

%!TEX root = Animismus_in_Anime.tex
Das lateinische Wort \emph{anima}\footnote{Das kann man auch anders übersetzten!! wiki sagt: Wind, Hauch} für Seele lässt den Animismus wurde von Stahl erstmals eingeführt und durch Edward Tylor in \emph{Primitive Culture} in 1871 eingeführt. Wie bei vielen Begriffen in der Religionswissenschaft, trägt der Begriff mehrere Bedeutung und kann verschieden verwendet werden. In der Regel wird bei einer animistischen Religion von einer schriftlosen Religion aus. Früher wurden die gerne als Natur-, achaische oder primitive Religionen bezeichnet. In diesem Zusammenhang, aber nicht deckend, versteht man unter Animismus auch den Glauben an eine beseelte Umwelt. Somit ist der Mensch nicht das einzige beseelte Wesen, sonder auch Tiere und Naturobjekte können beseelt sein. Letztlich kann mit Animismus auch einfach der Glaube an Geister und Seelen verstanden werden.~\footnote{RGG 1: 504} 
Religionssoziologie\todo{sollte vielleicht erwähnt werden?}

Es soll an dieser Stelle zunächst ein kurzer Historischer Abriss des Animismus gegeben werden. Danach wird Graham Harveys Animismus (\emph{Respecting the Living World}) als Vertreter eines modernen Animismus vorgestellt. Schliesslich verlassen wir das ausdrückliche Gebiet des Animismus um eine ganz andere Perspektive auf Religion zu haben und wenden uns noch der kognitiven Religionswissenschaft zu. 

\subsection{Der alte Animismus}

Der erste, welcher den Begriff Animismus verwendet hat war Georg Ernst Stahl gewesen. Er stellte die Theorie auf, dass es ein physikalisches Element gäbe, welches belebt. Eine lebendige Person hat demensprechend viel anima, während eine tote Person, oder ein Stein kein anima (mehr) hat. Dabei gibt es eine Abstufung, so dass Tiere und Pflanzen auch anima besitzen, jedoch weniger als der Mensch.~\footnote{\textsc{Harvey 06: 3-4}}

James Frazer (1854-1914) stellt die Theorie auf, dass die Wilden (savage) Pflanzen und Tiere genau so beseelt glaubten wie die Menschen. Der Animismus werde dann zum Polytheismus, wenn dann die Wilden beginnen zu glauben, dass Pflanzen und Tiere nur temporär durch eine andere Wesenheit beseelt seien.~\footnote{\textsc{Harvey 06: 3-4}} 

Edward Tylor (1932-1917) beschreibt in seinem Werk \glqq Primitive Culture\grqq (Die Anfänge der Cultur) den Animismus als der Ursprung der Religion. Der Animismus würde dann, im Laufe der Weiterentwicklung und Zivilisierung einer Kultur durch verschiedene andere Formen der Religion abgelöst werden. Doch auch in einer hochentwickelten und komplexen Religion würden sich noch Überreste der alten Religion in Form von Aberglaube finden.

Robert R. Marett (1866-1943) kritisierte Tylors Religionstheorie welche den Animismus als Ursprungsreligion setzten weil diese Phänomene wie Ehrfurcht vor Tieren, Blut oder Naturgewalten nicht berücksichtigten.Marett führt die Dichotomie vom Alltäglichen und vom Ausseraltäglichen ein, wobei letzteres durch Religion erklärt und verarbeitet würde. Das Ausseralltägliche teilt er weiter in die Begriffe Mana und Tabu. Mana beschreibe die Begegnung mit einer übermenschlichen Macht, während Tabu für Furcht und Kontaktvermeidung aufgrund von Gefahr steht. Er ordnet dann Religion dem Mana an, während er das was mit Tabu verbunden wird als Magie bezeichnet.\footnote{\texttt{https://de.wikipedia.org/wiki/Robert\_Ranulph\_Marett} 25.08.15} 

Emil Durkheim (1858-1917) setzt Anstelle des Animismus den Totmismus als Ursprungsreligion. Seiner Meinung nach ständen soziale Aspekte über den Erfahrungen eines Individuums.

Zusammenfassend kann man sagen, dass der \glqq alte\grqq Animismus als Vorstufe für eine bessere Religion gesehen wurde. In der Geburtsstunde der Religionssoziologie und der Anthropologie war der Westen überzeugt, dass es eine lineare Entwicklung gibt, wobei der Westen ganz oben an dieser Skala steht. Diese Ansicht gilt heute als veraltet und somit haben auch frühere Werke über Animismus ihre Bedeutung in der heutigen Religionswissenschaft eingebüsst. Es gibt aber auch Versuche den Animismus neu zu definieren, - ihm eine Bedeutung in der Moderne zu geben. 

\subsection{Shinto: Animismus in Japan?}
TODO
Traditioneller Glaube von Japan ist ein Geister, Vorfahre und Naturglaube bla.

\subsection{Harvey Graham: Ansätze für einen Modernen Animismus}
\subsubsection*{Moderner Animismus}
Der Animismus steckt heute in so fern in einer Krise, da auf den \glqq alten\grqq Animismus nicht weiter aufgebaut werden kann. Anderseits sind die Phänomene des Animismus weiterhin interkulturell präsent. Die Phänomene sind weiter hin von Ethnologen und Psychologen ernst genommen worden. Als Konsequenz werden sie als kognitiver Fehler\footnote{In diesem Zusammenhang siehe nächstes Kapitel.}, als Projektion, als Produktion einer überproduktiven Phantasie oder einer mangelnden Trennung von subjektiver und objektiver Welt eingeschätzt. 

Auf der anderen Seite wirkt die Moderne zu Gunsten des Animismus. Früher wurden Kulturen belächelt und man bezeichnete sie als primitiv, wenn sie an Naturgeister glauben und diese anbeteten. Hundert Jahre später sehen wir unsere Existenzgrundlage bedroht, weil wir unseren Umgebung rücksichtslos ausgenommen haben. Es ist daher verständlich, dass Haltungen welche die Natur in ein Gegenüber stellt, mit dem man (respektvoll) interagieren kann eine gewisse Sympathie erfährt.

\subsubsection*{Graham Harvey}
\begin{quote}
	Animists are people who recognise that the world is full of persons, only some of whom are human, and that life is always lived in relationship with others. Animism is lived out in various ways that are all about learning to act respectfully towards and among other persons.\cite{harvey06}
\end{quote}

Mit dieser Aussage beginnt Graham Harvey sein Buch \emph{Animism. Respecting the Living World.}\footnote{\textsc{harvey06}}. Es ist hier bereits erkennbar, dass es Harvey in erster Linie darum geht eine Lebenshaltung zu postulieren. Er erklärt anhand vieler Beispiel welche er im Laufe seiner Forschung bei den *** gemacht wie dieser Animismus zu verstehen ist. Aber wichtiger scheint es zu sein, dass sein Anliegen, eine Aufforderung zu mehr Respekt auf dem Gegenüber was wir nicht (er)kennen zu zollen, beim Leser ankommt. So redet er auch weniger vom Animismus, als von Animisten.

Harvey macht ebenfalls eine Unterscheidung vom \glqq alten\grqq und \glqq neuen\grqq Animismus. In der alten Vorstellung ging man davon aus, dass Animisten Menschen sind, welche nicht zwischen Objekten und Subjektion unterscheiden konnten oder wollten. Neue Animisten hingegen suchen Wege und Ansichten wie sie mit andern Leuten richtig und respektvoll interagieren können. Zentral in Harvey Buch ist das Zusammenfassen von Menschen (humans) und Andere-als-Menschen zu einer Übergruppe von Leuten (people). Es gibt also Leute, welche nicht Menschen sind. Es handelt sich aber dennoch um Leute mit denen man interagieren kann (oder muss). Unter den Leuten gibt es hinterlistige und verschlagene Personen (Menschen und Andere-als-Menschen) und es ist wichtig allfällige Masken, Täuschungen und falsche Aussagen zu durchschauen zu können. Im Wissen, dass es Leute gibt, welche uns gerne essen möchten, ist es weise sowohl vorsichtig als auch konstruktiv im respektvollen Umgang mit andern zu sein.

Ich werde hier auf zwei seiner Beispiele eingehen. Das erste handelnd von den Ojibwe, einem Nordamerikanischen Indianerstamm. Seine Überlegungen stützen sich hauptsächlich auf Irving Hallowells Beobachtungen und Untersuchungen wobei die Sprache im Zentrum steht.~\footnote{Zitiere Hallowells Buch} Als zweites soll Harveys ÜBerlegungen zu Maori Kunst dargestellt werden. Nebst seinen persönlichen Erfahrungen\todo{Stimmt das?} benutzt er XX\todo{wer bitte?} als primäre Quelle.

\subsubsection*{Die Sprache der Ojibwa}
Die Ojibwa geben uns ein Beispiel dafür, dass sich Animismus in der Grammatik der Sprache zeigen kann. So wie es im Deutschen (und ähnlichen Sprachen) eine Untescheidung zwischen männlich und weiblich gibt, unterscheidet die ojibwe Grammatik zwischen belebt (animated) und leblos (inanimated). Die Unterscheidung ist aber nicht selbstredend. So wie wir bei uns \glqq die Tasse\grqq oder \glqq der Hund\grqq sagen, ist das nicht immer eine eindeutige Aussage über das Geschlecht des Beschriebenen. In der ojibwe Sprache geschreibt die Grammatik die Steine als animiert. Doch als Antwort auf die Frage ob denn alle Steine leben würde, antwortete ein alter Ojibwe mit: \glqq Nein. Aber ein paar schon.\grqq\footnote{\textsc{Harvey 06: 33}}. Aus diesem Beispiel geht hervor, dass der Animismus hier kein dogmatisches Glaubenssystem ist. Es ist möglich, dass ein Stein animiert ist, jedoch lässt sich diese Aussage nicht auf alle Steine übertragen. Für diese Animisten ist also nicht grundsätzlich alles belebt.

Eine weitere Anekdote erzählt von einem Stein, der durch einem weissen Händler ausgegraben wurden. Der Händler dachte dass er zu einer zeremoniellen Pavillon gehöre, also suchte er einen Ojibwa namens John auf. John beuge sich zum Stein und frage den Stein leise, ob er zu diesem Pavillon gehöre. Laut John antwortete der Stein, dass dem nicht so sei. Das wichtige was wir hier sehen ist, dass mit dem Stein wie mit einer Person umgegangen wurde. John sprach nicht \emph{zu} sondern \emph{mit} dem Stein.\footnote{\textsc{Harvey 06: 37}} 

Es gibt auch Erzählungen bei den Ojibwa von Steinen welche belebt sind und anthropomorphe Merkmale besitzen. Zum Beispiele Steine die so geformt sind, dass es aussieht als ob sie einen Mund, oder Augen haben. Solche Merkmale werden aber nicht zwingend als Hinweis zur Beseeltheit des Steines gelesen. Das Aussehen kann trügen. Ein Stein gilt als animiert, wenn mit ihm gesprochen werden kann. Wenn man mit ihm wie mit andern Personen interagieren kann.

Ein anderes und weitaus abstrakteres Beispiel findet sich bei den Saisongeschichten (Seasonal Stories). Der Umgang mit diesen Geschichten entspricht dem respektvollen Umgang mit einer Person. Tatsächlich werden diese Geschichten Grossvater genannt und sind entsprechend auch ehrwürdig.\footnote{\textsc{Harvey 06: 42}} Man beschäftigt sich nicht leichtfertig mit diesen Geschichten, und auch wenn sie auch lustig sein können, so nimmt man sie doch ernst. Sie vermitteln Dinge grosser Wichtigkeit, wenn man sich ihnen respektvoll annähert.

People people: \texttt{https://orionmagazine.org/article/forget-shorter-showers/}

\subsubsection*{Die Kunst der Maori}
Maori sind für ihre kunstvollen Schnitzereien von Pounamu Steinen\footnote{Sammelnbezeichnung der Maori für Nephrit-Jade und Bowenit. Im Englischen werden diese Steine schlicht \emph{greenstone} genannt.}, Knochen und Holz berühmt. Harvey möchte zeigen, dass diese Kunstwerke selbst (durch den Macher) beseelt sind. 

Die Maori fühlen eine tiefe Verwandtschaft mit dem Ort an dem sie leben. Ein junger Mensch entwickelt sich in Abhängigkeit seiner Familie und seines Clans, aber auch die Natur gehört zu seinen Vorvätern. Das Land wird als Quelle der Identität betrachtet. Es gehört und wird kontinuierlich geteilt von den Toten, den Lebenden und den Ungeborenen.\footnote{\texttt{http://www.justice.govt.nz/publications/publications-archived/2001/he-hinatore-ki-te-ao-maori-a-glimpse-into-the-maori-world/part-1-traditional-maori-concepts/whenua}} Es ist zum Beispiel Brauch, dass bei der Geburt eines Kindes die Plazenta vergraben wird. Somit ist das Neugeborene mit dem Ort verbunden.

Die Maori sehen in der Süsskartoffel nahe Verwandte, ohne deren Hilfe den Maori eine wichtige Nahrungsgrundlage fehlen würde. Ohne die Hilfe der MAori würde die Pflanze jedoch gar nicht erst wachsen und gedeien können. Die Kartoffeln aus zugraben und zu essen grenzt daher an Kannibalismus.\footnote{Kannibalismus ist unter Maori durchaus üblich. Dabei geht es in keiner Weise darum sich vom Menschenfleisch zu ernähren. Die Einverleibung fand von Freunden und Feinden statt.}

Das Schnitzen von Knochen, welche in jedem Menschen vorhanden sind, stellt keinen grösseren Eingriff dar, als das Fällen und Schnitzen von Bäumen und das Schnitzen von Holz. Eine Schnitzerei steht somit immer im Zusammenhang dem Nehmen von Leben. Die Überreste einer Schnitzerei werden jeweils zurückgegeben. Die kunstvolle Schnitzerei ist nicht dafür da, um davon abzulenken. Durch das Schnitzen findet eine Transformation statt, in der der Künstler das Potential das im Holz, Stein oder Knochen schlummert hervor bringt. Ein Pounamu Anhänger ist belebt und nicht einfach nur Schmuck oder Identität für den Träger. Er hat ein Geschlecht, einen Namen und verdient Respekt. 

Ein Wharenui, das Gemeinschaftshaus einer Maori Gruppe, gilt als belebt. Durch die kunstvollen Schnitzereien wird der Vorfahre enthüllt und präsentiert. Das Haus \emph{isst} jene, welche eintreten und transformiert das \emph{tapu}-Neuheit in eine \emph{noa}-Normalität.\footnote{Tapu und Noa?} Der Gast ist somit nicht zum Einheimischen geworden, aber statt der Befindlichkeit findet man eine Normalität, selbst wenn diese nicht alltäglich ist. 

\smallskip
Die beiden aufgezeigten Kulturen in welchen Harvey von Animismus redet zeigen, dass es sehr grosse Unterschiede darin gibt, wie Animisten mit der Welt um sie herum agieren. Dabei decken diese beiden Beispiele nur einen sehr kleinen Teil der Aspekte ab, welche Harvey dem Begriff Animismus zusammenfasst. Weitere Beispiele sollen bei der Filmanalyse an passender Stelle gezeigt werden.





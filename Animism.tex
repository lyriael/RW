%!TEX root = Animismus_in_Anime.tex
Der Ausdruck \emph{Animismus}, von lateinisch \emph{anima} für Seele\footnote{Auch: Atem, Leben.}, wurde von Georg Ernst Stahl (1659-1734) erstmals verwendet und im Jahr 1871 von Edward Tylor in Primitive Culture eingeführt. Wie bei vielen Begriffen in der Religions\-wissenschaft, trägt auch dieser mehrere Bedeutungen und kann unterschiedlich verwendet werden. In der Regel wird unter einer animistischen Religion eine schriftlose Religion verstanden. Früher wurden diese gerne als Naturreligionen oder als achaische oder primitive Religionen bezeichnet. In diesem Zusammenhang, aber nicht deckend, versteht man unter Animismus auch den Glauben an eine beseelte Umwelt. Somit ist der Mensch nicht das einzige beseelte Wesen. Auch Tiere und Naturobjekte können beseelt sein. Letztlich kann mit Animismus auch einfach der Glaube an Geister und Seelen verstanden werden.\footnote{RGG$^4$ 1 (2008). Animismus: 504-506.}

Es soll an dieser Stelle zunächst ein kurzer historischer Abriss des Animismus gegeben werden. Danach wird Graham Harvey als Vertreter eines modernen Animismus vorgestellt. Schliesslich verlassen wir das ausdrückliche Gebiet des Animismus um eine ganz andere Perspektive auf Religion einzunehmen. Dazu wenden wir uns der kognitiven Religionswissenschaft nach Pascal Boyer zu. 

\subsection{Der alte Animismus}
Die nachfolgende Zusammenfassung basiert primär auf jener von Graham Harvey in \emph{Animism. Respecting the Living World}. 

Wie schon festgehalten, wird der Begriff des Animismus erstmals von Stahl verwendet. Er stellt die Theorie auf, dass es ein physikalisches Element gäbe, welches belebt. Je mehr Anima vorhanden ist, desto belebter ist ein Objekt. Während eine tote Person oder ein Stein keine Anima (mehr) aufweist, besitzt eine lebendige Person viel Anima. Auch Tiere und Pflanzen besitzen demnach Anima, jedoch weniger als der Mensch.

James Frazer (1854-1914) stellt später die Theorie auf, dass die Wilden (\mbox{savage}) glauben, dass Pflanzen und Tiere gleichermassen beseelt seien wie die Menschen. Wenn nun die Wilden zu glauben beginnen, dass Pflanzen und Tiere nur temporär und durch eine andere Wesenheit beseelt seien, entwickelt sich der Animismus in einer Folgestufe zum Polytheismus.

Edward Tylor (1832-1917) beschreibt in seinem Werk \emph{Primitive Culture} (Die Anfänge der Cultur) den Animismus als der Ursprung der Religion. Der Animismus würde im Laufe der Weiterentwicklung und Zivilisierung einer Kultur durch höhere Formen der Religion abgelöst werden. Nichtsdestotrotz würden sich auch in einer hochentwickelten und komplexen Religion noch Überreste der alten Religion in Form von Aberglaube finden.

Diese Religionstheorie Tylors kritisiert Robert R. Marett (1866-1943), weil Phänomene wie Ehrfurcht vor den Tieren, vor Blut oder Naturgewalten nicht berücksichtigt würden. Stattdessen führt Marett eine Dichotomie vom Alltäg\-lichen und vom Ausseralltäglichen ein, wobei Letzteres durch Religion erklärt und verarbeitet würde. Das Ausseralltägliche teilt er weiter in die Begriffe Mana und Tabu ein. Mana beschreibt dabei die Begegnung mit einer übermenschlichen Macht, während Tabu für Furcht und Kontaktvermeidung aufgrund von Gefahr steht. Folglich ordnet er Religion dem Mana zu, während er das, was mit Tabu verbunden wird, als Magie bezeichnet.\footnote{Ruel, M. J. (2008). \glqq Marett, Robert Ranulph\grqq. \emph{International Encyclopedia of the Social Sciences}, in: Cengage Learning. (2015).\\ \url{http://www.encyclopedia.com/doc/1G2-3045000765.html} 07. September 2015.} 

Ebenso wie Tylor, geht Emil Durkheim (1858-1917) von einer Ursprungsreligion aus. Im Unterschied dazu setzt er jedoch den Totemismus an die Stelle des Animismus. Er tut dies, weil er die sozialen Aspekte über die Erfahrungen des Individuums stellt.

Zusammenfassend kann gesagt werden, dass der \emph{alte} Animismus als Vorstufe für eine höher entwickelte Religion gesehen wurde. In der Geburtsstunde der Religionssoziologie und der Anthropologie war der Westen überzeugt, dass es eine lineare Entwicklung gebe, wobei der Westen auf der höchsten Stufe stehe. Weil diese Ansicht heutzutage als veraltet gilt, haben viele der früheren Werke über den Animismus ihre Bedeutung für die heutige Religionswissenschaft eingebüsst. Es gibt jedoch Versuche den Animismus neu zu definieren und ihm auf diese Weise eine neue Bedeutung in der Moderne zu geben. 

\subsubsection{Animismus und Shintoismus}
Shinto, oder auch Shintoismus bezeichnet die native japanische Religion. Ähnlich wie auch das Wort Bud\={o} setzt sich das Wort aus zwei japanischen Schrift\-zeichen zusammen, wobei das Zweite jeweils mit Weg übersetzt werden kann. Das erste Zeichen steht für \emph{Götter} oder \emph{Geister}, japanisch Kami oder ehrfürchtig Kami-sama. Dabei handelt es sich um übernatürliche Wesen, die in japanischen Mythen und Legenden vorkommen. Sie können auch natürlichen Phänomenen innewohnen oder Schutzpatron für eine jeweilige Region sein. Wird Kami als Fachbegriff übernommen, kann Shinto als der \emph{Weg der Kami} übersetzt werden.\footnote{RGG$^4$ 7 (2004). Sintoismus: 1283-1286.}

Es wäre sicherlich interessant einen Abgleich zwischen dem Shintoismus und den animistischen Elemente, welche man in den Filmen findet, zu machen. Man könnte vergleichen, wie nahe sich der historische Shintoismus und die moderne japanische Vorstellung sehen. Das liegt jedoch nicht im Sinne dieser Arbeit, da hier der Animismus losgelöst von seiner Kultur gesucht wird. Deswegen wird der Shinto hier nur der Vollständigkeit halber erwähnt, spielt aber für die weitere Untersuchung eine nebensächliche Rolle.

\subsection{Harvey Graham: Ansätze für einen modernen Animismus}
Der Animismus steckt heute insofern in einer Krise, als dass auf den \emph{alten} Animismus nicht einfach aufgebaut werden kann. Andrerseits sind die Phänomene des Animismus weiterhin interkulturell präsent. Die Phänomene an sich werden weiterhin von Ethnologen und Psychologen ernst genommen. Als Konsequenz werden sie jedoch als kognitiver Fehler, als Projektion, als Produkt einer überproduktiven Phantasie oder einer mangelnden Trennung von subjektiver und objektiver Welt eingeschätzt. 

Auf der anderen Seite wirkt die Moderne zu Gunsten des Animismus. Früher wurden fremde Kulturen belächelt. Man bezeichnete die Völker als primitiv, die an Naturgeister glaubten und diese anbeteten. Hundert Jahre später sehen wir unsere Existenzgrundlage bedroht, weil wir unsere Umwelt rücksichtslos ausgebeutet haben. Es ist daher verständlich, dass Haltungen, welche die Natur in ein Gegenüber stellen und so eine respektvolle Interaktion ermöglichen, eine gewisse Sympathie erfahren. 

\subsubsection*{Graham Harvey}
Ein Beispiel dafür ist bei Graham Harvey zu finden. Er sagt: 

\begin{quote}
	\glqq Animists are people who recognise that the world is full of persons, only some of whom are human, and that life is always lived in relationship with others. Animism is lived out in various ways that are all about learning to act respectfully towards and among other persons.\grqq ~(Harvey 2006: xi.)
\end{quote}

Mit dieser Aussage beginnt Harvey sein Buch \emph{Animism. Respecting the \mbox{Living World}}. Es ist bereits an dieser Stelle erkennbar, dass es ihm in erster Linie darum geht eine bestimmte Lebenshaltung zu postulieren. Anhand zahlreicher Beispiele, welche er im Laufe seiner Forschung in Neuseeland, Australien, Hawaii, Neufundland, Niger, und Amerkia gesammelt hat, erklärt er, wie der Animismus zu verstehen sei. 

Harvey nimmt eine Unterscheidung zwischen \emph{altem} und \emph{neuem} Animismus vor. In der alten Vorstellung wird davon ausgegangen, dass Animisten Menschen sind, die nicht zwischen Objekt und Subjekt unterscheiden, sei es, weil sie es nicht können, sei es, weil sie es nicht wollen. Im Unterschied dazu suchen Neue Animisten Wege und Ansichten, wie sie mit andern Personen richtig und respektvoll interagieren können. Zentral in Harveys Buch ist das Zusammenfassen von Menschen (humans) und Anderen-als-Menschen zu einer Übergruppe von Personen (people). Es gibt also Personen, welche nicht Menschen sind. Dennoch kann und muss mit ihnen interagiert werden. Allerdings gibt es unter den Personen auch hinterlistige und verschlagene (Menschen und Andere-als-Menschen). Deshalb ist es wichtig allfällige Masken, Täuschungen und falsche Aussagen durchschauen zu können. Im Wissen etwa, dass es Personen gibt, die uns fressen möchten, ist es weise, sowohl vorsichtig, als auch konstruktiv im Umgang mit den Anderen zu sein. 

Ich werde hier auf zwei Beispiele Harveys eingehen. Das erste handelt von den Ojibwa, einem nordamerikanischen Indianerstamm. Harveys Überlegungen stützen sich dabei hauptsächlich auf Irving Hallowells Beobachtungen\footnote{Nach Graham Harvey: Hallowell, A. Irving. (1960). \glqq Ojibwa Ontology, Behavior, and World View\grqq. \emph{Culture in History: Essays in Honor of Paul Radin}, hg. v. Stanley Diamond. Columbia University Press.} und Untersuchungen, wobei die Sprache im Zentrum steht. Als zweites sollen Harveys Überlegungen zur Maori-Kunst dargestellt werden.

\subsubsection*{Die Sprache der Ojibwa}
Die Ojibwa geben uns ein Beispiel dafür, dass sich Animismus in der Grammatik der Sprache zeigen kann. So wie es im Deutschen (und in verwandten Sprachen) eine Untescheidung zwischen männlich und weiblich gibt, unterscheidet die Ojibwe-Grammatik zwischen belebt (animated) und leblos (inanimated). Diese Unterscheidung ist keineswegs selbstredend. So wie wir „die Tasse“ oder „der Hund“ sagen, entspricht das grammatische Geschlecht nicht immer mit dem Geschlecht des Beschriebenen überein. In der Ojibwe-Sprache beschreibt die Grammatik auch die Steine als animiert. Doch als Antwort auf die Frage ob denn alle Steine leben würde, antwortete ein alter Ojibwe mit: „Nein. Aber ein paar schon.“\footnote{\textsc{Harvey} 06: 33.} Aus diesem Beispiel geht hervor, dass der Animismus hier kein dogmatisches Glaubenssystem darstellt. Es ist möglich, dass ein Stein animiert ist, und doch lässt sich diese Aussage nicht auf alle Steine übertragen. Für diese Animisten ist also nicht grundsätzlich alles belebt. 

Eine weitere Anekdote erzählt von einem Stein, der durch einen weissen Händler ausgegraben wurde. Der Händler dachte, er gehöre zu einem zeremoniellen Pavillon. Also suchte er einen Ojibwa namens John auf. John beugte sich zum Stein und frage den Stein leise, ob er zu diesem Pavillon gehöre. Laut John antwortete der Stein, dass dem nicht so sei. Dieses Beispiel zeigt, dass mit dem Stein wie mit einer Person umgegangen wird. John spricht nicht \emph{zu}, sondern \emph{mit} dem Stein.\footnote{\textsc{Harvey} 06: 37.} 

Weiter gibt es bei den Ojibwa auch Erzählungen von Steinen, die belebt sind und darüberhinaus anthropomorphe Merkmale besitzen. Dies sind zum Beispiel Steine, die so geformt sind, dass es aussieht, als ob sie einen Mund oder Augen hätten. Gleichzeitig sind solche Merkmale nicht zwingend als Hinweis zur Beseeltheit des jeweiligen Steines zu verstehen. Das Aussehen kann trügen. Ein Stein gilt als animiert, wenn mit ihm gesprochen werden kann. Wenn man mit ihm wie mit anderen Personen interagieren kann. 

Ein nächstes und weitaus abstrakteres Beispiel findet sich bei den Saison\-geschichten (Seasonal Stories). Der Umgang mit diesen Geschichten entspricht dem respektvollen Umgang mit einer Person. Tatsächlich werden diese Geschichten \emph{Grossvater} genannt und gelten entsprechend als ehrwürdig.\footnote{\textsc{Harvey 06: 42.}} Mit diesen Geschichten beschäftigt man sich nicht leichtfertig. Und auch wenn sie mitunter lustig sein können, so nimmt man sie doch ernst. Sie vermitteln Dinge von grosser Wichtigkeit, wenn man sich ihnen respektvoll annähert. 

\subsubsection*{Die Kunst der Maori}
Maori sind für ihre kunstvollen Schnitzereien von Pounamu-Steinen\footnote{Sammelbezeichnung der Maori für Nephrit-Jade und Bowenit. Im Englischen werden diese Steine schlicht \emph{greenstone} genannt.}, Knochen und Holz berühmt. Harvey möchte zeigen, dass diese Kunstwerke selbst (durch den Macher) beseelt sind.

Die Maori fühlen eine tiefe Verwandtschaft mit dem Ort an dem sie leben. Ein junger Mensch entwickelt sich nicht nur in Abhängigkeit seiner Familie und seines Clans, auch die Natur gehört zu seinen Vorvätern. Das Land wird als Quelle der Identität betrachtet. Es gehört den Toten, den Lebenden und den Ungeborenen, und wird mit ihnen geteilt.\footnote{New Zealand Ministry of Justice. (ohne Jahr). \emph{Whenua}. 

	\url{http://www.justice.govt.nz/publications/publications-archived/2001/he-hinatore-ki-te-ao-maori-a-glimpse-into-the-maori-world/part-1-traditional-maori-concepts/whenua} 07. September 2015.}
Es ist zum Beispiel Brauch, dass bei der Geburt eines Kindes die Plazenta vergraben wird. Somit ist das Neugeborene mit dem Ort seiner Geburt verbunden.

Die Maori sehen in der Süsskartoffel nahe Verwandte, ohne deren Hilfe den Maori eine wichtige Nahrungsgrundlage fehlen würde. Ohne die Hilfe der Maori würde die Pflanze jedoch gar nicht erst wachsen und gedeihen können. Die Kartoffeln auszugraben und zu essen grenzt daher an Kannibalismus.\footnote{Kannibalismus war unter Maori durchaus üblich. Dabei geht es in keiner Weise darum sich vom Menschenfleisch zu ernähren. Die Einverleibung fand von Freunden und Feinden statt.}

Ebenso stellt das Schnitzen von Knochen, welche ja in jedem Menschen vorhanden sind, keinen grösseren Eingriff dar, als das Fällen und Schnitzen von Bäumen und das Schnitzen von Holz. Eine Schnitzerei steht somit immer im Zusammenhang mit dem Nehmen von Leben. Die Überreste einer Schnitzerei werden jeweils zurückgegeben. Die kunstvolle Schnitzerei ist nicht dafür da, um davon abzulenken. Durch das Schnitzen findet eine Transformation statt, in der der Künstler das Potential, das im Holz, Stein oder Knochen schlummert, hervor bringt. Ein Pounamu-Anhänger ist belebt und nicht einfach nur Schmuck oder Identität des Trägers. Er hat ein Geschlecht, einen Namen und er verdient Respekt. 

\medskip
Die beiden aufgezeigten Kulturen, in welchen Harvey von Animismus redet, zeigen, dass es sehr grosse Unterschiede darin gibt, wie Animisten mit der Welt um sie herum agieren. Dabei decken diese beiden Beispiele nur einen sehr kleinen Teil der Aspekte ab, welche Harvey unter dem Begriff Animismus zusammenfasst.
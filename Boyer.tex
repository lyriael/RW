%!TEX root = Animismus_in_Anime.tex
In dieser Unterdisziplin der Religionswissenschaft werden Religion bzw. religiöse Phänomene aus der Perspektive der Kognitions- und Evolutionswissenschaft betrachtet. Es wird versucht zu erklären, weshalb religiöse Praktiken und Denk\-weisen universell verbreitet sind. Die kognitive Religionswissenschaft ist eine eher junge Disziplin, die sich erst gegen Ende des zwanzigsten Jahrhunderts etabliert hat. An dieser Stelle soll Pascal Boyers \emph{Und Mensch schuf Gott} als Stellvertreter für die kognitiven Ansätze dienen um die Animationsfilme auf Animismus zu untersuchen.\footnote{RGG$^4$ 4 (2008). Kognitionswissenschaft: 1470-1471.}

Boyer erklärt in der Einleitung seines 2001 erschienenen Buches, dass Religion im Geiste des Menschen zu suchen sei. Denn jeder menschliche Geist habe das Zeug religiös zu sein. Es ist somit nicht Boyers Ziel zu beweisen, dass es Gott nicht gebe und nur ein Produkt unserer Fantasie sei, er möchte erklären können, warum religiöse Menschen glauben was sie glauben.

Bei der Frage nach dem Ursprung der Religion tauchen immer wieder ähnliche intuitive Begründungen auf: \emph{Die Religion bietet Erklärungen. Die Religion spen\-det Trost. Die Religion sichert die gesellschaftliche Ordnung. Die Religion ist eine kognitive Täuschung}.\footnote{\textsc{Boyer 04: 14-15.}} Laut Boyer sind diese intuitiven Gewissheiten in ihrer Existenz zwar berechtigt, jedoch nicht dienlich, wenn es darum geht den Ursprung zu finden. Ein Ursprung im Sinne eines historischen Ereignisses ist freilich eine Wunschvorstellung, welche dem Wunsch entspringt eine Ursache zu haben, aus der sich alle weiteren Phänomene ableiten lassen. Boyer zeigt in aller Ausführlichkeit, wie man in jeder dieser Vorstellungen Widersprüche fin\-det, oder dass sie schlicht nicht befriedigend sind. Beides macht sie als Ursprung ungeeignet. Für jedes Gebiet, das er abarbeitet, fügt er am Ende einen anderen Blickwinkel hinzu, in dem er aus kognitiver Sicht den Wert dieser intuitiven Annahme beschreibt.

Ein Beispiel für den Fehler im ersten Punkt (Religion bietet Erklärungen) ist das Donnergrollen. Boyer zufolge gibt es zu wenig her, anzunehmen, dass das Grollen die Reaktion von Geistern, Göttern oder Ahnen auf ein Fehlverhalten des Menschen sei. Der Aufwand für ein an und für sich simples Phänomen wäre unverhältnismässig gross. Denn um laute, grollende, dumpfe Geräusche bei Stürmen zu erklären, muss eine komplette imaginäre Welt mit übernatürlichen Mächten vorausgesetzt werden. Dies wirft an sich noch mehr Fragen auf: Woher sind diese Wesen gekommen? Wo sind sie? Warum sieht man sie nicht? Haben sie einen riesigen Mund um diese Geräusche zu erzeugen? Nur wenn ein solcher Glaube verbreitet ist, finden sich zu diesen Fragen auch Antworten. Diese Antworten sind aber teilweise so weit her geholt, dass sie die Ausgangslage, die Geräusche eines Gewitters erklären zu wollen, unwahrscheinlich machen. 

In seiner Erklärung geht Boyer von der Funktionsweise des Geistes im Allgemeinen aus, welche unabhängig von der Kultur überall gleich ist.\footnote{\textsc{Boyer 04: 11-12.}} Das scheint zu nächst eine Sackgasse, da kulturell grosse Unterschiede bezüglich religiöser Praktiken und Vorstellungen zu finden sind. Das geniale hier sei, argumentiert er, dass sich etwas so vielschichtiges wie Religion durch etwas erklären lasse, was überall gleich sei (d. h. das Gehirn). Es ist jedoch notwendig zunächst mehr darüber zu wissen, wie das Gehirn Informationen aufnimmt und verarbeitet. Denn die Arbeit, die ein Gehirn leistet, ist lange unterschätzt worden. Einerseits muss man von der verbreiteten Annahme wegkommen, es handle sich beim Geist um ein leeres Gefäss, welches beliebig mit Informationen (Erziehung, Bildung und persönliche Erlebnisse) gefüllt werden könne. Andrerseits gilt es von der Idee wegzukommen, der Geist könne wahllos mit Informationen gefüllt werden. Das Gehirn kann sich aus gutem Grund nicht alles Beliebige merken. Das Gehirn muss die relevanten Informationen aus der Umwelt identifizieren und auf eine spezifische Weise verarbeiten. 

\subsubsection{Meme}
Boyer geht nun der Frage nach, auf welche Weise religiöse Konzepte überhaupt entstehen. Die Bezeichnung Mem als Kulturelement, also Vorstellungen, Werte, Geschichten und dergleichen, die die Menschen in ihrem Handeln beeinflussen und die weitergegeben werden, wurde vom Evolutionsbiologen Richard Dawkins erstmals vorgestellt. Ein Mem bezeichnet demnach einen Bewusstseinsinhalt, welcher durch Kommunikation in der Gesellschaft weitergegeben und somit vervielfältigt werden kann.\footnote{Mittelstrass, Jürgen (Hrsg.). (2005-2014). \emph{Enzyklopädie Philosophie und Wissenschaftstheorie. Gesamtwerk in acht Bänden.} 2 Aufl., Bd. 5. Stuttgart: Metzler, als online-PDF:\\ \url{http://www.uni-konstanz.de/FuF/Philo/Philosophie/philosophie/files/mem.pdf} 10.~September 2015.} Es ist das soziokulturelle Pendant zu den bio\-logischen Genen in der Evolution. Sodann lassen sich die Meme ähnlich wie die Gene beschreiben: Information wird durch Kommunikation weiter gegeben (repliziert). Dadurch werden die Inhalte nicht einfach verbreitet, sondern auch abgeändert (mutiert).\footnote{Die Information ändert sich nicht erst durch deren Weitergabe. Etwas, das wir erfahren, wird bei verschiedenen Menschen bereits anders verarbeitet. Somit können zwei Personen genau das gleiche hören und eine andere Version des Inhalts in ihrem Bewusstsein haben.} Schliesslich werden nur die einprägsamen bzw. relevanten Meme tatsächlich weiter gegeben, sie werden selektiert.

Boyer beschreibt die Meme zwar als eine wunderbare Ausgangslage, er will ihnen aber nicht mehr als genau das zugestehen. Seine Kritik zielt darauf ab, dass es keine Replikation der eigentlichen Informationen gebe. Boyer zufolge werden Inhalte gerade nicht faktisch übergeben, sondern jeweils neu konstruiert. Zwei Menschen können zwei faktisch identische Aussagen machen, aber jeder hat die Information, welche er wiedergibt auf seine eigene Art rekonstruiert. Entsprechend stellt Boyer als nächstes seine Theorie zum Einfangen von Vorstellungen durch Schablonen vor.

\subsubsection{Schablonen}
Ein grosser Teil dessen, was wir wissen, musste uns niemand faktisch erzählen. Eine erstaunliche Eigenschaft unserer geistigen Fähigkeit ist es, durch die Kombination bereits existierenden Wissens und der Hinzugabe einer neuen Information zusätzliches Wissen zu generieren. Boyer zeigt dies auf, indem er das Kind als eine Person aufzeigt, dessen Wissen fortlaufend erweitert wird. Wird einem Kind zum ersten Mal ein Seehund mit Namen gezeigt, so hat es abgesehen vom Namen und dem äusseren Erscheinungsbild des Seehundes keine weiteren Informationen darüber. Dennoch wird das Kind erwarten, dass der Seehund isst, schläft und dass er sich fortplfanzt. Diese Informationen über den Seehund hat das Kind geschlussfolgert, indem es eine Annahme gemacht hat: Der Seehund ist ein Säugetier. Säugetiere essen, schlafen und pflanzen sich fort. Folglich be\-zeichnet Boyer das Säugetier als eine \emph{Schablone}. Mit dieser Säugetier-Schablone hat das Kind eine Seehund-Vorstellung gebildet.\footnote{\textsc{Boyer 04: 59.}} 

Über diesen Schablonen, die verschiedene Konzepte zusammenfassen und aus einem Informationsstück mehr Information schaffen, stehen die ontolo\-gischen Kategorien. Boyer zählt fünf auf: PERSON, TIER, PFLANZE, NA\-TUR\-OB\-JEKT, WERKZEUG. Die Schablonen, und im stärkerem Masse die ontolo\-gischen Kategorien, seien das, was über Kulturen hinweg universell gültig sei. Erst bei den konkreten Konzepten ergebe sich eine Varianz. Die Vorstellungen, die von Angehörigen einer gleichen Gruppe anhand einer Schablone hergestellt werden, sind sich in der Regel ähnlich. Die Vorstellungen einer anderen Gruppe kann davon jedoch stark abweichen, obwohl die gleiche Schablone benutzt wurde. 

Dieses System der Schablonen, welches mit Hilfe von Schlussfolgerungen zu Vorstellungen bzw. zu Konzepten führen, überträgt Boyer nun auf die Religion. Demnach gibt es Schablonen für religiöse Vorstellungen. Diese Schablonen werden universell geteilt, wenn auch die Konzepte regional stark variieren können. Darüber, dass übernatürliche Kräfte unsichtbar sein können, ist man sich weitgehend einig. Wenn es aber darum geht, was und warum übernatürliche Kräfte etwas tun, so gehen die Vorstellungen weit auseinander. 

Boyer betont, es müsse letztlich berücksichtigt werden, dass die kulturelle Varianz in der Regel geringer sei, als man allgemein annimmt. Beim Übermitteln findet durch die Schablonen ein Filtern der gegebenen Informationen statt, so dass daraus voraussagbare Strukturen gebaut werden (\textsc{Boyer 04: 65}).

\subsubsection{Beschaffenheit des Übernatürlichen}
Es folgt also die Suche nach dem mentalen Rezept für religiöse Vorstellung\-en. Mit einem Versuch verschiedener mehr oder weniger potenten religiösen Aussagen versucht Boyer dem Leser zu zeigen, dass man der Intuition folgend gewisse Aussagen über übernatürliche Wesen direkt ausschliessen kann, während man bei anderen sofort glauben würde, dass es sich um eine existierende religiöse Vorstellung handelt. Folgende Aufzählung der Kriterien für eine erfolgreiche (religiöse) Vorstellung nach Boyer basiert auf Robin Hanson.\footnote{Hanson, Robin. (2001). \emph{A Review of Religion Explained: The Evolutionary Origins of Religious Thought}. \\ \url{http://mason.gmu.edu/~rhanson/religion.html} 10. September 2015.} 

Erstens habe jedes übernatürliche Konzept die Tendenz, eine seiner ontologischen Annahmen zu verletzen. Ein Geist gehört zur ontologischen Kategorie PERSON, doch dass er keinen physikalischen Körper hat, bricht mit der Ontologie. Ein anderes Beispiel für diese Kontraintuitivität findet sich in der Natur, so etwa bei einer Raupe, welche nach der Metamorphose zu einem Schmetterling wird. Die Erwartung für ein TIER ist, dass es sich im Laufe des Wachstums nur durch Grösse und Masse verändert, jedoch nicht, dass es zu einem anderen TIER wird. Dabei ist es wichtig, dass tatsächlich gegen die ontologische Ka\-tegorie verstossen wird und es sich nicht nur um eine Merkwürdigkeit handelt. Eine PERSON, die ihre Hautfarbe ändert, ist dem zu Folge weniger erfolgreich, als eine PERSON, welche durch Wände gehen kann. 

Zweitens hat religiöses Denken die Tendenz auf Leute-Ähnliche übernatür\-liche Wesen zu fokussieren, welche Zugang zu sozial-relevanten Informationen haben. Der Austausch von Informationen ist für den Menschen kritisch, so \mbox{Boyer}. Wir sind darauf angewiesen, dass andere in der Gruppe Dinge wissen und uns dieses Wissen übertragen können. Bei religiösen Vorstellungen wird in der Regel davon ausgegangen, dass ein Wesen die moralische Haltung eines Individuums oder auch der ganzen Gruppe teilt. Dieses Wesen weiss Bescheid darüber, wenn schlechte Dinge geschehen. Die Person erwartet folglich, dass dieses Wesen wertet und allenfalls böse wird und die Person für deren Verhalten bestraft. Eine solche Vorstellung macht diese Wesen zu wichtigen Subjekten für Gedanken und Diskussionen in einer Gruppe. 

Im dritten Punkt werden religiöse Rituale aus den Reinigungsritualen her\-geleitet. Unser mentales System behandelt Krankheit mit Abscheu zum ei\-genen Schutz vor einer unsichtbaren Gefahr. So sollen auch Rituale in religiösen Vorstellungen vor allem zum Schutz vor unsichtbaren Gefahren und Mächten bestehen. 

Letztlich setzt Boyer noch den Fokus auf die Leiche und wie Menschen damit umgehen. Im Prinzip sind Leichen ein Spezialfall des ersten Punktes, wo es um die Verletzung der ontologischen Kategorie geht. Der Mensch, den wir sehen oder sogar gekannt haben, wird in unserem System der ontologischen Kategorie PERSON zugeordnet. Gleichzeitig verletzt die Leiche dieses Menschens jegliche Kriterien der PERSON-Kategorie und entspräche demnach der Kategorie NATUROBJEKT und sollte unsere Abscheu vor Krankheit wecken. Dieser Widerspruch macht die menschliche Leiche zum Prototyp für religiöse Objekte überall auf der Welt.
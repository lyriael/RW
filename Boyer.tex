%!TEX root = Animismus_in_Anime.tex
In dieser Unterdisziplin der Religionswissenschaft werden Religion bzw. religiöse Phänomene aus der Perspektive der Kognitions- und Evolutionswissenschaft betrachtet. Es wird versucht zu erklären, weshalb religiöse Praktiken und Denkweisen universell verbreitet sind.\todo{wikipedia: \texttt{https://en.wikipedia.org/wiki/Cognitive\_science\_of\_religion} und \texttt{https://de.wikipedia.org/wiki/Kognitive\_Religionswissenschaft} 24.08.15}

\subsection*{Boyer: Und Mensch schuf Gott}
Die kognitive Religionswissenschaft ist eine eher junge Disziplin, die sich erst gegen Ende des zwanzigsten Jahrhunderts etabliert hat. Zu ihren Begründern gehören unter anderen E. Thomas Lawson und Robert McCauley (\emph{Rethinking Religion: Connecting Cognition and Culture and Bringing Ritual to Mind: Psychological Foundations of Cultural Forms}), Pascal Boyer (\emph{Naturalness of Religous Ideas}) und Guthrie (\emph{Faces in the cloud}). An dieser Stelle soll Pascal Boyers \emph{Und Mensch schuf Gott}1 als Stellvertreter für die kognitiven Ansätze dienen um die Animationsfilme auf Animismus zu untersuchen.\todo{boyer}

Boyer erklärt in der Einleitung seines 2001 erschienenen Buches, dass Religion im Geiste des Menschen zu suchen sei. Denn jeder menschliche Geist habe das Zeug religiös zu sein. Es ist somit nicht Boyers Ziel zu beweisen, dass es Gott nicht gebe und nur ein Produkt unserer Fantasie sei, er möchte erklären können, warum religiöse Menschen glauben was sie glauben.

Bei der Frage nach dem Ursprung der Religion tauchen immer wieder ähnliche intuitive Begründungen auf: \emph{Die Religion bietet Erklärungen, Die Religion spendet Trost, Die Religion sichert die gesellschaftliche Ordnung, Die Religion ist eine kognitive Täuschung}\todo{Boyer 04: 14-15} usw. usf. Laut Boyer sind diese intuitiven Gewissheiten in ihrer Existenz zwar berechtigt, jedoch nicht dienlich, wenn es darum geht den Ursprung zu finden. Einen Ursprung im Sinne eines historischen Ereignis ist freilich eine Wunschvorstellung, welche aus dem Wunsch entspringt eine Ursache zu haben, aus der sich alle weiteren Phänomene ableiten lassen. Boyer zeigt in aller Ausführlichkeit, wie man in jeder dieser Vorstellung Widersprüche findet, oder dass sie schlicht nicht befriedigend sind. Beides macht sie als Ursprung ungeeignet. Für jedes Gebiet, das er abarbeitet, fügt er am Ende „[einen] andere[n] Blickwinkel“ hinzu, in dem er aus kognitiver Sicht den Wert dieser intuitiven Annahme beschreibt: 

\begin{itemize}
	\item Das Erkenntnissystem des Gehirns produziert Erklärungen, oft ohne dass wir uns dessen gewahr sind. - Religion als Erklärung.
	\item Emotionale Programme sind für uns lebenswichtig~\todo{Hier das Beispiel der Angst vor einem Raubtier. [S.34]} und somit ein Aspekt unseres entwicklungsgeschichtlichen Erbes. - Emotionen in der Religion.
	\item Die Untersuchung des sozialen Bewusstseins (soziale Intelligenz) kann Antworten auf die Frage nach den Erwartungen an das gesellschaftliche Leben und Moral geben. - Religion, Moral und Gesellschaft.
	\item Bei all den Übernatürlichen Informationen welcher der Geist bekommt, werden nur manche als plausibel erachtet und so angeeignet. - Religion und Denken.
\end{itemize}

Boyer ist sehr grosszügig mit einleuchtenden Beispielen, allerdings wird nicht immer klar, wieviel von dem, was er sagt, wissenschaftlich erwiesen ist und wieviel davon spekulativer Natur ist. 

Ein Beispiel für den ersten Punkt ist das Donnergrollen. Boyer zufolge gibt es zu wenig her, anzunehmen, dass das Grollen die Reaktion von Geistern, Göttern oder Ahnen auf ein Fehlverhalten des Menschen sei. Der Aufwand für ein an und für sich simples Phänomen wäre unverhältnis gross. Denn um laute, grollende, dumpfe Geräusche bei Stürmen zu erklären, muss eine komplette imaginäre Welt mit übernatürlichen Mächten vorausgesetzt werden. Dies wirft an sich noch mehr Fragen auf: Woher sind diese Wesen gekommen? Wo sind sie? Warum sieht man sie nicht? Haben sie einen riesigen Mund um diese Geräusche zu erzeugen? Nur wenn ein solcher Glaube verbreitet ist, finden sich zu diesen Fragen auch Antworten. Diese Antworten sind aber teilweise so weit her geholt, dass es die Ausgangslage, die Geräusche eines Gewitters erklären zu wollen unwahrscheinlich macht. 

Boyer führt nun weitere Konzepte ein, die für seinen Ansatz unausschliessbar sind. Er zählt diese als den Inhalt eines Werkzeugkasten auf. Für die weitere Arbeit werden diese Werkzeuge eine zentrale Rolle spielen, da diese aus der Sicht der kognitiven Religionswissenschaft die Methode zur Analyse liefern. 


In seiner Erklärung geht Boyer von der Funktionsweise des Geistes im Allgemeinen aus, welche unabhängig von der Kultur überall gleich ist. Das scheint zu nächst eine Sackgasse, da kulturell grosse Unterschiede bezüglich religiöser Praktiken und Vorstellungen zu finden sind. Das geniale hier sei, argumentiert er, dass sich etwas so vielschichtiges wie Religion durch etwas erklären lasse, was überall gleich sei (d. h. das Gehirn). Es ist jedoch notwendig zunächst mehr darüber zu wissen, wie das Gehirn Informationen aufnimmt und verarbeitet.\todo{[S.11]} Denn die Arbeit, die ein Gehirn leistet, ist lange unterschätzt worden. Einerseits muss man von der verbreiteten Annahme wegkommen, es handle sich beim Geist um ein leeres Gefäss, welches beliebig mit Informationen (Erziehung, Bildung und persönlichen Erlebnissen) gefüllt werden könne. Anderseits gilt es von der Idee wegzukommen, der Geist könne wahllos mit Informationen gefüllt werden. Das Gehirn kann sich aus gutem Grund nicht alles Beliebige merken. Das Gehirn muss die relevanten Informationen aus der Umwelt identifizieren und auf eine spezifische Weise verarbeiten.\todo{[S. 12]} 

\subsubsection*{Meme}
Boyer geht nun der Frage nach, auf welche Weise religiöse Konzepte überhaupt entstehen. Die Bezeichnung Mem als Kulturelement, also Vorstellungen, Werte, Geschichten und dergleichen, die die Menschen in ihrem Handeln beeinflussen und die weitergegeben werden\todo{\textsc{Boyer} 04: 50}, wurde vom Evolutionsbiologen Richard Dawkins erstmals vorgestellt. Ein Mem bezeichnet demnach einen Bewusstseinsinhalt\todo{wiki: \texttt{https://de.wikipedia.org/wiki/Mem} 24.08.15}, welcher durch Kommunikation in der Gesellschaft weitergegeben und somit vervielfältigt werden kann. Es ist das soziokulturelle Pendant zu den biologischen Genen in der Evolution. Sodann lassen sich die Meme ähnlich wie die Gene beschreiben: Information wird durch Kommunikation weiter gegeben (repliziert). Dadurch werden die Inhalte aber nicht einfach verbreitet, sondern auch leicht (oder schwerwiegend) abgeändert (mutiert).\footnote{Die Information ändert sich nicht erst durch deren Weitergabe. Etwas das wir erfahren wird bei verschiedenen Menschen bereits anders verarbeitet. Somit können zwei Personen genau das gleiche hören und eine andere Version des Inhalts in ihrem Bewusst sein haben.} Schliesslich werden nur jene die einprägsamen bzw. relevanten Meme tatsächlich weiter gegeben, sie werden selektiert.

Boyer beschreibt die Meme zwar als eine wunderbare Ausgangslage, er will ihnen aber nicht mehr als genau das zugestehen. Seine Kritik liegt darauf, dass es keine Replikation der eigentlichen Informationen gebe. Boyer zufolge werden Inhalte gerade nicht faktisch übergeben, sondern jeweils neu konstruiert. Zwei Menschen können zwei faktisch identische Aussagen machen, aber jeder hat die Information, welche er wiedergibt auf seine eigene Art rekonstruiert. Entsprechend stellt Boyer als nächstes seine Theorie zum Einfangen von Vorstellungen durch Schablonen vor.

\subsubsection*{Schablonen, Vorstellungen und das Schlussfolgerungssystem}
Ein grosser Teil dessen, was wir wissen, musste uns niemand faktisch erzählen. Eine erstaunliche Eigenschaft unserer geistigen Fähigkeit ist es, durch die Kombination bereits existierenden Wissens und der Hinzugabe einer neuen Information zusätzliches Wissen zu generieren. Boyer zeigt dies auf, indem er das Kind als eine Person aufführt, deren Wissen fortlaufend erweitert wird. Wird einem Kind zum ersten Mal ein Seehund gezeigt, so hat es abgesehen vom Namen und dem äusseren Erscheinungsbild des Seehundes keine weiteren Informationen darüber. Dennoch wird das Kind erwarten, dass der Seehund isst, schläft und dass er sich fortplfanzt. Diese Informationen über das Seehund hat das Kind geschlussfolgert, indem es eine Annahme gemacht hat: Der Seehund ist ein Säugetier. Säugeteire essen, schlafen und pflanzen sich fort. Folglich bezeichnet Boyer das Säugetier als eine \emph{Schablone}. Mit dieser Säugetier-Schablone hat das Kind eine Seehund-Vorstellung gebildet.\todo{\textsc{Boyer 04: 59}} 

Über diesen Schablonen, die verschiedene Konzepte zusammenfassen und aus einem Informationsstück mehr Information schaffen, stehen die ontologischen Kategorien. Boyer zählt fünf auf: \textsc{Person, Tier, Pflanze, Naturobjekt, Werkzeug}. Die Schablonen, und im stärkerem Masse die ontologischen Kategorien, seien das, was über Kulturen hinweg universell gültig sei. Erst bei den konkreten Konzepten ergebe sich eine Varianz. Die Vorstellungen, die von Angehörigen einer gleichen Gruppe anhand einer Schablone hergestellt werden, sind sich in der Regel ähnlich. Die Vorstellungen einer anderen Gruppe kann davon jedoch stark abweichen, obwohl die gleiche Schablone benutzt wurde. 

Dieses System der Schablonen, welches mit Hilfe von Schlussfolgerungen zu Vorstellungen bzw. zu Konzepten führen, überträgt Boyer nun auf die Religion. Demnach gibt es Schablonen für religiöse Vorstellungen. Diese Schablonen werden universell geteilt, wenn auch die Konzepte regional stark variieren können. Darüber, dass übernatürliche Kräfte unsichtbar sein können, ist man sich weitgehend einig. Wenn es aber darum geht, was und warum übernatürliche Kräfte etwas tun, so gehen die Vorstellungen weit auseinander. 

Boyer betont, es müsse letztlich berücksichtigt werden, dass die kulturelle Varianz in der Regel geringer sei, als man allgemein annimmt. Beim Übermitteln findet durch die Schablonen ein Filtern der gegebenen Informationen statt, so dass daraus voraussagbare Strukturen gebaut werden.\todo{[65]}

\subsubsection*{Beschaffenheit des Übernatürlichen}
Es folgt also die Suche nach dem mentalen Rezept für religiöse Vorstellungen. Mit einem Versuch verschiedener mehr oder weniger potenten religiösen Aussagen versucht Boyer dem Leser zu zeigen, dass man der Intuition folgend gewisse Aussagen über übernatürliche Wesen direkt ausschliessen kann, während man bei andern sofort glauben würde, dass es sich um eine existierende religiöse Vorstellung handelt. Folgende Kriterien für eine erfolgreiche (religiöse) Vorstellung kommen dabei heraus:\todo{Zusammenfassung: http://serendip.brynmawr.edu/exchange/node/1581 
Zusammenfassung 2: http://mason.gmu.edu/~rhanson/religion.html} 

Erstens habe jedes übernatürliche Konzept die Tendenz, eine seiner ontologischen Annahmen zu verletzen. Ein Geist gehört zur ontologischen Kategorie PERSON, doch dass er keinen physikalischen Körper hat bricht mit der Ontologie. Ein anderes Beispiel für diese Kontraintuitivität findet sich auch in der Natur, so etwa bei einer Raupe, welche nach der Metamorphose zu einem Schmetterling wird. Die Erwartung für ein TIER ist, dass es sich im Laufe des Wachstums nur durch Grösse und Masse verändert, jedoch nicht, dass es zu einem anderen TIER wird. Dabei ist es wichtig, dass tatsächlich gegen die ontologische Kategorie verstossen wird und es sich nicht nur um eine Merkwürdigkeit handelt. Eine PERSON, die ihre Hautfarbe ändert, ist dem zu Folge weniger erfolgreich, als eine PERSON, welche durch Wände gehen kann.\todo{fix Satz} 

Zweitens hat religiöses Denken die Tendenz auf Leute-Ähnliche übernatürliche Wesen zu fokussieren, welche Zugang zu sozial-relevanten Informationen haben. Der Austausch von Informationen ist für den Menschen kritisch, so Boyer. Wir sind darauf angewiesen, dass andere in der Gruppe Dinge wissen und uns diese übertragen können. Bei religiösen Vorstellungen wird in der Regel davon ausgegangen, dass ein Wesen die moralische Haltung eines Individuums oder auch der ganzen Gruppe teilt. Dieses Wesen weiss Bescheid darüber, wenn schlechte Dinge geschehen. Die Person erwartet folglich, dass dieses Wesen wertet und allenfalls böse wird und die Person für deren Verhalten bestraft. Eine solche Vorstellung macht diese Wesen zu wichtigen Subjekten für Gedanken und Diskussionen in einer Gruppe. 

Im dritten Punkt werden religiöse Rituale aus den Reinigungsritualen hergeleitet. Unser mentales System behandelt Krankheit mit Abscheu zum eigenen Schutz vor einer unsichtbaren Gefahr. So sollen auch Rituale in religiösen Vorstellungen vor allem zum Schutz vor unsichtbaren Gefahren und Mächten bestehen. 

Letztlich setzt Boyer noch den Fokus auf die Leiche und wie Menschen damit umgehen. Im Prinzip sind Leichen ein Spezialfall des ersten Punktes, wo es um die Verletzung der ontologischen Kategorie geht. Der Mensch, den wir sehen oder sogar gekannt haben, wird in unserem System der ontologischen Kategorie PERSON zugeordnet. Gleichzeitig verletzt die Leiche dieses Menschens jegliche Kriterien der PERSON-Kategorie und entspräche demnach der Kategorie NATUROBJEKT und sollte unsere Abscheu vor Krankheit wecken. Dieser Widerspruch macht die menschliche Leiche zum Prototyp für religiöse Objekte überall auf der Welt.
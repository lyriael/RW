%!TEX root = Animismus_in_Anime.tex
In dieser Underdisziplin der Religionswissenschaft wird Religion oder religiöse Phänomene aus der Perspektive der Kognitions- und Evolutionswissenschaft betrachtet. Es wird versucht zu erklären, weshalb religiöse Praktiken und und Denkweisen universell verbreitet sind.~\footnote{wikipedia: \texttt{https://en.wikipedia.org/wiki/Cognitive\_science\_of\_religion} und \texttt{https://de.wikipedia.org/wiki/Kognitive\_Religionswissenschaft} 24.08.15}

Die kognitive Religionswissenschaft ist eine eher junge Disziplin, welche sich erst Ende des zwanzigsten Jahrhunderts etabliert hat. Zu ihren Begründern gehören unter anderen E. Thomas Lawson und Robert McCauley (\emph{Rethinking Religion: Connecting Cognition and Culture and Bringing Ritual to Mind: Psychological Foundations of Cultural Forms}), Pascal Boyer (\emph{Naturalness of Religous Ideas}) und Guthrie (\emph{Faces in the cloud}). An dieser Stelle soll Pascal Boyers \emph{Und Mensch schuf Gott}~\footnote{\cite{boyer04}} als Stellvertreter für andere kognitive Ansätze dienen um die Animationsfilme auf Animismus zu untersuchen. In den nächsten Abschnitten 

\subsection*{Boyer: Und Mensch schuf Gott}
Boyer erklärt in der Einleitung seines Buches, dass es im Geiste des Menschen sei zu suchen. Jeder menschlicher Geist habe das Zeug religiös zu sein. Es ist somit nicht sein Ziel zu beweisen, dass es Gott nicht gibt und nur ein Produkt unserer Fantasie ist, sonder er möchte erklären können, warum wir das glauben was wir glauben.

Bei der Frage nach dem Ursprung der Religion tauchen immer wieder ähnliche intuitive Begründungen auf: \emph{Die Religion bietet Erklärungen, Die Religion spendet Trost, Die Religion sichert die gesellschaftliche Ordnung, Die Religion ist eine kognitive Täuschung}~\footnote{\textsc{Boyer 04: 14-15}}. Laut Boyer sind diese intuitiven Gewissheiten in ihrer Existenz zwar berechtigt, jedoch nicht dienlich dabei den Ursprung zu finden. Einen Ursprung im Sinne eines historischen Ereignis ist eine Wunschvorstellung, welche aus dem Wunsch entspringt eine Ursache zu haben, aus der sich alle weiteren Phänomene ableiten lassen würden. Ausführlich zeigt Boyer, wie man in jeder dieser Vorstellung Widersprüche findet oder sie schlicht nicht befriedigend sind, welche sie als Ursprung ungeeignet machen. Doch für jedes Gebiet, das er abarbeitet fügt er am Ende \glqq[einen] andere[n] Blickwinkel\grqq  hinzu, in dem er aus kognitiver Sicht den Wert dieser intuitiven Annahme beschreibt:

\begin{itemize}
	\item Das Erkenntnissystem des Gehirns produziert Erklärungen, oft ohne dass wir uns dessen gewahr sind. - Religion als Erklärung.
	\item Emotionale Programme sind für uns lebenswichtig~\footnote{Hier das Beispiel der Angst vor einem Raubtier. [S.34]} und somit ein Aspekt unseres entwicklungsgeschichtlichen Erbes. - Emotionen in der Religion.
	\item Die Untersuchung des sozialen Bewusstseins (soziale Intelligenz) kann Antworten auf die Frage nach den Erwartungen an das gesellschaftliche Leben und Moral geben. - Religion, Moral und Gesellschaft.
	\item Bei all den Übernatürlichen Informationen welcher der Geist bekommt, werden nur manche als plausibel erachtet und so angeeignet. - Religion und Denken.
\end{itemize}

Für den ersten Punkt das Beispiel vom Donnergrollen, welches das Grollen der Geister/Götter/Ahnen über ein Fehlverhalten gibt für Boyer zu wenig her für einen zu grossen Aufwand. Um laute, grollende, dumpfe Geräusche bei Stürmen zu erklären, muss eine komplett imaginäre Welt mit übernatürlichen Mächten vorausgesetzt werden, welche selbst noch mehr Fragen auf werfen: Woher sind sie gekommen? Wo sind sie? Warum sieht man sie nicht? Haben sie einen riesigen Mund um diese Geräusche zu erzeugen? Ist ein solcher Glauben verbreitet, so finden sich zu diesen Fragen auch Antworten. Diese Antworten sind aber teilweise so weit her geholt, dass es die Ausgangslage, die Geräusche eines Gewitters erklären zu wollen unwahrscheinlich macht.

Boyer ist sehr grosszügig mit einleuchtenden Beispielen, jedoch ist nicht immer ganz klar, wie viel von dem was er sagt wissenschaftlich erwiesen ist, und wie viel davon Spekulation ist. 

Jedenfalls führt er nun weitere Konzepte ein, welche für seinen Ansatz unausschliessbar sind. Er zählt diese als den Inhalt eines Werkzeugkasten auf. Für die weitere Arbeit werden diese Werkzeuge eine zentrale Rolle spielen, da diese die Methode zur Analyse liefern aus Sicht der Kognitiven Religionswissenschaft.

In seiner Erklärung geht Boyer von der Funktionsweise des Geistes im Allgemeinen aus, welche unabhängig von der Kultur überall gleich ist. Das scheint zu nächst eine Sackgasse, da kulturell grosse Unterschiede bezüglich religiöser Praktiken und Vorstellungen zu finden sind. Das geniale hier sei, argumentiert er, dass sich etwas so \emph{vielschichtiges} wie Religion durch etwas erklären lässt, was überall gleich ist (das Gehirn). Es ist jedoch notwendig zunächst mehr darüber zu wissen, wie das Gehirn Informationen aufnimmt und verarbeitet.[S.11]

Die Arbeit die ein Gehirn leistet ist lange unterschätzt worden. Einerseits muss man von der verbreiteten Annahme wegkommen, dass es sich beim Geist um ein leeres Gefäss handelt welches beliebig mit Informationen (Erziehung, Bildung und persönliche Erlebnissen) gefüllt werden kann. Anderseits auf von der Idee, dass der Geist mit wahllosen Informationen abgefüllt werden kann. Wir können uns bei weitem nicht alles merken und das ist auch gut so. Es braucht also etwas im Hirn, das relevante Informationen aus der Umwelt identifiziert und auf eine spezifische Weise zu verarbeiten mag. [S. 12]

\subsubsection*{Meme}
Boyer geht nun der Frage nach, wie religiöse Konzepte überhaupt entstehen. Die Bezeichnung Meme als Kulturelemente, also Vorstellungen, Werte, Geschichten und dergleichen, welche Menschen in ihrem Handeln beeinflusst und weitergegeben werden,~\footnote{\textsc{Boyer} 04: 50} wurde vom Evolutionsbiologen Richard Dawkins erstmals vorgestellt. Ein Meme bezeichnet demnach einen Bewusstseinsinhalt\footnote{wiki: \texttt{https://de.wikipedia.org/wiki/Mem} 24.08.15}, welcher durch Kommunikation in der Gesellschaft weitergegeben und somit vervielfältigt werden kann. Es ist das soziokulturelle Pendant zu den biologischen Genen in der Evolution. So dann lassen sich die Meme auch ähnlich wie die Gene beschreiben: Information wird durch Kommunikation weiter gegeben (replizieren). Dadurch werden die Inhalte aber nicht einfach verbreitet, sonder auch leicht (oder schwerwiegend) abgeändert (mutieren).~\footnote{Die Information ändert sich nicht erst durch deren Weitergabe. Etwas das wir erfahren wird bei verschiedenen Menschen bereits anders verarbeitet. Somit können zwei Personen genau das gleiche hören und eine andere Version des Inhalts in ihrem Bewusst sein haben.} Schliesslich werden nur jene Memes tatsächlich weiter gegeben oder überhaupt erst erinnert, welche einprägsam sind (selektieren).

Boyer beschreibt die Meme als wunderbare Ausgangslage, will ihnen aber nicht mehr als genau das zugestehen. Seine Kritik liegt darauf, dass es keine Replikation der eigentlichen Informationen gibt. Inhalte werden nicht faktisch übergeben, sondern jeweils neu konstruiert. Zwei Menschen können zwei faktisch identische Aussagen machen, aber jeder hat die Information, welche er wiedergibt auf seine eigene Art rekonstruiert. Entsprechend stellt Boyer als nächstes seine Theorie zum Einfangen von Vorstellungen durch Schablonen vor. 

\subsubsection*{Schablonen, Vorstellungen und das Schlussfolgerungssystem}
Ein grosser Teil dessen, was wir wissen musste uns niemand faktisch erzählen. Eine erstaunliche Eigenschaft unserer unserer geistigen Fähigkeit ist es durch die Kombination bereits existierendem Wissen und der Hinzugabe einer neuen Information mehr Wissen zu generieren. Boyer zeigt dies anhand eines Beispiels mit einem Kind auf, mit dem Kind als eine Person, dessen Wissen erweitert wird. Zeigen wir einen Kind zum ersten Mal ein Seehund, so hat es keine weiteren Informationen darüber als den Namen und seine äussere Erscheinung. Dennoch wird das Kind erwartet, dass das Seehund isst, schläft und dass es Kinder haben kann. Diese Information über das Seehund hat das Kind geschlussfolgert, indem es eine Annahme gemacht hat: Das Seehund ist ein Säugetier. Diese essen, schlafen und bekommen Kinder. Boyer bezeichnet das Säugetier als eine \emph{Schablone}. Mit dieser Säugetier-\emph{Schablone} hat das Kind eine Seehund-Vorstellung gebildet.\footnote{\textsc{Boyer 04: 59}}   

Über diesen Schablonen, welche dazu da sind verschiedene Konzepte zusammen zufassen und dabei helfen aus einem Informationsstück mehr Information zu schaffen, stehen die ontologischen Kategorien. Boyer nennt fünf davon: \emph{Person, Tier, Pflanze, Naturobjekt, Werkzeug}. ie Schablonen, und im stärkerem Masse die ontologischen Kategorien seinen das, was über Kulturen hinweg universell sei. Erst bei den konkreten Konzepten ergäbe sich die Varianz. Die Vorstellungen welche von Angehörigen einer gleichen Gruppe anhand einer Schablone hergestellt werden sind sich in der Regel ähnlich. Die Vorstellungen einer anderen Gruppe kann stark davon abweichen, trotzdem die gleiche Schablone benutzt wurde.

Dieses System der Schablonen welche mit Schlussfolgerungen zu Vorstellungen beziehungsweise zu Konzepten führen überträgt Boyer nun auf die Religion. Es gibt demnach Schablonen für religiöse Vorstellungen. Die Schablonen von religiösen Vorstellungen teilen wir uns universell, doch die Konzepte variieren stark regional. Dass übernatürliche Kräfte unsichtbar sind/sein können ist man sich breitgänig einig. Wenn es aber darum geht, was und warum übernatürliche Kräfte etwas tun, so gehen die Vorstellungen weit auseinander.

Letztlich müsse berücksichtigt werden, dass die kulturelle Varianz in der Regel geringer ist, als man annimmt. Beim Übermitteln findet durch die Schablonen ein Filtern der gegebenen Informationen statt, so dass daraus voraussagbare Strukturen gebaut werden.[65]

\subsubsection*{Beschaffenheit des Übernatürlichen}
Es folgt also die Suche nach dem mentalen Rezept für religiöse Vorstellungen. Mit einem Versuch verschiedener mehr oder weniger potenten religiösen Aussagen versucht Boyer dem Leser zu zeigen, dass man der Intuition folgend gewisse Aussagen über übernatürliche Wesen direkt ausschliessen kann, während man bei andern sofort glauben würde, dass es sich um eine existierende religiöse  Vorstellung handelt. 

Folgende Kriterien für eine erfolgreiche (religiöse) Vorstellung kommen dabei heraus:

Erstens habe jedes übernatürliche Konzept die Tendenz, eine seiner ontologischen Annahmen zu verletzen. Ein Geist gehört zur ontologischen Kategorie PERSON, doch dass er keinen physikalischen Körper hat bricht diese Ontologie. Ein anderes Beispiel für diese Kontraintuitivität findet sich auch in der Natur, bei einer Raupe, welche nach der Metamorphose zu einem Schmetterling wird. Die Erwartung für ein TIER ist, dass es sich im Laufe des Wachstums nur durch Grösse und Masse verändert, jedoch nicht, dass es zu einem anderen TIER wird. Dabei ist es wichtig, dass tatsächlich gegen die ontologische Kategorie verstossen wird und es sich nicht nur um eine Merkwürdigkeit handelt. Eine PERSON welche ihre Hautfarbe ändert, ist dem zu Folge weniger erfolgreich, als eine PERSON, welche durch Wände gehen kann. 

Zweitens hat religiöses Denken die Tendenz auf Leute-ähnliche übernatürliche Wesen zu fokussieren, welche Zugang zu sozial-relevanten Informationen haben. Der Austausch von Informationen ist für den Menschen kritisch so Boyer. Wir sind darauf angewiesen, dass andere in der Gruppe Dinge wissen und uns diese übertragen können. Bei religiösen Vorstellungen wird in der Regel davon ausgegangen, dass dieses Wesen die moralische Haltung eines Individiums oder auch der ganzen Gruppe teilt. Dieses Wesen weiss Bescheid darüber, wenn schlechte Dinge geschehen. Die Person erwartet dann, dass dieses Wesen wertet und allenfalls böse wird und die Person für deren Verhalten bestraft. Eine solche Vorstellung macht diese Wesen zu wichtigen Subjekten für Gedanken und Diskussionen in einer Gruppe.

Im dritten Punkt werden religiöse Rituale mit aus den Reinigungsritualen hergeleitet. Unser mentales System behandelt Krankheit mit Abscheu zum eigenen Schutz vor einer unsichtbaren Gefahr. So sollen auch Rituale in religiösen Vorstellungen vor allem zum Schutz vor unsichtbaren Gefahren und Mächten sein.

Letztlich setzt Boyer noch eine Fokus auf die Leiche und wie Menschen damit umgehen. Im Prinzip sind Leichen eine Spezialfall des ersten Punktes, wo es um die Verletzung der ontologischen Kategorie geht. Der Mensch, den wir sehen oder sogar kannten wird in unserem System der ontologischen Kategorie PERSON zugeordnet. Gleichzeitig verletzt die Leiche dieses Menschens jegliche Kriterien der PERSON Kategorie und entspräche demnach der Kategorie NATUROBJEKT und sollte unsere Abscheu vor Krankheit wecken. Dieser Widerspruch macht die menschliche Leichte zum Prototyp für religiöse Objekte überall auf der Welt.

\smallskip
Zusammenfassendes bla?

Zusammenfassung: http://serendip.brynmawr.edu/exchange/node/1581
Zusammenfassung 2: http://mason.gmu.edu/~rhanson/religion.html
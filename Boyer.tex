%!TEX root = Animismus_in_Anime.tex
In dieser Underdisziplin der Religionswissenschaft wird Religion oder religiöse Phänomene aus der Perspektive der Kognitions- und Evolutionswissenschaft betrachtet. Es wird versucht zu erklären, weshalb religiöse Praktiken und und Denkweisen universell verbreitet sind.~\footnote{wikipedia: \texttt{https://en.wikipedia.org/wiki/Cognitive\_science\_of\_religion} und \texttt{https://de.wikipedia.org/wiki/Kognitive\_Religionswissenschaft} 24.08.15}

Die kognitive Religionswissenschaft ist eine eher junge Disziplin, welche sich erst Ende des zwanzigsten Jahrhunderts etabliert hat. Zu ihren Begründern gehören unter anderen E. Thomas Lawson und Robert McCauley (\emph{Rethinking Religion: Connecting Cognition and Culture and Bringing Ritual to Mind: Psychological Foundations of Cultural Forms}), Pascal Boyer (\emph{Naturalness of Religous Ideas}) und Guthrie (\emph{Faces in the cloud}).

An dieser Stelle soll Pascal Boyers \emph{Und Mensch schuf Gott}~\footnote{\cite{boyer04}} als Stellvertreter für andere kognitive Ansätze dienen um die Animationsfilme auf Animismus zu untersuchen. 

Im Geiste des Menschen ist zu suchen. Jeder menschlicher Geist hat das Zeug religiös zu sein. Boyers Theorien stützen sich auf Funktionieren des Geistes im Allgemeinen, unabhänig von Kultur. Das scheint zu nächst im Widerspruch, da es ja gerade kulturell grosse Unterschiede bezüglich religiöser Praktiken und Vorstellungen gibt. Das geniale hier sei, dass sich etwas so \emph{vielschichtiges} wie Religion durch etwas erklären lässt, was überall gleich ist (das Gehirn). Es ist jedoch notwendig zunächst mehr darüber zu wissen, wie das Gehirn Informationen aufnimmt und verarbeitet.[S.11]

Die Arbeit die ein Gehirn leistet ist lange unterschätzt worden. Einerseits muss man von der verbreiteten Annahme wegkommen, dass es sich beim Geist um ein leeres Gefäss handelt welches beliebig mit Informationen (Erziehung, Bildung und persönliche Erlebnissen) gefüllt werden kann. Anderseits auf von der Idee, dass der Geist mit wahllosen Informationen abgefüllt werden kann. Wir können uns bei weitem nicht alles merken und das ist auch gut so. Es braucht also etwas im Hirn, das relevante Informationen aus der Umwelt identifiziert und auf eine spezifische Weise zu verarbeiten mag. [S. 12]

Die relevanten Informationen werden nicht mit den Genen weiter gegeben. Aber das System, welches die Arbeit welche dahinter liegt verrichtet schon. Denn wenn man ein normale menschliches Gehirn besitzt kann daraus noch nicht geschlossen werden, dass dieser Mensch auch Religion hat. Es bedeutet lediglich, dass sich dieser Mensch Religion zu eigen machen kann. Daher ist die wichtige Frage: Wie muss der Nährboden für Informationen aussehen, damit sie als relevant gelten und erfolgreich verarbeitet werden. 

Bei der Frage nach dem Ursprung der Religion tauchen immer wieder ähnliche intuitive Begründungen auf: \emph{Die Religion bietet Erklärungen, Die Religion spendet Trost, Die Religion sichert die gesellschaftliche Ordnung, Die Religion ist eine kognitive Täuschung}~\footnote{\textsc{Boyer 04: 14-15}}. Laut Boyer sind diese intuitiven Gewissheiten in ihrer Existenz zwar berechtigt, jedoch nicht dienlich dabei den Ursprung zu finden. Einen Ursprung im Sinne eines historischen Ereignis ist eine Wunschvorstellung, welche aus dem Wunsch entspringt eine Ursache zu haben, aus der sich alle weiteren Phänomene ableiten lassen würden. Ausführlich zeigt Boyer, wie man in jeder dieser Vorstellung Widersprüche findet oder sie schlicht nicht befriedigend sind, welche sie als Ursprung ungeeignet machen. Doch für jedes Gebiet, das er abarbeitet fügt er am Ende \glqq[einen] andere[n] Blickwinkel\grqq  hinzu, in dem er aus kognitiver Sicht den Wert dieser intuitiven Annahme beschreibt:

\begin{itemize}
	\item Das Erkenntnissystem des Gehirns produziert Erklärungen, oft ohne dass wir uns dessen gewahr sind. - Religion als Erklärung.
	\item Emotionale Programme sind für uns lebenswichtig~\footnote{Hier das Beispiel der Angst vor einem Raubtier. [S.34]} und somit ein Aspekt unseres entwicklungsgeschichtlichen Erbes. - Emotionen in der Religion.
	\item Die Untersuchung des sozialen Bewusstseins (soziale Intelligenz) kann Antworten auf die Frage nach den Erwartungen an das gesellschaftliche Leben und Moral geben. - Religion, Moral und Gesellschaft.
	\item Bei all den Übernatürlichen Informationen welcher der Geist bekommt, werden nur manche als plausibel erachtet und so angeeignet. - Religion und Denken.
\end{itemize}

Boyer ist sehr grosszügig mit einleuchtenden Beispielen, jedoch ist nicht immer ganz klar, wie viel von dem was er sagt wissenschaftlich erwiesen ist, und wie viel davon Spekulation ist. 

Jedenfalls führt er nun weitere Konzepte ein, welche für seinen Ansatz unausschliessbar sind. Er zählt diese als den Inhalt eines Werkzeugkasten auf. Für die weitere Arbeit werden diese Werkzeuge eine zentrale Rolle spielen, da diese die Methode zur Analyse liefern aus Sicht der Kognitiven Religionswissenschaft.

\subsubsection*{Meme}
Die Bezeichnung Meme als Kulturelemente, also Vorstellungen, Werte, Geschichten und dergleiche, welche Menschen in ihrem Handeln beeinflusst und weitergegeben werden,~\footnote{\textsc{Boyer} 04: 50} wurde vom Evolutionsbiologen Richard Dawkins erstmals vorgestellt. Ein Meme bezeichnet demnach einen Bewusstseinsinhalt\footnote{wiki: \texttt{https://de.wikipedia.org/wiki/Mem} 24.08.15}, welcher durch Kommunikation in der Gesellschaft weitergegeben und somit vervielfältigt werden kann. Es ist das soziokulturelle Pendant zu den biologischen Genen in der Evolutions. So dann lassen sich die Meme auch ähnlich wie die Gene beschreiben: Information wird durch Kommunikation weiter gegeben (replizieren). Dadurch werden die Inhalte aber nicht einfach verbreitet, sonder auch leicht (oder schwerwiegend) abgeändert (mutieren).~\footnote{Die Information ändert sich nicht erst durch deren Weitergabe. Etwas das wir erfahren wird bei verschiedenen Menschen bereits anders verarbeitet. Somit können zwei Personen genau das gleiche hören und eine andere Version des Inhalts in ihrem Bewusst sein haben.} Schliesslich werden nur jene Memes tatsächlich weiter gegeben oder überhaupt erst erinnert, welche einprägsam sind (selektieren). Die Frage wäre nun, wo oder was entscheidet welche Memes weitergegeben werden und welche durch das Raster fallen, weil sie nicht relevant sind? 

Boyer beschreibt die Meme als wunderbare Ausgangslage, will ihnen aber nicht mehr als genau das zugestehen. Seine Kritik liegt darauf, dass es keine Replikation der eigentlichen Informationen gibt. Inhalte werden nicht faktisch übergeben, sondern jeweils neu konstruiert. Zwei Menschen können zwei faktisch identische Aussagen machen, aber jeder hat die Information, welche er wiedergibt auf seine eigene Art rekonstruiert. Entsprechend stellt Boyer als nächstes seine Theorie zum einfangen von Vorstellungen durch Schablonen vor. 

\subsubsection*{Schablonen und Vorstellungen}
Ein grosser Teil dessen, was wir wissen musste uns niemand faktisch erzählen. Eine erstaunliche Eigenschaft unserer unserer geistigen Fähigkeit ist es durch die Kombination bereits existierendem Wissen und der Hinzugabe einer neuen Information mehr Wissen zu generieren. Boyer zeigt dies anhand eines Beispiels mit einem Kind auf, mit dem Kind als eine Person, dessen Wissen erweitert wird. Zeigen wir einen Kind zum ersten Mal ein Walross, so hat es keine weiteren Informationen darüber als den Namen und seine äussere Erscheinung. Dennoch wird das Kin erwartet, dass das Walross isst, schläft und dass es Kinder haben kann. Diese Information über das Walross hat das Kind geschlussfogert, indem es eine Annahme gemacht hat: Das Walross ist ein TIER. Tiere essen, schlafen und bekommen Kinder. Boyer bezeichnet TIER als eine \emph{Schablone}. Mit dieser TIER-\emph{Schablone} hat das Kind eine Walross-Vorstellung gebildet.\footnote{\textsc{Boyer 04: 59}}   

Boyer stellt es zur Grundannahme, dass es deutlich weniger Schablonen\footnote{Beispiele die er nennt und auch öfters gebraucht: TIER, WERKZEUG, UNREINE SUBSTANZ, NATUROBJEKT, PERSON, PFLANZE. Bei diesen Schablonen handelt es sich vorwiegend um konkrete Dinge. Boyer erwähnt aber auch GESICHT als eine Schablone im abstrakten Sinne (\emph{das Gesicht verlieren}). \textsc{Boyer 04: 61}} als Vorstellungen gibt und dass diese Schablonen universell sind, im Gegensatz zu den Vorstellungen. 

\subsubsection*{epidemiologisches Modell}
Als weiteres Element in seinem Werkzeugkasten stellt Boyer die Kulturepidemie~\footnote{\textsc{Boyer 04: 62ff}} vor. Religiöse Vorstellungen und Phänomene betreffen in der Regel eine beliebig grosse Gruppe von Menschen.
\todo{Haaaa?}
Man kann also Religion als eine besondere Form der mentalen Epidemie erklären. Durch die Ausbreitung der Epidemie formen Menschen (auf Basis von unterschiedlichen Informationen) ähnlich strukturierte Formen religiöser Vorstellungen und Normen.\footnote{\textsc{Boyer 04: 64}} Hier setzt nun das Konzept der Schablonen und Vorstellungen an. Die Vorstellungen welche von Angehörigen einer gleichen Gruppe anhand einer Schablone hergestellt werden sind sich in der Regel ähnlich. Die Vorstellungen einer anderen Gruppe kann stark davon abweichen, trotzdem die gleiche Schablone benutzt wurde. So lasse sich das Beispiel mit den Tier Vorstellungen auch auf religiöse Vorstellungen übertragen. Demnach gibt es eine Schablone für religiöse Vorstellungen. Wie bei der Tierschablone können religiöse Vorstellungen übereinstimmen (einigermassen ähnlich strukturiert sein), obwohl die Information auf deren sie aufbaut von Mensch zu Mensch verschieden ist. Letztlich muss berücksichtig werden, dass die kulturelle Varianz in der Regel geringer ist, als man annimmt. Beim Übermitteln findet durch die Schablonen ein Filtern der gegebenen Informationen statt, so dass daraus voraussagbare Strukturen gebaut werden.[65]

\subsubsection*{Beschaffenheit des übernatürlichen}
Es folgt also die Suche nach dem mentalen Rezept für religiöse Vorstellungen. Mit einem Versuch verschiedener mehr oder weniger potenten religiösen Aussagen versucht Boyer dem Leser zu zeigen, dass man der Intuition folgend gewisse Aussagen direkt ausschliessen kann, während andere absolut denkbar wären als religiöse Vorstellung. 

Folgende Anleitung kristallisiert sich mit der Zeit heraus: In einer Aussage wird ein Vertreter einer ontologischen Kategorie\footnote{Whats that precious?} gewählt und mit einem Merkmal/Bemerkung behaftet, welche kontraintuitive bezüglich der ontologischen Kategorie ist. Ein Beispiel von Kontraintuitivität ist zum Beispiel die Raupe, welche nach der Metamorphose zu einem Schmetterling wird. Die Erwartung für ein TIER ist, dass es sich im Laufe des Wachstums nur durch grösse und Masse verändert, jedoch nicht, dass es zu einem anderen TIER wird. Als fiktives Beispiel einer religiösen Aussage nennt Boyer hier. 
\begin{quote}Manche Ebenholzbäume behalten Gespräche in Erinnerung, die Menschen in ihrem Schatten geführt haben.\end{quote} 

Hier ist der Ebenholzbaum ein Vertreter der onologischen Kategorie PFLANZE und hat das kontraintuitiv Merkmal eine geistige Präsenz zu haben. Dabei ist es wichtig, dass tatsächlich gegen die ontologische Kategorie verstossen wird und nicht nur einfach eine Merkwürdigkeit. Eine PERSON welche ihre Hautfarbe ändert, ist dem zu Folge weniger erfolgreich, als eine PERSON, welche durch Wände gehen kann. 

Diese Verstösse bilden den Kern der religiösen Aussage. Es ist aber auch üblich, den ontologischen Kategorienverstoss mit weiteren Verstössen auszuschmücken, welche aber nicht mehr kontraintuitiv bezüglich der Ontologischen Kategorie sind.

\begin{itemize}
	\item Nur eine Stufe der Ontologie brechen. (PERSON zu TIER, NATUR OBJEKT zu TIER)
	\item Wichtigkeit der Information. INformation die von der Gruppe von Menschen verwaltet wird. Wichtigkeit der Kommunikation. Wichtigkeit andere Menschen zu verstehen.
\end{itemize}
\subsubsection*{komplexität des Hirns: was damit gemacht wird}

TODO\todo{Fertig machen -> Google Drive: Boyer Methodik}
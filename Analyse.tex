%!TEX root = Animismus_in_Anime.tex
Für Völker und Ethnien, von welchen beliebte Beispiele von Geistern, Hexen und anderen übernatürlichen Phänomenen kommen, sind diese Phänomene real. So nimmt man als westlicher\todo{wie lässt sich das beschreiben?} Betrachter eine distanzierte analytische Position ein. Es wird versucht zu erklären ohne dass man das was man untersucht glaubt.\todo{formulieren!} In Filmen, wo sich der Betrachter mit den Charakteren identifiziert und in der fiktiven Umgebungswelt lebt, schaffen den Rahmen zumindest temporär in ein System von übernatürlichen und phantastischen Phänomenen abzutauchen, ohne seine rationale und faktische Weltsicht aufzugeben zu müssen. Die Vorstellung von Animismus konzentriert sich nach wie vor auf die Untersuchung von Gruppen und Ethnien, welche nicht von westlichen Kolonialismus überrannt wurden und sich ihre Kultur und somit auch Teile ihre Religion erhalten konnten. Das Studium von Animismus ist somit fast zwangsläufig eines, das einem fort führt aus seiner eigenen Welt.\todo{Umgebung/Kultur/Glaube/Weltanschauung} Doch durch die Analyse von Filmen (im weiteren Sinne auch von Literatur) gibt uns die Möglichkeit uns selbst zu studieren.

Es mag nach dieser Argumentation etwas widersprüchlich scheinen, dass ausgerechnet japanische Animationsfilme Gegenstand der Untersuchung sind. Es gibt im Bereich der Animationsfilme genügend Alternativen\footnote{Um ein Beispiel zu nennen: \textsc{Song of the Sea} (2014) und \textsc{The Book of Kells} (2009) produziert von Cartoon Saloon, einem irischen Animations Studio.}, welche kulturell näher wären. Doch am Ende ist die kulturelle Prägung des Rezeptionisten für diese Untersuchung entscheidend und nicht die des Filmes.

In den Filmen \emph{Prinziessin Mononoke} und \emph{Das wandelnde Schloss} haben wir einerseits ist da das alte, Mythen und Legenden reiche Japan, in dem Götter und Geister welche zwischen Sterblichen wandeln. Wenn auch die Welt angereichert ist mit phantastischen Wesen, so erkennen wir darin doch unserer eigene Vergangenheit in dieser Welt. Anderseits haben wir eine Märchenwelt, ein gutes Jahrhundert jüngeres Europa, wo Magie und Zauberei zum Alltag gehören. Diese Welt ist uns zeitlich zwar näher, doch gerade der breite Gebrauch von Magie in einer Zeit an die wir uns noch zu erinnern glauben, entfremdet sie für den Betrachter. In beide Welten ist das Übernatürliche Grund zur Furcht oder Faszination, jedoch ist niemand über das Vorhandensein vom diesem überrascht. \par

\medskip

In der folgenden Untersuchung werden beide Filme
In der folgenden Untersuchung der beiden Filme werden zwei Schwerpunkte gesetzt: Erstens wird untersucht wo und wie man Animismus im Sinne von Graham Harveys findet und zweitens sollen die Filme nach Elementen untersucht werden, welche Pascal Boyers Kriterien genügen.

\subsection{Mononoke}
\subsubsection{Konflikt als Resultat fehlenden Respekts}
Durch seinen historisch Hintergrund fügt sich die Welt des Films von sich aus in dem Shinto und somit in einen geschichtlichen Animismus ein. Die Welt vom Mononoke ist voll von verschiedenen \emph{Leuten}. Die meisten davon sind Menschen aber nicht wenige sind Nicht-Menschen-Leute. Es gibt verschiedene Interessensgruppen und entscheiden ist dabei nicht, ob es sich um Menschen oder Andere-als-Menschen handelt. Der Konflikt zwischen den Eisenschmiedebewohner und den Tieren des Waldes ist nicht grösser als der zwischen den Bewohnern der Eisenschmiede und den Samurai des Königs und die Affengötter werfen mit Steinen nach den Wolfsgöttern. Der Respektvolle Umgang zwischen den verschiedenen Menschen und Andere-als-Menschen Gruppe steht im Zentrum der Geschichte. Der rücksichtslose Umgang mit den andern ist der Grund der vielen Konflikte. Am Ende ist das Schlimmste zwar abgewandt, aber die Grundproblematik besteht und einzig ein gegenseitiger Respekt kann weiterhin den Frieden halten.

Miyazakis Botschaft in Mononoke ist ein Schuss ins Blaue für den Animismus den Harvey plädiert. Es ist hingegen schwieriger Harveys Beispiele für Animismus in Mononoke anzuwenden. Das Übernatürliche ist so real und fassbar in der Welt von Mononoke, dass es kaum Platz für das stille Belebte lässt wie zum Beispiel ein geschnitztes Anhänger der Maori oder ein Stein der Ojibwa. Eine Ausnahme dazu sind höchstens die Kodama, die kleine stummen Geister, welche Zeichen für einen gesunden Wald sind.\todo{mehr dazu?} 

\subsubsection{Die ontologische Kategorie TIER}
Boyers Ansatz um die Faszination der übernatürlichen zu Erklären bringt zunächst keine überraschende Resultate. Viel mehr haben wir mit den Tiergöttern einfache vorzeige Beispiele des ontologischen Verstosses der Kategorie TIER. Die Tiergötter des Waldes, die Wölfe, die Keiler und auch die Affen, handeln, sprechen und denken wie eine PERSON. Angereichert werden sie durch spezielle Merkmale, welche aber nur einen Verstoss der Schablone für ihre jeweilige Art darstellt. Die Tiergötter sind alle grösser als ihre natürlichen Verwandten, die Wölfin Moro besitzt zwei Schwänze und der Keiler Okkoto hat eine zusätzliche Reihe von Hauern. Jedoch sind das nicht die ausschlaggebenden Elemente. Das sehen wir am Beispiel von Ashitakas Reittier. Yakul hat ein kräftiges rot-oranges Fell welches cremig weiss ist an seiner Unterseite. Er hat den Körperbau eines Hirsches, ist aber kräftiger und grösser gebaut und er hat zwei lang Hörner anstelle eines Geweihs. Das Tier passt also zu keiner konkreten Vorstellung die wir haben, aber er passt zur Schablone welche eben Rehe, Gämsen und Elche einschliesst. Damit ist er höchsten besonders, aber er reiht sich nicht in die Vorstellungen religiöser Phänomene ein.

Nach ausführlichem Suchen könnte man der Wolfgöttin Moro, beziehungsweise dem Mädchen San einen weiteren Bruch zuschreiben, da die beiden in einer Mutter-Tochter Beziehung stehen und das über die ontologische Kategorie von TIER beziehungsweise PERSON drüber hinaus. Da es sich aber explizit um eine Adoption handelt, sollte dieser Punkt nicht für sich alleine stehen und in das PERSON sein von Moro eingeschlossen werden.

Interessanter und zugleich ungleich schwieriger für die Untersuchung mit Boyers Methoden ist der Waldgott Shishigami. Es ist nicht ganz einfach den Waldgott einzuordnen. Während dem Tag streift er mit einer Herde Rehen durch den Wald. Bei Nacht wechselt er die Form und kann ohne Widerstand durch den Wald laufen. Er ähnelt einem Hirsch, hat aber ein menschliches Gesicht. Er spricht nicht, aber mit seiner Berührung kann er Leben nehmen oder geben. Er läuft im Wald und ist zugleich der Geist des Waldes. Shishigami scheint voller Brüche und Verstosse in unseren Schablonen und Kategorien. Er bricht so viele Annahmen, dass es schwer ist ihn überhaupt einer ontologischen Kategorie zu zuordnen. Letztlich ist jedoch weniger sein (nicht ausfindbarer) Kategoriebruch das was ihn seiner Darstellung von einem religiösen Phänomen ausmacht. Wichtiger ist die Annahme, dass der Shishigami Dinge \emph{weiss}. Als Rezeption und als auch die Charakter im Film fragt man sich, wie der Waldgott entscheidet, wen er tötet und wen er heilt. Es könnte pure Willkür sein, doch der Gedanke liegt näher, dass der Waldgott über \emph{Informationen} verfügt. Wie Boyer das beschreibt zeigt sich dann auch, dass dieses Wissen eine Wichtige Rolle im sozialen Umfeld spielt. Als der Keilergott halb Wahnsinnig zum Waldgott stürmt, geht er davon aus, dass Shishigami \emph{weiss}, dass er und seine Krieger tapfer gekämpft haben und den Wald verteidigen.

Im Ritual, welches die Seherin, aus dem Dorf von dem Ashitaka stammt, soll der böse Geist des gefallenen Dämon besänftigt und das Dorf somit geschützt werden. Somit entspricht dieses Ritual Boyer Beschreibung von religiösen Ritualen, da es im Kern darum geht sich präventiv vor etwas zu schützen. Um den unheimlichen Eindruck zu verstärken den man sowie so schon vor Leichen hat, kommt hier hinzu, dass der Dämon obwohl er schon gestorben ist noch einen Fluch für alle hörbar spricht.\todo{Besseren Abschluss/übergang für nächstes Kapitel?}

\subsection{Wandelndes Schloss}
\subsubsection{Alles lebt}
Aus Sicht des alten Animismus ist es etwas schwieriger in diesem Film anzusetzen als es bei Mononoke war. Abgesehen von der Magie gleicht die Welt in der die Geschichte spielt unserer Welt vor nicht all zu langer Zeit. Eine Zeit in welcher man anderen Kulturen nach gesagt hat, dass sie \emph{primitiv} sind, weil sie an eine beseelte Welt glauben. Natürlich lässt sich unsere Vergangenheit nicht mit der Märchenwelt aus dem Film gleichsetzen, aber es soll gezeigt sein, dass der Animismus hier anders zu finden ist als in Mononoke.

Calcifer redet und interagiert mit den anderen Bewohnern des Schlosses. Das macht ihn, wenn man Harveys Ansatz folgt, eindeutig zu einer Person. Um es für den Betrachter einfacher zu machen, das Feuer als Person zu sehen bekommt Calcifer zwei grosse Glupschaugen und ein Mund von variabler Grösse, in das er sich gerne Holzstücke reinstopft. Die Augen sind in der Regel rund, weisse Kreise mit schwarzen Pupillen. Doch mit dem flackern seines ganzen Körpers verändert sich die Form der Augen zwischen durch unmerklich, so dass sie schlitzförmiger werden und einen gefährlichen (dämonischen) Eindruck machen. Seine Flammenform, welche keine feste Umrisslinie hat flackert beständig und von Zeit zu Zeit lösen sich kleine Flämmchen von ihm. Calcifer gehört also sicherlich zu jenen Anders-als-Menschen mit denen mal alleine der Vorsicht wegen mit Respekt behandeln sollte. Obwohl das Feuer freundlich sein kann, schimmert immer wieder seine dämonische und unberechenbare Seite hindurch.

Aus Sicht von Harvey macht es wenig Sinn zwischen dem Feuer Calcifer und dem wandelnden Schloss zu unterscheiden, da beides das gleiche ist.

Sophie zeigt (fast) allen gegenüber einen respektvollen, hilfsbereiten Umgang. Sei dies nun Hauro, Markl (Hauros junger Assisten) oder der Hexe aus dem Ödland\todo{es gibt hier verschiedene Situationen.} als Menschen oder dem Rübenkopf und Calcifer als Andere-als-Menschen. Was einem nur klar wird, wenn man auch die Buchvorlage von Jones gelesen hat, ist, dass Sophie selbst eine magische Fähigkeit hat Dinge zu \emph{beleben}. Es ist daher also nicht wunderlich, dass Sophie vorsichtshalber mit allem einen anständigen Umgang pflegt. 

\subsubsection{Ist ein Haus ein WERKZEUG?}
Ist ein Haus ein Werkzeug? Im ersten Moment scheint die Idee etwas abwegig, doch das liegt wohl daran dass das gleiche System welches Boyer beschreibt bei der Reflextion des Wortes Werkzeug die Schablone \emph{Handwerkzeuge} nimmt. Ein Schraubenzieher, ein Hammer, vielleicht auch ein Handmixer - das alles passt zur Schablone der Handwerkzeuge. Erst mit einem Schritt zurück (oder mit Hilfe des Ausschlusverfahren), kommt man zum Schluss, dass auch ein Haus ein Werkzeug ist. Ein Werkzeug ist etwas (in der Regel) Menschengemachtes das eine Funktion oder einen Zweck erfüllen soll. Mit dieser Beschreibung können wir also getrost sagen, dass ein Haus zur ontologischen Kategorie WERKZEUG gehört. Für das wandelnde Schloss erwarten wir nach Boyer jetzt einen Bruch, welche diese Kategorie verletzt. Eine weitere Eigenschaft von Werkzeugen ist auch, dass die (so fern sie funktionieren) die Absicht eines Menschen erfüllen. Werkzeuge haben keine Selbstbestimmung. Und genau darüber verfügt das wandelnde Schloss. Es lässt sich argumentieren, dass das Schloss vom Feuerdämon Calcifer belebt und gesteuert wird. Die Verletzung der ontologischen Kategorie bleibt jedoch die gleiche mit dem Unterschied, dass es statt dem WERKZEUG zu dem das Haus gehört nun dem NATUR OBJEKT ist. In beiden Fällen wird der Eindruck dadurch verstärkt, dass Eigenschaften einer anderen ontologischen Kategorie in das Aussehen einfliessen. 
%!TEX root = Animismus_in_Anime.tex
\subsection{Animismus in Anime}
Alt oder neuer Animismus. Japanischer Shinto. Wo zu finden und was heisst das?

\subsubsection*{Mononoke}
Da historisch passt sehr gut in Shinto und somit in den Geschichtlichen Animismus (in wie fern den alten). Harveys Ansatz vom respektvollen Leben mit andern people ist zentrale Botschaft des Filmes. Boyer lässt sich zwar Anwendung, aber ohne überraschende Erkenntnis. Tiergötter sind TIERE mit PERSONEN-Eigenschaften. Moro: ontologischer Bruch: hat eine Menschentochter. Interessant hier Shishigami: Auch wenn die gründe nicht erklärt werden, so geht man davon aus, dass der Gott entscheidetet wen er leben lässt und wen er tötet. Er muss also eine Art von Information besitzen. Anderes Beispiel aus Boyer: Tote und das Ritual -> Nago. Das NATUR OBJEKT toter Nago bewegt sich nicht mehr, aber er spricht! Die Seherin vollführt ein Ritual um sich und ihr Dorf vor dem Zorn des Gefallen Gottes zu schützen. Diese beiden helfen aber nur ein religiöses Phänomen zu identifizieren, jedoch nicht umbedingt zu erklären, warum das auch für uns westler so interessant ist. Oder ist es gerade, dass das Ritual anders ist, aber die Notwendigkeit (Schablone davon) in uns allen vorhanden und wir es deswegen verstehen? Letztlich noch kurz die falschen Wildschweine: Unheimlicfh auch für den Betrachter. Umgekehrt von den Tiergöttern: PERSON die aussieht wie TIER.

\subsubsection*{Wandelndes Schloss}
Typisch Animistisch gibt es wenig hier anzusetzten. Aus einer Zeit (Europa Epoche) wo man den Animismus nicht mehr gefolgt ist. Animismus gehört nicht mehr zur Narration. Man kann höchstens Animistische Elemente identifizieren. 

Wir wissen, dass Calcifier das Schloss belebt - animiert. Doch wie wird dieser Eindruck, dass eine Konstruktion belebt ist an den Rezeptionist vermittelt?

Obwohl Calcifer das Schloss gebaut hat, steuert und auch seine Emotionen durch das Schloss geäussert werden, so es für den Rezipienten doch ein eigenes Wesen.


Calcifer redet und interagiert mit den anderen Bewohnern des Schlosses. Das macht ihn, wenn man Harveys Ansatz folgt eindeutig zu einer Person. Um es für den Betrachter einfacher zu machen, das Feuer als Person zu sehen bekommt Calcifer zwei grosse Glupschaugen und ein Mund von variabler Grösse, in das er sich gerne Holzstücke reinstopft. Die Augen sind in der Regel rund, weisse Kreise mit schwarzen Pupillen. Doch mit dem flackern seines ganzen Körpers verändert sich die Form der Augen zwischen durch unmerklich, so dass sie schlitzförmiger werden und einen gefährlichen (dämonischen) Eindruck machen. Seine Flammenform, welche keine feste Umrisslinie hat flackert beständig und von Zeit zu Zeit lösen sich kleine Flämmchen von ihm.

\subsection{Kognitiver Ansatz}
Welten unterscheidung: Fiktive Welt ist eventuell der Welt welcher viele Beispiele entnommen wurden näher. Unser rationales westliches Denken schliesst übernatürliches in der Regel aus. In ethnischen Gruppen kann Magie und übernatürliches zum Alltag gehören. Daher eignet sich Boyers Methode also trotzdem, oder erst recht?

\subsubsection*{Wandelnde Schloss}
Haus als zur ontologischen Kategorie Werkzeug gehörig. Damit erreichen wir den Bruch/Verletzung dadurch, dass das Haus eine Selbstbestimmung hat.

Schablone vom Gebäude, Fixe Umgebung welche sich dennoch bewegt. Noch deutlicher wird das beim Rübenkopf, welcher in keiner Weise die Lebendigkeit des Schlosses zeigt (Keine Mimik, nichts bewegliches) aber dennoch in die gleiche Schublage gehört, da er sich zumindest selbständig bewegen kann.
Obwohl ihm vorigen Kapitel mehrheitlich auf das Schloss und nicht auf Calcifer eingegangen wurde, trotzdem noch eine Bemerkung: In der Szene, wo man sieht, wie die Lichter vom Himmel fallen, von denen auch Calcifer eines ist, verpuffen die meisten. Die springen vielleicht noch ein paar wenige Male vom Boden ab, bevor sie endgültig verglühen. Diese Lichter alleine reichen nicht aus, um unsere Aufmerksamkeit auf sich zu ziehen (trotz unheimlich schöner Animation), da ihnen der Bruch ihrer Ontologie fehlt. \emph{Farbige Lichter fallen vom Himmel und springen in Gestalt von Menschen ein paar Mal auf dem Boden, bevor sie verglühen.} Anders verhält sich das mit Calcifer, welcher nachdem ihm von Hauro das Herz gegeben wurde und später durch Sophie ein Leben ohne Herz ermöglicht wurde.

Abgesehen von den äusseren Merkmale welche dem Schloss tierähnlicher machen kommen seine Bewegungen und seine Reaktionen. Durch das individuelle Bewegen und langsame Vergrössern und Verkleinern einzelner Teile gibt dem Schloss etwas organisches. 

Es stellt sich natürlich die Frage, inwiefern sich aus Calcifers Beseeltheit auch die des Schlosses schliessen lässt, da das Schloss von Calcifer gebaut und gesteuert wird, ist es im Prinzip Spiegel von Calcifer Zustand.

\subsubsection*{Mononoke} 
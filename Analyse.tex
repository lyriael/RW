%!TEX root = Animismus_in_Anime.tex
\subsection{Animismus in Anime}
Alt oder neuer Animismus. Japanischer Shinto. Wo zu finden und was heisst das?

\subsubsection*{Mononoke}
Da historisch passt sehr gut in Shinto und somit in den Geschichtlichen Animismus (in wie fern den alten). Harveys Ansatz vom respektvollen Leben mit andern people ist zentrale Botschaft des Filmes. Boyer lässt sich zwar Anwendung, aber ohne überraschende Erkenntnis. Tiergötter sind TIERE mit PERSONEN-Eigenschaften. Moro: ontologischer Bruch: hat eine Menschentochter. Interessant hier Shishigami: Auch wenn die gründe nicht erklärt werden, so geht man davon aus, dass der Gott entscheidetet wen er leben lässt und wen er tötet. Er muss also eine Art von Information besitzen. Anderes Beispiel aus Boyer: Tote und das Ritual -> Nago. Das NATUR OBJEKT toter Nago bewegt sich nicht mehr, aber er spricht! Die Seherin vollführt ein Ritual um sich und ihr Dorf vor dem Zorn des Gefallen Gottes zu schützen. Diese beiden helfen aber nur ein religiöses Phänomen zu identifizieren, jedoch nicht umbedingt zu erklären, warum das auch für uns westler so interessant ist. Oder ist es gerade, dass das Ritual anders ist, aber die Notwendigkeit (Schablone davon) in uns allen vorhanden und wir es deswegen verstehen? Letztlich noch kurz die falschen Wildschweine: Unheimlicfh auch für den Betrachter. Umgekehrt von den Tiergöttern: PERSON die aussieht wie TIER.

\subsubsection*{Wandelndes Schloss}
Typisch Animistisch gibt es wenig hier anzusetzten. Aus einer Zeit (Europa Epoche) wo man den Animismus nicht mehr gefolgt ist. Animismus gehört nicht mehr zur Narration. Man kann höchstens Animistische Elemente identifizieren. 


\subsection{Kognitiver Ansatz}
Welten unterscheidung: Fiktive Welt ist eventuell der Welt welcher viele Beispiele entnommen wurden näher. Unser rationales westliches Denken schliesst übernatürliches in der Regel aus. In ethnischen Gruppen kann Magie und übernatürliches zum Alltag gehören. Daher eignet sich Boyers Methode also trotzdem, oder erst recht?

\subsubsection*{Wandelnde Schloss}
Haus als zur ontologischen Kategorie Werkzeug gehörig. Damit erreichen wir den Bruch/Verletzung dadurch, dass das Haus eine Selbstbestimmung hat.

\subsubsection*{Mononoke} 